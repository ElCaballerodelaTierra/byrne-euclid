\documentclass[booklanguage=spanish]{byrnebook}
%\usepackage{lua-visual-debug}

\begin{document}

\thispagestyle{empty}

\begin{center}

\Large \uppercase{Los primeros seis libros de}

\LARGE \uppercase{Los elementos de Euclides}

\vskip 0.5\baselineskip

\normalsize \uppercase{en el que se utilizan diagramas y símbolos coloreados en lugar de letras para facilitar el aprendizaje}

\vskip 0.75\baselineskip

\Large \uppercase{Por Oliver Byrne}

%{\uppercase{SURVEYOR OF HER MAJESTY'S SETTLEMENTS IN THE FALKLAND ISLANDS AND AUTHOR OF NUMEROUS MATHEMATICAL WORKS}}

\defineNewPicture{
textLabels := false;
scaleFactor := 7/6;
angleScale := 4/3;
pair A, B, C, D, E, F, G, H, I, J, K, L, M, d[];
A := (0, 0);
B := A shifted (-7/10u, -8/7u);
C = whatever[A, A shifted ((A-B) rotated 90)] = whatever[B, B shifted dir(0)];
d1 := (B-A) rotated -90;
D := A shifted d1;
E := B shifted d1;
d2 := (A-C) rotated -90;
F := C shifted d2;
G := A shifted d2;
d3 := (C-B) rotated -90;
H := B shifted d3;
I := C shifted d3;
J = whatever[A, A shifted dir(90)];
J = whatever[B, C];
K = whatever[A, A shifted dir(90)];
K = whatever[H, I];
L = whatever[B, F];
L = whatever[A, C];
M = whatever[A, I];
M = whatever[B, C];
draw byPolygon(A,B,E,D)(byblack);
draw byPolygon(L,A,G,F)(byred);
draw byPolygon(C,L,F)(byred);
draw byPolygon(J,M,I,K)(byyellow);
draw byPolygon (M,C,I)(byyellow);
draw byPolygon(B,J,K,H)(byblue);
byAngleDefine(F, C, A, byyellow, SOLID_SECTOR);
byAngleDefine(B, C, I, byblue, SOLID_SECTOR);
byAngleDefine(A, C, B, byblack, SOLID_SECTOR);
draw byNamedAngleResized();
draw byLineFull(A, K, byred, 1, 0)(I, I, 1, 1, -1);
draw byLineFull(B, F, byblack, 0, 0)(G, G, 1, 1, -1);
draw byLineFull(A, I, byblack, 0, 0)(K, K, 1, 1, 1);
byLineDefine(C, F, byblue, DASHED_LINE, REGULAR_WIDTH);
byLineDefine(C, I, byblack, DASHED_LINE, REGULAR_WIDTH);
draw byNamedLineSeq(0)(CF,CI);
byLineDefine(A, B, byyellow, SOLID_LINE, REGULAR_WIDTH);
byLineDefine(B, C, byred, SOLID_LINE, REGULAR_WIDTH);
byLineDefine(C, A, byblue, SOLID_LINE, REGULAR_WIDTH);
draw byNamedLineSeq(1)(AB,BC,CA);
byLineDefine.CAb(C, A, byblack, SOLID_LINE, REGULAR_WIDTH);
byLineStylize (M, M, 1, 0, -1) (CAb);
byLineDefine.AMb(A, M, byblack, SOLID_LINE, REGULAR_WIDTH);
byLineStylize (C, C, 0, 1, -1) (AMb);
byLineDefine.BCb(B, C, byblack, SOLID_LINE, REGULAR_WIDTH);
byLineStylize (L, L, 0, 1, -1) (BCb);
byLineDefine.BLb(L, B, byblack, SOLID_LINE, REGULAR_WIDTH);
byLineStylize (C, C, 1, 0, -1) (BLb);
draw byLabelsOnPolygon(B, E, D, A, G, F, C, I, K, H)(ALL_LABELS, -1);
draw byLabelsOnPolygon(A, J, C)(OMIT_FIRST_LABEL+OMIT_LAST_LABEL, 1);
}

\vfill\vfill

~\hfill\drawCurrentPicture\hfill~

\vfill\vfill\vfill

\large github.com/jemmybutton
\vskip 0.25\baselineskip

\Large 2025 ed.\,0.8-latex-alpha
\vskip \baselineskip

\ccbysa 
\vskip 0.25\baselineskip

\footnotesize Esta traducción de \emph{Los primeros seis libros de los elementos de Euclides} de Oliver Byrne es realizada por Slyusarev Sergey y se distribuye bajo licencia CC-BY-SA~4.0

\vskip -\baselineskip

\normalsize
\end{center}
\pagebreak

\part*{Introducción}

\charspacing{-2}{\regularLettrine{L}{as} artes y ciencias se han vuelto tan extensas, que facilitar su adquisición es de tanta importancia como extender sus límites. La ilustración, si no acorta el tiempo de estudio, al menos lo hará más agradable. Este trabajo tiene un objetivo mayor que la mera ilustración; no introducimos colores con el propósito de entretener, o para entretener \emph{con ciertas combinaciones de matiz y forma}, % no sé de dónde es esto
pero para ayudar a la mente en sus investigaciones de la verdad, para aumentar las facilidades de introducción y para difundir el conocimiento permanente. Si quisiéramos autoridades para probar la importancia y utilidad de la geometría, podríamos citar a todos los filósofos desde el día de Platón. Entre los griegos, en la antigüedad, como en la escuela de Pestalozzi y otros en tiempos recientes, la geometría fue adoptada como el mejor gimnasio de la mente. De hecho, los Elementos de Euclides se han convertido, por consentimiento común, en la base de la ciencia matemática en todo el mundo civilizado. Pero esto no parecerá extraordinario, si consideramos que esta ciencia sublime no solo está mejor calculada que cualquier otra para despertar el espíritu de investigación, elevar la mente y fortalecer las facultades de razonamiento, sino que también constituye la mejor introducción a la mayoría de las vocaciones útiles e importantes de la vida humana. La aritmética, la topografía, la hidrostática, la neumática, la óptica, la astronomía física, \&c.\ dependen todas de las proposiciones de la geometría.}

\charspacing{-1}{Mucho sin embargo depende de la primera comunicación de cualquier ciencia a un alumno, aunque los métodos mejores y más fáciles rara vez se adoptan. Se presentan proposiciones a un estudiante, a quien, aunque tenga una comprensión suficiente, se le dice tan poco sobre ellas al entrar en el umbral mismo de la ciencia, que se le da una predisposición muy desfavorable para su futuro estudio de este tema delicioso; o \enquote{las formalidades y parafernalia del rigor se presentan de manera tan ostentosa, que casi ocultan la realidad. Las repeticiones interminables y desconcertantes, que no confieren mayor exactitud al razonamiento, hacen que las demostraciones sean complejas y oscuras, y ocultan a la vista del estudiante la sucesión de la evidencia.} % la cita parece ser de The First Six Books of the Elements of Euclid de Dionysius Lardner https://books.google.ru/books?id=YnkAAAAAMAAJ
Así se crea una aversión en la mente del alumno, y un tema tan calculado para mejorar las facultades de razonamiento y dar el hábito de pensar con atención se degrada por un curso de instrucción seco y rígido en un ejercicio poco interesante de la memoria. Despertar la curiosidad y despertar las facultades letárgicas y latentes de las mentes jóvenes debería ser el objetivo de todo maestro; pero donde faltan ejemplos de excelencia, los intentos de alcanzarla son pocos, mientras que la eminencia excita la atención y produce imitación. El objetivo de esta obra es introducir un método de enseñanza de la geometría que ha sido muy aprobado por muchos hombres de ciencia en este país, así como en Francia y América. El plan aquí adoptado apela fuertemente al ojo, el más sensible y el más completo de nuestros órganos externos, y su preeminencia para grabar su tema en la mente está respaldada por el axioma incontrovertible expresado en las conocidas palabras de Horacio:—}

\begin{center}
\emph{Segnius irritant animos demissa per aurem\\
Quam quae sunt oculis subjecta fidelibus}\\
\vskip 0.5\baselineskip
A feebler impress through the ear is made,\\
Than what is by the faithful eye conveyed.
\end{center}

\charspacing{-1}{Todo el lenguaje consiste en signos representativos, y los mejores signos son aquellos que cumplen sus propósitos con la mayor precisión y rapidez. Tales, para todos los propósitos comunes, son los signos audibles llamados palabras, que todavía se consideran audibles, ya sea que se dirijan inmediatamente al oído o a través del medio de las letras al ojo. Los diagramas geométricos no son signos, sino los materiales de la ciencia geométrica, cuyo objeto es mostrar las cantidades relativas de sus partes mediante un proceso de razonamiento llamado Demostración. Este razonamiento se ha llevado a cabo generalmente con palabras, letras y diagramas negros o incoloros; pero como el uso de símbolos, signos y diagramas coloreados en las artes y ciencias lineales hace que el proceso de razonamiento sea más preciso y la adquisición más rápida, se han adoptado en este caso en consecuencia.}

Tal es la expedición de este modo atractivo de comunicar conocimientos, que los Elementos de Euclides pueden adquirirse en menos de un tercio del tiempo habitualmente empleado, y la retención por la memoria es mucho más permanente; estos hechos han sido comprobados por numerosos experimentos realizados por el inventor y varios otros que han adoptado sus planes. Los detalles de los cuales son pocos y obvios; las letras anexas a puntos, líneas u otras partes de un diagrama son en realidad meros nombres arbitrarios y los representan en la demostración; en lugar de estos, las partes, al estar coloreadas de manera diferente, se nombran a sí mismas, ya que sus formas en colores correspondientes las representan en la demostración.

Para dar una mejor idea de este sistema y de las ventajas obtenidas por su adopción, tomemos un triángulo rectángulo y expresemos algunas de sus propiedades tanto con colores como con el método generalmente empleado.

\defineNewPicture{
	pair A, B, C;
	B := (0, 0);
	A := B shifted (dir(-145)*3u);
	C = whatever[A, A shifted (1,0)] = whatever[B, B shifted dir(-145+90)];
		byAngleDefine(A, B, C, byyellow, SOLID_SECTOR);
		byAngleDefine(B, C, A, byblue, SOLID_SECTOR);
		byAngleDefine(C, A, B, byred, SOLID_SECTOR);
		draw byNamedAngleResized();
		byLineDefine(A, B, byblue, SOLID_LINE, REGULAR_WIDTH);
		byLineDefine(B, C, byred, SOLID_LINE, REGULAR_WIDTH);
		byLineDefine(C, A, byyellow, SOLID_LINE, REGULAR_WIDTH);
		draw byNamedLineSeq(0)(AB,BC,CA);
		label.top(btex B etex, B);
		label.rt(btex C etex, C);
		label.lft(btex A etex, A);
	angleScale := 4/5;
}

\begin{center}
\drawCurrentPictureInMargin \emph{Algunas de las propiedades del triángulo rectángulo ABC, expresadas por el método generalmente empleado:}
\end{center}

\vskip 0.5\baselineskip

\begin{enumerate}
\item El ángulo BAC, junto con los ángulos BCA y ABC, es igual a dos ángulos rectos, o al doble del ángulo ABC.
\item El ángulo CAB sumado al ángulo ACB será igual al ángulo ABC.
\item El ángulo ABC es mayor que cualquiera de los ángulos BAC o BCA.
\item El ángulo BCA o el ángulo CAB es menor que el ángulo ABC.
\item Si del ángulo ABC se resta el ángulo BAC, el resto será igual al ángulo ACB.
\item El cuadrado de AC es igual a la suma de los cuadrados de AB y BC.
\end{enumerate}

\vskip 0.5\baselineskip

\begin{center}
\emph{Las mismas propiedades expresadas coloreando las diferentes partes:}
\end{center}

\vskip 0.5\baselineskip

\begin{enumerate}
\item $\drawAngle{A} + \drawAngle{B} + \drawAngle{C} = 2 \drawAngle{B} = \drawTwoRightAngles$. \\ Es decir, el ángulo rojo sumado al ángulo amarillo sumado al ángulo azul, es igual al doble del ángulo amarillo, igual a dos ángulos rectos.
\item $\drawAngle{A} + \drawAngle{C} = \drawAngle{B}$. \\ O en otras palabras, el ángulo rojo sumado al ángulo azul, es igual al ángulo amarillo.
\item $\drawAngle{B} > \drawAngle{A} \mbox{ or } > \drawAngle{C}$. \\ El ángulo amarillo es mayor que cualquiera de los ángulos rojo o azul.
\item $\drawAngle{A} \mbox{ or } \drawAngle{C} < \drawAngle{B}$. \\ El ángulo rojo o azul es menor que el ángulo amarillo.
\item $\drawAngle{B} - \drawAngle{C} = \drawAngle{A}$. \\ En otros términos, el ángulo amarillo menos el ángulo azul es igual al ángulo rojo.
\item $\drawUnitLine{CA}^2 = \drawUnitLine{AB}^2 + \drawUnitLine{BC}^2$. \\ Es decir, el cuadrado de la línea amarilla es igual a la suma de los cuadrados de las líneas azul y roja.
\end{enumerate}

En las demostraciones orales obtenemos con los colores esta ventaja importante: el ojo y el oído pueden ser atendidos al mismo tiempo, de modo que para enseñar geometría y otras artes y ciencias lineales en clases, el sistema es el mejor que se ha propuesto jamás; esto es evidente por los ejemplos dados.

\charspacing{-2}{De donde se deduce que una referencia del texto al diagrama es más rápida y segura, al dar las formas y colores de las partes, o al nombrar las partes y sus colores, que al nombrar las partes y las letras en el diagrama. Además de la simplicidad superior, este sistema también es conspicuous por su concentración, y excluye totalmente la práctica perjudicial, aunque prevalente, de permitir que el estudiante memorice la demostración; hasta que la razón, el hecho y la prueba solamente dejen impresiones en la comprensión.}

De nuevo, al dar una conferencia sobre los principios o propiedades de las figuras, si mencionamos el color de la parte o partes a las que nos referimos, como al decir, el ángulo rojo, la línea azul o las líneas, \&c, la parte o partes así nombradas serán vistas inmediatamente por toda la clase al mismo instante; no así si decimos el ángulo ABC, el triángulo PFQ, la figura EGKt, y así sucesivamente; porque las letras deben trazarse una por una antes de que los estudiantes organicen en sus mentes la magnitud particular a la que se refieren, lo que a menudo ocasiona confusión y error, así como pérdida de tiempo. Además, si las partes que se dan como iguales tienen los mismos colores en cualquier diagrama, la mente no se desviará del objeto que tiene ante sí; es decir, tal disposición presenta una demostración ocular de las partes que se deben probar como iguales, y el alumno retiene los datos durante todo el razonamiento. Pero cualesquiera que sean las ventajas del plan actual, si no se sustituye, siempre puede ser un poderoso auxiliar de los otros métodos, con el propósito de introducción, o de una reminiscencia más rápida, o de una retención más permanente por la memoria.

\charspacing{-2.5}{La experiencia de todos los que han formado sistemas para impresionar hechos en la comprensión, concuerda en probar que las representaciones coloreadas, como imágenes, grabados, diagramas, \&c.\ son más fáciles de fijar en la mente que las meras oraciones sin ninguna peculiaridad. Curioso como pueda parecer, los poetas parecen ser más conscientes de este hecho que los matemáticos; muchos poetas modernos aluden a este sistema visible de comunicar conocimiento, uno de ellos se ha expresado así:}

\vskip 0.5\baselineskip

\begin{center} % Once again, the same verse by Horace, this time translated by Isaac Watts https://archive.org/stream/improvementofmin00wattuoft#page/198/mode/2up
Los sonidos que se dirigen al oído se pierden y mueren\\
En una hora corta, pero estos que golpean el ojo,\\
Viven mucho en la mente, la vista fiel\\
Graba el conocimiento con un rayo de luz.
% I'd put something like this instead:
% For man loves knowledge, and the beams of Truth
% More welcome touch his understanding's eye,
% Than all the blandishments of sound his ear,
% Than all of taste his tongue...
% from The Pleasures Of Imagination by Mark Akenside (1721–1770) https://archive.org/stream/pleasuresofimagi00aken#page/50/mode/2up
\end{center}

\vskip 0.5\baselineskip

Esto quizás pueda considerarse la única mejora que ha recibido la geometría plana desde los días de Euclides, y si hubo algún geómetra de renombre antes de esa época, el éxito de Euclides ha eclipsado por completo su memoria, e incluso ha hecho que todas las cosas buenas de ese tipo se le atribuyan a él; como \AE sop entre los escritores de fábulas. También cabe señalar que, dado que los diagramas tangibles proporcionan el único medio a través del cual la geometría y otras artes lineales pueden enseñarse a los ciegos, el sistema visible es igualmente adecuado para las exigencias de los sordomudos.

\charspacing{-1}{Se debe tener cuidado de mostrar que el color no tiene nada que ver con las líneas, ángulos o magnitudes, excepto meramente para nombrarlos. Una línea matemática, que es longitud sin anchura, no puede poseer color, sin embargo, la unión de dos colores en el mismo plano da una buena idea de lo que se entiende por una línea matemática; recordemos que estamos hablando familiarmente, dicha unión debe entenderse y no el color, cuando decimos la línea negra, la línea roja o las líneas, \&c.}

Los colores y los diagramas coloreados pueden parecer al principio un método torpe para transmitir nociones adecuadas de las propiedades y partes de las figuras y magnitudes matemáticas, sin embargo, se encontrarán que ofrecen un medio más refinado y extenso que cualquiera que se haya propuesto hasta ahora.

Aquí definiremos un punto, una línea y una superficie, y demostraremos una proposición para mostrar la verdad de esta afirmación.

Un punto es aquello que tiene posición, pero no magnitud; o un punto es solo posición, abstraído de la consideración de longitud, anchura y grosor. Quizás la siguiente descripción esté mejor calculada para explicar la naturaleza del punto matemático a aquellos que no han adquirido la idea, que la definición anterior, aunque especiosa.

\defineNewPicture{
	angleScale := 2;
	pair O, A, B, C;
	O := (0, 0);
	A := dir(0) scaled 3u;
	B := dir(120) scaled 3u;
	C := dir(240) scaled 3u;
		draw byAngle(A, O, B, byred, SOLID_SECTOR);
		draw byAngle(B, O, C, byblue, SOLID_SECTOR);
		draw byAngle(C, O, A, byyellow, SOLID_SECTOR);
}
Que tres colores \drawCurrentPictureInMargin se encuentren y cubran una porción del papel, donde se encuentran no es azul, ni es amarillo, ni es rojo, ya que no ocupa ninguna porción del plano, porque si lo hiciera, pertenecería a la parte azul, roja o amarilla; sin embargo, existe y tiene posición sin magnitud, de modo que con un poco de reflexión, esta unión de tres colores en un plano da una buena idea de un punto matemático.

Una línea es longitud sin anchura. Con la ayuda de colores, casi de la misma manera que antes, se puede dar así una idea de una línea:—

\defineNewPicture{
	pair A, B, C, D, E, F;
	A := (0, 0);
	B := (5/2u, ypart(A));
	C := (xpart(A), -2u);
	D := (xpart(B), ypart(C));
	E := 1/2[A, C];
	F := 1/2[B, D];
		draw byPolygon(A,B,F,E)(byred);
		draw byPolygon(C,D,F,E)(byblue);
}
\drawCurrentPictureInMargin
Que dos colores se encuentren y cubran una porción del papel; donde se encuentran no es rojo, ni es azul; por lo tanto, la unión no ocupa ninguna porción del plano, y por lo tanto no puede tener anchura, sino solo longitud: de lo que podemos formarnos fácilmente una idea de lo que se entiende por una línea matemática. A efectos de ilustración, un color que difiera del color del papel, o del plano sobre el que está dibujado, habría sido suficiente; de ahí en adelante, si decimos la línea roja, la línea azul o las líneas, \&c.\ se entiende que son las uniones con el plano sobre el que están dibujadas.

\defineNewPicture{
	pair A', A'', A''', B', B'', B''', C', C'', C''', D', D'', D''', d[];
	d1 := (3/2u, 0);
	d2 := (-3/4u, -2/3u);
	d3 := (0, -3/2u);
	A' := (0, 0);
	B' := A' shifted d1;
	C' := A' shifted d2;
	D' := C' shifted d1;
	A'' := A' shifted d3;
	B'' := B' shifted d3;
	C'' := C' shifted d3;
	D'' := D' shifted d3;
	A''' := A'' shifted d3;
	B''' := B'' shifted d3;
	C''' := C'' shifted d3;
	D''' := D'' shifted d3;
		draw byPolygon(A',B',B'',A'',C'',C')(byred);
		draw byPolygon(A'',B'',D'',C'')(byblue);
		draw byPolygon(C'',D'',B'',B''',D''',C''')(byyellow);
		draw byLine(A''', B''', white, SOLID_LINE, 2);
		draw byLine(A''', C''', white, SOLID_LINE, 2);
		draw byLine(A''', A', white, SOLID_LINE, 2);
		draw byLine(D', C', white, SOLID_LINE, THIN_WIDTH);
		draw byLine(D', B', white, SOLID_LINE, THIN_WIDTH);
		draw byLine(D', D''', white, SOLID_LINE, THIN_WIDTH);
		label.lft(btex P etex, C');
		label.lft(btex R etex, C'');
		label.rt(btex S etex, B'');
		label.rt(btex Q etex, B''');
}
\drawCurrentPictureInMargin
Una superficie es aquello que tiene longitud y anchura sin grosor.

Cuando consideramos un cuerpo sólido (PQ), percibimos de inmediato que tiene tres dimensiones, a saber: longitud, anchura y grosor; supongamos que una parte de este sólido (PS) es roja, y la otra parte (QR) amarilla, y que los colores son distintos sin mezclarse, la superficie azul (RS) que separa estas partes, o que es lo mismo, que divide el sólido sin pérdida de material, debe carecer de grosor, y solo posee longitud y anchura; esto aparece claramente por un razonamiento similar al que se acaba de emplear para definir, o más bien describir, un punto y una línea.

La proposición que hemos seleccionado para elucidar la manera en que se aplican los principios, es la quinta del primer Libro.

\defineNewPicture[1/4]{
angleScale := 5/6;
pair A, B, C, D, E;
A := (0, 0);
B := A shifted (u, -2u);
C := B xscaled -1;
D := 9/5[A,B];
E := 9/5[A,C];
byAngleDefine(B, A, C, byblack, SOLID_SECTOR);
byAngleDefine(A, B, C, byblue, SOLID_SECTOR);
byAngleDefine(B, C, A, byblue, SOLID_SECTOR);
byAngleDefine(C, B, E, byyellow, SOLID_SECTOR);
byAngleDefine(D, C, B, byyellow, SOLID_SECTOR);
byAngleDefine(B, D, C, byred, SOLID_SECTOR);
byAngleDefine(C, E, B, byred, SOLID_SECTOR);
byAngleDefine(E, B, D, byblack, ARC_SECTOR);
byAngleDefine(D, C, E, byblack, ARC_SECTOR);
draw byNamedAngleResized(BAC,ABC,BCA,CBE,DCB,BDC,CEB);
byLineDefine(B, D, byyellow, SOLID_LINE, REGULAR_WIDTH);
byLineDefine(C, E, byyellow, SOLID_LINE, REGULAR_WIDTH);
byLineDefine(B, E, byblue, SOLID_LINE, REGULAR_WIDTH);
byLineDefine(C, D, byblue, SOLID_LINE, REGULAR_WIDTH);
byLineDefine(A, B, byred, SOLID_LINE, REGULAR_WIDTH);
byLineDefine(A, C, byred, SOLID_LINE, REGULAR_WIDTH);
byLineDefine(B, C, byblack, SOLID_LINE, REGULAR_WIDTH);
draw byNamedLineSeq(0)(CD,noLine,BC,noLine,BE,CE,AC,AB,BD);
label.top(btex A etex, A);
label.lft(btex C etex, C);
label.rt(btex B etex, B);
label.lft(btex E etex, E);
label.rt(btex D etex, D);
}
En un triángulo isósceles ABC, los ángulos internos en la base ABC, ACB son iguales, y cuando los lados AB, AC se prolongan, los ángulos externos en la base BCE, CBD también son iguales.

\begin{center}
\drawCurrentPictureInMargin Produce \drawUnitLine{AB} y \drawUnitLine{AC},\\
haz $\drawUnitLine{BD} = \drawUnitLine{CE}$, dibuja \drawUnitLine{BE} y \drawUnitLine{CD}.\\
En
\drawFromCurrentPicture{
draw byNamedAngle(BAC);
startAutoLabeling;
draw byNamedLineSeq(0)(BE,CE,AC,AB);
stopAutoLabeling;
}
and
\drawFromCurrentPicture{
draw byNamedAngle(BAC);
startAutoLabeling;
draw byNamedLineSeq(0)(BD,CD,AC,AB);
stopAutoLabeling;
}\\
we have $\drawUnitLine{AB,BD} = \drawUnitLine{AC,CE}$,\\
\drawAngle{BAC} common and $\drawUnitLine{AB} = \drawUnitLine{AC}$:\\
$\therefore \drawAngle{BCA,DCB} = \drawAngle{ABC,CBE}$, $\drawUnitLine{BE} = \drawUnitLine{CD}$\\
and $\drawAngle{CEB} = \drawAngle{BDC}$ \byref{prop:I.IV}.\\
Again in \drawLine{BC,BE,CE} and \drawLine{BC,BD,CD},\\
$\drawUnitLine{BD} = \drawUnitLine{CE}$, $\drawAngle{CEB} = \drawAngle{BDC}$\\
and $\drawUnitLine{BE} = \drawUnitLine{CD}$;\\
$\therefore \drawAngle{DCE,DCB} = \drawAngle{EBD,CBE}$
and $\drawAngle{DCB} = \drawAngle{CBE}$ \byref{prop:I.IV}\\
But $\drawAngle{BCA,DCB} = \drawAngle{ABC,CBE}$, $\therefore \drawAngle{BCA} = \drawAngle{ABC}$.
\end{center}

\qedNB

\begin{center}
\emph{By annexing Letters to the Diagram.}
\end{center}

Sea las lados iguales AB y AC producidos a través de los extremos BC, del tercer lado, y en la parte producida BD de cualquiera, sea asumido cualquier punto D, y del otro sea cortado AE igual a AD \byref{prop:I.III}. Sean los puntos E y D, así tomados en los lados producidos, conectados por líneas rectas DC y BE con los extremos alternos del tercer lado del triángulo.

En los triángulos DAC y EAB los lados DA y AC son respectivamente iguales a EA y AB, y el ángulo incluido A es común a ambos triángulos. Por lo tanto \byref{prop:I.IV} la línea DC es igual a BE, el ángulo ADC al ángulo AEB, y el ángulo ACD al ángulo ABE; si de las líneas iguales AD y AE se toman los lados iguales AB y AC, los residuos BD y CE serán iguales. Por lo tanto en los triángulos BDC y CEB, los lados BD y DC son respectivamente iguales a CE y EB, y los ángulos D y E incluidos por esos lados también son iguales. Por lo tanto \byref{prop:I.IV} los ángulos DBC y ECB, que son los incluidos por el tercer lado BC y las producciones de los lados iguales AB y AC son iguales. También los ángulos DCB y EBC son iguales si esos iguales se toman de los ángulos DCA y EBA antes probados iguales, los residuos, que son los ángulos ABC y ACB opuestos a los lados iguales, serán iguales.

\emph{Por lo tanto en un triángulo isósceles,} \&c.

\qedNB

\charspacing{-2}{Nuestro objeto en este lugar es introducir el sistema en lugar de enseñar cualquier conjunto particular de proposiciones, por lo tanto hemos seleccionado las anteriores del curso regular. Para escuelas y otros lugares públicos de instrucción, las tizas teñidas servirán para describir los diagramas, \&c.\ para uso privado los lápices de colores resultarán muy convenientes.}

Estamos felices de encontrar que los Elementos de las Matemáticas ahora forman una parte considerable de toda educación femenina sólida, por lo tanto llamamos la atención de aquellos interesados o comprometidos en la educación de las damas a este modo muy atractivo de comunicar conocimiento, y al trabajo sucesivo para su futuro desarrollo.

\charspacing{-2.5}{Nosotros concluiremos por el momento observando, ya que los sentidos de la vista y el oído pueden ser tan fuertemente e instantáneamente abordados por igual con mil como con uno, \emph{el millón} podría ser enseñado geometría y otras ramas de las matemáticas con gran facilidad, esto avanzaría el propósito de la educación más que cualquier cosa \emph{podría} ser nombrada, porque enseñaría a la gente cómo pensar, y no qué pensar; es en este particular donde se origina el gran error de la educación.}

\chapter*{Elucidaciones}

La geometría tiene como objetos principales la exposición y explicación de las propiedades de la \emph{figura}, y la figura se define como la relación que existe entre los límites del espacio. El espacio o la magnitud es de tres clases, \emph{lineal}, \emph{superficial} y \emph{sólido}.

\defineNewPicture{
	pair A, B, C;
	numeric s;
	s := 3/2u;
	A := (0, s);
	B := (1/2s, 0);
	C := B xscaled -1;
		draw byAngle.A(B, A, C, byyellow, SOLID_SECTOR);
		byLineDefine(B, A, byblue, SOLID_LINE, REGULAR_WIDTH);
		byLineDefine(C, A, byred, SOLID_LINE, REGULAR_WIDTH);
		draw byNamedLineSeq(0)(CA,BA);
		label.urt(btex A etex, A);
}\drawCurrentPictureInMargin
Los ángulos podrían considerarse apropiadamente como una cuarta especie de magnitud. La magnitud angular evidentemente consta de partes, y por lo tanto debe admitirse que es una especie de cantidad. El estudiante no debe suponer que la magnitud de un ángulo es afectada por la longitud de las líneas rectas que lo incluyen y de cuya divergencia mutua es la medida. El \emph{vértice} de un ángulo es el punto donde se encuentran los \emph{lados} de las \emph{patas} del ángulo, como A.

\defineNewPicture{
	pair B, C, D, E, F, G, H;
	numeric s;
	s := 5/4u;
	C := (0, 0);
	B := dir(0)*s;
	D := dir(50)*s;
	E := dir(-30)*s;
	F := E scaled -1;
	G := D scaled -1;
	H := B scaled -1;
	angleScale := 4/3;
		draw byAngle(E, C, B, byyellow, SOLID_SECTOR);
		draw byAngle(B, C, D, byblack, SOLID_SECTOR);
		draw byAngle(D, C, F, byblue, SOLID_SECTOR);
		draw byAngle(F, C, H, byred, SOLID_SECTOR);
		draw byAngle(H, C, G, byyellow, ARC_SECTOR);
		draw byAngle(G, C, E, byblue, ARC_SECTOR);
		draw byLine(B, H, byblue, SOLID_LINE, REGULAR_WIDTH);
		draw byLine(D, G, byred, SOLID_LINE, REGULAR_WIDTH);
		draw byLine(E, F, byblack, SOLID_LINE, REGULAR_WIDTH);
		label.bot(btex C etex, C shifted (0, -3pt));
		label.bot(btex B etex, B);
		label.lrt(btex D etex, D);
		label.llft(btex F etex, F);
		label.bot(btex H etex, H);
		label.lrt(btex G etex, G);
		label.llft(btex E etex, E);
}
\drawCurrentPictureInMargin
\charspacing{-2}{Un ángulo se designa a menudo por una sola letra cuando sus lados son las únicas líneas que se encuentran en su vértice. Así, las líneas roja y azul forman el ángulo amarillo, que en otros sistemas se llamaría ángulo A. Pero cuando más de dos líneas se encuentran en el mismo punto, era necesario por métodos anteriores, para evitar confusiones, emplear tres letras para designar un ángulo alrededor de ese punto, la letra que marcaba el vértice del ángulo se colocaba siempre en el medio. Así, las líneas negra y roja que se encuentran en C, forman el ángulo azul, y ha sido usualmente denominado el ángulo FCD o DCF. Las líneas FC y CD son los lados del ángulo; el punto C es su vértice. De manera similar, el ángulo negro se designaría el ángulo DCB o BCD. Los ángulos rojo y azul sumados, o el ángulo HCF sumado a FCD, forman el ángulo HCD; y así de otros ángulos.}

Cuando los lados de un ángulo se producen o prolongan más allá de su vértice, los ángulos formados por ellos en ambos lados del vértice se dicen que son \emph{opuestos por el vértice} entre sí: así, los ángulos rojo y amarillo se dicen que son ángulos opuestos por el vértice.

\charspacing{-2}{\emph{La superposición} es el proceso por el cual una magnitud puede concebirse colocada sobre otra, de modo que la cubra exactamente, o de modo que cada parte de cada una coincida exactamente.}

Una línea se dice que es \emph{producida}, cuando es extendida, prolongada, o tiene su longitud aumentada, y el aumento de longitud que recibe se llama \emph{parte producida}, o su \emph{producción}.

\charspacing{-2.5}{La longitud total de la línea o líneas que encierran una figura, se llama su \emph{perímetro}. Los primeros seis libros de Euclides tratan solo de figuras planas. Una línea trazada desde el centro de un círculo hasta su circunferencia, se llama \emph{radio}. El lado de un triángulo rectángulo, que es opuesto al ángulo recto, se llama \emph{hipotenusa}. Un oblongo se define en el segundo libro, y se llama \emph{rectángulo}. Se supone que todas las líneas consideradas en los primeros seis libros de los Elementos están en el mismo plano.}

\charspacing{-3}{La \emph{regla} y el \emph{compás} son los únicos instrumentos cuyo uso está permitido en Euclides, o en la Geometría plana. Declarar esta restricción es el objeto de los \emph{postulados.}}

Los \emph{axiomas} de la geometría son ciertas proposiciones generales, cuya verdad se toma como autoevidente e incapaz de ser establecida por demostración.

Las \emph{proposiciones} son aquellos resultados que se obtienen en geometría por un proceso de razonamiento. Hay dos especies de proposiciones en geometría, \emph{problemas} y \emph{teoremas}.

Un \emph{problema} es una proposición en la que se propone hacer algo; como una línea que se va a trazar bajo algunas condiciones dadas, un círculo que se va a describir, alguna figura que se va a construir, \&c.

La \emph{solución} del problema consiste en mostrar cómo se puede hacer la cosa requerida con la ayuda de la regla o straight-edge y el compás.

La \emph{demostración} consiste en probar que el proceso indicado en la solución alcanza el fin requerido.

Un \emph{teorema} es una proposición en la que se afirma la verdad de algún principio. Este principio debe deducirse de los axiomas y definiciones, u otras verdades previamente y de forma independiente establecidas. Mostrar esto es el objeto de la demostración.

Un \emph{problema} es análogo a un postulado.

Un \emph{teorema} se asemeja a un axioma.

Un \emph{postulado} es un problema, cuya solución se asume.

Un \emph{axioma} es un teorema, cuya verdad se concede sin demostración.

Un \emph{corolario} es una inferencia deducida inmediatamente de una proposición.

Un \emph{escolio} es una nota u observación sobre una proposición que no contiene una inferencia de importancia suficiente para merecer el nombre de \emph{corolario}.

\charspacing{-2}{Un \emph{lema} es una proposición introducida meramente con el propósito de establecer una proposición más importante.}

\chapter*{Símbolos y abreviaturas}

\symb{$\therefore$}
expresa la palabra \emph{por lo tanto}.

\symb{$\because$}
expresa la palabra \emph{porque}.

\symb{$=$}
expresa la palabra \emph{igual}. Este signo de igualdad puede leerse \emph{igual a}, o \emph{es igual a}, o \emph{son iguales a}; pero la discrepancia con respecto a la introducción de los verbos auxiliares \emph{es}, \emph{son}, \&c.\ no puede afectar el rigor geométrico.

\symb{$\neq$}
significa lo mismo que si se escribieran las palabras \emph{‘no igual’}.

\symb{$>$}
significa \emph{mayor que}.

\symb{$<$}
significa \emph{menor que}.

\symb{$\ngtr$}
significa \emph{no mayor que}.

\symb{$\nless$}
significa \emph{no menor que}.

\symb{$+$}
se lee \emph{más}, el signo de adición; cuando se interpone entre dos o más magnitudes, significa su suma.

\symb{$-$}
se lee \emph{menos}, significa la resta; y cuando se coloca entre dos cantidades, denota que la última se toma de la primera.

\symb{$\times$}
este signo expresa el producto de dos o más números cuando se coloca entre ellos en aritmética y álgebra; pero en geometría se usa generalmente para expresar un \emph{rectángulo}, cuando se coloca entre \enquote{dos líneas rectas que contienen uno de sus ángulos rectos.} Un \emph{rectángulo} también puede representarse colocando un punto entre dos de sus lados contiguos.

\symb{$:\ ::\ :$}
expresa una \emph{analogía} o \emph{proporción}; así, si A, B, C y D representan cuatro magnitudes, y A tiene con B la misma razón que C tiene con D, la proporción se escribe brevemente así

$A : B :: C : D$, $A : B = C : D$, or $\dfrac{A}{B} = \dfrac{C}{D}$.

Esta igualdad o semejanza de razón se lee,

como A es a B, así es C a D;

o A es a B, como C es a D.

\symb{$\parallel$}
significa \emph{paralelo a}.

\symb{$\perp$}
significa \emph{perpendicular a}.

\defineNewPicture{
	pair A, B, C, D;
	numeric s;
	s := 3/2u;
	A := (0, 0);
	B := dir(0)*s;
	C := dir(50)*s;
	D := dir(90)*s;
	byAngleDefine(B, A, C, byblack, ARC_SECTOR);
	byAngleDefine(B, A, D, byblack, ARC_SECTOR);
	byPointLabelRemove(A,B,C,D);
}

\symb{\drawAngle{BAC}}
significa \emph{ángulo}.

\symb{\drawAngle{BAD}}
significa \emph{ángulo recto}.

\symb{\drawTwoRightAngles}
significa \emph{dos ángulos rectos}.

\defineNewPicture{
	pair A, B, C, D;
	A := (0, -1/4u);
	B := (u, 0);
	C := (-u, 0);
	D := (0, u);
		byLineDefine (A, D, byblack, SOLID_LINE, REGULAR_WIDTH);
		byLineDefine (B, D, byblack, SOLID_LINE, REGULAR_WIDTH);
		byLineDefine (C, D, byblack, SOLID_LINE, REGULAR_WIDTH);
	byPointLabelRemove(A,D);
}

\symb{\drawFromCurrentPicture{
draw byNamedLine(AD);
draw byNamedLineSeq(0)(BD,CD);
}
or
\drawFromCurrentPicture{
draw byNamedLineSeq(0)(AD,BD);
}}
briefly designates a \emph{point}.

The square described on a line is concisely written thus, $\drawUnitLine{AD}^2$.

In the same manner twice the square of, is expressed by $2 \cdot \drawUnitLine{AD}^2$.

\symb{\indefstr}
significa \emph{definición}.

\symb{\inpoststr}
significa \emph{postulado}.

\symb{\inaxstr}
significa \emph{axioma}.

\symb{hyp.}
significa \emph{hipótesis}. Puede ser necesario aquí remarkar que \emph{hipótesis} es la condición asumida o dada por sentada. Así, la hipótesis de la proposición dada en la Introducción, es que el triángulo es isósceles, o que sus lados son iguales.

\symb{\conststr}
significa \emph{construcción}. La \emph{construcción} es el cambio hecho en la figura original, dibujando líneas, haciendo ángulos, describiendo círculos, \&c.\ para adaptarla al argumento de la demostración o a la solución del problema. Las condiciones bajo las cuales se hacen estos cambios, son tan indiscutibles como las contenidas en la hipótesis. Por ejemplo, si hacemos un ángulo igual a un ángulo dado, estos dos ángulos son iguales por construcción.

\symb{\qedstr}
significa \emph{Quod erat demonstrandum}. Lo que se quería demostrar.

\part{Libro I}

\chapter*{Definiciones}

\startdefinition{}\label{def:I.I}

\begin{center}
Un \emph{punto} es el que no tiene partes.
\end{center}

\startdefinition{}\label{def:I.II}

\begin{center}
Una \emph{línea} es longitud sin ancho.
\end{center}

\startdefinition{}\label{def:I.III}

\begin{center}
Los extremos de una línea son puntos.
\end{center}

\startdefinition{}\label{def:I.IV}

\begin{center}
Una línea recta o recta es la que se encuentra uniformemente entre sus extremos.
\end{center}

\startdefinition{}\label{def:I.V}

\begin{center}
Una superficie es aquella que sólo tiene longitud y ancho.
\end{center}

\startdefinition{}\label{def:I.VI}

\begin{center}
Los extremos de una superficie son líneas.
\end{center}

\startdefinition{}\label{def:I.VII}

\begin{center}
Una superficie plana es la que se encuentra uniformemente entre sus extremos.
\end{center}

\startdefinition{}\label{def:I.VIII}

\defineNewPicture{
	pair A, B, C;
	A := (0, 0);
	B := (2/3u, 2/3u);
	C := (u, ypart(A));
		byAngleDefine(B, A, C, byyellow, SOLID_SECTOR);
		draw byNamedAngleResized();
		byLineDefine(A, B, byblue, SOLID_LINE, REGULAR_WIDTH);
		byLineDefine(A, C, byred, SOLID_LINE, REGULAR_WIDTH);
		draw byNamedLineSeq(0)(AC,AB);
}
\drawCurrentPictureInMargin
\begin{center}
Un ángulo plano es la inclinación de dos líneas entre sí, que se encuentran en un plano, pero no están en la misma dirección.
\end{center}

\startdefinition{}\label{def:I.IX}

\defineNewPicture{
	pair A, B, C, D;
	A := (0, 0);
	B := (u, 0);
	C := (0, 2/3u);
	D := (-u, 0);
		byAngleDefine(B, A, C, byblack, ARC_SECTOR);
		byAngleDefine(D, A, C, byblack, ARC_SECTOR);
		draw byNamedAngleResized();
		draw byLine(D, B, byblack, SOLID_LINE, REGULAR_WIDTH);
		draw byLine(A, C, byblack, SOLID_LINE, REGULAR_WIDTH);
}
\drawCurrentPictureInMargin
\begin{center}
Un ángulo rectilíneo plano es la inclinación de dos líneas rectas entre sí, que se encuentran, pero no están en la misma línea recta.
\end{center}

\startdefinition{}\label{def:I.X}

\defineNewPicture{
	pair A, B, C;
	A := (0, 0);
	B := (-2/3u, 2/3u);
	C := (u, ypart(A));
		byAngleDefine(B, A, C, byred, SOLID_SECTOR);
		draw byNamedAngleResized();
		byLineDefine(A, B, byyellow, SOLID_LINE, REGULAR_WIDTH);
		byLineDefine(A, C, byblue, SOLID_LINE, REGULAR_WIDTH);
		draw byNamedLineSeq(0)(AC,AB);
}
\drawCurrentPictureInMargin
\begin{center}
Cuando una línea recta parada sobre otra línea recta hace iguales los ángulos adyacentes, cada uno de estos ángulos es llamado \emph{ángulo recto}, y se dice que cada una de estas líneas es \emph{perpendicular} a la otra.
\end{center}

\startdefinition{}\label{def:I.XI}

\defineNewPicture{
	pair A, B, C;
	A := (0, 0);
	B := (2/3u, 2/3u);
	C := (u, ypart(A));
		byAngleDefine(B, A, C, byblue, SOLID_SECTOR);
		draw byNamedAngleResized();
		byLineDefine(A, B, byyellow, SOLID_LINE, REGULAR_WIDTH);
		byLineDefine(A, C, byred, SOLID_LINE, REGULAR_WIDTH);
		draw byNamedLineSeq(0)(AC,AB);
}
\drawCurrentPictureInMargin
\begin{center}
Un ángulo obtuso es un ángulo mayor que un ángulo recto.
\end{center}

\startdefinition{}\label{def:I.XII}

\begin{center}
Un ángulo agudo es menor que un ángulo recto.
\end{center}

\startdefinition{}\label{def:I.XIII}

\begin{center}
Un borde o límite es el extremo de cualquier cosa.
\end{center}

\startdefinition{}\label{def:I.XIV}

\defineNewPicture{
	pair O, A, B, C, D, E;
	numeric r;
	r := 1/2u;
	O := (0, 0);
	A := dir(0) scaled r;
	B := dir(60) scaled r;
	C := dir(130) scaled r;
	D := dir(180) scaled r;
	E := dir(-60) scaled r;
		draw byLine(O, B)(byblack, SOLID_LINE, REGULAR_WIDTH);
		draw byLine(O, C)(byred, SOLID_LINE, REGULAR_WIDTH);
		draw byLine(O, E)(byyellow, SOLID_LINE, REGULAR_WIDTH);
		draw byLine(D, A)(byblue, SOLID_LINE, REGULAR_WIDTH);
		draw byCircleR(O, r, byred, 0, 0, 0);
}
\drawCurrentPictureInMargin
\begin{center}
Una figura es una superficie encerrada en todos los lados por una línea o líneas.
\end{center}

\startdefinition{}\label{def:I.XV}

\begin{center}
Un  es una figura plana, delimitada por una línea continua, llamada circunferencia o periferia; y tiene un cierto punto dentro de él, desde el cual todas las líneas rectas dibujadas a su circunferencia son iguales.
\end{center}

\startdefinition{}\label{def:I.XVI}

\defineNewPicture{
	pair O, A, B;
	numeric r;
	r := 1/2u;
	O := (0, 0);
	A := dir(0) scaled r;
	B := dir(180) scaled r;
		draw byLine(A, B)(byyellow, SOLID_LINE, REGULAR_WIDTH);
		draw byCircleR(O, r, byred, 0, 0, 0);
}
\drawCurrentPictureInMargin
\begin{center}
Este punto (desde el cual se dibujan las líneas iguales) se llama el centro del círculo.
\end{center}

\startdefinition{}\label{def:I.XVII}

\defineNewPicture{
	pair O, A, B;
	numeric r;
	r := 1/2u;
	O := (0, 0);
	A := dir(0) scaled r;
	B := dir(180) scaled r;
		draw byLine(A, B)(byblue, SOLID_LINE, REGULAR_WIDTH);
		draw byArc(O, A, B)(r, byyellow, 0, 0, 0, 0);
		draw byArc(O, B, A)(r, byyellow, 1, 0, 0, 0);
}
\drawCurrentPictureInMargin
\begin{center}
El  es una línea recta dibujada a través del centro, terminada en ambos sentidos en la circunferencia.
\end{center}

\startdefinition{}\label{def:I.XVIII}

\defineNewPicture{
	pair O, A, B;
	path P;
	numeric r;
	r := 1/2u;
	P := fullcircle scaled 2r;
	O := (0, 0);
	A := point 1 of P;
	B := point 3 of P;
		draw byLine(A, B)(byred, SOLID_LINE, REGULAR_WIDTH);
		draw byArc(O, A, B)(r, byblue, 0, 0, 0, 0);
		draw byArc(O, B, A)(r, byblue, 1, 0, 0, 0);
}
\drawCurrentPictureInMargin
\begin{center}
Un  es la figura contenida por el diámetro, y la parte del círculo cortada por el diámetro.
\end{center}

\startdefinition{}\label{def:I.XIX}

\begin{center}
Un  es una figura contenida por una línea recta y la parte de la circunferencia que corta.
\end{center}

\startdefinition{}\label{def:I.XX}

\begin{center}
Una figura contenida solo por líneas rectas, se llama figura rectilínea.
\end{center}

\startdefinition{}\label{def:I.XXI}

\defineNewPicture{
	pair A, B, C, D;
	A := (0, 0);
	B := (u, 1/2u);
	C := (-1/2u, -4/3u);
	D := (4/3u, -u);
		draw byLine(C, B)(byred, SOLID_LINE, REGULAR_WIDTH);
		draw byLine(A, D)(byblue, SOLID_LINE, REGULAR_WIDTH);
		byLineDefine(A, B, byyellow, SOLID_LINE, REGULAR_WIDTH);
		byLineDefine(A, C, byyellow, SOLID_LINE, REGULAR_WIDTH);
		byLineDefine(B, D, byyellow, SOLID_LINE, REGULAR_WIDTH);
		byLineDefine(C, D, byblack, SOLID_LINE, REGULAR_WIDTH);
		draw byNamedLineSeq(0)(BD,CD,AC,AB);
		draw byLabelsOnPolygon(A, B, D, C)(ALL_LABELS, 0);
}
\drawCurrentPictureInMargin
\begin{center}
Un triángulo es una figura rectilínea contenida por tres lados.
\end{center}

\startdefinition{}\label{def:I.XXII}

\begin{center}
Una figura cuadrilátera es aquella que está limitada por cuatro lados. Las líneas rectas \drawUnitLine{AD} y \drawUnitLine{CB} que conectan los vértices de los ángulos opuestos de una figura cuadrilátera, son denominadas sus diagonales.
\end{center}

\startdefinition{}\label{def:I.XXIII}

\defineNewPicture{
	pair A, B, C;
	A := dir(-30) scaled 1/2u;
	B := dir(-150) scaled 1/2u;
	C := dir(90) scaled 1/2u;
		byLineDefine(A, B, byblue, SOLID_LINE, REGULAR_WIDTH);
		byLineDefine(B, C, byred, SOLID_LINE, REGULAR_WIDTH);
		byLineDefine(C, A, byyellow, SOLID_LINE, REGULAR_WIDTH);
		draw byNamedLineSeq(0)(AB,BC,CA);
}
\drawCurrentPictureInMargin
\begin{center}
Un polígono es una figura rectilínea delimitada por más de cuatro lados.
\end{center}

\startdefinition{}\label{def:I.XXIV}

\defineNewPicture{
	pair A, B, C;
	A := dir(-60) scaled 1/2u;
	B := dir(-120) scaled 1/2u;
	C := dir(90) scaled 1/2u;
		byLineDefine(A, B, byblue, SOLID_LINE, REGULAR_WIDTH);
		byLineDefine(B, C, byred, SOLID_LINE, REGULAR_WIDTH);
		byLineDefine(C, A, byred, SOLID_LINE, REGULAR_WIDTH);
		draw byNamedLineSeq(0)(AB,BC,CA);
}
\drawCurrentPictureInMargin
\begin{center}
Un triángulo cuyos tres lados son iguales, se dice ser equilátero.
\end{center}

\startdefinition{}\label{def:I.XXV}

\begin{center}
Un triángulo que tiene  es llamado triángulo isósceles.
\end{center}

\startdefinition{}\label{def:I.XXVI}

\defineNewPicture{
	pair A, B, C;
	A := (0, 0);
	B := (-u, 0);
	C := (0, 3/4u);
		byLineDefine(A, B, byred, SOLID_LINE, REGULAR_WIDTH);
		byLineDefine(B, C, byyellow, SOLID_LINE, REGULAR_WIDTH);
		byLineDefine(C, A, byblue, SOLID_LINE, REGULAR_WIDTH);
		draw byNamedLineSeq(0)(AB,BC,CA);
}
\drawCurrentPictureInMargin
\begin{center}
Un triángulo escaleno es uno que no tiene dos lados iguales.
\end{center}

\startdefinition{}\label{def:I.XXVII}

\defineNewPicture{
	pair A, B, C;
	A := (-1/4u, 0);
	B := (-u, 0);
	C := (0, 3/4u);
		byLineDefine(A, B, byred, SOLID_LINE, REGULAR_WIDTH);
		byLineDefine(B, C, byblue, SOLID_LINE, REGULAR_WIDTH);
		byLineDefine(C, A, byyellow, SOLID_LINE, REGULAR_WIDTH);
		draw byNamedLineSeq(0)(AB,BC,CA);
}
\drawCurrentPictureInMargin
\begin{center}
Un triángulo rectángulo es el que tiene un .
\end{center}

\startdefinition{}\label{def:I.XXVIII}

\defineNewPicture{
	pair A, B, C;
	A := (0, 0);
	B := (-u, 0);
	C := (-1/4u, 3/4u);
		byLineDefine(A, B, byblue, SOLID_LINE, REGULAR_WIDTH);
		byLineDefine(B, C, byyellow, SOLID_LINE, REGULAR_WIDTH);
		byLineDefine(C, A, byred, SOLID_LINE, REGULAR_WIDTH);
		draw byNamedLineSeq(0)(AB,BC,CA);
}
\drawCurrentPictureInMargin
\begin{center}
Un triángulo obtusángulo es aquel que tiene un .
\end{center}

\startdefinition{}\label{def:I.XXIX}

\defineNewPicture{
	pair A, B, C, D;
	numeric s;
	s := u;
	A := (0, 0);
	B := (s, 0);
	C := (0, s);
	D := (s, s);
		byLineDefine(A, B, byred, SOLID_LINE, REGULAR_WIDTH);
		byLineDefine(A, C, byblue, SOLID_LINE, REGULAR_WIDTH);
		byLineDefine(B, D, byyellow, SOLID_LINE, REGULAR_WIDTH);
		byLineDefine(C, D, byblack, SOLID_LINE, REGULAR_WIDTH);
		draw byNamedLineSeq(0)(AB,AC,CD,BD);
}
\drawCurrentPictureInMargin
\begin{center}
Un triángulo acutángulo es aquel que tiene tres ángulos agudos.
\end{center}

\startdefinition{}\label{def:I.XXX}

\defineNewPicture{
	pair A, B, C, D;
	numeric s;
	s := u;
	A := (0, 0);
	B := (s, 0);
	C := A shifted (dir(80) scaled s);
	D := B shifted (dir(80) scaled s);
		byLineDefine(A, B, byred, SOLID_LINE, REGULAR_WIDTH);
		byLineDefine(A, C, byblue, SOLID_LINE, REGULAR_WIDTH);
		byLineDefine(B, D, byyellow, SOLID_LINE, REGULAR_WIDTH);
		byLineDefine(C, D, byblack, SOLID_LINE, REGULAR_WIDTH);
		draw byNamedLineSeq(0)(AB,AC,CD,BD);
}
\drawCurrentPictureInMargin
\begin{center}
De las figuras de cuatro lados, un cuadrado es aquel que tiene todos sus lados iguales y todos sus ángulos, ángulos rectos.
\end{center}

\startdefinition{}\label{def:I.XXXI}

\defineNewPicture{
	pair A,B,C,D;
	numeric s;
	s := u;
	A := (0, 0);
	B := (4/3s, 0);
	C := (0, 3/4s);
	D := (4/3s, 3/4s);
		byLineDefine(A, B, byblue, SOLID_LINE, REGULAR_WIDTH);
		byLineDefine(A, C, byred, SOLID_LINE, REGULAR_WIDTH);
		byLineDefine(B, D, byred, SOLID_LINE, REGULAR_WIDTH);
		byLineDefine(C, D, byblue, SOLID_LINE, REGULAR_WIDTH);
		draw byNamedLineSeq(0)(AB,AC,CD,BD);
}
\drawCurrentPictureInMargin
\begin{center}
Un rombo es aquel que tiene todos sus lados iguales, pero sus ángulos no son ángulos rectos.
\end{center}

\startdefinition{}\label{def:I.XXXII}

\defineNewPicture{
	pair A, B, C, D;
	numeric s;
	s := u;
	A := (0, 0);
	B := (s, 0);
	C := (1/4s, 3/4s);
	D := (s + 1/4s, 3/4s);
		byLineDefine(A, B, byblue, SOLID_LINE, REGULAR_WIDTH);
		byLineDefine(A, C, byred, SOLID_LINE, REGULAR_WIDTH);
		byLineDefine(B, D, byred, SOLID_LINE, REGULAR_WIDTH);
		byLineDefine(C, D, byblue, SOLID_LINE, REGULAR_WIDTH);
		draw byNamedLineSeq(0)(AB,AC,CD,BD);
}
\drawCurrentPictureInMargin
\begin{center}
Un oblongo es aquel que tiene todos sus ángulos, ángulos rectos, pero no tiene todos sus lados iguales.
\end{center}

\startdefinition{}\label{def:I.XXXIII}

\begin{center}
Un romboide es aquel que tiene sus lados opuestos iguales entre sí, pero todos sus lados no son iguales, ni sus ángulos son ángulos rectos.
\end{center}

\startdefinition{}\label{def:I.XXXIV}

\defineNewPicture{
	pair A, B, C, D;
	numeric s;
	s := u;
	A := (0, 0);
	B := (4/3s, 0);
	C := (0, 1/2s);
	D := (4/3s, 1/2s);
		draw byLine(A, B, byred, SOLID_LINE, REGULAR_WIDTH);
		draw byLine(C, D, byyellow, SOLID_LINE, REGULAR_WIDTH);
}
\drawCurrentPictureInMargin
\begin{center}
Todas las demás figuras cuadriláteras se llaman trapecios.
\end{center}

\startdefinition{}\label{def:I.XXXV}

\begin{center}
Las líneas rectas paralelas son como las que se encuentran en el mismo plano, y que prolongadas continuamente en ambas direcciones, nunca se encontrarán.
\end{center}

\chapter*{Postulates}

\startpostulate{}\label{post:I.I}

\begin{center}
Deje que sea aceptado que una línea recta puede ser dibujada desde cualquier punto a cualquier otro punto.
\end{center}

\startpostulate{}\label{post:I.II}

\begin{center}
Deje que sea aceptado que una línea recta finita puede ser prolongada a cualquier longitud en una línea recta.
\end{center}

\startpostulate{}\label{post:I.III}

\begin{center}
Deje que sea aceptado que un círculo puede ser trazado con cualquier centro a cualquier distancia de ese centro.
\end{center}

\chapter*{Axioms}

\startaxiom{}\label{ax:I.I}

\begin{center}
Las magnitudes que son iguales a lo mismo son iguales entre sí.
\end{center}

\startaxiom{}\label{ax:I.II}

\begin{center}
Si se suma igual a igual, las sumas serán iguales.
\end{center}

\startaxiom{}\label{ax:I.III}

\begin{center}
Si se quita igual a igual, el residuo será igual.
\end{center}

\startaxiom{}\label{ax:I.IV}

\begin{center}
Si se agregan iguales a desiguales, las sumas serán desiguales.
\end{center}

\startaxiom{}\label{ax:I.V}

\begin{center}
Si se eliminan iguales de desiguales, el residuo será desigual.
\end{center}

\startaxiom{}\label{ax:I.VI}

\begin{center}
Los dobles de lo mismo o magnitudes iguales son iguales.
\end{center}

\startaxiom{}\label{ax:I.VII}

\begin{center}
Las mitades de lo mismo o magnitudes iguales son iguales.
\end{center}

\startaxiom{}\label{ax:I.VIII}

\begin{center}
Las magnitudes que coinciden entre sí, o que llenan exactamente el mismo espacio, son iguales.
\end{center}

\startaxiom{}\label{ax:I.IX}

\begin{center}
El todo es mayor que su parte.
\end{center}

\startaxiom{}\label{ax:I.X}

\begin{center}
Dos líneas rectas no pueden contener un espacio.
\end{center}

\startaxiom{}\label{ax:I.XI}

\begin{center}
Todos los ángulos rectos son iguales.
\end{center}

\startaxiom{}\label{ax:I.XII}

\defineNewPicture{
	pair A, B, C, D, E, F, G, H;
	numeric s;
	s := 3/2u;
	A := (0, 0);
	B := (4/3s, 0);
	C := (0, s);
	D := (4/3s, s);
	E := (1/3s, 8/6s);
	F := (xpart(E), -2/6s);
	G = whatever[A, B] = whatever[E, F];
	H = whatever[C, D] = whatever[E, F];
		byAngleDefine(B, G, E, byred, SOLID_SECTOR);
		byAngleDefine(D, H, F, byyellow, SOLID_SECTOR);
		draw byNamedAngleResized();
		draw byLine(A, B, byblue, SOLID_LINE, REGULAR_WIDTH);
		draw byLine(C, D, byred, SOLID_LINE, REGULAR_WIDTH);
		draw byLine(E, F, byblack, SOLID_LINE, REGULAR_WIDTH);
		draw byLabelLine(0)(AB, CD, EF);
		draw byLabelsOnPolygon(E, H, D)(OMIT_FIRST_LABEL+OMIT_LAST_LABEL, 0);
		draw byLabelsOnPolygon(B, G, F)(OMIT_FIRST_LABEL+OMIT_LAST_LABEL, 0);
}
\drawCurrentPictureInMargin
\begin{center}
Si dos líneas rectas
                    (\drawUnitLine{AB}) se encuentran con una tercera línea recta
                    (\drawUnitLine{CD})
                    para hacer que los dos ángulos interiores
                    (\drawUnitLine{EF}
                    y
                    \drawAngle{H})
                    en el mismo lado sean menores que dos ángulos rectos, estas dos líneas rectas se encontrarán si se prolongan en el lado en el que los ángulos son menos de dos ángulos rectos.

El duodécimo axioma puede ser expresado en cualquiera de las siguientes maneras:

La geometría tiene por objeto principal la exposición y explicación de las propiedades de la \emph{figura}, y la figura se define como la relación que subsiste entre los límites del espacio. El espacio o magnitud es de tres tipos, \emph{lineal}, \emph{superficial} y \emph{sólido}.

Los ángulos pueden considerarse propiamente como una cuarta especie de magnitud. La magnitud angular evidentemente consiste en partes y, por lo tanto, debe admitirse que es una especie de cantidad. El alumno no debe suponer que la magnitud de un ángulo se ve afectada por la longitud de las líneas rectas que lo incluyen y de cuya divergencia mutua es la medida. El \emph{vértice} de un ángulo es el punto donde los \emph{lados} o las \emph{patas} del ángulo se encuentran, como \drawAngle{G}.

Un ángulo a menudo se designa con una sola letra cuando sus patas son las únicas líneas que se unen en su vértice. Por lo tanto, las líneas roja y azul forman el ángulo amarillo, que en otros sistemas sería llamado el . Pero cuando más de dos líneas se encuentran en el mismo punto, era necesario por métodos anteriores, para evitar confusiones, emplear tres letras para designar un ángulo sobre ese punto, la letra que marcaba el vértice del ángulo siempre se colocaba en la mitad. Por lo tanto, las líneas  y  que se juntan en  forman el  y se ha denominado habitualmente ángulo . Las líneas  y  son las patas del ángulo; El punto  es su vértice. Del mismo modo, el  se designaría como el ángulo . Los ángulos  y  sumados, o el  agregado a , hacen el ; y así de los otros ángulos.

Cuando las patas de un ángulo se dirigen o se prolongan más allá de su vértice, los ángulos hechos por ellas en ambos lados del vértice, se dicen ser \emph{verticalmente opuestos} entre sí: por lo tanto, los ángulos rojo y amarillo se dicen ser ángulos verticalmente opuestos.

La \emph{superposición} es el proceso por el cual una magnitud puede ser concebida para ser colocada sobre otra, para cubrirla exactamente, o para que cada parte de cada una coincida exactamente.

Se dice que se \emph{prolonga} una línea cuando esta es extendida, prolongada o su longitud ha aumentado, y el aumento de longitud que recibe se denomina su \emph{parte prolongada} o \emph{prolongación}.

La longitud total de la línea o líneas que encierran una figura es llamada \emph{perímetro}. Los primeros seis libros de Euclides tratan solo de figuras planas. Una línea dibujada desde el centro de un círculo hasta su circunferencia es llamada \emph{radio}. Las líneas que contienen una figura son llamadas \emph{lados}. Ese lado de un triángulo rectángulo, que es opuesto al ángulo recto, se llama \emph{hipotenusa}. Un \emph{oblongo} se define en el segundo libro, llamado \emph{rectángulo}. Todas las líneas que se consideran en los primeros seis libros de los Elementos se supone que están en el mismo plano.

La \emph{regla} y el \emph{compás} son los únicos instrumentos, cuyo uso está permitido en Euclides, o geometría plana. Declarar esta restricción es el objeto de los \emph{postulados}.

Los \emph{axiomas} de la geometría son ciertas proposiciones generales, cuya verdad se considera evidente e incapaz de establecerse mediante demostración.

Las \emph{proposiciones} son aquellos resultados que se obtienen en geometría mediante un proceso de razonamiento. Hay dos especies de proposiciones en geometría, \emph{problemas} y \emph{teoremas}.

Un \emph{problema} es una proposición en la que se propone hacer algo; como una línea para ser dibujada bajo ciertas condiciones, un círculo para ser trazado, alguna figura para ser construida, etcétera.

La \emph{solución} del problema consiste en mostrar cómo se puede hacer la cosa requerida con la ayuda de la regla y el compás.

La \emph{demostración} consiste en probar que el proceso indicado en la solución realmente alcanza el fin requerido.

Un \emph{teorema} es una proposición en la cual la verdad de algún principio es afirmada. Este principio debe deducirse de los axiomas y definiciones, u otras verdades establecidas previa e independientemente. Mostrar esto es el objeto de la demostración.

Un \emph{problema} es análogo a un postulado.

Un \emph{teorema} se asemeja a un axioma.

Un \emph{postulado} es un problema, cuya solución se supone.

Un \emph{axioma} es un teorema, cuya verdad se otorga sin demostración.

Un \emph{corolario} es una inferencia deducida inmediatamente de una proposición.

Un \emph{escolio} es una nota u observación sobre una proposición que no contiene una inferencia de importancia suficiente para darle el nombre de un \emph{corolario}.

Un \emph{lema} es una proposición simplemente introducida para el propósito de establecer una proposición más importante
\end{center}

\chapter*{Propositions}

\startproblem{Prop. I. Prob.}\label{prop:I.I}

\defineNewPicture[1/2]{
	pair A, B, C;
	path P[];
	numeric r;
	r := 3/2u;
	A := (0, 0);
	B := (r, 0);
	P1 := fullcircle scaled 2r;
	P2 := fullcircle scaled 2r shifted B;
	C := P1 intersectionpoint P2;
		byLineDefine(A, B, byblack, SOLID_LINE, REGULAR_WIDTH);
		byLineDefine(B, C, byred, SOLID_LINE, REGULAR_WIDTH);
		byLineDefine(C, A, byyellow, SOLID_LINE, REGULAR_WIDTH);
		draw byNamedLineSeq(-1)(AB,CA,BC);
		draw byCircle.A(A, B, byblue, 0, 0, 1/2);
		draw byCircle.B(B, A, byred, 0, 0, 1/2);
		draw byLabelsOnPolygon(A, C, B)(ALL_LABELS, 1);
}
\drawCurrentPictureInMargin
\problem{E}{n}{ una línea recta finita dada (\drawUnitLine{AB}) para trazar un triángulo equilátero.}

\begin{center}
Traza \offsetPicture{15pt}{0pt}{\drawFromCurrentPicture{
draw byNamedLine(AB);
draw byNamedCircle(A);
draw byLabelLineEnd(A, B, 0);
draw byLabelLineEnd(B, A, 1);
}} y \offsetPicture{15pt}{0pt}{\drawFromCurrentPicture{
draw byNamedLine(AB);
draw byNamedCircle(B);
draw byLabelLineEnd(A, B, 1);
draw byLabelLineEnd(B, A, 0);
}} \byref{post:I.III};\\
dibuja \drawUnitLine{CA} y \drawUnitLine{BC} \byref{post:I.I}.\\
Entonces \drawLine[bottom][triangleABC]{AB,CA,BC} será equilátero.

Para \drawUnitLine{BC} $=$ \drawLine[bottom][triangleABC]{AB,CA,BC} \byref{def:I.XV};
y \drawUnitLine{AB} $=$ \drawUnitLine{CA} \byref{def:I.XV},
$\therefore$ \drawUnitLine{AB} $=$ \drawUnitLine{BC} \byref{ax:I.I};

y por lo tanto \drawUnitLine{CA} es el triángulo equilátero requerido.

Q. E. D. \drawUnitLine{BC} \triangleABC
\end{center}

\qed

\startproblem{Prop. II. Prob.}\label{prop:I.II}

\defineNewPicture{
pair A, B, C, D, E, F;
path P[];
numeric r[];
A := (0, 0);
B := (-3/5u, -3/5u);
C := (-2u, -1/3u);
r1 := abs(A-B);
D := (fullcircle scaled 2r1 shifted A) intersectionpoint (fullcircle scaled 2r1 shifted B);
r2 := abs(B-C);
r3 := r1 + r2;
P1 := fullcircle scaled 2r2 shifted B;
P2 := fullcircle scaled 2r3 shifted D;
E := (D -- 10[D, B]) intersectionpoint P1;
F := (D -- 10[D, A]) intersectionpoint P2;
byLineDefine(A, B, byblack, DASHED_LINE, REGULAR_WIDTH);
byLineDefine(B, C, byblack, SOLID_LINE, REGULAR_WIDTH);
byLineDefine(B, D, byred, SOLID_LINE, REGULAR_WIDTH);
byLineDefine(D, A, byred, SOLID_LINE, REGULAR_WIDTH); % improvement: change style of either BD or DA
byLineDefine(B, E, byyellow, SOLID_LINE, REGULAR_WIDTH);
byLineDefine(A, F, byblue, SOLID_LINE, REGULAR_WIDTH);
draw byNamedLineSeq(0)(AB,BC);
draw byNamedLineSeq(0)(BE,BD,DA,AF);
draw byCircle.A(D, E, byred, 0, 0, 1/2);
draw byCircle.B(B, C, byblue, 0, 0, -1/2);
draw byLabelsOnPolygon(E, D, A, F)(OMIT_FIRST_LABEL+OMIT_LAST_LABEL, 1);
draw byLabelsOnPolygon(E, B, C)(OMIT_FIRST_LABEL+OMIT_LAST_LABEL, 1);
draw byLabelsOnCircle(C)(B);
draw byLabelsOnCircle(E, F)(A);
}
\drawCurrentPictureInMargin
\problem{D}{e}{un punto dado (\drawFromCurrentPicture[middle][pointA]{
startGlobalRotation(-lineAngle.DA);
draw byNamedPointLines(A,"AB");
stopGlobalRotation;
}), dibujar una línea recta igual a una línea recta finita dada (\drawUnitLine{BC}).}

\begin{center}
Dibuja \drawUnitLine{AB} \byref{post:I.I}, traza \drawFromCurrentPicture[bottom]{
startAutoLabeling;
startTempScale(scaleFactor*3);
startGlobalRotation(180-lineAngle.AB);
draw byNamedLineSeq(0)(AB,BD,DA);
stopGlobalRotation;
stopTempScale;
stopAutoLabeling;
} \byref{prop:I.I}, prolonga \drawUnitLine{BD} \byref{post:I.II}, traza \drawFromCurrentPicture{
draw byNamedLine (BC);
draw byNamedCircle(B);
draw byLabelLineEnd(B, C, 0);
draw byLabelLineEnd(C, B, 0);
} \byref{post:I.III}, y \drawFromCurrentPicture{
draw byNamedLineSeq(0)(BD, BE);
draw byNamedCircle(A);
draw byLabelLineEnd(D, E, 0);
draw byLabelLineEnd(E, D, 1);
} \byref{post:I.III}; prolonga \drawUnitLine{DA} \byref{post:I.II}, entonces \drawUnitLine{AF} es la línea requerida.

Para \drawUnitLine{BE,BD} $=$ \drawUnitLine{DA,AF} \byref{def:I.XV}, y \drawUnitLine{BD} $=$ \drawUnitLine{DA} (conſt.), $\therefore$ \drawUnitLine{BE} $=$ \drawUnitLine{AF} \byref{\constref}, pero \byref{post:I.III} \drawUnitLine{BC} $=$ \drawUnitLine{BE} $=$ \drawUnitLine{AF}; $\therefore$ \drawUnitLine{AF} dibujada de un punto dado (\pointA), es igual a la línea dada \drawUnitLine{BC}.

Q. E. D. \byref{def:I.XV}
\end{center}

\qed

\startproblem{Prop. III. Prob.}\label{prop:I.III}

\defineNewPicture{
pair A, B, C, D, E, F;
path P;
numeric r;
A := (0, 0);
r := 7/4u;
B := A shifted (r, 0);
C := A shifted (4/3r, 0);
D := A shifted dir(30)*r;
E := A shifted (7/6r, -1/6r);
F := A shifted (7/6r, -7/6r);
byLineDefine(A, B, byblack, SOLID_LINE, REGULAR_WIDTH);
byLineDefine(B, C, byblack, DASHED_LINE, REGULAR_WIDTH);
byLineDefine(A, D, byred, SOLID_LINE, REGULAR_WIDTH);
draw byNamedLineSeq(0)(BC,AB,AD);
draw byLine(E, F, byblue, SOLID_LINE, REGULAR_WIDTH);
draw byCircle.A(A, D, byblue, 0, 0, 0);
draw byLabelsOnPolygon(B, A, D)(OMIT_FIRST_LABEL+OMIT_LAST_LABEL, 1);
draw byLabelLineEnd(D, A, 0);
draw byLabelLineEnd(C, A, 0);
draw byLabelPoint(B, angle(B-A) + 45, 2);
draw byLabelsOnPolygon(E, F)(ALL_LABELS, 0);
}
\drawCurrentPictureInMargin
\problem{D}{esde}{ la mayor (\drawUnitLine{AB,BC}) de dos líneas rectas dadas, cortar una parte igual a la menor (\drawUnitLine{EF}).}

\begin{center}
Dibuja \drawUnitLine{AD} $=$ \drawUnitLine{EF} \byref{prop:I.II}, traza \drawFromCurrentPicture{
draw byNamedLine (AD);
draw byNamedCircle(A);
draw byLabelLineEnd(D, A, 0);
draw byLabelLineEnd(A, D, 0);
} \byref{post:I.III}, entonces \drawUnitLine{EF} $=$ \drawUnitLine{AB}.

Para \drawUnitLine{AD} $=$ \drawUnitLine{AB} \byref{def:I.XV},
\drawUnitLine{EF} $=$ \drawUnitLine{AD} (conſt.);
$\therefore$ \drawUnitLine{EF} $=$ \drawUnitLine{AB} \byref{\constref}.

Q. E. D. \byref{ax:I.I}
\end{center}

\qed

\starttheorem{Prop. IV. Theor.}\label{prop:I.IV}

\defineNewPicture[1/5]{
pair A, B, C, D, E, F, d;
A := (0, 0);
B := A shifted (-5/2u, -7/2u);
C := A shifted (1/2u, -3u);
d := (0, -4u);
D := A shifted d;
E := B shifted d;
F := C shifted d;
byAngleDefine(B, A, C, byyellow, SOLID_SECTOR);
byAngleDefine(A, B, C, byblue, SOLID_SECTOR);
byAngleDefine(B, C, A, byred, SOLID_SECTOR);
draw byNamedAngleResized(BAC, ABC, BCA);
byLineDefine(A, B, byred, SOLID_LINE, REGULAR_WIDTH);
byLineDefine(B, C, byblack, SOLID_LINE, REGULAR_WIDTH);
byLineDefine(C, A, byblue, SOLID_LINE, REGULAR_WIDTH);
draw byNamedLineSeq(0)(CA,BC,AB);
byAngleDefine(E, D, F, byyellow, SOLID_SECTOR);
byAngleDefine(D, E, F, byblue, SOLID_SECTOR);
byAngleDefine(E, F, D, byred, SOLID_SECTOR);
draw byNamedAngleResized(EDF, DEF, EFD);
byLineDefine(D, E, byred, SOLID_LINE, THIN_WIDTH);
byLineDefine(E, F, byblack, SOLID_LINE, THIN_WIDTH);
byLineDefine(F, D, byblue, SOLID_LINE, THIN_WIDTH);
draw byNamedLineSeq(0)(FD,EF,DE);
draw byLabelsOnPolygon(F, E, D)(ALL_LABELS, 0);
draw byLabelsOnPolygon(B, A, C)(ALL_LABELS, 1);
}
\drawCurrentPictureInMargin
\problem{S}{i}{ dos triángulos tienen dos lados del uno respectivamente igual a dos lados del otro, (\drawUnitLine{AB} para \drawUnitLine{DE} y \drawUnitLine{CA} para \drawUnitLine{FD}) y los ángulos (\drawAngle{A} y \drawAngle{D}) contenidos por esos lados iguales, también iguales; entonces sus bases o sus lados (\drawUnitLine{BC} y \drawUnitLine{EF}) también son iguales, y los ángulos restantes opuestos a lados iguales son respectivamente iguales (\drawAngle{B} $=$ \drawAngle{E} y \drawAngle{C} $=$ \drawAngle{F}) y los triángulos son iguales en todos los aspectos.}

\begin{center}
Deja que los dos triángulos sean concebidos para ser colocados, que el vértice de uno de los ángulos iguales, \drawAngle{A} o \drawAngle{D}; caerá sobre él del otro, y \drawUnitLine{AB} coincida con \drawUnitLine{DE}, entonces \drawUnitLine{CA} coincidirá con \drawUnitLine{FD}, si aplica; consecuentemente \drawUnitLine{BC} coincidirá con \drawUnitLine{EF}, o dos líneas rectas encerrarán un espacio, lo cual es imposible \byref{ax:I.X}, por lo tanto \drawUnitLine{BC} $=$ \drawUnitLine{EF}, \drawAngle{B} $=$ \drawAngle{E} y \drawAngle{C} $=$ \drawAngle{F}, y como los triángulos \drawLine{CA,BC,AB} y \drawLine{FD,EF,DE} coinciden, cuando aplican, son iguales en todos los aspectos.

Q. E. D.
\end{center}

\qed

\starttheorem{Prop. V. Theor.}\label{prop:I.V}

\defineNewPicture{
pair A, B, C, D, E;
A := (0, 0);
B := A shifted (u, -2u);
C := B xscaled -1;
D := 9/5[A,B];
E := 9/5[A,C];
byAngleDefine(B, A, C, byblack, SOLID_SECTOR);
byAngleDefine(A, B, C, byblue, SOLID_SECTOR);
byAngleDefine(B, C, A, byblue, SOLID_SECTOR);
byAngleDefine(C, B, E, byyellow, SOLID_SECTOR);
byAngleDefine(D, C, B, byyellow, SOLID_SECTOR);
byAngleDefine(B, D, C, byred, SOLID_SECTOR);
byAngleDefine(C, E, B, byred, SOLID_SECTOR);
byAngleDefine(E, B, D, byblack, ARC_SECTOR);
byAngleDefine(D, C, E, byblack, ARC_SECTOR);
draw byNamedAngleResized(BAC,ABC,BCA,CBE,DCB,BDC,CEB);
byLineDefine(B, D, byyellow, SOLID_LINE, REGULAR_WIDTH);
byLineDefine(C, E, byyellow, SOLID_LINE, REGULAR_WIDTH);
byLineDefine(B, E, byblue, SOLID_LINE, REGULAR_WIDTH);
byLineDefine(C, D, byblue, SOLID_LINE, REGULAR_WIDTH);
byLineDefine(A, B, byred, SOLID_LINE, REGULAR_WIDTH);
byLineDefine(A, C, byred, SOLID_LINE, REGULAR_WIDTH);
byLineDefine(B, C, byblack, SOLID_LINE, REGULAR_WIDTH);
draw byNamedLineSeq(0)(CD,noLine,BC,noLine,BE,CE,AC,AB,BD);
draw byLabelsOnPolygon(E, C, A, B, D, C, B)(ALL_LABELS, 0);
}
\drawCurrentPictureInMargin
\problem{E}{n}{ cualquier triángulo isósceles \drawLine[bottom]{BC,AC,AB} si se prolongan los lados iguales, los ángulos externos en la base son iguales, y los ángulos internos en la base también son iguales.}

\begin{center}
Prolonga \drawUnitLine{AB}, y \drawUnitLine{AC}, \byref{post:I.II}, toma \drawUnitLine{BD} $=$ \drawUnitLine{CE}, \byref{prop:I.III}, dibuja \drawUnitLine{BE} y \drawUnitLine{CD}.

Entonces en \drawFromCurrentPicture{
startAutoLabeling;
draw byNamedAngle(BAC);
draw byNamedLineSeq(0)(BE,CE,AC,AB);
stopAutoLabeling;
} y \drawFromCurrentPicture{
startAutoLabeling;
draw byNamedAngle(BAC);
draw byNamedLineSeq(0)(BD,CD,AC,AB);
stopAutoLabeling;
} tenemos,
                     \\ \drawUnitLine{AB,BD} $=$ \drawUnitLine{AC,CE} (const.), \drawAngle{BAC} es común a
                     \\ ambos, y \drawUnitLine{AB} $=$ \drawUnitLine{AC} (hip.), $\therefore$ \drawAngle{BCA,DCB} $=$ \drawAngle{ABC,CBE} ,
 \\ \drawUnitLine{BE} $=$ \drawUnitLine{CD} and \drawAngle{CEB} $=$ \drawAngle{BDC} \byref{\constref}.

De nuevo en \drawLine{BE,CE,BC} y \drawLine{BD,CD,BC} tenemos \drawUnitLine{BD} $=$ \drawUnitLine{CE},
 \\ \drawAngle{CEB} $=$ \drawAngle{BDC} y \drawUnitLine{BE} $=$ \drawUnitLine{CD},
 \\ $\therefore$ \drawAngle{DCE,DCB} $=$ \drawAngle{EBD,CBE} y \drawAngle{DCB} $=$ \drawAngle{CBE} \byref{\hypref} pero
                     \\ \drawAngle{BCA,DCB} $=$ \drawAngle{ABC,CBE} , $\therefore$ \drawAngle{BCA} $=$ \drawAngle{ABC} \byref{prop:I.IV}

Q. E. D. \byref{prop:I.IV} \byref{ax:I.III}
\end{center}

\qed

\starttheorem{Prop VI. Theor.}\label{prop:I.VI}

\defineNewPicture[1/4]{
pair A, B, C, D;
A := (0, 0);
B := A shifted (7/2u, 0);
D := 1/2[A,B] shifted (0, 3u);
C := 2/3[A, D];
byAngleDefine(B, A, D, byyellow, SOLID_SECTOR);
byAngleDefine(A, B, D, byblack, SOLID_SECTOR);
draw byNamedAngleResized();
byLineDefine(B, C, byyellow, SOLID_LINE, REGULAR_WIDTH);
byLineDefine(A, B, byred, SOLID_LINE, REGULAR_WIDTH);
byLineDefine(B, D, byblue, SOLID_LINE, REGULAR_WIDTH);
byLineDefine(C, A, byblack, SOLID_LINE, REGULAR_WIDTH);
byLineDefine(C, D, byblack, DASHED_LINE, REGULAR_WIDTH);
draw byNamedLine(BC);
draw byNamedLineSeq(0)(CA,CD,BD,AB);
draw byLabelsOnPolygon(A, C, D, B)(ALL_LABELS, 0);
}
\drawCurrentPictureInMargin
\problem{E}{n}{ cualquier triángulo (\drawLine[bottom][triangleABD]{CA,CD,BD,AB}) si dos ángulos (\drawAngle{A} y \drawAngle{B}) son iguales, los lados opuestos (\drawUnitLine{CA,CD} y \drawUnitLine{BD}) a ellos también son iguales.}

\begin{center}
Si los lados no son iguales, deje que uno de ellos \drawUnitLine{CA,CD} sea mayor que el otro \drawUnitLine{BD}, y de él corte \drawUnitLine{CA} $=$ \drawUnitLine{BD} \byref{prop:I.III}, dibuje \drawUnitLine{BC}.

Entonces \drawLine[bottom]{BC,AB,CA} y \triangleABD, \drawUnitLine{CA} $=$ \drawUnitLine{BD}, (const.) \drawAngle{A}$=$ \drawAngle{B} (hip.) y \drawUnitLine{AB} común, $\therefore$ los triángulos son iguales \byref{\constref} una parte igual al todo, lo cual es absurdo; $\therefore$ ninguno de los lados \drawUnitLine{CA,CD} or \drawUnitLine{BD} es mayor que el otro, $\therefore$ por eso son iguales.

Q. E. D. \byref{\hypref} \byref{prop:I.IV}
\end{center}

\qed

\starttheorem{Prop VII. Theor.}\label{prop:I.VII}

\defineNewPicture{
pair A, B, C, D, E, F, G, H;
A := (0, 0);
B := A shifted (4u, 0);
C := A shifted (u, 3u);
D := C shifted (7/4u, 0);
E := 1/2[C, D] yscaled -0.7;
F := E shifted (0, -2u);
G := 5/4[A, E];
H := 5/4[A, F];
byAngleDefine.C(B, C, A, byblack, SOLID_SECTOR);
byAngleDefine(D, C, B, byred, SOLID_SECTOR);
byAngleDefine.D(A, D, B, byyellow, SOLID_SECTOR);
byAngleDefine(C, D, A, byblue, SOLID_SECTOR);
byAngleDefine(B, F, H, byblack, SOLID_SECTOR);
byAngleDefine(B, F, E, byred, SOLID_SECTOR);
byAngleDefine(B, E, G, byyellow, SOLID_SECTOR);
byAngleDefine(G, E, F, byblue, SOLID_SECTOR);
draw byNamedAngleResized();
draw byLine(C, D, byblack, DASHED_LINE, REGULAR_WIDTH);
draw byLine(E, F, byblack, DASHED_LINE, REGULAR_WIDTH);
draw byLine(A, B, byblack, SOLID_LINE, REGULAR_WIDTH);
byLineDefine(B, C, byblue, SOLID_LINE, REGULAR_WIDTH);
byLineDefine(C, A, byred, SOLID_LINE, REGULAR_WIDTH);
byLineDefine(B, D, byblue, SOLID_LINE, REGULAR_WIDTH);
byLineDefine(D, A, byred, SOLID_LINE, REGULAR_WIDTH);
byLineDefine(B, E, byblue, SOLID_LINE, REGULAR_WIDTH);
byLineDefine(E, A, byred, SOLID_LINE, REGULAR_WIDTH);
byLineDefine(B, F, byblue, SOLID_LINE, REGULAR_WIDTH);
byLineDefine(F, A, byred, SOLID_LINE, REGULAR_WIDTH);
byLineDefine(E, G, byred, DASHED_LINE, REGULAR_WIDTH);
byLineDefine(F, H, byred, DASHED_LINE, REGULAR_WIDTH);
draw byNamedLine(EG,FH);
draw byNamedLineSeq(0)(BC,CA,EA,BE);
draw byNamedLineSeq(0)(BD,DA,FA,BF);
byPointLabelDefine(F, "C");
byPointLabelDefine(E, "D");
draw byLabelsOnPolygon(F, A, C, D, B, F, noPoint)(OMIT_FIRST_LABEL+OMIT_LAST_LABEL, 0);
draw byLabelsOnPolygon(A, E, B)(OMIT_FIRST_LABEL+OMIT_LAST_LABEL, 0);
draw byLabelsOnPolygon(H, F, A)(OMIT_FIRST_LABEL+OMIT_LAST_LABEL, 0);
}
\drawCurrentPictureInMargin
\problem{E}{n}{ la misma base (\drawUnitLine{AB}) y en el mismo lado de la misma, no puede haber dos triángulos que tengan sus lados adyacentes (\drawUnitLine{CA} y \drawUnitLine{DA}, \drawUnitLine{BC} y \drawUnitLine{BD}) en ambos extremos de la base, iguales entre sí.}

\begin{center}
Cuando dos \emph{triángulos} se están situados en la misma base, y en el mismo lado, el vértice de uno debe caer fuera del otro triángulo o dentro de él; o, finalmente, en uno de sus lados.

Si es posible, deje que los dos triángulos se construyan de modo que
                


\drawUnitLine{CA} $=$ \drawUnitLine{DA}
\drawUnitLine{BC} $=$ \drawUnitLine{BD}


,
                entonce dibuja \drawUnitLine{CD} y,

\drawAngle{C,DCB} $=$ \drawAngle{CDA} \byref{prop:I.V}
$\therefore$ \drawAngle{DCB} < \drawAngle{CDA} y




$\therefore$
\drawAngle{DCB} <
\drawAngle{CDA,D} 

                                        pero \byref{prop:I.V}
                                        \drawAngle{DCB} $=$
\drawAngle{CDA,D} 



                            lo que es absurdo,

por lo tanto, los dos triángulos no pueden tener sus lados adyacentes iguales en ambos extremos de la base.

Q. E. D.
\end{center}

\qed

\starttheorem{Prop VIII. Theor.}\label{prop:I.VIII}

\defineNewPicture{
pair A, B, C, D, E, F, d;
A := (0, 0);
B := A shifted (-u, -4u);
C := A shifted (3/2u, -3u);
d := (0, -9/2u);
D := A shifted d;
E := B shifted d;
F := C shifted d;
byAngleDefine(F, D, E, byblack, SOLID_SECTOR);
byAngleDefine(C, A, B, byblack, SOLID_SECTOR);
draw byNamedAngleResized();
byLineDefine(A, B, byred, SOLID_LINE, REGULAR_WIDTH);
byLineDefine(B, C, byblack, SOLID_LINE, REGULAR_WIDTH);
byLineDefine(C, A, byblue, SOLID_LINE, REGULAR_WIDTH);
byLineDefine(D, E, byred, SOLID_LINE, THIN_WIDTH);
byLineDefine(E, F, byblack, SOLID_LINE, THIN_WIDTH);
byLineDefine(F, D, byblue, SOLID_LINE, THIN_WIDTH);
draw byNamedLineSeq(0)(CA,BC,AB);
draw byNamedLineSeq(0)(FD,EF,DE);
draw byLabelsOnPolygon(C, B, A)(ALL_LABELS, 0);
draw byLabelsOnPolygon(F, E, D)(ALL_LABELS, 0);
}
\drawCurrentPictureInMargin
\problem{S}{i}{dos triángulos tienen dos lados del uno, respectivamente, iguales a dos lados del otro (\drawUnitLine{CA} $=$ \drawUnitLine{FD} y \drawUnitLine{AB} $=$ \drawUnitLine{DE}), y también sus bases (\drawUnitLine{BC} $=$ \drawUnitLine{EF}), iguales; entonces los ángulos (\drawFromCurrentPicture{
startAutoLabeling;
startGlobalRotation(-getAttribute("angle","Direction","A"));
draw byNamedAngleWithDummySides(A);
stopGlobalRotation;
stopAutoLabeling;
} y \drawFromCurrentPicture{
startAutoLabeling;
startGlobalRotation(-getAttribute("angle","Direction","D"));
draw byNamedAngleWithDummySides(D);
stopGlobalRotation;
stopAutoLabeling;
}) contenidos por sus lados iguales son también iguales.}

\begin{center}
Si las bases iguales \drawUnitLine{BC} y \drawUnitLine{EF} se conciben para colocarse sobre la otra, de modo que los triángulos se encuentren en el mismo lado de ellos, y que los lados iguales \drawUnitLine{AB} y \drawUnitLine{DE}, \drawUnitLine{CA} y \drawUnitLine{FD} sean contiguos, el vértice de uno debe caer sobre el vértice del otro; pues suponer que no son coincidentes contradeciría la última proposición.

Por consiguiente, los lados \drawUnitLine{AB} y \drawUnitLine{CA}, son coincidentes con \drawUnitLine{DE} y \drawUnitLine{FD},
 \\ $\therefore$ \drawAngle{A} $=$ \drawAngle{D}.

Q. E. D.
\end{center}

\qed

\startproblem{Prop IX. Prob.}\label{prop:I.IX}

\defineNewPicture{
pair A, B, C, D, E, F;
A := (0, 5/3u);
B := (-4/3u, 0);
C := B xscaled -1;
D = whatever[B, B shifted ((C-B) rotated -60)] = whatever[C, C shifted ((B-C) rotated 60)];
E := 5/4[A, B];
F := 5/4[A, C];
byAngleDefine(B, A, D, byblue, SOLID_SECTOR);
byAngleDefine(C, A, D, byyellow, SOLID_SECTOR);
draw byNamedAngleResized();
byLineDefine(B, C, byyellow, SOLID_LINE, REGULAR_WIDTH);
byLineDefine(A, D, byblack, SOLID_LINE, REGULAR_WIDTH);
byLineDefine(D, B, byblue, SOLID_LINE, REGULAR_WIDTH);
byLineDefine(C, D, byblue, SOLID_LINE, REGULAR_WIDTH);
byLineDefine(A, B, byred, SOLID_LINE, REGULAR_WIDTH);
byLineDefine(C, A, byred, SOLID_LINE, REGULAR_WIDTH);
byLineDefine(B, E, byred, DASHED_LINE, REGULAR_WIDTH);
byLineDefine(C, F, byred, DASHED_LINE, REGULAR_WIDTH);
draw byNamedLine(BC,AD);
draw byNamedLineSeq(0)(DB,CD);
draw byNamedLineSeq(0)(BE,AB,CA,CF);
draw byLabelsOnPolygon(D, B, A, C)(ALL_LABELS, 0);
}
\drawCurrentPictureInMargin
\problem{P}{ara}{ bisecar un ángulo rectilíneo dado (\drawAngle{BAD,CAD}).}

\begin{center}
Toma \drawUnitLine{AB} $=$ \drawUnitLine{CA} \byref{prop:I.III} dibuja \drawUnitLine{BC}, sobre la cual se traza
                     \\ \drawLine{CD,DB,BC} \byref{prop:I.I}, dibuja \drawUnitLine{AD}.

Porque \drawUnitLine{AB} $=$ \drawUnitLine{CA} (const.) y \drawUnitLine{AD} es común a los dos 
                     \\ triángulos y \drawUnitLine{CD} $=$ \drawUnitLine{DB} (const.),
                     \\ $\therefore$ \drawAngle{BAD} $=$ \drawAngle{CAD} \byref{\constref}

Q. E. D. \byref{\constref} \byref{prop:I.VIII}
\end{center}

\qed

\startproblem{Prop X. Prob.}\label{prop:I.X}

\defineNewPicture{
pair A, B, C, D;
A := (0, 3u);
B := (-ypart(A)/sqrt(3), 0);
C := B xscaled -1;
D := 1/2[B, C];
byAngleDefine(B, A, D, byblue, SOLID_SECTOR);
byAngleDefine(C, A, D, byyellow, SOLID_SECTOR);
draw byNamedAngleResized();
draw byLine(A, D, byred, SOLID_LINE, REGULAR_WIDTH);
byLineDefine(D, B, byblack, SOLID_LINE, REGULAR_WIDTH);
byLineDefine(C, D, byblack, DASHED_LINE, REGULAR_WIDTH);
byLineDefine(A, B, byyellow, SOLID_LINE, REGULAR_WIDTH);
byLineDefine(C, A, byblue, SOLID_LINE, REGULAR_WIDTH);
draw byNamedLineSeq(0)(AB,CA,CD,DB);
draw byLabelsOnPolygon(B, A, C, D)(ALL_LABELS, 0);
}
\drawCurrentPictureInMargin
\problem{P}{ara}{ bisecar una línea recta finita dada (\drawUnitLine{DB,CD}).}

\begin{center}
Construye \drawLine[bottom]{AB,CA,CD,DB} \byref{prop:I.I},
                     \\ dibuje \drawUnitLine{AD}, haciendo \drawAngle{BAD} $=$ \drawAngle{CAD} \byref{prop:I.IX}.

Entonce \drawUnitLine{BD} $=$ \drawUnitLine{DC} por \byref{prop:I.IV},
                     \\ \drawUnitLine{AB} $=$ \drawUnitLine{AC} (const.) \drawAngle{BAD} $=$ \drawAngle{CAD} y
                     \\ \drawUnitLine{AD} común a los dos triángulos.

Por lo tanto, la línea dada es bisecada.

Q. E. D. \byref{\constref}
\end{center}

\qed

\startproblem{Prop XI. Prob.}\label{prop:I.XI}

\defineNewPicture{
pair A, B, C, D, E, F;
A := (0, 5/2u);
B := (-ypart(A)/sqrt(3), 0);
C := B xscaled -1;
D := 1/2[B, C];
E := 3/2[D, B];
F := 3/2[D, C];
byAngleDefine(A, D, B, byred, SOLID_SECTOR);
byAngleDefine(C, D, A, byblue, SOLID_SECTOR);
draw byNamedAngleResized();
draw byLine(A, D, byyellow, SOLID_LINE, REGULAR_WIDTH);
byLineDefine(A, B, byblue, SOLID_LINE, REGULAR_WIDTH);
byLineDefine(C, A, byblue, SOLID_LINE, REGULAR_WIDTH);
draw byNamedLineSeq(0)(AB,CA);
byLineDefine(D, B, byblack, SOLID_LINE, REGULAR_WIDTH);
byLineDefine(B, E, byblack, DASHED_LINE, REGULAR_WIDTH);
byLineDefine(C, D, byred, SOLID_LINE, REGULAR_WIDTH);
byLineDefine(F, C, byred, DASHED_LINE, REGULAR_WIDTH);
draw byNamedLineSeq(0)(BE,DB,CD,FC);
draw byLabelsOnPolygon(F, C, D, B, E)(OMIT_FIRST_LABEL+OMIT_LAST_LABEL, 0);
draw byLabelsOnPolygon(B, A, C)(OMIT_FIRST_LABEL+OMIT_LAST_LABEL, 0);
}
\drawCurrentPictureInMargin
\problem{D}{esde}{ un punto dado (\drawUnitLine[3/2cm]{BD,DC}), en una línea recta dada (\drawUnitLine{DB}), para dibujar una perpendicular.}

\begin{center}
Toma cualquier punto (\drawUnitLine{CD}) en la línea dada,
                     \\ corta \drawLine[bottom]{AB,CA,CD,DB} $=$ \drawUnitLine{AD} \byref{prop:I.III}, 
                     \\ construye \drawUnitLine{AB} \byref{prop:I.I},
                     \\ dibuja \drawUnitLine{CA} y esta será perpendicular a la línea dada.

Para \drawUnitLine{CD} $=$ \drawUnitLine{DB} (const.)
                     \\ \drawUnitLine{AD} $=$ \drawAngle{ADB} (const.)
                     \\ \drawAngle{CDA} común a los dos triángulos.

Por consiguiente, \drawUnitLine{AD} $=$ \drawUnitLine{DB,CD} \byref{\constref}
                     \\ $\therefore$  ⊥  \byref{\constref}.

Q. E. D. \byref{prop:I.VIII} \byref{def:I.X}
\end{center}

\qed

\startproblem{Prop XII. Prob.}\label{prop:I.XII}

\defineNewPicture{
pair A, B, C, D, E, F;
path c;
numeric r, a[];
A := (0, 2u);
B := (-7/4u, 0);
C := B xscaled -1;
D := 1/2[B, C];
E := 4/3[D, B];
F := 4/3[D, C];
r := abs(A-B);
c := fullcircle scaled 2r shifted A;
a1 := xpart(c intersectiontimes (F--1/2[B, C]));
a2 := xpart(c intersectiontimes (E--1/2[B, C]));
byAngleDefine(A, D, B, byyellow, SOLID_SECTOR);
byAngleDefine(C, D, A, byblue, SOLID_SECTOR);
draw byNamedAngleResized();
draw byLine(A, D, byred, SOLID_LINE, REGULAR_WIDTH);
byLineDefine(A, B, byblue, SOLID_LINE, REGULAR_WIDTH);
byLineDefine(C, A, byblue, SOLID_LINE, REGULAR_WIDTH);
draw byNamedLineSeq(0)(AB,CA);
draw byArc.O(A, B, C)(r, byred, 0, 0, 0, 0);
draw byArcBE.Ol(A, a2-1/4, a2, r, byred, 1, 0, 0, 0);
draw byArcBE.Or(A, a1, a1+1/4, r, byred, 1, 0, 0, 0);
byLineDefine(D, B, byblack, SOLID_LINE, REGULAR_WIDTH);
byLineDefine(B, E, byblack, DASHED_LINE, REGULAR_WIDTH);
byLineDefine(C, D, byyellow, SOLID_LINE, REGULAR_WIDTH);
byLineDefine(F, C, byyellow, DASHED_LINE, REGULAR_WIDTH);
draw byNamedLineSeq(1)(BE, DB, CD,FC);
draw byLabelsOnPolygon(B, A, C)(OMIT_FIRST_LABEL+OMIT_LAST_LABEL, 0);
draw byLabelLineEnd(B, A, 0);
draw byLabelLineEnd(D, A, 0);
draw byLabelLineEnd(C, A, 0);
}
\drawCurrentPictureInMargin
\problem{P}{ara}{ dibujar una línea recta perpendicular a una línea recta indefinida dada (\drawUnitLine[1.2cm]{DB,CD}) desde un punto dado (\drawUnitLine{DB}).}

\begin{center}
Con el punto dado \drawUnitLine{DB} como centro, a un lado de las línea, y cualquier distancia \drawUnitLine{CD} capaz de extenderse al otro lado, traza \drawUnitLine{AB},

Haz \drawUnitLine{CA} $=$ \drawUnitLine{AD} \byref{prop:I.X}
                     \\ dibuja \drawUnitLine{AD}, \drawUnitLine{DB,CD} y \drawUnitLine{DB}.
 \\ entonces \drawUnitLine{CD} ⊥ \drawUnitLine{AD}.

Para \byref{prop:I.VIII} ya que \drawUnitLine{AB} $=$ \drawUnitLine{CA} (const.)
                     \\ \drawAngle{ADB} común a ambas
                     \\ y \drawAngle{CDA} $=$ \drawUnitLine{AD} \byref{\constref}

$\therefore$ \drawUnitLine{DB,CD} $=$ , y
                     \\ $\therefore$  ⊥  \byref{def:I.XV}.

Q. E. D. \byref{def:I.X}
\end{center}

\qed

\starttheorem{Prop XIII. Theor.}\label{prop:I.XIII}

\defineNewPicture{
pair A, B, C, D, E;
A := (0, 5/2u);
B := (-7/4u, 0);
C := B xscaled -1;
D := (xpart(A), ypart(B));
E := (2/3xpart(C), 2/3ypart(A));
byAngleDefine(A, D, B, byyellow, SOLID_SECTOR);
byAngleDefine(E, D, A, byred, SOLID_SECTOR);
byAngleDefine(C, D, E, byblue, SOLID_SECTOR);
draw byNamedAngleResized();
draw byLine(A, D, byblack, SOLID_LINE, REGULAR_WIDTH);
draw byLine(E, D, byyellow, SOLID_LINE, REGULAR_WIDTH);
draw byLine(B, C, byred, SOLID_LINE, REGULAR_WIDTH);
draw byLabelsOnPolygon(C, D, B, noPoint)(ALL_LABELS, 0);
draw byLabelLineEnd(E, D, 0);
draw byLabelLineEnd(A, D, 0);
}
\drawCurrentPictureInMargin
\problem{C}{uando}{ una línea recta (\drawUnitLine{ED}) se apoya sobre otra línea recta (\drawUnitLine{BC}) forma ángulos con ella; que son dos ángulos rectos o juntos equivalen a dos ángulos rectos.}

\begin{center}
Si \drawUnitLine{ED} es ⊥ a \drawUnitLine{BC} entonces,
                     \\ \drawAngle{ADB,EDA} y \drawAngle{CDE} $=$ 
\drawTwoRightAngles


 \byref{def:I.X},

Pero si \drawTwoRightAngles no es ⊥ a \drawUnitLine{ED} entonces,
                     \\ dibuja \drawUnitLine{BC} ⊥ \drawUnitLine{AD}; \byref{prop:I.XI}
                     \\ \drawUnitLine{BC} + \drawAngle{ADB} $=$ 
\drawTwoRightAngles


 (const.),
                     \\ \drawAngle{CDE,EDA} $=$ \drawTwoRightAngles $=$ \drawAngle{ADB} + \drawAngle{CDE,EDA}  \\ $\therefore$ \drawAngle{EDA} + \drawAngle{CDE} $=$ \drawAngle{ADB} + \drawAngle{CDE,EDA} + \drawAngle{ADB} \byref{\constref}
                     \\ $=$ \drawAngle{EDA} + \drawAngle{CDE} $=$ 
\drawTwoRightAngles


 .

Q. E. D. \drawAngle{ADB,EDA} \drawAngle{CDE} \drawTwoRightAngles \byref{ax:I.II}
\end{center}

\qed

\starttheorem{Prop XIV. Theor.}\label{prop:I.XIV}

\defineNewPicture[1/4]{
pair A, B, C, D, E;
A := (u, 5/2u);
B := (-7/4u, 0);
C := B xscaled -1;
D := (0, 0);
E := (xpart(C), -1/2ypart(A));
byAngleDefine(B, D, A, byyellow, SOLID_SECTOR);
byAngleDefine(C, D, A, byblue, SOLID_SECTOR);
byAngleDefine(E, D, C, byred, SOLID_SECTOR);
draw byNamedAngleResized();
draw byLine(A, D, byred, SOLID_LINE, REGULAR_WIDTH);
draw byLine(E, D, byyellow, SOLID_LINE, REGULAR_WIDTH);
draw byLine(B, D, byblue, SOLID_LINE, REGULAR_WIDTH);
draw byLine(C, D, byblack, SOLID_LINE, REGULAR_WIDTH);
draw byLabelsOnPolygon(E, D, B, noPoint)(ALL_LABELS, 0);
draw byLabelsOnPolygon(C, B, noPoint)(OMIT_LAST_LABEL, 0);
draw byLabelLineEnd(A, D, 0);
}
\drawCurrentPictureInMargin
\problem{S}{i}{ hay dos líneas rectas (\drawUnitLine{BD} y \drawUnitLine{DC}), que se encuentran con una tercera línea recta (\drawUnitLine{AD}), en el mismo punto y en lados opuestos, hacen con este ángulos adyacentes (\offsetPicture{0pt}{15pt}{\drawAngle{BDA}} y \drawAngle{CDA}) iguales a dos ángulos rectos; estas líneas rectas se encuentran en una línea recta continua.}

\begin{center}
Si es posible, deja \drawAngle{CDA}, y no \drawUnitLine{ED},\\
ser la continuación de \drawUnitLine{DC},\\
entonces $\drawAngle{BDA} + \drawAngle{CDA,EDC} = \drawTwoRightAngles$

pero por la hipótesis $\drawAngle{BDA} + \drawAngle{CDA} = \drawTwoRightAngles$

$\therefore\drawAngle{CDA,EDC} = \drawAngle{CDA}$ \byref{ax:I.III}; lo cual es absurdo \byref{ax:I.IX}.

$\therefore \drawUnitLine{ED}$ no es la continuación de \drawUnitLine{BD}, y se puede demostrar algo similar de cualquier otra línea recta excepto \drawUnitLine{DC}, $\therefore \drawUnitLine{DC}$ es la continuación de \drawUnitLine{BD}.
\end{center}

\qed

\starttheorem{Prop XV. Theor.}\label{prop:I.XV}

\defineNewPicture{
pair A, B, C, D, E;
A := (7/4u, 3/2u);
B := A scaled -1;
C := A xscaled -1;
D := C scaled -1;
E := (A--B) intersectionpoint (C--D);
byAngleDefine(B, E, C, byyellow, SOLID_SECTOR);
byAngleDefine(C, E, A, byred, SOLID_SECTOR);
byAngleDefine(A, E, D, byblack, SOLID_SECTOR);
byAngleDefine(D, E, B, byblue, SOLID_SECTOR);
draw byNamedAngleResized();
draw byLine(A, B, byred, SOLID_LINE, REGULAR_WIDTH);
draw byLine(C, D, byblack, SOLID_LINE, REGULAR_WIDTH);
draw byLabelsOnPolygon(C, E, A, noPoint)(ALL_LABELS, 0);
draw byLabelPoint(B, lineAngle.AB + 90, 1);
draw byLabelPoint(D, lineAngle.CD - 90, 1);
}
\drawCurrentPictureInMargin
\problem{S}{i}{ dos líneas rectas (\drawUnitLine{AB} y \drawUnitLine{CD}) se cruzan entre sí, los ángulos verticales \drawAngle{BEC} y \drawAngle{AED}, \drawAngle{CEA} y \drawAngle{DEB} son iguales.}

\begin{center}
$\drawAngle{BEC} + \drawAngle{CEA} = \drawTwoRightAngles$

$\drawAngle{AED} + \drawAngle{CEA} = \drawTwoRightAngles$

$\therefore \drawAngle{BEC} = \drawAngle{AED}$.

De la misma manera, se puede demostrar que
\\ \drawAngle{AEC} $=$ \drawAngle{DEB}

\qed \byref{ax:I.III}
\end{center}

\qed

\starttheorem{Prop XVI. Theor.}\label{prop:I.XVI}

\defineNewPicture[1/4]{
pair A, B, C, D, E, F, G;
A := (0, 0);
B := A shifted (u, 7/2u);
C := A shifted (3u, 0);
D := B shifted (3u, 0);
E = whatever[A, D] = whatever[B, C];
F := (xpart(D), ypart(A));
G := 4/3[B, C];
byAngleDefine(B, A, C, byblue, SOLID_SECTOR);
byAngleDefine(C, B, A, byblack, SOLID_SECTOR);
byAngleDefine(A, E, B, byyellow, SOLID_SECTOR);
byAngleDefine(D, E, C, byyellow, SOLID_SECTOR);
byAngleDefine(E, C, D, byblack, SOLID_SECTOR);
byAngleDefine(G, C, A, byred, SOLID_SECTOR);
byAngleDefine(D, C, F, byblack, ARC_SECTOR);
draw byNamedAngleResized();
byLineDefine(C, F, byblack, DASHED_LINE, REGULAR_WIDTH);
byLineDefine(C, G, byblack, SOLID_LINE, REGULAR_WIDTH);
byLineDefine(B, E, byblue, SOLID_LINE, REGULAR_WIDTH);
byLineDefine(E, C, byblue, DASHED_LINE, REGULAR_WIDTH);
byLineDefine(A, E, byred, SOLID_LINE, REGULAR_WIDTH);
byLineDefine(E, D, byred, DASHED_LINE, REGULAR_WIDTH);
byLineDefine(A, B, byyellow, DASHED_LINE, REGULAR_WIDTH);
byLineDefine(A, C, byblack, SOLID_LINE, REGULAR_WIDTH);
byLineDefine(C, D, byyellow, SOLID_LINE, REGULAR_WIDTH);
draw byNamedLineSeq(0)(AE,ED,CD);
draw byNamedLineSeq(0)(EC,CG,noLine,CF,AC,AB,BE);
draw byLabelsOnPolygon(F, A, B, E, D, C)(OMIT_FIRST_LABEL+OMIT_LAST_LABEL, 0);
draw byLabelsOnPolygon(F, C, G, noPoint)(ALL_LABELS, 0);
}
\drawCurrentPictureInMargin
\problem{S}{i}{un lado de un triángulo (\drawLine[bottom]{BE,EC,AC,AB}) es prolongado, el ángulo externo (\drawFromCurrentPicture[middle][anglesECDpDCF]{
startAutoLabeling;
draw byNamedAngleSides(ECD,DCF)(CF);
stopAutoLabeling;
}) es mayor que cualquiera de los ángulos opuestos internos (\drawAngle{B} o \drawAngle{A}).}

\begin{center}
Haz \drawUnitLine{BE} $=$ \drawUnitLine{EC} \byref{prop:I.X}.
                     \\ Dibuja \drawUnitLine{AE} y prolóngala hasta
                     \\ \drawUnitLine{ED} $=$ \drawUnitLine{AE}; dibuja \drawUnitLine{CD}.

En \drawLine{BE,AE,AB} y \drawLine{EC,ED,CD}; \drawUnitLine{BE} $=$ \drawUnitLine{EC}  \\ \drawAngle{AEB} $=$ \drawAngle{DEC} y \drawUnitLine{AE} $=$ \drawUnitLine{ED} (const. pr. 15.),
                     \\ $\therefore$ \drawAngle{B} $=$ \drawAngle{ECD} \byref{\constref,prop:I.XV},
                     \\ $\therefore$ \drawAngle{B} > \drawUnitLine{BC}.

De igual manera se puede mostrar, que si \drawAngle{GCA}  \\ son prolongadas, \drawAngle{A} > \drawAngle{GCA}, y por lo tanto
                     \\ \drawAngle{A} el cual es $=$  es > .

Q. E. D. \byref{prop:I.IV}
\end{center}

\qed

\starttheorem{Prop XVII. Theor.}\label{prop:I.XVII}

\defineNewPicture{
pair A, B, C, D;
A := (0, 0);
B := A shifted (2u, 5/2u);
C := A shifted (5/2u, 0);
D := C shifted (3/4u, 0);
byAngleDefine(B, A, C, byblue, SOLID_SECTOR);
byAngleDefine(A, B, C, byblack, SOLID_SECTOR);
byAngleDefine(A, C, B, byred, SOLID_SECTOR);
byAngleDefine(B, C, D, byyellow, SOLID_SECTOR);
draw byNamedAngleResized();
byLineDefine(A, B, byred, SOLID_LINE, REGULAR_WIDTH);
byLineDefine(B, C, byblue, SOLID_LINE, REGULAR_WIDTH);
byLineDefine(A, C, byblack, SOLID_LINE, REGULAR_WIDTH);
byLineDefine(C, D, byblack, SOLID_LINE, REGULAR_WIDTH);
draw byNamedLineSeq(0)(noLine,BC,AB,AC,CD);
draw byLabelsOnPolygon(D, C, A, B)(ALL_LABELS, 0);
}
\drawCurrentPictureInMargin
\problem{C}{ualquiera}{ de los dos ángulos de un triángulo \drawLine[bottom]{AB,BC,AC} juntos son menos que dos ángulos rectos.}

\begin{center}
Prolonga \drawUnitLine{AC}, entonces será\\
$\drawAngle{ACB} + \drawAngle{BCD} = \drawTwoRightAngles$.

Pero $\drawAngle{BCD} > \drawAngle{A}$ \byref{prop:I.XVI}

$\therefore \drawAngle{A} + \drawAngle{ACB} < \drawTwoRightAngles$,

\noindent y de la misma manera se puede demostrar que cualesquiera otros dos ángulos del triángulo tomados juntos son menos que dos ángulos rectos.

\qed
\end{center}

\qed

\starttheorem{Prop XVIII. Theor.}\label{prop:I.XVIII}

\defineNewPicture[1/4]{
pair A, B, C, D;
numeric a;
A := (0, 0);
B := A shifted (5/2u, 2u);
C := B shifted (-3/2u, 2u);
D := C shifted (unitvector(A-C) scaled abs(B-C));
a := angle(B-D);
forsuffixes i=A, B, C, D:
i := i rotated -a;
endfor;
byAngleDefine(C, D, B, byblue, SOLID_SECTOR);
byAngleDefine(D, B, C, byblack, SOLID_SECTOR);
byAngleDefine(A, B, D, byred, SOLID_SECTOR);
byAngleDefine(B, A, D, byyellow, SOLID_SECTOR);
draw byNamedAngleResized();
draw byLine(D, B, byyellow, SOLID_LINE, REGULAR_WIDTH);
byLineDefine(D, C, byred, SOLID_LINE, REGULAR_WIDTH);
byLineDefine(B, C, byblue, SOLID_LINE, REGULAR_WIDTH);
byLineDefine(B, A, byblack, SOLID_LINE, REGULAR_WIDTH);
byLineDefine(A, D, byred, DASHED_LINE, REGULAR_WIDTH);
draw byNamedLineSeq(0)(DC,BC,BA,AD);
draw byLabelsOnPolygon(D, C, B, A)(ALL_LABELS, 0);
}
\drawCurrentPictureInMargin
\problem{E}{n}{ cualquier triángulo \drawLine{DC,BC,BA,AD} si un lado \drawUnitLine{AD,DC} es mayor que otro \drawUnitLine{BC}, el ángulo opuesto del lado mayor es mayor que el ángulo opuesto del menor, por ejemplo \drawAngle{DBC,ABD} > \drawAngle{A}.}

\begin{center}
Haz \drawUnitLine{DC} $=$ \drawUnitLine{BC} \byref{prop:I.III}, dibuja \drawUnitLine{DB}.

Entonces será \drawAngle{D} $=$ \drawAngle{DBC} \byref{prop:I.V};
                     \\ pero \drawAngle{D} > \drawAngle{A} \byref{prop:I.XVI};
                     \\ $\therefore$ \drawAngle{DBC} > \drawAngle{A} y mucho más
                     \\ es \drawAngle{DBC,ABD} > \drawAngle{A}.

Q. E. D.
\end{center}

\qed

\starttheorem{Prop XIX. Theor.}\label{prop:I.XIX}

\defineNewPicture[1/4]{
pair A, B, C;
A := (0, 0);
B := A shifted (7/2u, 0);
C := A shifted (u, 3u);
byAngleDefine(C, A, B, byblue, SOLID_SECTOR);
byAngleDefine(A, B, C, byred, SOLID_SECTOR);
draw byNamedAngleResized();
byLineDefine(A, B, byblack, SOLID_LINE, REGULAR_WIDTH);
byLineDefine(B, C, byblue, SOLID_LINE, REGULAR_WIDTH);
byLineDefine(C, A, byred, SOLID_LINE, REGULAR_WIDTH);
draw byNamedLineSeq(0)(CA,BC,AB);
draw byLabelsOnPolygon(B, A, C)(ALL_LABELS, 0);
}
\drawCurrentPictureInMargin
\problem{S}{i}{ en cualquier triángulo \drawLine[bottom]{CA,BC,AB} un ángulo \drawAngle{A} es mayor que otro \drawAngle{B} el lado \drawUnitLine{BC} que es opuesto al ángulo mayor, es mayor que el lado \drawUnitLine{CA} opuesto al menor.}

\begin{center}
Si \drawUnitLine{BC} no es mayor que \drawUnitLine{CA} entonces debe
                     \\ \drawUnitLine{BC} $=$ o < \drawUnitLine{CA}.

Si \drawUnitLine{BC} $=$ \drawUnitLine{CA} entonces
                     \\ \drawAngle{A} $=$ \drawAngle{B} \byref{prop:I.V};
                     \\ lo cual es contrario a la hipótesis.
                     \\ \drawUnitLine{BC} no es menor que \drawUnitLine{CA}; porque si fuera,
                     \\ \drawAngle{A} < \drawAngle{B} \byref{prop:I.XVIII}
                     \\ lo cual es contrario a la hipótesis:
                     \\ $\therefore$ \drawUnitLine{BC} > \drawUnitLine{CA}.

Q. E. D.
\end{center}

\qed

\starttheorem{Prop XX. Theor.}\label{prop:I.XX}

\defineNewPicture{
pair A, B, C, D;
A := (0, 0);
B := A shifted (7/2u, 0);
D := A shifted (4/3u, 3/2u);
C := ((fullcircle scaled 2arclength(D--B)) shifted D) intersectionpoint (D--10[A, D]);
byAngleDefine(B, C, A, byred, SOLID_SECTOR);
byAngleDefine(C, B, D, byblue, SOLID_SECTOR);
byAngleDefine(D, B, A, byyellow, SOLID_SECTOR);
draw byNamedAngleResized();
byLineDefine(B, D, byred, SOLID_LINE, REGULAR_WIDTH);
byLineDefine(A, B, byblack, SOLID_LINE, REGULAR_WIDTH);
byLineDefine(B, C, byyellow, SOLID_LINE, REGULAR_WIDTH);
byLineDefine(C, D, byblue, DASHED_LINE, REGULAR_WIDTH);
byLineDefine(D, A, byblue, SOLID_LINE, REGULAR_WIDTH);
draw byNamedLineSeq(0)(BD);
draw byNamedLineSeq(0)(DA,CD,BC,AB);
draw byLabelsOnPolygon(D, C, B, A)(ALL_LABELS, 0);
}
\drawCurrentPictureInMargin
\problem{C}{ualquiera}{ dos lados \drawUnitLine{DA} y \drawUnitLine{BD} de un triángulo \drawLine[bottom]{DA,BD,AB} tomados juntos son mayores que el tercer lado (\drawUnitLine{AB}).}

\begin{center}
Prolonga \drawUnitLine{DA}, y
                     \\ haz \drawUnitLine{CD} $=$ \drawUnitLine{BD} \byref{prop:I.III};
                     \\ dibuja \drawUnitLine{BC}.

Entonces porque \drawUnitLine{CD} $=$ \drawUnitLine{BD} (conſt.),

\drawAngle{CBD} $=$ \drawAngle{C} \byref{\constref}
$\therefore$ \drawAngle{CBD,DBA} > \drawAngle{C} \byref{prop:I.V}

$\therefore$ \drawUnitLine{DA} + \drawUnitLine{CD} > \drawUnitLine{AB} \byref{ax:I.IX}
                     \\ y $\therefore$ \drawUnitLine{DA} + \drawUnitLine{BD} > \drawUnitLine{AB}.

Q. E. D. \byref{prop:I.XIX}
\end{center}

\qed

\starttheorem{Prop XXI. Theor.}\label{prop:I.XXI}

\defineNewPicture[1/5]{
pair A, B, C, D, E;
A := (0, 0);
B := A shifted (7/2u, 0);
C := A shifted (3u, 4u);
D := 1/2[1/2[A, B], C];
E = whatever[A, D] = whatever[B, C];
byAngleDefine(B, D, A, byred, SOLID_SECTOR);
byAngleDefine(B, E, D, byblue, SOLID_SECTOR);
byAngleDefine(B, C, A, byyellow, SOLID_SECTOR);
draw byNamedAngleResized();
byLineDefine(B, D, byyellow, SOLID_LINE, REGULAR_WIDTH);
byLineDefine(A, D, byblack, SOLID_LINE, REGULAR_WIDTH);
byLineDefine(D, E, byblack, DASHED_LINE, REGULAR_WIDTH);
byLineDefine(A, B, byblue, DASHED_LINE, REGULAR_WIDTH);
byLineDefine(B, E, byred, DASHED_LINE, REGULAR_WIDTH);
byLineDefine(E, C, byred, SOLID_LINE, REGULAR_WIDTH);
byLineDefine(C, A, byblue, SOLID_LINE, REGULAR_WIDTH);
draw byNamedLine(BD);
draw byNamedLineSeq(0)(AD,DE);
draw byNamedLineSeq(0)(CA,EC,BE,AB);
draw byLabelsOnPolygon(A, C, E, B)(ALL_LABELS, 0);
draw byLabelsOnPolygon(A, D, E)(OMIT_FIRST_LABEL+OMIT_LAST_LABEL, 0);
}
\drawCurrentPictureInMargin
\problem{S}{i}{ desde cualquier punto (\drawLine[bottom]{CA,EC,BE,AB}) dentro de un triángulo \drawAngle{E} se dibujan líneas rectas a las extremidades de un lado (\drawAngle{C}), estas líneas deben estar menos juntas que los otros dos lados, pero deben contener un ángulo mayor.}

\begin{center}
Prolonga \drawAngle{D},
 \\ \drawAngle{E} + \drawAngle{D} > \drawAngle{C} \byref{prop:I.XX},
                     \\ agrega  a cada una,
                     \\  +  >  +  \byref{ax:I.IV}

De igual manera se puede mostrar que
                     \\  +  >  + ,
 \\ $\therefore$  +  >  + ,
 \\ que debía ser probado.

De nuevo  >  \byref{prop:I.XVI},
y también  >  \byref{prop:I.XVI},
$\therefore$  > .

Q. E. D.
\end{center}

\qed

\startproblem{Prop XXII. Prob.}\label{prop:I.XXII}

\defineNewPicture[1/2]{
numeric r[], d;
pair A, B, C, D, E, LI, LII, LIII, LIV, LV, LVI;
path q[];
r1 := 2u;
r2 := 4/3u;
r3 := (1/2)*(r1+r2);
d := 1/3u;
A := (0, 0);
B := A shifted (r3, 0);
q1 := (fullcircle scaled 2r1) shifted A;
q2 := (fullcircle scaled 2r2) shifted B;
C := q1 intersectionpoint q2;
D := point 11/2 of q1;
E := point 3/4 of q2;
LI := (xpart(point 0 of q2), ypart(point 6 of q1) - 1/2d);
LII := LI shifted (-r3, 0);
LIII := LI shifted (0, -d);
LIV := LIII shifted (-r2, 0);
LV := LIII shifted (0, -d);
LVI := LV shifted (-r1, 0);
draw byCircle.A(A, D, byblue, 0, 0, 0);
byLineDefine(A, D, byblue, SOLID_LINE, REGULAR_WIDTH);
byLineDefine(B, E, byred, SOLID_LINE, REGULAR_WIDTH);
byLineDefine(A, B, byblack, SOLID_LINE, REGULAR_WIDTH);
byLineDefine(B, C, byyellow, SOLID_LINE, REGULAR_WIDTH);
byLineDefine(C, A, byyellow, DASHED_LINE, REGULAR_WIDTH);
draw byNamedLineSeq(0)(BC,CA);
draw byNamedLineSeq(0)(AD,AB,BE);
draw byLineWithName (LII, LI, byblack, 1, 0)(L');
draw byLineWithName (LIV, LIII, byred, 1, 0)(L'');
draw byLineWithName (LVI, LV, byblue, 1, 0)(L''');
draw byCircle.B(B, E, byred, 0, 0, 0);
draw byLabelsOnPolygon(D, A, C)(OMIT_FIRST_LABEL+OMIT_LAST_LABEL, 0);
draw byLabelsOnPolygon(E, B, A)(OMIT_FIRST_LABEL+OMIT_LAST_LABEL, 0);
draw byLabelsOnPolygon(A, C, B)(OMIT_FIRST_LABEL+OMIT_LAST_LABEL, 0);
draw byLabelsOnCircle(D)(A);
draw byLabelsOnCircle(E)(B);
draw byLabelLine(0)(L', L'', L''');
}
\drawCurrentPictureInMargin
\problem[4]{D}{adas}{tres líneas rectas $\vcenter{\hbox{\drawFromCurrentPicture{
draw byLine(LII, LI, byblack, SOLID_LINE, REGULAR_WIDTH);
draw byLine(LIV, LIII, byred, SOLID_LINE, REGULAR_WIDTH);
draw byLine(LVI, LV, byblue, SOLID_LINE, REGULAR_WIDTH);
}}}$ la suma de dos cualesquiera mayores que la tercera, para construir un triángulo cuyos lados serán respectivamente iguales a las líneas dadas.}

\begin{center}
Asume \offsetPicture{12pt}{0pt}{\drawFromCurrentPicture{
draw byNamedLine(BE); draw byNamedCircle(B);
draw byLabelLineEnd(B, E, 0);
draw byLabelLineEnd(E, B, 0);
}} \byref{prop:I.III}.



Dibuja \drawLine[bottom]{CA,BC,AB} $=$ 
y  $=$ 



                            \byref{prop:I.II}.

Con  y  como radios,
                     \\ dibuja  y  \byref{post:I.III};
                     \\ dibuja  y ,
 \\ entonces  será el triángulo requerido.

Para  $=$ ,
 $=$  $=$ ,
y  $=$  $=$ .



                    (conſt.)

Q. E. D. \byref{\constref}
\end{center}

\qed

\startproblem{Prop XXIII. Prob.}\label{prop:I.XXIII}

\defineNewPicture{
pair A, B, C, D, E, F, G, H, J, d;
A := (0, 0);
B := A shifted (7/2u, 0);
C := A shifted (3u, 11/5u);
D := 5/4[A, B];
E := 7/6[A, C];
d := (0, -3u);
F := A shifted d;
G := B shifted d;
H := C shifted d;
J := D shifted d;
byAngleDefine(B, A, C, byred, SOLID_SECTOR);
byAngleDefine(G, F, H, byblue, SOLID_SECTOR);
draw byNamedAngleResized();
byLineDefine(B, D, byblack, DASHED_LINE, THIN_WIDTH);
byLineDefine(C, E, byblue, DASHED_LINE, THIN_WIDTH);
byLineDefine(A, B, byblack, SOLID_LINE, THIN_WIDTH);
byLineDefine(C, A, byblue, SOLID_LINE, THIN_WIDTH);
draw byLine(B, C, byred, SOLID_LINE, THIN_WIDTH);
draw byNamedLineSeq(0)(CE,CA,AB,BD);
byLineDefine(G, J, byblack, DASHED_LINE, REGULAR_WIDTH);
byLineDefine(F, G, byblack, SOLID_LINE, REGULAR_WIDTH);
byLineDefine(G, H, byred, SOLID_LINE, REGULAR_WIDTH);
byLineDefine(H, F, byyellow, SOLID_LINE, REGULAR_WIDTH);
draw byNamedLineSeq(0)(noLine,GH,HF,FG,GJ);
draw byLabelsOnPolygon(D, B, A, C, E)(OMIT_FIRST_LABEL+OMIT_LAST_LABEL, 0);
draw byLabelsOnPolygon(noPoint,J, G, F, H, G)(OMIT_FIRST_LABEL+OMIT_LAST_LABEL, 0);
}
\drawCurrentPictureInMargin
\problem{E}{n}{ un punto dado (\drawUnitLine{FG,GJ}), en una línea recta dada \drawAngle{A}), hacer un ángulo igual a un ángulo rectilíneo dado (\drawUnitLine{BC}).}

\begin{center}
Dibuja \drawLine[bottom]{HF,GH,FG} entre dos puntos en las patas del ángulo dado.

Construye \drawUnitLine{FG} \byref{prop:I.XXII}. de modo que
                     \\ \drawUnitLine{AB} $=$ \drawUnitLine{HF}, \drawUnitLine{CA} $=$ \drawUnitLine{GH}  \\ y \drawUnitLine{BC} $=$ \drawAngle{A}.

Entonces \drawAngle{F} $=$  \byref{prop:I.VIII}.

Q. E. D.
\end{center}

\qed

\starttheorem{Prop XXIV. Theor.}\label{prop:I.XXIV}

\defineNewPicture{
byAngleMacroName[3] := "byAngleMArcs";
vardef byAngleMArcs (expr angleArc, angleColor)(suffix angleOptionalColors) =
    save p;
    path p;
    p := angleArc scaled ((angleSize*angleScale) - lineWidthThin);
    image(
        for i := 1 step -1/floor((angleSize*angleScale)/2mm) until (2mm/(angleSize*angleScale)):
            draw (p scaled i) withpen pencircle scaled lineWidthThin withcolor angleColor;
        endfor;
    )
enddef;
pair A, B, C, D, E, F, G, d;
A := (0, 0);
B := A shifted (u, -5/2u);
C := A shifted (-u, -7/2u);
D := ((fullcircle scaled 2abs(C-A)) shifted A) intersectionpoint (B--B shifted (-2abs(C-A), 0));
d := (0, -4u);
E := A shifted d;
F := B shifted d;
G := C shifted d;
byAngleDefine(B, A, C, byblack, DASHED_ARC_SECTOR);
byAngleDefine(C, A, D, byblack, 3);
byAngleDefine(F, E, G, byblack, DASHED_ARC_SECTOR);
byAngleDefine(B, D, A, byblue, SOLID_SECTOR);
byAngleDefine(C, D, B, byred, SOLID_SECTOR);
byAngleDefine(D, C, A, byyellow, SOLID_SECTOR);
byAngleDefine(A, C, B, byblack, SOLID_SECTOR);
draw byNamedAngleResized();
byLineDefine(A, B, byblue, SOLID_LINE, REGULAR_WIDTH);
byLineDefine(B, C, byblack, DASHED_LINE, REGULAR_WIDTH);
draw byLine(C, A, byred, SOLID_LINE, REGULAR_WIDTH);
draw byLine(B, D, byblack, SOLID_LINE, REGULAR_WIDTH);
byLineDefine(A, D, byred, DASHED_LINE, REGULAR_WIDTH);
byLineDefine(C, D, byblue, DASHED_LINE, REGULAR_WIDTH);
draw byNamedLineSeq(0)(AB,AD,CD,BC);
byLineDefine(E, F, byblue, SOLID_LINE, THIN_WIDTH);
byLineDefine(F, G, byyellow, SOLID_LINE, THIN_WIDTH);
byLineDefine(G, E, byred, SOLID_LINE, THIN_WIDTH);
draw byNamedLineSeq(0)(EF,FG,GE);
draw byLabelsOnPolygon(D, A, B, C)(ALL_LABELS, 0);
draw byLabelsOnPolygon(G, E, F)(ALL_LABELS, 0);
}
\drawCurrentPictureInMargin
\problem{S}{i}{dos triángulos tienen dos lados de uno respectivamente igual a dos lados del otro (\drawUnitLine{AB} a \drawUnitLine{EF} y \drawUnitLine{AD} a \drawUnitLine{GE}), y si uno de los ángulos (\drawFromCurrentPicture[bottom]{
startAutoLabeling;
draw byNamedAngleSides(BAC,CAD)(AB,CA,AD);
stopAutoLabeling;
}) contenido por los lados iguales es mayor que el otro (\drawFromCurrentPicture[bottom][angleFEG]{
startAutoLabeling;
draw byNamedAngleSides(FEG)(EF, GE);
stopAutoLabeling;
}), el lado (\drawUnitLine{DB}) que es opuesto al ángulo más grande es mayor que el lado (\drawUnitLine{FG}) que es opuesto al ángulo menor.}

\begin{center}
Haz \drawFromCurrentPicture[bottom][angleBAC]{
startAutoLabeling;
draw byNamedAngleSides(BAC)(AB, CA);
stopAutoLabeling;
} $=$ \drawUnitLine{CA} \byref{prop:I.XXIII},
                 \\ y \drawUnitLine{GE} $=$ \drawUnitLine{CD} \byref{prop:I.III},
                 \\ dibuja \drawUnitLine{BC} y \drawUnitLine{CA}.
 \\ Porque \drawUnitLine{AD} $=$ \drawAngle{BDA,CDB} \byref{ax:I.I,\hypref,\constref}
                 \\ $\therefore$ \drawAngle{DCA} $=$ \drawAngle{CDB} \byref{prop:I.V}
                 \\ pero \drawAngle{DCA} < \drawAngle{CDB}  \\ y $\therefore$ \drawAngle{DCA,ACB} < \drawUnitLine{DB},
 \\ $\therefore$ \drawUnitLine{BC} > \drawUnitLine{BC} \byref{prop:I.XIX}
                 \\ pero \drawUnitLine{FG} $=$ \drawUnitLine{DB} \byref{prop:I.IV}
                 \\ $\therefore$ \drawUnitLine{FG} > .

Q. E. D.
\end{center}

\qed

\starttheorem{Prop XXV. Theor.}\label{prop:I.XXV}

\defineNewPicture{
pair A, B, C, D, E, F, d;
A := (0, 0);
B := A shifted (u, -3u);
C := A shifted (-7/4u, -4u);
d := (0, -9/2u);
D := A shifted d;
E := ((B shifted -A) rotated -10) shifted d;
F := C shifted d;
byAngleDefine(B, A, C, byyellow, SOLID_SECTOR);
byAngleDefine(E, D, F, byblack, SOLID_SECTOR);
draw byNamedAngleResized();
byLineDefine(A, B, byblue, SOLID_LINE, REGULAR_WIDTH);
byLineDefine(B, C, byblack, SOLID_LINE, REGULAR_WIDTH);
byLineDefine(C, A, byred, SOLID_LINE, REGULAR_WIDTH);
draw byNamedLineSeq(0)(AB,BC,CA);
byLineDefine(D, E, byblue, SOLID_LINE, THIN_WIDTH);
byLineDefine(E, F, byyellow, SOLID_LINE, THIN_WIDTH);
byLineDefine(F, D, byred, SOLID_LINE, THIN_WIDTH);
draw byNamedLineSeq(0)(DE,EF,FD);
draw byLabelsOnPolygon(A, B, C)(ALL_LABELS, 0);
draw byLabelsOnPolygon(D, E, F)(ALL_LABELS, 0);
}
\drawCurrentPictureInMargin
\problem{S}{i}{ dos triángulos tienen dos lados (\drawUnitLine[0.7cm]{AB} y \drawUnitLine[0.7cm]{CA}) de uno respectivamente igual a dos lados (\drawUnitLine{DE} y \drawUnitLine{FD}) del otro, pero sus bases son desiguales, el ángulo subtendido por la base mayor (\drawUnitLine{BC}) de uno, debe ser mayor que el ángulo subtendido por la base menor (\drawUnitLine{EF}) del otro.}

\begin{center}
\drawAngle{A} $=$, > o < \drawAngle{D} \drawAngle{A} no es igual a \drawAngle{D}  \\ por si \drawAngle{A} $=$ \drawAngle{D} entonces \drawUnitLine{CB} $=$ \drawUnitLine{FE} \byref{prop:I.IV}
                     \\ lo cual es contrario a la hipótesis;

\drawAngle{A} no es menor que \drawAngle{D}  \\ por si \drawAngle{A} < \drawAngle{D}  \\ entonces \drawUnitLine{CB} < \drawUnitLine{FE} \byref{prop:I.XXIV},
                     \\ lo cual también es contrario a la hipótesis:

$\therefore$ \drawAngle{A} > \drawAngle{D}.

Q. E. D.
\end{center}

\qed

\starttheorem{Prop XXVI. Theor.}\label{prop:I.XXVI}

\defineNewPicture{
pair A, B, C, D, E, F, G, d;
A := (0, 0);
B := A shifted (3u, 0);
C := A shifted (5/2u, 3u);
d := (0, -4u);
D := A shifted d;
E := B shifted d;
F := C shifted d;
G := 3/4[D, F];
byAngleDefine(B, A, C, byyellow, SOLID_SECTOR);
byAngleDefine(C, B, A, byred, SOLID_SECTOR);
byAngleDefine(A, C, B, byblack, ARC_SECTOR);
draw byNamedAngleResized(BAC, CBA, ACB);
byLineDefine(A, B, byblue, SOLID_LINE, REGULAR_WIDTH);
byLineDefine(B, C, byblack, SOLID_LINE, REGULAR_WIDTH);
byLineDefine(C, A, byred, SOLID_LINE, REGULAR_WIDTH);
draw byNamedLineSeq(0)(CA,BC,AB);
byAngleDefine(E, D, F, byyellow, SOLID_SECTOR);
byAngleDefine(G, E, D, byblack, SOLID_SECTOR);
byAngleDefine(F, E, G, byblue, SOLID_SECTOR);
byAngleDefine(D, F, E, byblack, ARC_SECTOR);
draw byNamedAngleResized(EDF, GED, FEG, DFE);
draw byLine(E, G, byyellow, SOLID_LINE, THIN_WIDTH);
byLineDefine(D, E, byblue, SOLID_LINE, THIN_WIDTH);
byLineDefine(E, F, byblack, SOLID_LINE, THIN_WIDTH);
byLineDefine(F, G, byred, DASHED_LINE, THIN_WIDTH);
byLineDefine(G, D, byred, SOLID_LINE, THIN_WIDTH);
draw byNamedLineSeq(0)(GD,FG,EF,DE);
draw byLabelsOnPolygon(C, B, A)(ALL_LABELS, 0);
draw byLabelsOnPolygon(D, G, F, E)(ALL_LABELS, 0);
}
\drawCurrentPictureInMargin
\problem{S}{i}{dos triángulos tienen dos ángulos de uno respectivamente igual a dos ángulos del otro, (\drawAngle{BAC} $=$ \drawAngle{EDF} y \drawAngle{CBA} $=$ \drawAngle{GED,FEG}), y un lado de uno igual a un lado del otro colocado de manera similar con respecto a los ángulos iguales, los lados y ángulos restantes son respectivamente iguales entre sí.}

\startsubproposition{Caso I.}
\begin{center}
Deja que \drawUnitLine{AB} y \drawUnitLine{DE} que se encuentran entre los ángulos iguales sean iguales,
                         \\ entonces \drawUnitLine{CA} $=$ \drawUnitLine{GD,FG}.
 \\ Por si es posible, deja una de ellas \drawUnitLine{GD,FG} ser mayor que la otra;
                         \\ haz \drawUnitLine{CA} $=$ \drawUnitLine{GD}, dibuja \drawUnitLine{EG}.

En \drawLine[bottom]{CA,BC,AB} y \drawLine[bottom]{GD,EG,DE} tenemos \drawUnitLine{CA}  \\ $=$ \drawUnitLine{GD}, \drawAngle{BAC} $=$ \drawAngle{EDF}, \drawUnitLine{AB} $=$ \drawUnitLine{DE};
 \\ $\therefore$ \drawAngle{CBA} $=$ \drawAngle{GED} \byref{\hypref}
                         \\ pero \drawAngle{CBA} $=$ \drawAngle{GED,FEG} \byref{\hypref}
                         \\ y por consiguiente \drawAngle{GED} $=$ \drawAngle{GED,FEG}, lo cual es absurdo; por lo tanto ninguno de los lados \drawUnitLine{CA} y \drawUnitLine{GD,FG} es mayor que el otro; y $\therefore$ son iguales;
                         \\ $\therefore$ \drawUnitLine{BC} $=$ \drawUnitLine{EF}, y \drawAngle{ACB} $=$ \drawAngle{DFE}, \byref{prop:I.IV}.
\end{center}

\vfill\pagebreak

\defineNewPicture{
pair A, B, C, D, E, F, G, d;
d := (0, -4u);
A := (0, 0);
B := A shifted (3u, 0);
C := A shifted (1/2u, 3u);
D := A shifted d;
E := B shifted d;
F := C shifted d;
G := 3/4[D, E];
byAngleDefine(B, A, C, byyellow, SOLID_SECTOR);
byAngleDefine(C, B, A, byred, SOLID_SECTOR);
draw byNamedAngleResized(BAC, CBA);
byLineDefine(A, B, byblue, SOLID_LINE, REGULAR_WIDTH);
byLineDefine(B, C, byblack, SOLID_LINE, REGULAR_WIDTH);
byLineDefine(C, A, byred, SOLID_LINE, REGULAR_WIDTH);
draw byNamedLineSeq(0)(CA,AB,BC);
byAngleDefine(F, D, E, byyellow, SOLID_SECTOR);
byAngleDefine(F, G, D, byblack, SOLID_SECTOR);
byAngleDefine(F, E, D, byred, SOLID_SECTOR);
draw byNamedAngleResized(FDE, FGD, FED);
draw byLine(F, G, byyellow, SOLID_LINE, THIN_WIDTH);
byLineDefine(D, G, byblue, SOLID_LINE, THIN_WIDTH);
byLineDefine(G, E, byblue, DASHED_LINE, THIN_WIDTH);
byLineDefine(E, F, byblack, SOLID_LINE, THIN_WIDTH);
byLineDefine(F, D, byred, SOLID_LINE, THIN_WIDTH);
draw byNamedLineSeq(0)(FD,EF,GE,DG);
draw byLabelsOnPolygon(C, B, A)(ALL_LABELS, 0);
draw byLabelsOnPolygon(D, F, E, G)(ALL_LABELS, 0);
}
\drawCurrentPictureInMargin

\startsubproposition{Caso II.}

\begin{center}
De nuevo, deja \drawUnitLine{CA} $=$ \drawUnitLine{FD}, que se encuentran frente a los ángulos iguales \drawAngle{B} y \drawAngle{E}. Si esto es posible, deja \drawUnitLine{DG,GE} > \drawUnitLine{AB}, entonces toma \drawUnitLine{DG} $=$ \drawUnitLine{AB}, dibuja \drawUnitLine{FG}.

De nuevo, deja \drawLine[bottom]{CA,BC,AB} y \drawLine[bottom]{FD,FG,DG} tenemos \drawUnitLine{CA} $=$ \drawUnitLine{FD},
 \\ \drawUnitLine{AB} $=$ \drawUnitLine{DG} y \drawAngle{BAC} $=$ \drawAngle{FDE},
 \\ $\therefore$ \drawAngle{CBA} $=$ \drawAngle{FGD} \byref{prop:I.IV}
                         \\ pero \drawAngle{CBA} $=$ \drawAngle{FED} \byref{\hypref}
                         \\ $\therefore$ \drawAngle{FGD} $=$ \drawAngle{FED} lo cual es absurdo \byref{prop:I.XVI}.

En consecuencia, ninguno de los lados \drawUnitLine{AB} o \drawUnitLine{DG,GE} es mayor que el otro, por lo tanto, deben ser iguales. Se deduce (por la pr. 4.) que los triángulos son iguales en todos los aspectos.

Q. E. D. \byref{prop:I.XVI} \byref{prop:I.IV}
\end{center}

\qed

\starttheorem{Prop XXVII. Theor.}\label{prop:I.XXVII}

\defineNewPicture{
pair A, B, C, D, E, F, G, H, I, d;
A := (0, 0);
B := A shifted (8/3u, 0);
d := (0, -7/4u);
C := A shifted d;
D := B shifted d;
E := 1/3[A, B];
F := 2/3[C, D];
G := 3/2[F, E];
H := 3/2[E, F];
I := 1/2[A, C] shifted (-2u, 0);
byAngleDefine(A, E, H, byyellow, SOLID_SECTOR);
byAngleDefine(H, E, B, byred, SOLID_SECTOR);
byAngleDefine(C, F, G, byblue, SOLID_SECTOR);
byAngleDefine(G, F, D, byyellow, SOLID_SECTOR);
draw byNamedAngleResized();
byLineDefine(I, A, byblue, SOLID_LINE, REGULAR_WIDTH);
byLineDefine(A, B, byblue, SOLID_LINE, REGULAR_WIDTH);
byLineDefine(I, C, byred, SOLID_LINE, REGULAR_WIDTH);
byLineDefine(C, D, byred, SOLID_LINE, REGULAR_WIDTH);
draw byNamedLineSeq(0)(CD,IC,IA,AB);
draw byLine(G, H, byblack, SOLID_LINE, REGULAR_WIDTH);
draw byLabelLine(0)(AB, CD, GH);
draw byLabelsOnPolygon(G, E, B)(OMIT_FIRST_LABEL+OMIT_LAST_LABEL, 0);
draw byLabelsOnPolygon(H, F, C)(OMIT_FIRST_LABEL+OMIT_LAST_LABEL, 0);
}
\drawCurrentPictureInMargin
\problem{S}{i}{ una línea recta (\drawUnitLine{GH}) que se encuentra con otras dos líneas rectas, ((\drawUnitLine{CD} y \drawUnitLine{AB}) forma con ellas los ángulos alternos (\drawAngle{CFG} y \drawAngle{HEB}; \drawAngle{GFD} y \drawAngle{AEH}) iguales, estas dos líneas rectas son paralelas.}

\begin{center}
Si \drawUnitLine{CD} no es paralela a \drawUnitLine{AB} se encontrarán cuando se prolonguen.

Si es posible, deje que esas líneas no sean paralelas, sino que se encuentren cuando se prolonguen; entonces el ángulo externo \drawAngle{HEB} es mayor \drawAngle{CFG} \byref{prop:I.XVI}, pero ellos también son igual (hip.), lo cual es absurdo; de la misma manera se puede demostrar que no pueden encontrarse en el otro lado; $\therefore$ son paralelos.

Q. E. D. \byref{\hypref}
\end{center}

\qed

\starttheorem{Prop XXVIII. Theor.}\label{prop:I.XXVIII}

\defineNewPicture{
pair A, B, C, D, E, F, G, H, d;
A := (0, 0);
B := A shifted (7/2u, 0);
d := (0, -3/2u);
C := A shifted d;
D := B shifted d;
E := 9/20[A, B];
F := 11/20[C, D];
G := 7/4[F, E];
H := 4/3[E, F];
byAngleDefine(G, E, A, byblack, SOLID_SECTOR);
byAngleDefine(B, E, G, byyellow, SOLID_SECTOR);
byAngleDefine(A, E, H, byred, SOLID_SECTOR);
byAngleDefine(H, E, B, byblue, SOLID_SECTOR);
byAngleDefine(C, F, G, byblue, SOLID_SECTOR);
byAngleDefine(G, F, D, byred, SOLID_SECTOR);
draw byNamedAngleResized();
draw byLine(A, B, byred, SOLID_LINE, REGULAR_WIDTH);
draw byLine(C, D, byyellow, SOLID_LINE, REGULAR_WIDTH);
draw byLine(G, H, byblack, SOLID_LINE, REGULAR_WIDTH);
draw byLabelLine(0)(AB, CD, GH);
draw byLabelsOnPolygon(G, E, B)(OMIT_FIRST_LABEL+OMIT_LAST_LABEL, 0);
draw byLabelsOnPolygon(H, F, C)(OMIT_FIRST_LABEL+OMIT_LAST_LABEL, 0);
}
\drawCurrentPictureInMargin
\problem{S}{i}{ una línea recta (\drawUnitLine{GH}), corta otras dos líneas rectas (\drawUnitLine{AB} y \drawUnitLine{CD}), hace que el ángulo externo sea igual al ángulo interno y opuesto, en el mismo lado de la línea de corte (es decir, \drawAngle{GEA} $=$ \drawAngle{CFG} o \drawAngle{BEG} $=$ \drawAngle{GFD}), o si hace que juntos los dos ángulos internos en el mismo lado (\drawAngle{GFD} y \drawAngle{HEB}, o \drawAngle{CFG} y \drawAngle{AEH}) iguales a dos ángulos rectos, esas dos líneas rectas son paralelas.}

\begin{center}
Primero, si \drawAngle{GEA} $=$\drawAngle{CFG}, entonces \drawAngle{GEA} $=$ \drawAngle{HEB} \byref{prop:I.XV},
                     \\ $\therefore$ \drawAngle{CFG} $=$ \drawAngle{HEB} $\therefore$ \drawUnitLine{AB} ∥ \drawUnitLine{CD} \byref{prop:I.XXVII}.

Seguno, si \drawAngle{CFG} + \drawAngle{AEH} $=$ 
\drawTwoRightAngles


,
 \\ entonces \drawTwoRightAngles + \drawAngle{AEH} $=$ 
\drawTwoRightAngles


 \byref{prop:I.XIII},
                     \\ $\therefore$ \drawAngle{HEB} + \drawTwoRightAngles $=$ \drawAngle{CFG} + \drawAngle{AEH} \byref{ax:I.III}
                     \\ $\therefore$ \drawAngle{AEH} $=$ \drawAngle{HEB}  \\ $\therefore$ \drawAngle{CFG} ∥ \drawAngle{HEB} \byref{prop:I.XXVII}

Q. E. D. \drawUnitLine{AB} \drawUnitLine{CD}
\end{center}

\qed

\starttheorem{Prop XXIX. Theor.}\label{prop:I.XXIX}

\defineNewPicture{
pair A, B, C, D, E, F, G, H, I, J, d[];
A := (0, 0);
B := A shifted (7/2u, 0);
d1 := (0, -2u);
C := A shifted d1;
D := B shifted d1;
E := 11/20[A, B];
F := 9/20[C, D];
G := 7/4[F, E];
H := 4/3[E, F];
d2 := (3/2u, 1/2u);
I := E shifted -d2;
J := E shifted d2;
byAngleDefine(I, E, A, byblue, SOLID_SECTOR);
byAngleDefine(H, E, I, byyellow, SOLID_SECTOR);
byAngleDefine(B, E, H, byblack, SOLID_SECTOR);
byAngleDefine(G, E, B, byred, SOLID_SECTOR);
byAngleDefine(G, F, D, byblack, SOLID_SECTOR);
draw byNamedAngleResized();
draw byLine(I, E, byblack, SOLID_LINE, REGULAR_WIDTH);
draw byLine(E, J, byblack, DASHED_LINE, REGULAR_WIDTH);
draw byLine(A, B, byyellow, SOLID_LINE, REGULAR_WIDTH);
draw byLine(C, D, byred, SOLID_LINE, REGULAR_WIDTH);
draw byLine(G, H, byblue, SOLID_LINE, REGULAR_WIDTH);
draw byLabelLine(0)(AB, CD, GH);
draw byLabelsOnPolygon(A, E, G)(OMIT_FIRST_LABEL+OMIT_LAST_LABEL, 0);
draw byLabelsOnPolygon(H, F, C)(OMIT_FIRST_LABEL+OMIT_LAST_LABEL, 0);
draw byLabelPoint(I, lineAngle.IE - 90, 1);
draw byLabelPoint(J, lineAngle.EJ + 90, 1);
}
\drawCurrentPictureInMargin
\problem{U}{na}{ línea recta (\drawUnitLine{GH}) que cae sobre dos líneas rectas paralelas (\drawUnitLine{AB} y \drawUnitLine{CD}), hace que los ángulos alternos sean iguales entre sí; y también el externo igual al ángulo interno y opuesto en el mismo lado; y los dos ángulos internos en el mismo lado juntos equivalen a dos ángulos rectos.}

\begin{center}
Por si los ángulos alternos \drawAngle{IEA,HEI} y \drawAngle{GFD}, dibuja \drawUnitLine{IE} haciendo \drawAngle{HEI} $=$ \drawAngle{GFD} \byref{prop:I.XXIII}.

Por lo tanto \drawUnitLine{IE,EJ} ∥ \drawUnitLine{CD} \byref{prop:I.XXVII} y, por lo tanto, dos líneas rectas que se cruzan son paralelas a la misma línea recta, lo cual es imposible \byref{ax:I.XII}.

Por lo tanto, los ángulos alternos \drawAngle{IEA,HEI} y \drawAngle{GFD} no son desiguales, es decir, son iguales: \drawAngle{IEA,HEI} $=$ \drawAngle{GEB} \byref{prop:I.XV}; $\therefore$ \drawAngle{GEB} $=$ \drawAngle{GFD}, el ángulo externo igual al interno y opuesto en el mismo lado: si \drawAngle{BEH} se agrega a ambos, entonces \drawAngle{GFD} + \drawAngle{BEH} $=$ \drawAngle{BEH,GEB} $=$ 
\drawTwoRightAngles


 \byref{prop:I.XIII}. Es decir, los dos ángulos internos en el mismo lado de la línea de corte son iguales a dos ángulos rectos.

Q. E. D. \drawTwoRightAngles
\end{center}

\qed

\starttheorem{Prop XXX. Theor.}\label{prop:I.XXX}

\defineNewPicture{
pair A, B, C, D, E, F, G, H, I, J, K, d;
A := (0, 0);
B := A shifted (7/2u, 0);
d := (0, -u);
C := A shifted d;
D := B shifted d;
E := C shifted d;
F := D shifted d;
G := 13/20[A, B];
H := 7/20[E, F];
I := (G--H) intersectionpoint (C--D);
J := 3/2[H, G];
K := 5/4[G, H];
byAngleDefine(B, G, J, byyellow, SOLID_SECTOR);
byAngleDefine(D, I, J, byblue, SOLID_SECTOR);
byAngleDefine(F, H, J, byred, SOLID_SECTOR);
draw byNamedAngleResized();
draw byLine(A, B, byred, SOLID_LINE, REGULAR_WIDTH);
draw byLine(C, D, byyellow, SOLID_LINE, REGULAR_WIDTH);
draw byLine(E, F, byblue, SOLID_LINE, REGULAR_WIDTH);
draw byLine(J, K, byblack, SOLID_LINE, REGULAR_WIDTH);
draw byLabelLine(0)(AB, CD, EF, JK);
draw byLabelsOnPolygon(J, G, A)(OMIT_FIRST_LABEL+OMIT_LAST_LABEL, 0);
draw byLabelsOnPolygon(J, I, C)(OMIT_FIRST_LABEL+OMIT_LAST_LABEL, 0);
draw byLabelsOnPolygon(J, H, E)(OMIT_FIRST_LABEL+OMIT_LAST_LABEL, 0);
}
\drawCurrentPictureInMargin
\problem{L}{as}{ líneas rectas (\drawUnitLine{AB} y \drawUnitLine{EF}) que son paralelas a la misma línea recta (\drawUnitLine{CD}), son paralelas entre sí.}

\begin{center}
Deja que \drawUnitLine{JK} interseque $\left\{\vcenter{\nointerlineskip\hbox{\drawUnitLine{AB}}\nointerlineskip\hbox{\drawUnitLine{CD}}\nointerlineskip\hbox{\drawUnitLine{EF}}}\right\}$;

Entonces, $\drawAngle{G} = \drawAngle{I} = \drawAngle{H}$ \byref{prop:I.XXIX},

$\therefore \drawAngle{G} = \drawAngle{H}$

$\therefore \drawUnitLine{AB} \parallel \drawUnitLine{EF}$ \byref{prop:I.XXVIII}.
\end{center}

\qed

\startproblem{Prop XXXI. Prob.}\label{prop:I.XXXI}

\defineNewPicture[1/6]{
pair A, B, C, D, E, F, d;
A := (0, 0);
B := A shifted (7/2u, 0);
d := (0, -3u);
C := A shifted d;
D := B shifted d;
E := 4/5[A, B];
F := 1/5[C, D];
byAngleDefine(F, E, A, byyellow, SOLID_SECTOR);
byAngleDefine(E, F, D, byred, SOLID_SECTOR);
draw byNamedAngleResized();
draw byLine(E, F, byblack, SOLID_LINE, REGULAR_WIDTH);
byLineDefine(A, E, byred, SOLID_LINE, REGULAR_WIDTH);
byLineDefine(E, B, byred, DASHED_LINE, REGULAR_WIDTH);
byLineDefine(C, F, byblue, SOLID_LINE, REGULAR_WIDTH);
byLineDefine(F, D, byblue, SOLID_LINE, REGULAR_WIDTH);
draw byNamedLineSeq(0)(AE, EB);
draw byNamedLineSeq(0)(CF, FD);
draw byLabelsOnPolygon(A, E, B, noPoint, D, F, C, noPoint)(ALL_LABELS, 0);
}
\drawCurrentPictureInMargin
\problem{D}{esde}{ un punto dado \drawPointL[middle][EB]{E} para dibujar una línea recta paralela a una línea recta dada (\drawUnitLine{CF,FD}).}

\begin{center}
Dibuja \drawUnitLine{EF} desde el punto \drawPointL[middle][EB]{E} a cualquier punto \drawPointL[middle][CF]{F} en \drawUnitLine{CF,FD},
 \\ haz $\drawAngle{E} = \drawAngle{F}$ \byref{prop:I.XXIII},
 \\ entonces $\drawUnitLine{AE,EB} \parallel \drawUnitLine{CF,FD}$ \byref{prop:I.XXVII}.
Q. E. D.
\end{center}

\qed

\starttheorem{Prop XXXII. Theor.}\label{prop:I.XXXII}

\defineNewPicture[1/6]{
pair A, B, C, D, E;
A := (0, 0);
B := A shifted (-1u, -3u);
C := A shifted (5/2u, -3u);
D := 4/3[B, A];
E := A shifted (unitvector(C-B) scaled 3/2u);
byAngleDefine(D, A, E, byred, SOLID_SECTOR);
byAngleDefine(E, A, C, byblack, SOLID_SECTOR);
byAngleDefine(C, A, B, byblue, SOLID_SECTOR);
byAngleDefine(A, B, C, byyellow, SOLID_SECTOR);
byAngleDefine(B, C, A, byblack, SOLID_SECTOR);
draw byNamedAngleResized();
draw byLine(A, E, byblue, SOLID_LINE, REGULAR_WIDTH);
byLineDefine(A, D, byblack, DASHED_LINE, REGULAR_WIDTH);
byLineDefine(A, B, byblack, SOLID_LINE, REGULAR_WIDTH);
byLineDefine(B, C, byred, SOLID_LINE, REGULAR_WIDTH);
byLineDefine(C, A, byyellow, SOLID_LINE, REGULAR_WIDTH);
draw byNamedLineSeq(0)(CA,noLine,AD,AB,BC);
draw byLabelsOnPolygon(C, B, A, D, noPoint, E, A)(ALL_LABELS, 0);
}
\drawCurrentPictureInMargin
\problem{S}{i}{ cualquier lado (\drawUnitLine{AB}) de un triángulo es prolongado, el ángulo externo (\drawAngle{DAE,EAC}) es igual a la suma de los dos ángulos internos opuestos (\drawAngle{B} y \drawAngle{C}), y los tres ángulos internos de cada triángulo juntos son iguales a dos ángulos rectos.}

\begin{center}
A través del punto \drawUnitLine{AE} dibuja
                     \\ \drawUnitLine{BC} ∥ \drawAngle{DAE} \byref{prop:I.XXXI}.

Entonces
                    


\drawAngle{B} $=$ \drawAngle{EAC}
\drawAngle{C} $=$ \drawAngle{B}



                    \byref{prop:I.XXIX},

$\therefore$ \drawAngle{C} + \drawAngle{DAE,EAC} $=$ \drawAngle{B} \byref{ax:I.II},
                     \\ y por lo tanto,
                     \\ \drawAngle{CAB} + \drawAngle{C} + \drawAngle{DAE,EAC,CAB} $=$ \drawTwoRightAngles $=$ 
\drawTwoRightAngles


 \byref{prop:I.XIII}.

Q. E. D.
\end{center}

\qed

\starttheorem{Prop XXXIII. Theor.}\label{prop:I.XXXIII}

\defineNewPicture[1/4]{
pair A, B, C, D, d[];
d1 := (5/2u, 0);
d2 := (-7/8u, -3u);
A := (0, 0);
B := A shifted d1;
C := A shifted d2;
D := C shifted d1;
byAngleDefine(B, A, D, byyellow, SOLID_SECTOR);
byAngleDefine(D, A, C, byred, SOLID_SECTOR);
byAngleDefine(C, D, A, byblack, SOLID_SECTOR);
byAngleDefine(A, D, B, byblue, SOLID_SECTOR);
draw byNamedAngleResized();
draw byLine(A, D, byblack, SOLID_LINE, REGULAR_WIDTH);
byLineDefine(A, B, byred, SOLID_LINE, REGULAR_WIDTH);
byLineDefine(C, D, byred, DASHED_LINE, REGULAR_WIDTH);
byLineDefine(A, C, byblue, SOLID_LINE, REGULAR_WIDTH);
byLineDefine(B, D, byyellow, SOLID_LINE, REGULAR_WIDTH);
draw byNamedLineSeq(0)(AB,BD,CD,AC);
draw byLabelsOnPolygon(A, B, D, C)(ALL_LABELS, 0);
}
\drawCurrentPictureInMargin
\problem{L}{as}{líneas rectas (\drawUnitLine{AC} y \drawUnitLine{BD}) que unen los extremos adyacentes de dos líneas rectas iguales y paralelas (\drawUnitLine{AB} y \drawUnitLine{CD}), son ellas mismas iguales y paralelas.}

\begin{center}
Dibuja la diagonal \drawUnitLine{AD}.

\drawUnitLine{AB} $=$ \drawUnitLine{CD} (hip.)\\
\drawAngle{BAD} $=$ \drawAngle{CDA} \byref{\hypref}\\
y \drawUnitLine{AD} común a los dos triángulos;\\
$\therefore$ \drawUnitLine{AC} $=$ \drawUnitLine{BD}, y \drawAngle{ADB} $=$ \drawAngle{DAC} \byref{prop:I.IV};\\
y $\therefore$ \drawUnitLine{AC} ∥ \drawUnitLine{BD} \byref{prop:I.XXIX}.

Q. E. D. \byref{prop:I.XXVII}
\end{center}

\qed

\starttheorem{Prop XXXIV. Theor.}\label{prop:I.XXXIV}

\defineNewPicture{
pair A, B, C, D, d[];
d1 := (5/2u, 0);
d2 := (-7/8u, -3u);
A := (0, 0);
B := A shifted d1;
C := A shifted d2;
D := C shifted d1;
byAngleDefine(B, A, D, byblue, SOLID_SECTOR);
byAngleDefine(D, A, C, byred, SOLID_SECTOR);
byAngleDefine(C, D, A, byyellow, SOLID_SECTOR);
byAngleDefine(A, D, B, byred, SOLID_SECTOR);
byAngleDefine(A, C, D, byblack, SOLID_SECTOR);
byAngleDefine(D, B, A, byblack, SOLID_SECTOR);
draw byNamedAngleResized();
draw byLine(A, D, byblack, SOLID_LINE, REGULAR_WIDTH);
byLineDefine(A, B, byred, SOLID_LINE, REGULAR_WIDTH);
byLineDefine(C, D, byred, DASHED_LINE, REGULAR_WIDTH);
byLineDefine(A, C, byyellow, SOLID_LINE, REGULAR_WIDTH);
byLineDefine(B, D, byblue, SOLID_LINE, REGULAR_WIDTH);
draw byNamedLineSeq(0)(AB,BD,CD,AC);
draw byLabelsOnPolygon(A, B, D, C)(ALL_LABELS, 0);
}
\drawCurrentPictureInMargin
\problem{L}{os}{lados y ángulos opuestos de cualquier paralelogramo son iguales, y la diagonal (\drawUnitLine{AD}) lo divide en dos partes iguales.}

\begin{center}
Desde $\left\{ \begin{aligned} \drawAngle{BAD} &= \drawAngle{CDA} \\ \drawAngle{DAC} &= \drawAngle{ADB} \end{aligned} \right\}$ \byref{prop:I.XXIX} y \drawUnitLine{AD} común a los dos triángulos.

$\therefore \left\{ \begin{aligned} \drawUnitLine{AB} &= \drawUnitLine{CD} \\ \drawUnitLine{AC} &= \drawUnitLine{BD} \\ \drawAngle{B} &= \drawAngle{C} \end{aligned} \right\}$ \byref{prop:I.XXVI}

y $\drawAngle{BAD,DAC} = \drawAngle{CDA,ADB}$ \byref{ax:I.II}.

Por lo tanto, los lados y ángulos opuestos del paralelogramo son iguales: y como los triángulos \drawLine{AD,CD,AC} y \drawLine{AB,BD,AD} son iguales en todos los aspectos \byref{prop:I.IV}, la diagonal divide el paralelogramo en dos partes iguales.

Q. E. D.
\end{center}

\qed

\starttheorem{Prop XXXV. Theor.}\label{prop:I.XXXV}

\defineNewPicture{
pair A, B, C, D, E, F, G, d[];
d1 := (7/4u, 0);
d2 := (u, -3u);
d3 := (-2u, -3u);
A := (0, 0);
B := A shifted d1;
C := A shifted d2;
D := C shifted d1;
E := C shifted -d3;
F := D shifted -d3;
G := (B--D) intersectionpoint (C--E);
draw byPolygon(A,B,G,C)(byyellow);
draw byPolygon(E,F,D,G)(byyellow);
draw byPolygon(B,E,G)(byyellow);
draw byPolygon(C,D,G)(byblue);
byAngleDefine(E, A, C, byred, SOLID_SECTOR);
byAngleDefine(F, B, D, byblue, SOLID_SECTOR);
byAngleDefine(A, E, C, byblack, SOLID_SECTOR);
byAngleDefine(B, F, D, white, SOLID_SECTOR);
draw byNamedAngleResized();
draw byNamedAngleDummySides(BFD);
byLineDefine(A, C, byblue, SOLID_LINE, REGULAR_WIDTH);
byLineDefine(B, D, byred, SOLID_LINE, REGULAR_WIDTH);
byLineDefine(C, D, byblack, SOLID_LINE, REGULAR_WIDTH);
draw byNamedLineSeq(0)(AC,CD,BD);
draw byLabelsOnPolygon(C, A, B, E, F, D)(ALL_LABELS, 0);
}
\drawCurrentPictureInMargin
\problem{P}{aralelogramos}{ en la misma base, y entre las mismos paralelas, son (en área) iguales.}

\begin{center}
A causa de las paralelas,
                    


\drawAngle{A} $=$ \drawAngle{B};
\drawAngle{E} $=$ \drawAngle{F};
\drawUnitLine{AC} $=$ \drawUnitLine{BD}





\byref{prop:I.XXIX}
\byref{prop:I.XXIX}
\byref{prop:I.XXXIV}

Pero, \drawFromCurrentPicture[middle][polygonABC]{
startAutoLabeling;
draw byNamedPolygon (ABGC, BEG);
stopAutoLabeling;
draw byNamedLine (AC);
} $=$ \drawFromCurrentPicture[middle][polygonEFD]{
startAutoLabeling;
draw byNamedPolygon (EFDG, BEG);
stopAutoLabeling;
draw byNamedLine (BD);
} \byref{prop:I.VIII}
                     \\ $\therefore$ \drawFromCurrentPicture[middle][polygonAFDC]{
startAutoLabeling;
draw byNamedPolygon (ABGC, EFDG, BEG, CDG);
stopAutoLabeling;
draw byNamedLine (AC);
} - \polygonEFD = \drawPolygon{ABGC, CDG}$,

y $\polygonAFDC - \polygonABC = \drawPolygon{EFDG, CDG}$;

$\therefore \drawPolygon{ABGC, CDG} = \drawPolygon{EFDG, CDG}$.

Q. E. D.
\end{center}

\qed

\starttheorem{Prop XXXVI. Theor.}\label{prop:I.XXXVI}

\defineNewPicture{
pair A, B, C, D, E, F, G, H, J, I, d[];
numeric h;
h := 3u;
d1 := (3/2u, 0);
d2 := (2/3u, -h);
d3 := (-8/3u, -h);
d4 := (-1/2u, -h);
A := (0, 0);
B := A shifted d1;
C := A shifted d2;
D := C shifted d1;
E := C shifted -d3;
F := D shifted -d3;
G := E shifted d4;
H := F shifted d4;
I := (B--D) intersectionpoint (C--E);
J := (E--G) intersectionpoint (D--F);
draw byPolygon(A,B,I,C)(byred);
draw byPolygon(C,D,I)(byred);
draw byPolygon(I,D,J,E)(byblue);
draw byPolygon(E,F,J)(byyellow);
draw byPolygon(J,F,H,G)(byyellow);
byLineDefine(C, E, byyellow, SOLID_LINE, REGULAR_WIDTH);
byLineDefine(D, F, byblack, DASHED_LINE, REGULAR_WIDTH);
byLineDefine(C, D, byblack, SOLID_LINE, REGULAR_WIDTH);
byLineDefine(E, F, byred, SOLID_LINE, REGULAR_WIDTH);
draw byNamedLineSeq(0)(CE,EF,DF,CD);
draw byLineFull(G, H, byblue, 0, 0)(E, F, 1, 1, 0);
draw byLabelsOnPolygon(A, B, D, C)(ALL_LABELS, 0);
draw byLabelsOnPolygon(E, F, H, G)(ALL_LABELS, 0);
}
\drawCurrentPictureInMargin
\problem{P}{aralelogramos}{ (\drawUnitLine{CE} and \drawUnitLine{DF}) en bases iguales, y entre las mismas paralelas, son iguales.}

\begin{center}
Dibuja \drawUnitLine{CD} y \drawUnitLine{GH},
 \\ \drawUnitLine{EF} $=$ \drawUnitLine{CD} $=$ \drawUnitLine{EF}, por \byref{prop:I.XXXIV,\hypref};
                     \\ $\therefore$ \drawUnitLine{CE}$=$ y ∥ \drawUnitLine{DF};
 \\ $\therefore$ \polygon $=$ y ∥ \polygon \byref{prop:I.XXXIII}

Y por lo tanto \polygon es un paralelogramo:
                     \\ pero \polygon $=$ \polygon $=$  \byref{prop:I.XXXV}
                     \\ $\therefore$  $=$  \byref{ax:I.I}.

Q. E. D.
\end{center}

\qed

\starttheorem{Prop XXXVII. Theor.}\label{prop:I.XXXVII}

\defineNewPicture{
pair A, B, C, D, E, F, G, H, I, d[];
d1 := (3/2u, 0);
d2 := (1/2u, -3u);
d3 := (-7/4u, -3u);
A := (0, 0);
B := A shifted d1;
C := A shifted d2;
D := C shifted d1;
E := C shifted -d3;
F := D shifted -d3;
G := (B--D) intersectionpoint (C--E);
H := 11/10[F, A];
I := 11/10[A, F];
draw byPolygon(A,B,C)(byblue);
draw byPolygon(B,C,G)(byyellow);
draw byPolygon(C,D,G)(byyellow);
draw byPolygon(D,G,E)(byblack);
draw byPolygon(E,F,D)(byred);
draw byLine(B, D, byred, SOLID_LINE, REGULAR_WIDTH);
draw byLine(E, C, byblue, SOLID_LINE, REGULAR_WIDTH);
byLineDefine(A, C, byred, DASHED_LINE, REGULAR_WIDTH);
byLineDefine(F, D, byblue, DASHED_LINE, REGULAR_WIDTH);
byLineDefine(C, D, byblack, SOLID_LINE, REGULAR_WIDTH);
draw byNamedLineSeq(0)(AC,CD,FD);
draw byLine(H, I, byblack, DASHED_LINE, REGULAR_WIDTH);
draw byLabelsOnPolygon(H, A, B, E, F, I, noPoint)(ALL_LABELS, 0);
draw byLabelsOnPolygon(F, D, C, A)(OMIT_FIRST_LABEL+OMIT_LAST_LABEL, 0);
}
\drawCurrentPictureInMargin
\problem{T}{riángulos}{ \drawUnitLine{CD} y \drawUnitLine{AC} en la misma base (\drawUnitLine{BD}) y entre las mismas paralelas son iguales.}

\begin{center}
Dibuja \drawUnitLine{FD} ∥ \drawUnitLine{EC}
\drawUnitLine{HI} ∥ \polygon



                \byref{prop:I.XXXI}

Prolonga \polygon.

\polygon y \polygon son paralelogramos en la misma base y entre las mismas paralelas, y por lo tanto iguales. \byref{prop:I.XXXV}

$\therefore$



\polygon $=$ dos veces \polygon
 $=$ dos veces 



                \byref{prop:I.XXXIV}

$\therefore$  $=$  .

Q. E. D.
\end{center}

\qed

\starttheorem{Prop XXXVIII. Theor.}\label{prop:I.XXXVIII}

\defineNewPicture{
pair A, B, C, D, E, F, G, H, J, I, d[];
numeric h;
h := 5/2u;
d1 := (3/2u, 0);
d2 := (3/4u, -h);
d3 := (7/3u, h);
d4 := (-1/4u, -h);
A := (0, 0);
B := A shifted d1;
C := A shifted d2;
D := C shifted d1;
E := C shifted d3;
F := D shifted d3;
G := E shifted d4;
H := F shifted d4;
I := (xpart(A), ypart(C));
J := (xpart(F), ypart(C));
draw byPolygon(A,B,C)(byyellow);
draw byPolygon(B,C,D)(byred);
draw byPolygon(E,F,H)(byblack);
draw byPolygon(E,G,H)(byblue);
draw byLine(B, D, byblue, SOLID_LINE, REGULAR_WIDTH);
draw byLine(E, G, byred, SOLID_LINE, REGULAR_WIDTH);
byLineDefine(A, C, byblue, DASHED_LINE, REGULAR_WIDTH);
byLineDefine(F, H, byred, DASHED_LINE, REGULAR_WIDTH);
byLineDefine(A, F, byblack, DASHED_LINE, REGULAR_WIDTH);
draw byNamedLineSeq(0)(AC,AF,FH);
draw byLine(I, J, byblack, DASHED_LINE, REGULAR_WIDTH);
draw byLabelsOnPolygon(A, B, E, F, noPoint)(ALL_LABELS, 0);
draw byLabelsOnPolygon(H, G, D, C, noPoint)(ALL_LABELS, 0);
}
\drawCurrentPictureInMargin
\problem{T}{riángulos}{ \drawUnitLine{AC} y \drawUnitLine{BD} en bases iguales y entre las mismas paralelas son iguales.}

\begin{center}
Dibuja \drawUnitLine{FH} ∥ \drawUnitLine{EG}
\polygon ∥ \polygon



                \byref{prop:I.XXXI}

\polygon $=$ \polygon \byref{prop:I.XXXVI};
                 \\ \polygon $=$ dos veces \polygon \byref{prop:I.XXXIV},
                 \\ y  $=$ dos veces  \byref{prop:I.XXXIV},
                 \\ $\therefore$  $=$  \byref{ax:I.VII}.

Q. E. D.
\end{center}

\qed

\starttheorem{Prop XXXIX. Theor.}\label{prop:I.XXXIX}

\defineNewPicture{
pair A, B, C, D, E, F, G;
A := (0, 0);
B := A shifted (5/2u, 0);
C := A shifted (3/4u, -5/2u);
D := C shifted (3u, 0);
E = whatever[A, D] = whatever[B, C];
F := 11/8[C, B];
G := 13/8[C, B];
draw byPolygon(A,B,E)(byred);
draw byPolygon(A,E,C)(byyellow);
draw byPolygon(B,E,D)(byblack);
draw byPolygon(E,C,D)(byyellow);
draw byPolygon(F,B,D)(byblue);
byLineDefine(A, F, byred, SOLID_LINE, REGULAR_WIDTH);
byLineDefine(D, F, byyellow, SOLID_LINE, REGULAR_WIDTH);
byLineDefine(A, B, byblue, SOLID_LINE, REGULAR_WIDTH);
byLineDefine(C, D, byblack, SOLID_LINE, REGULAR_WIDTH);
byLineDefine(C, G, byblack, DASHED_LINE, REGULAR_WIDTH);
draw byNamedLineSeq(-4/5)(AB,AF,DF,CD,CG);
draw byLabelsOnPolygon(F, D, C, A)(OMIT_FIRST_LABEL+OMIT_LAST_LABEL, 0);
draw byLabelsOnPolygon(A, B, F)(OMIT_FIRST_LABEL+OMIT_LAST_LABEL, 0);
draw byLabelsOnPolygon(A, F, G, noPoint)(ALL_LABELS, 0);
}
\drawCurrentPictureInMargin
\problem{S}{i}{ \drawUnitLine{AF}, que une los vértices de los triángulos, no sea ∥ \drawUnitLine{CD}, dibuja \drawUnitLine{CG} ∥ \drawUnitLine{DF} \byref{prop:I.XXXI}, encontrando \drawUnitLine{AF}.}

\begin{center}
Dibuja \drawUnitLine{CD}.

Porque \polygon ∥ \polygon (const.)
                     \\ \polygon $=$ \polygon \byref{\constref}:
                     \\ Pero \polygon $=$ \drawUnitLine{AF} (hip.);

$\therefore$ \drawUnitLine{CD} $=$ \drawUnitLine{AB}, una parte igual al todo, lo cual es absurdo.

$\therefore$ \drawUnitLine{CD} ∦ \drawUnitLine{AB}; y de la misma manera se puede demostrar que ninguna otra línea excepto
                     \\ \drawUnitLine{CD} es ∥ ; $\therefore$  ∥ .

Q. E. D. \byref{prop:I.XXXVII} \byref{\hypref}
\end{center}

\qed

\starttheorem{Prop XL. Theor.}\label{prop:I.XL}

\defineNewPicture{
pair A, B, C, D, E, F, G, H, d;
A := (0, 0);
B := A shifted (3/2u, 0);
C := A shifted (-3/2u, -9/4u);
d := (7/4u, 0);
D := C shifted d;
E := B shifted (-2/3u, -9/4u);
F := E shifted d;
G := 5/4[E, B];
H := 2[B, G];
draw byPolygon(A,C,D)(byyellow);
draw byPolygon(B,E,F)(byred);
draw byPolygon(G,B,F)(byblue);
draw byLine(E, H, byblack, DASHED_LINE, REGULAR_WIDTH);
byLineDefine(A, G, byred, SOLID_LINE, REGULAR_WIDTH);
byLineDefine(F, G, byyellow, SOLID_LINE, REGULAR_WIDTH);
byLineDefine(A, B, byblue, SOLID_LINE, REGULAR_WIDTH);
byLineDefine(C, D, byblack, SOLID_LINE, REGULAR_WIDTH);
byLineDefine(E, F, byblack, SOLID_LINE, REGULAR_WIDTH);
byLineDefine(D, E, byblue, DASHED_LINE, REGULAR_WIDTH);
draw byNamedLineSeq(-4/5)(AB,AG,FG,EF,DE,CD);
draw byLabelsOnPolygon(G, F, E, D, C, A)(OMIT_FIRST_LABEL+OMIT_LAST_LABEL, 0);
draw byLabelsOnPolygon(A, B, G)(OMIT_FIRST_LABEL+OMIT_LAST_LABEL, 0);
draw byLabelsOnPolygon(A, G, H, noPoint)(ALL_LABELS, 0);
}
\drawCurrentPictureInMargin
\problem{L}{os}{triángulos iguales sobre bases iguales (\drawFromCurrentPicture[bottom][polygonACD]{
startAutoLabeling;
draw byNamedPolygon (ACD);
stopAutoLabeling;
draw byNamedLineFull(A, A, 1, 1, 0, 0)(CD);
} y \drawFromCurrentPicture[bottom][polygonBEF]{
startAutoLabeling;
draw byNamedPolygon (BEF);
stopAutoLabeling;
draw byNamedLineFull(B, B, 1, 1,  0, 0)(EF);
}) y en el mismo lado, están entre las mismas paralelas.}

\begin{center}
Dibuja \polygon.
 \\ Porque  ∥  (const.)
                     \\  $=$  pero  $=$   \\ $\therefore$  $=$ , una parte igual al todo, lo cual es absurdo.
                     \\ $\therefore$  ∦ : y de la misma manera se puede demostrar que ninguna otra línea excepto
                     \\  es ∥ : $\therefore$  ∥ .

Q. E. D. \byref{\constref}
\end{center}

\qed

\starttheorem{Prop XLI. Theor.}\label{prop:I.XLI}

\defineNewPicture{
pair A, B, C, D, E, F, G, d;
A := (0, 0);
d := (2u, 0);
B := A shifted d;
C := B shifted (2u, 0);
D := A shifted (4/3u, -5/2u);
E := D shifted d;
F = whatever[B, E] = whatever[D, C];
G = whatever[A, E] = whatever[D, C];
draw byPolygon(A,B,F,G)(byyellow);
draw byPolygon(G,F,E)(byyellow);
draw byPolygon(A,G,D)(byblue);
draw byPolygon(D,E,G)(byblue);
draw byPolygon(C,F,E)(byred);
draw byLine(A, E, byred, SOLID_LINE, REGULAR_WIDTH);
draw byLineFull(A, C, byblack, 1, 0)(D, E, 1, 1, 0);
draw byLineFull(D, E, byblack, 0, 0)(A, C, 1, 1, 0);
draw byLabelsOnPolygon(E, D, A, B, C)(ALL_LABELS, 0);
}
\drawCurrentPictureInMargin
\problem{S}{i}{ un paralelogramo \drawUnitLine{DE} y un triángulo \drawUnitLine{AC} están sobre la misma base \drawUnitLine{DE} y entre las mismas paralelas \drawUnitLine{AE} y \polygon, el paralelogramo es el doble del triángulo.}

\begin{center}
Dibuja la diagonal \polygon;
                 \\ Entonces \polygon $=$ \polygon \byref{prop:I.XXXVII}
                 \\ \polygon $=$ dos veces  \byref{prop:I.XXXIV}
                 \\ $\therefore$  $=$ dos veces .

Q. E. D.
\end{center}

\qed

\startproblem{Prop XLII. Prob.}\label{prop:I.XLII}

\defineNewPicture{
pair A, B, C, D, E, F, G, H, I, J, d[];
A := (0, 0);
d1 := (8/5u, 0);
B := A shifted d1;
C := B shifted (3/2u, 0);
D := A shifted (u, -13/5u);
E := D shifted d1;
F := (B--E) intersectionpoint (D--C);
G := 2[D, E];
d2 := (-xpart(E)+xpart(D)-1/2u, 0);
H := B shifted d2;
I := E shifted d2;
J := D shifted d2;
draw byPolygon(A,B,F,D)(byyellow);
draw byPolygon(D,E,F)(byyellow);
draw byPolygon(C,F,E)(byblue);
draw byPolygon(E,C,G)(byblack);
byAngleDefine(B, E, D, byblue, SOLID_SECTOR);
byAngleDefine(H, I, J, byyellow, SOLID_SECTOR);
setAttribute("angle", "Standalone", "HIJ", 1);
draw byNamedAngleResized();
draw byLine(B, E, byred, SOLID_LINE, REGULAR_WIDTH);
byLineDefine(A, D, byred, DASHED_LINE, REGULAR_WIDTH);
byLineDefine(C, E, byyellow, SOLID_LINE, REGULAR_WIDTH);
byLineDefine(A, C, byblue, SOLID_LINE, REGULAR_WIDTH);
byLineDefine(D, E, byblack, SOLID_LINE, REGULAR_WIDTH);
byLineDefine(E, G, byblack, DASHED_LINE, REGULAR_WIDTH);
draw byNamedLineSeq(0)(CE,AC,AD,DE,EG,noLine);
draw byLabelsOnPolygon(G, E, D, A, B, C)(ALL_LABELS, 0);
startAutoLabeling;
draw byNamedAngleSidesFull(HIJ)();
stopAutoLabeling;
}
\drawCurrentPictureInMargin
\problem[3]{C}{onstruir}{un paralelogramo igual a un triángulo dado (\drawPolygon[bottom][polygonCDG]{DEF,CFE,ECG}) y que tenga un ángulo igual a un ángulo rectilíneo dado (\drawAngle{I}).}

\begin{center}
Haz $\drawUnitLine{DE} = \drawUnitLine{EG}$ \byref{prop:I.X}\\
Dibuja \drawUnitLine{CE}.

Haz $\drawAngle{E} = \drawAngle{I}$ \byref{prop:I.XXIII}\\
Dibuja $\left\{
	\begin{aligned}
		\drawUnitLine{AD} &\parallel \drawUnitLine{BE}\\
		\drawUnitLine{AC} &\parallel \drawUnitLine{DE}\\
	\end{aligned}
	\right\}$ \byref{prop:I.XXXI}

$\drawPolygon[bottom][polygonABED]{ABFD,DEF} = 2 \times \drawPolygon[bottom][polygonCED]{DEF,CFE}$ \byref{prop:I.XLI}\\
pero $\polygonCED = \drawPolygon[bottom][polygonDCG]{ECG}$ \byref{prop:I.XXXVIII}

$\therefore \polygonABED = \polygonCDG$.
\end{center}

\qed

\starttheorem{Prop XLIII. Theor.}\label{prop:I.XLIII}

\defineNewPicture[1/2]{
pair A, B, C, D, E, F, G, H, I, d[];
path q[];
d1 := (5/2u, 0);
d2 := (-u, -3u);
A := (0, 0);
B := A shifted d1;
C := A shifted d2;
D := C shifted d1;
E := 2/5[A, D];
q1 := (E shifted d1) -- (E shifted -d1);
q2 := (E shifted d2) -- (E shifted -d2);
F := q1 intersectionpoint (A--C);
G := q1 intersectionpoint (B--D);
H := q2 intersectionpoint (A--B);
I := q2 intersectionpoint (C--D);
draw byPolygon(A,E,H)(byyellow);
draw byPolygon(A,E,F)(byyellow);
draw byPolygon(H,B,G,E)(byblue);
draw byPolygon(F,C,I,E)(byblack);
draw byPolygon(I,D,E)(byred);
draw byPolygon(G,D,E)(byred);
draw byLabelsOnPolygon(C, F, A, H, B, G, D, I)(ALL_LABELS, 0);
draw byLabelsOnPolygon(F, E, H)(OMIT_FIRST_LABEL+OMIT_LAST_LABEL, 0);
}
\drawCurrentPictureInMargin
\problem{L}{os}{complementos de los paralelogramos que están alrededor de la diagonal de cualquier paralelogramo, son iguales entre sí.}

\begin{center}
$\drawPolygon[bottom][polygonHBEG]{HBEG} = \drawPolygon[bottom][polygonCIFE]{CIFE}$ \byref{prop:I.XXXIV}\\
y $\drawPolygon[bottom][polygonAEH]{AEH} = \drawPolygon[bottom][polygonAEF]{AEF}$ \byref{prop:I.XXXIV}\\
$\therefore \drawPolygon[bottom][polygonHBEG]{HBEG} = \drawPolygon[bottom][polygonCIFE]{CIFE}$ \byref{ax:I.III}.

Q. E. D.
\end{center}

\qed

\startproblem{Prop XLIV. Prob.}\label{prop:I.XLIV}

\defineNewPicture{
pair A, B, C, D, E, F, G, H, I, J, K, L, M, N, O, d[];
path q[];
d1 := (3u, 0);
d2 := (-3/2u, -3u);
d3 := (3/2u, 5/2u);
d4 := -d2 +1/2d1;
A := (0, 0);
B := A shifted d1;
C := A shifted d2;
D := C shifted d1;
E := 2/5[C, B];
q1 := (E shifted d1) -- (E shifted -d1);
q2 := (E shifted d2) -- (E shifted -d2);
F := q1 intersectionpoint (A--C);
G := q1 intersectionpoint (B--D);
H := q2 intersectionpoint (A--B);
I := q2 intersectionpoint (C--D);
J := A shifted d3;
K := J shifted (2(xpart(A)-xpart(H)), 0);
L := (xpart(1/3[J, K]), ypart(F)-ypart(A)+ypart(J));
M := A shifted d4;
N := F shifted d4;
O := E shifted d4;
draw byPolygon(J,K,L)(byred);
draw byPolygon(A,H,E,F)(byyellow);
draw byPolygon(E,G,D,I)(byblue);
byAngleDefine(A, F, E, byblue, SOLID_SECTOR);
byAngleDefine(F, E, I, byred, SOLID_SECTOR);
byAngleDefine(E, I, D, byblack, SOLID_SECTOR);
byAngleDefine(M, N, O, byyellow, SOLID_SECTOR);
setAttribute("angle", "Standalone", "MNO", 1);
draw byNamedAngleResized();
draw byLine(B, E, byred, SOLID_LINE, REGULAR_WIDTH);
draw byLine(E, C, byblack, SOLID_LINE, THIN_WIDTH);
byLineDefine(A, F, byred, DASHED_LINE, REGULAR_WIDTH);
byLineDefine(F, C, byblack, SOLID_LINE, THIN_WIDTH);
draw byLine(H, E, byblue, DASHED_LINE, REGULAR_WIDTH);
byLineDefine(B, G, byyellow, SOLID_LINE, REGULAR_WIDTH);
byLineDefine(A, H, byblue, SOLID_LINE, REGULAR_WIDTH);
byLineDefine(H, B, byblack, SOLID_LINE, REGULAR_WIDTH);
byLineDefine(F, E, byblack, DASHED_LINE, REGULAR_WIDTH);
byLineDefine(E, G, byblack, SOLID_LINE, REGULAR_WIDTH);
byLineDefine(C, D, byyellow, DASHED_LINE, REGULAR_WIDTH);
draw byNamedLineSeq(0)(FE,EG,BG,HB,AH,AF,FC,CD);
draw byLabelsOnPolygon(K, J, L)(ALL_LABELS, 0);
draw byLabelsOnPolygon(F, A, H, B, G, D, I, C)(ALL_LABELS, 0);
draw byLabelsOnPolygon(F, E, H)(OMIT_FIRST_LABEL+OMIT_LAST_LABEL, 0);
startAutoLabeling;
draw byNamedAngleSidesFull(MNO)();
stopAutoLabeling;
}
\drawCurrentPictureInMargin
\problem{A}{}{ una línea recta dada (\drawUnitLine{EG}) para aplicar un paralelogramo igual a un triángulo dado (\drawAngle{N}) y que tenga un ángulo igual a un ángulo rectilíneo dado (\polygon).}

\begin{center}
Haz \drawAngle{F} $=$ \drawAngle{N} con \drawUnitLine{FE} $=$ \drawUnitLine{EG} \byref{prop:I.XLII} y teniendo uno de sus lados \drawUnitLine{AH} limítrofe y en continuación de \drawUnitLine{BG}. Prolonga \drawUnitLine{HE} hasta que se encuentre \drawUnitLine{BE} ∥ \drawUnitLine{AF} dibuja \drawUnitLine{CD} y prolonga hasta que se encuentre \drawUnitLine{FE,EG} continuada; dibuja \drawUnitLine{BG} ∥ \drawUnitLine{HE} encuentra \polygon prolongada, y prolonga \polygon.

\polygon $=$ \polygon \byref{prop:I.XLIII}
                     \\ pero \polygon $=$ \drawAngle{F} (const.)
                     \\ $\therefore$ \drawAngle{E} $=$ \drawAngle{I}; y
                     \\ \drawAngle{N} $=$  $=$  $=$  \byref{\constref}

Q. E. D. \byref{prop:I.XXIX,\constref}
\end{center}

\qed

\startproblem{Prop XLV. Prob.}\label{prop:I.XLV}

\defineNewPicture{
pair A, B, C, D, E, F, G, H, I, J, K, L, M, N, O, P, d[];
numeric a, h[], b[], s[];
a := 15;
A := (0, 0);
B := A shifted (0, 2u);
C := A shifted (4/3u, u);
D := A shifted (2u, -3/2u);
E := A shifted (-6/5u, -u);
b1 := arclength(B--C);
h1 := distanceToLine(A, B--C);
s1 := (b1 * h1)/2;
b2 := arclength(C--D);
h2 := distanceToLine(A, C--D);
s2 := (b2 * h2)/2;
b3 := arclength(D--E);
h3 := distanceToLine(A, D--E);
s3 := (b3 * h3)/2;
d1 := (0, ypart(D)-u);
d2 := (0, -b3/2) rotated -a;
d3 := (h3*(1/cosd(a)), 0);
d6 := (u, 0);
F := (-u, 0) shifted d1;
G := F shifted d3;
H := F shifted d2;
I := G shifted d2;
d4 := (2*(s2/b3)*(1/cosd(a)), 0);
J := G shifted d4;
K := J shifted d2;
d5 := (2*(s1/b3)*(1/cosd(a)), 0);
L := J shifted d5;
M := L shifted d2;
N := L shifted d6;
O := M shifted d6;
P := K shifted d6;
draw byPolygon(A,B,C)(byred);
draw byPolygon(A,C,D)(byyellow);
draw byPolygon(A,D,E)(byblue);
byLineDefine(A, C, byblue, SOLID_LINE, REGULAR_WIDTH);
byLineDefine(A, D, byred, SOLID_LINE, REGULAR_WIDTH);
draw byNamedLineSeq(0)(AC,AD);
draw byPolygon(F,G,I,H)(byblue);
draw byPolygon(G,J,K,I)(byyellow);
draw byPolygon(J,L,M,K)(byred);
byAngleDefine(G, I, H, byyellow, SOLID_SECTOR);
byAngleDefine(J, K, I, byblack, SOLID_SECTOR);
byAngleDefine(L, M, K, byblue, SOLID_SECTOR);
byAngleDefine(N, O, P, byred, SOLID_SECTOR);
setAttribute("angle", "Standalone", "NOP", 1);
draw byNamedAngleResized();
draw byLine(G, I, byred, SOLID_LINE, REGULAR_WIDTH);
draw byLine(J, K, byblue, SOLID_LINE, REGULAR_WIDTH);
draw byLabelsOnPolygon(A, B, C, D, E)(ALL_LABELS, 0);
draw byLabelsOnPolygon(F, G, J, L, M, K, I, H)(ALL_LABELS, 0);
startAutoLabeling;
draw byNamedAngleSidesFull(NOP)();
stopAutoLabeling;
}
\drawCurrentPictureInMargin

\problem{P}{ara}{ construir un paralelogramo igual a una figura rectilínea dada (\drawAngle{O}) y que tenga un ángulo igual a un ángulo rectilíneo dado (\drawUnitLine{AD}).}

\begin{center}
Dibuja \drawUnitLine{AC} y \drawAngle{I} dividiendo la figura rectilínea en triángulos.

Construye \drawAngle{O} $=$ \drawUnitLine{GI}  \\ teniendo \drawAngle{K} $=$ \drawAngle{O}  \\ teniendo \polygon $=$ \polygon \byref{prop:I.XLIV}
                     \\ para \drawAngle{M} aplica \drawAngle{O} $=$   \\ teniendo  $=$  \byref{prop:I.XLIV}
                     \\ $\therefore$  $=$   \\ y  es un paralelogramo \byref{prop:I.XXIX,prop:I.XIV,prop:I.XXX}
                     \\ teniendo  $=$ .

Q. E. D.
\end{center}

\qed

\startproblem{Prop XLVI. Prob.}\label{prop:I.XLVI}

\defineNewPicture{
pair A, B, C, D;
numeric d;
d := 7/2u;
A := (0, 0);
B := A shifted (d, 0);
C := A shifted (0, -d);
D := A shifted (d, -d);
byAngleDefine(B, A, C, byblack, SOLID_SECTOR);
byAngleDefine(D, B, A, byblue, SOLID_SECTOR);
byAngleDefine(C, D, B, byred, SOLID_SECTOR);
byAngleDefine(A, C, D, byyellow, SOLID_SECTOR);
draw byNamedAngleResized();
byLineDefine(A, B, byred, SOLID_LINE, REGULAR_WIDTH);
byLineDefine(B, D, byyellow, SOLID_LINE, REGULAR_WIDTH);
byLineDefine(D, C, byblack, SOLID_LINE, REGULAR_WIDTH);
byLineDefine(C, A, byblue, SOLID_LINE, REGULAR_WIDTH);
draw byNamedLineSeq(0)(AB,BD,DC,CA);
draw byLabelsOnPolygon(A, B, D, C)(ALL_LABELS, 0);
}
\drawCurrentPictureInMargin
\problem{S}{obre}{ una línea recta dada \drawUnitLine{DC} construir un cuadrado.}

\begin{center}
Dibuja \drawUnitLine{CA} ⊥ y $=$ \drawUnitLine{DC} \byref{prop:I.XI,prop:I.III}

Dibuja \drawUnitLine{AB} ∥ \drawUnitLine{DC}, y encuentra \drawUnitLine{BD} dibujada ∥ \drawUnitLine{CA}.

En \drawFromCurrentPicture[bottom][polygonABDC]{
startTempAngleScale(angleScale*3/4);
draw byNamedAngle(A,B,C,D);
draw byNamedLineSeq(0)(AB,BD,DC,CA);
draw byLabelsOnPolygon(A, B, D, C)(ALL_LABELS, 0);
stopTempAngleScale;
} \drawUnitLine{CA} $=$ \drawUnitLine{DC} (const.)
                     \\ \drawAngle{C} $=$ un ángulo recto (const.)
                     \\ $\therefore$ \drawAngle{D} $=$ \drawAngle{C} $=$ un ángulo recto \byref{\constref},
                     \\ y los lados y ángulos restantes deben ser iguales. \byref{\constref}
                     \\ y $\therefore$ \polygon es un cuadrado. \byref{prop:I.XXIX}

Q. E. D. \byref{prop:I.XXXIV} \byref{def:I.XXX}
\end{center}

\qed

\starttheorem{Prop XLVII. Theor.}\label{prop:I.XLVII}

\defineNewPicture[1/2]{
pair A, B, C, D, E, F, G, H, I, J, K, L, M, d[];
%byPointLabelDefine(A, "α");
%byPointLabelDefine(B, "β");
%byPointLabelDefine(C, "γ");
%byPointLabelDefine(D, "δ");
%byPointLabelDefine(E, "ε");
%byPointLabelDefine(F, "ζ");
%byPointLabelDefine(G, "η");
%byPointLabelDefine(H, "θ");
%byPointLabelDefine(I, "ι");
%byPointLabelDefine(J, "κ");
%byPointLabelDefine(K, "λ");
%byPointLabelDefine(L, "μ");
%byPointLabelDefine(M, "ν");
A := (0, 0);
B := A shifted (-7/10u, -8/7u);
C = whatever[A, A shifted ((A-B) rotated 90)] = whatever[B, B shifted dir(0)];
d1 := (B-A) rotated -90;
D := A shifted d1;
E := B shifted d1;
d2 := (A-C) rotated -90;
F := C shifted d2;
G := A shifted d2;
d3 := (C-B) rotated -90;
H := B shifted d3;
I := C shifted d3;
J = whatever[A, A shifted dir(90)];
J = whatever[B, C];
K = whatever[A, A shifted dir(90)];
K = whatever[H, I];
L = whatever[B, F];
L = whatever[A, C];
M = whatever[A, I];
M = whatever[B, C];
draw byPolygon(A,B,E,D)(byblack);
draw byPolygon(L,A,G,F)(byred);
draw byPolygon(C,L,F)(byred);
draw byPolygon(J,M,I,K)(byblue);
draw byPolygon (M,C,I)(byblue);
draw byPolygon(B,J,K,H)(byyellow);
byAngleDefine(F, C, A, byyellow, SOLID_SECTOR);
byAngleDefine(B, C, I, byyellow, SOLID_SECTOR);
byAngleDefine(A, C, B, byblack, SOLID_SECTOR);
draw byNamedAngleResized();
draw byLineFull(A, K, byblack, 1, 0)(I, I, 1, 1, -1);
draw byLineFull(B, F, byblack, 0, 0)(G, G, 1, 1, -1);
draw byLineFull(A, I, byblack, 0, 0)(K, K, 1, 1, 1);
byLineDefine(C, F, byblue, DASHED_LINE, REGULAR_WIDTH);
byLineDefine(C, I, byred, DASHED_LINE, REGULAR_WIDTH);
draw byNamedLineSeq(0)(CF,CI);
byLineDefine(A, B, byyellow, SOLID_LINE, REGULAR_WIDTH);
byLineDefine(B, C, byred, SOLID_LINE, REGULAR_WIDTH);
byLineDefine(C, A, byblue, SOLID_LINE, REGULAR_WIDTH);
draw byNamedLineSeq(1)(AB,BC,CA);
byLineDefine.CAb(C, A, byblack, SOLID_LINE, REGULAR_WIDTH);
byLineStylize (M, M, 1, 0, -1) (CAb);
byLineDefine.AMb(A, M, byblack, SOLID_LINE, REGULAR_WIDTH);
byLineStylize (C, C, 0, 1, -1) (AMb);
byLineDefine.BCb(B, C, byblack, SOLID_LINE, REGULAR_WIDTH);
byLineStylize (L, L, 0, 1, -1) (BCb);
byLineDefine.BLb(L, B, byblack, SOLID_LINE, REGULAR_WIDTH);
byLineStylize (C, C, 1, 0, -1) (BLb);
draw byLabelsOnPolygon(B, E, D, A, G, F, C, I, K, H)(ALL_LABELS, -1);
draw byLabelsOnPolygon(A, J, C)(OMIT_FIRST_LABEL+OMIT_LAST_LABEL, 1);
}
\drawCurrentPictureInMargin
\problem{E}{n}{ un triángulo rectángulo \drawLine[bottom][triangleABC]{CA,BC,AB} el cuadrado de la hipotenusa \drawUnitLine{BC} es igual a la suma de los cuadrados de los lados, (\drawUnitLine{CA} y \drawUnitLine{AB}).}

\begin{center}
En \drawUnitLine{BC}, \drawUnitLine{CA} y \drawUnitLine{AB} dibuja cuadrados, \byref{prop:I.XLVI}

Dibuja \drawUnitLine{AK} ∥ \drawUnitLine{CI} \byref{prop:I.XXXI} tambien dibuja \drawUnitLine{BF} y \drawUnitLine{AI}.

\drawAngle{BCI} $=$ \drawAngle{FCA},

A cada uno agrega \drawAngle{ACB} $\therefore$ \drawAngle{BCI,ACB} $=$ \drawAngle{FCA,ACB},
 \\ \drawUnitLine{BC} $=$ \drawUnitLine{CI} y \drawUnitLine{CA} $=$ \drawUnitLine{CF};

$\therefore$ \drawFromCurrentPicture[middle][polygonAFC]{
draw byNamedPolygon(MCI);
draw byNamedAngle(ACB);
draw byNamedLine(CAb,AMb);
draw byLabelsOnPolygon(I,A,C)(OMIT_LABELS_AT_STRAIGHT_ANGLES, -1);
} $=$ \drawFromCurrentPicture[middle][polygonBLC]{
draw byNamedPolygon(CLF);
draw byNamedAngle(ACB);
draw byNamedLine(BCb,BLb);
draw byLabelsOnPolygon(B,F,C)(OMIT_LABELS_AT_STRAIGHT_ANGLES, -1);
}.

De nuevo, porque \drawUnitLine{AB} ∥ \drawUnitLine{CF}  \\ \polygon $=$ dos veces \polygon,
 \\ y \polygon $=$ dos veces \polygon;
 \\ $\therefore$  $=$ .

De la misma manera se puede mostrar
                     \\ que  $=$ ;
 \\ por lo tanto  $=$ .

Q. E. D.
\end{center}

\qed

\starttheorem{Prop XLVIII. Theor.}\label{prop:I.XLVIII}

\defineNewPicture{
pair A, B, C, D;
numeric d;
d := 7/4u;
A := (0, 0);
B := A shifted (0, 4u);
C := A shifted (d, 0);
D := A shifted (-d, 0);
byAngleDefine(B, A, C, byred, SOLID_SECTOR);
byAngleDefine(D, A, B, byyellow, SOLID_SECTOR);
draw byNamedAngleResized();
draw byLine(A, B, byblue, SOLID_LINE, REGULAR_WIDTH);
byLineDefine(A, C, byblack, SOLID_LINE, REGULAR_WIDTH);
byLineDefine(A, D, byblack, DASHED_LINE, REGULAR_WIDTH);
byLineDefine(B, C, byred, SOLID_LINE, REGULAR_WIDTH);
byLineDefine(B, D, byred, DASHED_LINE, REGULAR_WIDTH);
draw byNamedLineSeq(0)(AC,AD,BD,BC);
draw byLabelsOnPolygon(D, B, C, A)(ALL_LABELS, 0);
}
\drawCurrentPictureInMargin
\problem{S}{i}{ el cuadrado de un lado (\drawUnitLine{BC}) de un triángulo es igual a los cuadrados de los otros dos lados (\drawUnitLine{AB} y \drawUnitLine{AC}), el ángulo (\drawAngle{BAC}) subtendido por ese lado es un ángulo recto.}

\begin{center}
Dibuja \drawUnitLine{AD} ⊥ \drawUnitLine{AB} y $=$ \drawUnitLine{AC} (prs. 11. 3.)
 \\ y dibuja \drawUnitLine{BD} también.

Ya que \drawUnitLine{AD} $=$ \drawUnitLine{AC} (const.)
                     \\ \drawUnitLine{AD}2 $=$ \drawUnitLine{AC}2;
 \\ $\therefore$ \drawUnitLine{AD}2 + \drawUnitLine{AB}2 $=$ \drawUnitLine{AC}2 + \drawUnitLine{AB}2,
 \\ pero \drawUnitLine{AD}2 + \drawUnitLine{AB}2 $=$ \drawUnitLine{BD}2 \byref{prop:I.XI,prop:I.III},
                     \\ y \drawUnitLine{AC}2 + \drawUnitLine{AB}2 $=$ \drawUnitLine{BC}2 (hip.)
                     \\ $\therefore$ \drawUnitLine{BD}2 $=$ \drawUnitLine{BC}2,
 \\ $\therefore$ \drawUnitLine{BD} $=$ \drawUnitLine{BC};
 \\ y $\therefore$ \drawAngle{DAB} $=$ \drawAngle{BAC} \byref{\constref},
                     \\ por consiguiente \drawAngle{BAC} es un ángulo recto.

Q. E. D.

Un proyecto de Nicholas Rougeux     Licencia

¿Problemas de visualización? ¿Errores ortográficos? \byref{prop:I.XLVII} \byref{\hypref} \byref{prop:I.VIII}
\end{center}

\qed

\part{Book II}

\startdefinition{Definition I}\label{def:II.I}

\defineNewPicture{
pair A, B, C, D;
numeric w, h;
w := 7/2u;
h := 3u;
A := (0, 0);
B := (w, 0);
C := (0, h);
D := (w, h);
draw byPolygon(A,B,D,C)(byblue);
byLineDefine(A, B, byblack, SOLID_LINE, REGULAR_WIDTH);
byLineDefine(A, C, byred, SOLID_LINE, REGULAR_WIDTH);
draw byNamedLineSeq(0)(AB,AC);
draw byLabelsOnPolygon(A, C, D, B)(ALL_LABELS, 0);
}
\drawCurrentPictureInMargin
\begin{center}
Proposición I. Problema.

El rectángulo contenido por dos líneas rectas, una de las cuales está dividida en cualquier número de partes,
                
\drawUnitLine{AB} · \drawUnitLine{AC} $=$



\drawUnitLine{AB} · \drawUnitLine{AC}
+\drawUnitLine{AB} · \drawUnitLine{AC}
+\drawUnitLine{AB} · \drawUnitLine{AC}



                es igual a la suma de los rectángulos contenidos por la línea no dividida y las varias partes de la línea dividida.

Dibuja \drawUnitLine{AB} ⊥ \drawUnitLine{AC} y $=$ \drawUnitLine{AB} (prs. 2. y 3. L. 1.); completa los paralelogramos, es decir,

Dibuja
                    


\drawUnitLine{AC} ∥ \drawUnitLine{AC}
\drawUnitLine{AB} ∥ \drawUnitLine{AC}



                    

\drawUnitLine{AB} $=$  +  +   \\  $=$  ·   \\  $=$  · ,  $=$  · ,
 \\  $=$  · 

$\therefore$  ·  $=$  ·  +  ·  +  · .

Q. E. D.
\end{center}

\startdefinition{Definition II}\label{def:II.II}

\defineNewPicture{
pair A, B, C, D, E, F, G, H, I, d[];
d1 := (3u, 0);
d2 := (1/2u, 5/2u);
A := (0, 0);
B := A shifted d1;
C := A shifted d2;
D := A shifted d1 shifted d2;
E := 2/3[A, B];
F := 2/3[C, D];
G := 2/3[A, C];
H := 2/3[B, D];
I = whatever[E, F] = whatever[G, H];
draw byPolygon(A,E,I,G)(byblue);
draw byPolygon(E,B,H,I)(byyellow);
draw byPolygon(G,I,F,C)(byyellow);
draw byPolygon(I,H,D,F)(byred);
draw byLabelsOnPolygon(C, F, D, H, B, E, A, G)(ALL_LABELS,0);
draw byLabelsOnPolygon(G, I, F)(OMIT_FIRST_LABEL+OMIT_LAST_LABEL,0);
}
\drawCurrentPictureInMargin
\begin{center}
Proposición II. Teorema.

Si una línea recta se divide en dos partes , el cuadrado de la línea completa es igual a la suma de los rectángulos contenidos por la línea completa y cada una de sus partes.
                
2 $=$



 · 
+  · 

Traza  (L. 1. pr. 46.)
                     \\ Dibuja  paralela a  (L. 1. pr. 31.)
                     \\  $=$ 2
 \\  $=$  ·  $=$  ·   \\  $=$  ·  $=$  ·   \\  $=$  +   \\ $\therefore$ 2 $=$  ·  +  · .

Q. E. D.
\end{center}

\starttheorem{Prop. I. theor.}\label{prop:II.I}

\defineNewPicture[1/4]{
pair B, C, D, E, G, H, K, L, M, N;
numeric w, h;
w := 7/2u;
h := 3u;
G := (0, 0);
H := (w, 0);
B := (0, h);
C := (w, h);
K := 2/5[G, H];
D := 2/5[B, C];
L := 3/4[G, H];
E := 3/4[B, C];
M := G shifted (0, -2/3u);
N := M shifted (h, 0);
draw byPolygon(G,K,D,B)(byyellow);
draw byPolygon(K,L,E,D)(byblue);
draw byPolygon(L,H,C,E)(byred);
draw byLine(K, D, byblack, DASHED_LINE, REGULAR_WIDTH);
draw byLine(L, E, byblack, DASHED_LINE, REGULAR_WIDTH);
byLineDefine(G, B, byblack, SOLID_LINE, REGULAR_WIDTH);
draw byLine.A(M, N, byblack, SOLID_LINE, REGULAR_WIDTH); % improvement: the line wasn't there, but it should've been
byLineDefine(H, C, byblack, DASHED_LINE, REGULAR_WIDTH);
byLineDefine(G, K, byblue, SOLID_LINE, REGULAR_WIDTH);
byLineDefine(K, L, byred, SOLID_LINE, REGULAR_WIDTH);
byLineDefine(L, H, byyellow, SOLID_LINE, REGULAR_WIDTH);
byLineDefine(B, D, byblue, DASHED_LINE, REGULAR_WIDTH);
byLineDefine(D, E, byred, DASHED_LINE, REGULAR_WIDTH);
byLineDefine(E, C, byyellow, DASHED_LINE, REGULAR_WIDTH);
draw byNamedLineSeq(0)(GK,KL,LH,HC,EC,DE,BD,GB);
draw byLabelsOnPolygon(B, D, E, C, H, L, K, G)(ALL_LABELS, 0);
draw byLabelLine(0)(A);
}
\drawCurrentPictureInMargin
\problem{S}{i}{ una línea recta se divide en dos partes \polygon, el rectángulo contenido por la línea completa y cualquiera de sus partes es igual al cuadrado de esa parte, junto con el rectángulo bajo las partes.}

\begin{center}
\polygon · \polygon $=$ 2 +  · , o,  ·  $=$ 2 +  · .

Traza  \byref{prop:I.XI,prop:I.III}

Completa  \byref{prop:I.II}

Entonces  $=$  + , pero
                     \\  $=$  ·  y
                     \\  $=$ 2,  $=$  · ,
 \\ $\therefore$  ·  $=$ 2 +  · :

De manera similar, se puede demostrar fácilmente que  ·  $=$ 2 +  · .

Q. E. D. \byref{prop:I.XI} \byref{prop:I.XXXI}
\end{center}

\qed

\starttheorem{Prop. II. theor.}\label{prop:II.II}

\defineNewPicture{
pair A, B, C, D, E, F;
numeric w;
w := 7/2u;
A := (0, w);
B := (w, w);
C := 2/3[A, B];
D := (0, 0);
E := (w, 0);
F := 2/3[D, E];
draw byPolygon(A,C,F,D)(byred);
draw byPolygon(C,B,E,F)(byyellow);
draw byLine(C, F, byblack, SOLID_LINE, REGULAR_WIDTH);
byLineDefine(A, D, byblack, DASHED_LINE, REGULAR_WIDTH);
byLineDefine(B, E, byblack, DASHED_LINE, REGULAR_WIDTH);
byLineDefine(A, C, byblue, SOLID_LINE, REGULAR_WIDTH);
byLineDefine(C, B, byred, SOLID_LINE, REGULAR_WIDTH);
draw byNamedLineSeq(0)(AD,AC,CB,BE);
draw byLabelsOnPolygon(A, C, B, E, F, D)(ALL_LABELS, 0);
}
\drawCurrentPictureInMargin
\problem{S}{i}{ una línea recta se divide en dos partes \polygon, el cuadrado de la línea completa es igual a los cuadrados de las partes, junto con el doble del rectángulo contenido por las partes.}

\begin{center}
\polygon2 $=$ 2 + 2 + dos veces  · .

Traza  \byref{prop:I.XLVI}
                     \\ dibuja  \byref{prop:I.XXXI},
                     \\ y
                    


 ∥ 
 ∥ 



                    

 $=$  ,

 $=$  

$\therefore$  $=$ 

$\therefore$ por (prs. 6, 29, 34. L. 1.)  es un cuadrado $=$ 2.

Por las mismas razones  es un cuadrado $=$ 2,
                     \\  $=$  $=$  ·  
                     \\ pero  $=$  +  +  + ,
 \\ $\therefore$ 2 $=$ 2 + 2 + dos veces  · .

Q. E. D.
\end{center}

\qed

\starttheorem{Prop. III. theor.}\label{prop:II.III}

\defineNewPicture{
pair A, B, C, D, E, F;
numeric w, h;
w := -4u;
h := 11/4u;
A := (0, h);
B := (w, h);
C := (w+h, h);
D := (w+h, 0);
E := (w, 0);
F := (0, 0);
draw byPolygon(A,C,D,F)(byyellow);
draw byPolygon(C,B,E,D)(byred);
byLineDefine(D, F, byred, SOLID_LINE, REGULAR_WIDTH);
byLineDefine(B, C, byblue, SOLID_LINE, REGULAR_WIDTH);
byLineDefine(C, D, byblue, SOLID_LINE, REGULAR_WIDTH);
byLineDefine(D, E, byblue, SOLID_LINE, REGULAR_WIDTH);
byLineDefine(E, B, byblue, SOLID_LINE, REGULAR_WIDTH);
draw byNamedLineSeq(0)(CD,noLine,DF,DE,EB,BC);
draw byLabelsOnPolygon(B, C, A, F, D, E)(ALL_LABELS, 0);
}
\drawCurrentPictureInMargin
\problem{S}{i}{ una línea recta se divide \polygon en dos partes iguales y también \polygon en dos partes desiguales, el rectángulo contenido por las partes desiguales, junto con el cuadrado de la línea entre los puntos de sección, es igual al cuadrado de la mitad de esa línea.}

\begin{center}
\polygon · \polygon + \polygon2 $=$ 2 $=$ 2,

traza  \byref{prop:I.XLVI}, dibuja  y
                    



 ∥ 
 ∥ 
 ∥ 



                        \byref{prop:I.XXXI}

 $=$  (p. 36, L. 1.)

 $=$  (p. 43, L. 1.)

$\therefore$   $=$  $=$  ·   \\ pero  $=$ 2 (cor. pr. 4. L. 2.)
                     \\ y  $=$ 2 (const.)

$\therefore$   $=$   \\ $\therefore$  ·  + 2 $=$
 \\ 2 $=$ 2.

Q. E. D.
\end{center}

\qed

\starttheorem{Prop. IV. theor.}\label{prop:II.IV}

\defineNewPicture[1/4]{
pair A, B, C, D, E, F, G, H, K;
numeric w;
w := 9/2u;
A := (0, w);
B := (w, w);
C :=2/3[A, B];
D := (0, 0);
E := (w, 0);
F := 2/3[D, E];
H := 2/3[D, A];
K := 2/3[E, B];
G = whatever[H, K] = whatever[F, C];
draw byPolygon(A,C,G,H)(byyellow);
draw byPolygon(G,K,E,F)(byyellow);
draw byPolygon(D,H,G)(byblue);
draw byPolygon(G,B,C)(byred);
byAngleDefine(F, D, G, byyellow, SOLID_SECTOR);
byAngleDefine(F, G, D, byred, SOLID_SECTOR);
byAngleDefine(K, G, B, byblack, SOLID_SECTOR);
byAngleDefine(G, B, K, byblue, SOLID_SECTOR);
draw byNamedAngleResized();
draw byLine(H, G, byred, DASHED_LINE, REGULAR_WIDTH);
draw byLine(G, K, byred, SOLID_LINE, REGULAR_WIDTH);
draw byLine(C, G, byblue, DASHED_LINE, REGULAR_WIDTH);
draw byLine(F, G, byblue, SOLID_LINE, REGULAR_WIDTH);
byLineDefine(E, K, byblue, SOLID_LINE, REGULAR_WIDTH);
byLineDefine(F, E, byred, SOLID_LINE, REGULAR_WIDTH);
byLineDefine(D, F, byblue, SOLID_LINE, REGULAR_WIDTH);
byLineDefine(K, B, byred, SOLID_LINE, REGULAR_WIDTH);
byLineDefine(G, D, byblack, SOLID_LINE, REGULAR_WIDTH);
byLineDefine(B, G, byblack, DASHED_LINE, REGULAR_WIDTH);
draw byNamedLineSeq(1)(BG,GD,DF,FE,EK,KB);
byLineDefine(A, D, byblack, SOLID_LINE, REGULAR_WIDTH);
byLineStylize(B, E, 0, 0, -1)(AD);
byLineDefine(B, A, byblack, SOLID_LINE, REGULAR_WIDTH);
byLineStylize(E, D, 0, 0, -1)(BA);
draw byLabelsOnPolygon(A, C, B, K, E, F, D, H)(ALL_LABELS, 0);
draw byLabelsOnPolygon(H, G, C)(OMIT_FIRST_LABEL+OMIT_LAST_LABEL, 0);
}
\drawCurrentPictureInMargin
\problem{S}{i}{ una línea recta se biseca \drawLine[bottom][squareABED]{AD,BA,KB,EK,FE,DF} y se prolonga en cualquier punto \drawAngle{GBK}, el rectángulo contenido por toda la línea aumenta así, y la parte prolongada, junto con el cuadrado de la mitad de la línea, es igual al cuadrado de la línea formada por la mitad, y la parte prolongada.}

\begin{center}
\drawAngle{FDG} · \drawAngle{GBK} + \drawAngle{FGD}2 $=$ \drawAngle{FDG}2.

Traza \drawAngle{FGD} \byref{prop:I.XLVI}, dibuja \drawFromCurrentPicture[bottom][squareFGHD]{
draw byNamedPolygon(DHG);
draw byNamedLineFull(G, G, 1, 0,  0, -1)(DF);
draw byNamedLineFull(D, D, 0, 1,  0, -1)(FG);
draw byLabelsOnPolygon(D, H, G, F)(ALL_LABELS, 0);
}  \\ 

                        y
                        


\drawFromCurrentPicture[bottom][squareKBCG]{
draw byNamedPolygon(GBC);
draw byNamedLineFull(B, B, 1, 0,  0, -1)(GK);
draw byNamedLineFull(G, G, 0, 1,  0, -1)(KB);
draw byLabelsOnPolygon(G, C, B, K)(ALL_LABELS, 0);
} ∥ \drawFromCurrentPicture[bottom][squareABEDf]{
draw byNamedPolygon(GBC,DHG,ACGH,GKEF);
draw byNamedLineFull(G, G, 1, 0,  0, -1)(DF);
draw byNamedLineFull(D, D, 0, 1,  0, -1)(FG);
draw byNamedLineFull(B, B, 1, 0,  0, -1)(GK);
draw byNamedLineFull(G, G, 0, 1,  0, -1)(KB);
draw byLabelsOnPolygon(A, B, E, D)(ALL_LABELS, 0);
}
\square ∥ \square
 ∥ 



                        \byref{post:I.I}

 $=$  $=$  (prs. 36, 43, L. 1.)

$\therefore$  $=$  $=$  · ;
 \\  $=$ 2 (cor. 4, L. 2.)

$\therefore$  $=$ 2 $=$  (const. ax. 2.)

$\therefore$  ·  + 2 $=$ 2.

Q. E. D. \byref{prop:I.XXXI} \byref{prop:I.V} \byref{prop:I.XXIX} \byref{prop:I.VI,prop:I.XXIX,prop:I.XXXIV} \byref{prop:I.XLIII}
\end{center}

\qed

\starttheorem{Prop. V. theor.}\label{prop:II.V}

\defineNewPicture[1/4]{
pair A, B, C, D, E, F, G, H, K, L, M;
numeric h;
h := 6u;
A := (0, -h);
B := (0, 0);
C := 1/2[A, B];
D := 2/5[B, C];
E := (1/2h, -1/2h);
F := (1/2h, 0);
G := (xpart(F), ypart(D));
H = whatever[D, G] = whatever[B, E];
K := (xpart(H), ypart(A));
L := (xpart(H), ypart(C));
M := (xpart(H), ypart(B));
draw byPolygon(B,D,H,M)(byblue);
draw byPolygon(D,C,L,H)(byyellow);
draw byPolygon(C,L,K,A)(byblack);
draw byPolygon(M,H,G,F)(byyellow);
draw byPolygon(H,L,E,G)(byred);
draw byLine(H, G, byblack, DASHED_LINE, REGULAR_WIDTH);
draw byLine(D, H, byred, SOLID_LINE, REGULAR_WIDTH);
byLineDefine(L, M, byblack, DASHED_LINE, REGULAR_WIDTH);
byLineDefine(C, L, byred, SOLID_LINE, REGULAR_WIDTH);
byLineDefine(B, D, byred, SOLID_LINE, REGULAR_WIDTH);
byLineDefine(D, C, byblue, SOLID_LINE, REGULAR_WIDTH);
byLineDefine(C, A, byyellow, SOLID_LINE, REGULAR_WIDTH);
byLineDefine(E, B, byblack, SOLID_LINE, REGULAR_WIDTH);
byLineDefine(A, K, byred, DASHED_LINE, REGULAR_WIDTH);
byLineDefine(K, L, byyellow, SOLID_LINE, REGULAR_WIDTH);
byLineDefine(L, E, byblue, DASHED_LINE, REGULAR_WIDTH);
draw byNamedLineSeq(1)(BD,DC,CA,AK,KL,LM,CL,LE,EB);
draw byLabelsOnPolygon(A, C, D, B, M, F, G, E, L, K)(ALL_LABELS, 0);
draw byLabelsOnPolygon(M, H, G)(OMIT_FIRST_LABEL+OMIT_LAST_LABEL, 0);
}
\drawCurrentPictureInMargin
\problem{S}{i}{ una línea recta se divide en dos partes \square, los cuadrados de la línea completa y una de las partes son iguales al doble del rectángulo contenido por la línea completa y esa parte, junto con el cuadrado de las otras partes.}

\begin{center}
\square2 + 2 $=$ 2 ·  + 2

Traza , \byref{prop:I.XLVI}.

Dibuja  \byref{prop:I.XXXI},
                    
                        y
                        


 ∥ 
 ∥ 



                        \byref{prop:I.XXXVI}.

 $=$  \byref{prop:I.XLIII},
                     \\ agrega  $=$ 2 a ambas (cor. 4, L. 2.)

 $=$  $=$ ·   \\  $=$ 2 (cor. 4, L. 2.)

$\therefore$  +  +  $=$ 2 ·  + 2 $=$  + ;

$\therefore$ 2 + 2 $=$ 2 ·  + 2.

Q. E. D. \byref{ax:I.II} \byref{prop:II.IV} \byref{\constref} \byref{ax:I.II}
\end{center}

\qed

\starttheorem{Prop. VI. theor.}\label{prop:II.VI}

\defineNewPicture{
pair A, B, C, D, E, F, G, H, K, L, M;
numeric h, s;
h := 5u;
s := 2/5h;
A := (0, h);
B := (0, 0);
C := A shifted (0, -s);
D := A shifted (0, -2s);
E := (-h+s, h-s);
F := (-h+s, 0);
G := (xpart(F), ypart(D));
H = whatever[D, G] = whatever[B, E];
K := (xpart(H), ypart(A));
L := (xpart(H), ypart(C));
M := (xpart(H), ypart(B));
draw byPolygon(B,D,H,M)(byblue);
draw byPolygon(D,C,L,H)(byyellow);
draw byPolygon(C,L,K,A)(byblack);
draw byPolygon(M,H,G,F)(byyellow);
draw byPolygon(H,L,E,G)(byred);
draw byLine(H, G, byblue, DASHED_LINE, REGULAR_WIDTH);
draw byLine(D, H, byred, SOLID_LINE, REGULAR_WIDTH);
byLineDefine(L, M, byblack, DASHED_LINE, REGULAR_WIDTH);
byLineDefine(C, L, byred, SOLID_LINE, REGULAR_WIDTH);
byLineDefine(B, D, byred, SOLID_LINE, REGULAR_WIDTH);
byLineDefine(D, C, byblue, SOLID_LINE, REGULAR_WIDTH);
byLineDefine(C, A, byyellow, SOLID_LINE, REGULAR_WIDTH);
byLineDefine(E, B, byblack, SOLID_LINE, REGULAR_WIDTH);
byLineDefine(A, K, byred, DASHED_LINE, REGULAR_WIDTH);
byLineDefine(K, L, byyellow, SOLID_LINE, REGULAR_WIDTH);
byLineDefine(L, E, byblack, DASHED_LINE, REGULAR_WIDTH);
draw byNamedLineSeq(1)(BD,DC,CA,AK,KL,LM,CL,LE,EB);
draw byLabelsOnPolygon(F, G, E, L, K, A, C, D, B, M)(ALL_LABELS, 0);
draw byLabelsOnPolygon(L, H, D)(OMIT_FIRST_LABEL+OMIT_LAST_LABEL, 1);
}
\drawCurrentPictureInMargin
\problem{S}{i}{ una línea recta se divide en dos partes \square, el cuadrado de la suma de la línea completa y cualquiera de sus partes es igual a cuatro veces el rectángulo contenido por la línea completa, y esa parte junto con el cuadrado de la otra parte.}

\begin{center}
2 $=$ 4 ·  ·  + 2,

Prolonga  y haz  $=$ 

Construye  \byref{prop:I.XLVI};
                     \\ dibuja ,
                    










∥  








∥  



                        \byref{prop:I.XXXI}
                    
2 $=$ 2 + 2 + 2 ·  ·  \byref{prop:I.XXXVI,prop:I.XLIII}
                     \\ but 2 + 2 $=$ 2 ·  ·  + 2 \byref{prop:II.IV}
                     \\ $\therefore$ 2 $=$ 4 ·  ·  + 2.

Q. E. D. \byref{\constref,ax:I.II}
\end{center}

\qed

\starttheorem{Prop. VII. theor.}\label{prop:II.VII}

\defineNewPicture{
pair A, B, C, D, E, F, G, H, N;
numeric w;
w := 7/2u;
A := (0, w);
B := (w, w);
C := 3/5[A, B];
D := (0, 0);
E := (w, 0);
N := 3/5[D, E];
F := 3/5[E, B];
G = whatever[D, B] = whatever[N, C];
H := whatever[A, D] = whatever[F, G];
draw byPolygon(D,N,G,H)(byred);
draw byPolygon(N,E,F,G)(byblack);
draw byPolygon(H,G,C,A)(byyellow);
draw byPolygon(G,F,B,C)(byblue);
draw byLine(G, N, byblue, SOLID_LINE, REGULAR_WIDTH);
draw byLine(G, F, byred, SOLID_LINE, REGULAR_WIDTH);
draw byLine(G, H, byblack, DASHED_LINE, REGULAR_WIDTH);
draw byLine(G, C, byblack, DASHED_LINE, REGULAR_WIDTH);
byLineDefine(B, D, byblack, SOLID_LINE, REGULAR_WIDTH);
byLineDefine(D, N, byblue, SOLID_LINE, REGULAR_WIDTH);
byLineDefine(N, E, byred, SOLID_LINE, REGULAR_WIDTH);
byLineDefine(E, B, byyellow, SOLID_LINE, REGULAR_WIDTH);
draw byNamedLineSeq(1)(BD,DN,NE,EB);
draw byLabelsOnPolygon(D, H, A, C, B, F, E, N)(ALL_LABELS, 0);
draw byLabelsOnPolygon(H, G, C)(OMIT_FIRST_LABEL+OMIT_LAST_LABEL, 0);
}
\drawCurrentPictureInMargin
\problem{S}{i}{ una línea recta se divide en dos partes iguales \square, y también en dos partes desiguales , los cuadrados de las partes desiguales son juntos el doble de los cuadrados de la mitad de la línea y de la parte entre los puntos de sección.}

\begin{center}
2 + 2 $=$ 22 + 22.

Haz  ⊥ y $=$  o ,
 \\ Dibuja  y ,
 \\ ∥ ,  ∥ , y dibuja .

 $=$  \byref{prop:I.XLVI} $=$ mitad de un ángulo recto. (cor. pr. 32, L. 1.)
                     \\  $=$  \byref{post:I.I} $=$ mitad de un ángulo recto. (cor. pr. 32, L. 1.)
                     \\ $\therefore$ $=$ un ángulo recto.

 $=$  $=$  $=$  (prs. 5, 29, L. 1.).
                     \\ por lo tanto  $=$ ,  $=$  $=$  (prs. 6, 34, L. 1.)
                     \\ 

2 $=$



2 + 2, o + 2

$=$



2 $=$ 22 \byref{prop:I.XLIII}
2 $=$ 22






$\therefore$ 2 + 2 $=$ 22 + 22.

Q. E. D. \byref{prop:II.IV} \byref{prop:II.IV}
\end{center}

\qed

\starttheorem{Prop. VIII. theor.}\label{prop:II.VIII}

\defineNewPicture{
pair A, B, C, D, E, F, G, H, K, L, M, N, O, P, Q, R;
numeric w, d;
w := 7/2u;
d := u;
A := (0, w + d);
B := (w, w + d);
C := (w - d, w + d);
D := (w + d, w + d);
E := (0, 0);
F := (w + d, 0);
G := (w - d, w);
H := (w - d, 0);
K := (w, w);
L := (w, 0);
M := (0, w);
N := (w + d, w);
O := (0, w - d);
P := (w + d, w - d);
Q := (w - d, w - d);
R := (w, w - d);
draw byLine(C, Q, byblue, DASHED_LINE, REGULAR_WIDTH);
draw byLine(B, R, byblack, DASHED_LINE, REGULAR_WIDTH);
draw byLine(Q, H, byblue, SOLID_LINE, REGULAR_WIDTH);
draw byLine(R, L, byblue, SOLID_LINE, REGULAR_WIDTH);
draw byLine(M, G, byblack, DASHED_LINE, REGULAR_WIDTH);
draw byLine(O, Q, byred, DASHED_LINE, REGULAR_WIDTH);
draw byLine(G, N, byred, SOLID_LINE, REGULAR_WIDTH);
draw byLine(Q, P, byred, SOLID_LINE, REGULAR_WIDTH);
draw byLine(D, E, byblack, SOLID_LINE, REGULAR_WIDTH);
byLineDefine(E, H, byblue, SOLID_LINE, REGULAR_WIDTH);
byLineDefine(H, L, byred, SOLID_LINE, REGULAR_WIDTH);
byLineDefine(L, F, byyellow, SOLID_LINE, REGULAR_WIDTH);
byLineDefine(F, P, byblue, SOLID_LINE, REGULAR_WIDTH);
byLineDefine(P, N, byyellow, SOLID_LINE, REGULAR_WIDTH);
byLineDefine(N, D, byred, SOLID_LINE, REGULAR_WIDTH);
byLineDefine(A, D, byblack, SOLID_LINE, REGULAR_WIDTH);
byLineDefine(E, A, byblack, SOLID_LINE, REGULAR_WIDTH);
draw byNamedLineSeq(0)(EH,HL,LF,FP,PN,ND,AD,EA);
byLineDefine(D, F, byblack, SOLID_LINE, REGULAR_WIDTH);
byLineDefine(B, L, byblack, SOLID_LINE, THIN_WIDTH);
byLineDefine(C, H, byblack, SOLID_LINE, THIN_WIDTH);
byLineDefine(M, N, byblack, SOLID_LINE, THIN_WIDTH);
byLineDefine(O, P, byblack, SOLID_LINE, THIN_WIDTH);
draw byLabelsOnPolygon(A, C, B, D, N, P, F, L, H, E, O, M)(ALL_LABELS, 0);
}
\drawCurrentPictureInMargin
\problem{X}{X}{Si una línea recta \drawFromCurrentPicture[bottom]{
draw byNamedLine(BL,CH,MN,OP);
draw byNamedLineSeq(0)(LF,HL,EH,EA,AD,DF);
} se biseca y prolonga en cualquier punto , los cuadrados de toda la línea prolongada y de la parte prolongada son juntos el doble de los cuadrados de la media línea y de la línea formada por la mitad y la parte prolongada.}

\begin{center}
2 + 2 $=$ 22 + 22.

Haz  ⊥ y $=$ a  o ,
 \\ dibuja  y ,
 \\ 

                        y
                        


∥ 
∥ 



                        \byref{prop:I.XLVI};
                    
                    dibuja también .

 $=$  \byref{prop:I.XXXI} $=$ mitad de un ángulo recto (cor. pr. 32, L. 1.)
                     \\  $=$  \byref{prop:II.IV} $=$ mitad de un ángulo recto (cor. pr. 32, L. 1.)
                     \\ $\therefore$  $=$ un ángulo recto.
                     \\  $=$  $=$  $=$  $=$  $=$
 \\ la mitad de un ángulo recto (prs 5, 32, 29, 34, L. 1.),
                     \\ y  $=$ ,  $=$  $=$ , (prs. 6, 34, L. 1.). Consiguiente por \byref{prop:II.VII}
                    
2 $=$



2 + 2 or 2




+ 2 $=$ 22
+ 2 $=$ 22






$\therefore$ 2 + 2 $=$ 22 + 22.

Q. E. D.
\end{center}

\qed

\starttheorem{Prop. IX. theor.}\label{prop:II.IX}

\defineNewPicture{
pair A, B, C, D, E, F, G, H;
numeric w;
w := 5u;
A := (-1/2w, 0);
B := (1/2w, 0);
C := (0, 0);
D := (1/5w, 0);
E := (0, 1/2w);
F = whatever[E, B] = (xpart(D), whatever);
G = whatever[E, C] = (whatever, ypart(F));
H := 1/2[E, F];
byAngleDefine(E, A, B, byyellow, SOLID_SECTOR);
byAngleDefine(A, B, E, byblue, SOLID_SECTOR);
byAngleDefine(A, E, C, byyellow, SOLID_SECTOR);
byAngleDefine(C, E, B, byred, SOLID_SECTOR);
byAngleDefine(E, F, G, byred, SOLID_SECTOR);
byAngleDefine(D, F, B, byblack, SOLID_SECTOR);
draw byNamedAngleResized();
draw byLine(A, F, byblack, SOLID_LINE, REGULAR_WIDTH);
draw byLine(F, D, byred, DASHED_LINE, REGULAR_WIDTH);
draw byLine(G, F, byyellow, 0, -1);
byLineDefine(C, G, byblue, SOLID_LINE, REGULAR_WIDTH);
byLineDefine(G, E, byblue, DASHED_LINE, REGULAR_WIDTH);
draw byNamedLineSeq(0)(CG, GE);
byLineDefine(A, C, byblue, SOLID_LINE, REGULAR_WIDTH);
byLineDefine(C, D, byyellow, SOLID_LINE, REGULAR_WIDTH);
byLineDefine(D, B, byred, SOLID_LINE, REGULAR_WIDTH);
byLineDefine(F, B, byyellow, DASHED_LINE, REGULAR_WIDTH);
byLineDefine(H, F, byblack, SOLID_LINE, REGULAR_WIDTH);
byLineDefine(E, H, byblack, 2, REGULAR_WIDTH);
byLineDefine(A, E, byblack, DASHED_LINE, REGULAR_WIDTH);
draw byNamedLineSeq(0)(AC,CD,DB,FB,HF,EH,AE);
draw byLabelsOnPolygon(E, F, B, D, C, A)(ALL_LABELS, 0);
draw byLabelsOnPolygon(C, G, E)(OMIT_FIRST_LABEL+OMIT_LAST_LABEL, 0);
}
\drawCurrentPictureInMargin
\problem{P}{ara}{ dividir una línea recta \drawAngle{A} dada de tal manera, que el rectángulo contenido por la línea completa y una de sus partes puede ser igual al cuadrado de la otra.}

\begin{center}
\drawAngle{AEC} · \drawAngle{B} $=$ \drawAngle{DFB}2.

Traza \drawAngle{AEC,CEB} \byref{prop:I.V},
                     \\ haz \drawAngle{B} $=$ \drawAngle{CEB} \byref{prop:I.XXXII},
                     \\ dibuja \drawAngle{EFG},
                     \\ toma \drawAngle{DFB} $=$  \byref{prop:I.V},
                     \\ en  describe  \byref{prop:I.XXXII},

Prolonga  \byref{prop:I.V, prop:I.XXIX}.

Entonces, \byref{prop:I.VI,prop:I.XXXIV}  ·  + 2 $=$ 2 $=$ 2 $=$ 2 + 2 $\therefore$  ·  $=$ 2, o,
                     \\  $=$  $\therefore$  $=$  $\therefore$
 \\  ·  $=$ 2.

Q. E. D. \byref{prop:I.XLVII}
\end{center}

\qed

\starttheorem{Prop. X. theor.}\label{prop:II.X}

\defineNewPicture{
pair A, B, C, D, E, F, G;
numeric w;
w := 7/2u;
A := (0, 0);
B := (w, 0);
C := 1/2[A, B];
D := 4/3[A, B];
E := (w/2, w/2);
F := (xpart(D), ypart(E));
G = whatever[E, B] = whatever[F, D];
byAngleDefine(E, A, C, byblack, SOLID_SECTOR);
byAngleDefine(C, E, A, byyellow, SOLID_SECTOR);
byAngleDefine(B, E, C, byyellow, SOLID_SECTOR);
byAngleDefine(F, E, B, byblue, SOLID_SECTOR);
byAngleDefine(C, B, E, byred, SOLID_SECTOR);
byAngleDefine(D, B, G, byred, SOLID_SECTOR);
byAngleDefine(B, G, D, byblue, SOLID_SECTOR);
draw byNamedAngleResized();
draw byLine(C, E, byred, 0, -1);
draw byLine(A, C, byred, SOLID_LINE, REGULAR_WIDTH);
draw byLine(C, B, byyellow, SOLID_LINE, REGULAR_WIDTH);
draw byLine(B, D, byblue, SOLID_LINE, REGULAR_WIDTH);
byLineDefine(E, B, byblack, SOLID_LINE, REGULAR_WIDTH);
byLineDefine(B, G, byblack, DASHED_LINE, REGULAR_WIDTH);
draw byNamedLineSeq(0)(EB,BG);
byLineDefine(E, F, byyellow, DASHED_LINE, REGULAR_WIDTH);
byLineDefine(F, D, byred, SOLID_LINE, REGULAR_WIDTH);
byLineDefine(D, G, byred, DASHED_LINE, REGULAR_WIDTH);
byLineDefine(A, G, byblack, SOLID_LINE, REGULAR_WIDTH);
byLineDefine(A, E, byblue, DASHED_LINE, REGULAR_WIDTH);
draw byNamedLineSeq(0)(EF,FD,DG,AG,AE);
draw byLabelsOnPolygon(A, E, F, D, G)(ALL_LABELS, 0);
draw byLabelsOnPolygon(G, B, C, A)(OMIT_FIRST_LABEL+OMIT_LAST_LABEL, 0);
}
\drawCurrentPictureInMargin
\problem{E}{n}{ cualquier triángulo obtusángulo, el cuadrado del lado que subtiende el ángulo obtuso excede la suma de los cuadrados de los lados que contienen el ángulo obtuso, el doble del rectángulo contenido por cualquiera de estos lados y las partes prolongadas del mismo desde el ángulo obtuso a la perpendicular deja caer sobre él desde el ángulo agudo opuesto.}

\begin{center}
\drawAngle{A}2 > \drawAngle{CEA}2 + \drawAngle{CBE}2 by 2\drawAngle{BEC} · \drawAngle{CEA,BEC}.

Por pr. 4, L. 2.
 \\ \drawAngle{DBG}2 $=$ \drawAngle{CBE}2 + \drawAngle{BEC}2 + 2\drawAngle{FEB} · \drawAngle{G}:
 \\ agrega 2 a ambas
                     \\ 2 + 2 $=$ 2 \byref{prop:I.XXXI} $=$ 2 ·  ·  + 2 +



2
2



                    o + 2 \byref{prop:I.V}. Por lo tanto, 2 $=$ 2 ·  ·  + 2 + 2: por consiguiente 2 > 2 + 2 por 2 ·  · .

Q. E. D. \byref{prop:I.XXXII} \byref{prop:I.V} \byref{prop:I.XXXII} \byref{prop:I.V,prop:I.XXXII,prop:I.XXIX,prop:I.XXXIV} \byref{prop:I.VI,prop:I.XXXIV} \byref{prop:I.XLVII}
\end{center}

\qed

\startproblem{Prop. XI. prob.}\label{prop:II.XI}

\defineNewPicture{
pair A, B, C, D, E, F, G, H, K;
numeric w;
w := 7/2u;
A := (0, 0);
B := (w, 0);
C := (0, w);
D := (w, w);
E := 1/2[A, C];
F := E shifted (0, -abs(E-B));
G := F shifted (abs(F-A), 0);
H = whatever[A, B] = (xpart(G), whatever);
K = whatever[G, H] = whatever[C, D];
draw byPolygon(A,H,K,C)(byyellow);
draw byPolygon(H,B,D,K)(byblue);
draw byPolygon(A,H,G,F)(byblue);
draw byLine(K, G, byblack, DASHED_LINE, REGULAR_WIDTH);
byLineDefine(A, H, byred, SOLID_LINE, REGULAR_WIDTH);
byLineDefine(H, B, byred, DASHED_LINE, REGULAR_WIDTH);
byLineDefine(B, E, byblack, SOLID_LINE, REGULAR_WIDTH);
byLineDefine(C, E, byblue, DASHED_LINE, REGULAR_WIDTH);
byLineDefine(E, A, byblue, SOLID_LINE, REGULAR_WIDTH);
byLineDefine(A, F, byyellow, SOLID_LINE, REGULAR_WIDTH);
draw byNamedLineSeq(-1)(BE,HB,AH);
draw byNamedLineSeq(0)(CE,EA,AF);
draw byLabelsOnPolygon(F, A, E, C, K, D, B, H, G)(ALL_LABELS, 0);
}
\drawCurrentPictureInMargin
\problem{E}{n}{ cualquier triángulo, el cuadrado del lado que sostiene un ángulo agudo, es menor que la suma de los cuadrados de los lados que contienen ese ángulo, el doble del rectángulo contenido por cualquiera de estos lados, y la parte del mismo interceptada entre el pie de la perpendicular se deja caer desde el ángulo opuesto y el punto angular del ángulo agudo.}

\begin{center}
2 < 2 + 2 por 2 ·  · .

2 < 2 + 2 por 2 ·  · .

Primero, supongamos que la perpendicular cae dentro del triángulo, luego \byref{prop:I.XLVI}
                         \\ 2 + 2 $=$ 2 ·  ·  + 2,
                         \\ agrega a cada uno 2 entonces,
                         \\ 2 + 2 + 2 $=$ 2 ·  ·  + 2 + 2
 \\ $\therefore$ \byref{prop:I.X}
                         \\ 2 + 2 $=$ 2 ·  ·  + 2, and $\therefore$ 2 < 2 + 2 by 2 ·  · .

Luego suponga que la perpendicular cae fuera del triángulo, luego \byref{prop:I.III}
                         \\ 2 + 2 $=$ 2 ·  ·  + 2,
                         \\ agrega a cada uno 2 entonces
                         \\ 2 + 2 + 2 $=$ 2 ·  ·  + 2 + 2 $\therefore$ \byref{prop:I.XLVI},
                         \\ 2 + 2 $=$ 2 ·  ·  + 2, $\therefore$ 2 < 2 + 2 por 2 ·  · .

Q. E. D. \byref{post:I.II} \byref{prop:II.VI}
\end{center}

\qed

\starttheorem{Prop. XII. theor.}\label{prop:II.XII}

\defineNewPicture{
pair A, B, C, D;
numeric w, h;
w := 7/2u;
h := 3u;
A := (3/5w, 0);
B := (w, h);
C := (0, 0);
D := (w, 0);
draw byLine(B, A, byred, SOLID_LINE, REGULAR_WIDTH);
byLineDefine(C, A, byblack, SOLID_LINE, REGULAR_WIDTH);
byLineDefine(A, D, byblack, DASHED_LINE, REGULAR_WIDTH);
byLineDefine(D, B, byyellow, SOLID_LINE, REGULAR_WIDTH);
byLineDefine(B, C, byblue, SOLID_LINE, REGULAR_WIDTH);
draw byNamedLineSeq(0)(DB,AD,CA,BC);
draw byLabelsOnPolygon(C, B, D, A)(ALL_LABELS, 0);
}
\drawCurrentPictureInMargin
\problem{D}{ibujar una línea recta cuyo cuadrado sea igual a una figura rectilínea dada.}{ Para dibujar  tal que 2 $=$ }

\begin{center}
Haz  $=$  \byref{prop:II.IV},
                     \\ prolonga  hasta  $=$ ;
 \\ toma  $=$  \byref{prop:I.XLVII},

Traza  \byref{prop:I.XLVII},
                     \\ y prolonga  para encontrarlo: dibuja .
 \\ 2 o 2 $=$  ·  + 2 ,
                     \\ por 2 $=$ 2 + 2 ;
                     \\ $\therefore$ 2 + 2 $=$  ·  + 2
 \\ $\therefore$ 2 $=$  · , y
                     \\ $\therefore$ 2 $=$  $=$ 

Q. E. D.

Un proyecto de Nicholas Rougeux     Licencia

¿Problemas de visualización? ¿Errores ortográficos?
\end{center}

\qed

\starttheorem{Prop. XIII. theor.}\label{prop:II.XIII}
\defineNewPicture[1/4]{
pair A,B,C,D,E,F,G,H, d;
numeric w, h;
w := 3u;
h := 3u;
A := (2/5w, h);
B:= (0, 0);
C := (w, 0);
D = whatever[B, C] = (xpart(A), whatever);
d := (0, -h -4/3u);
E := (w, h) shifted d;
F := (0, 0) shifted d;
G := (2/5w, 0) shifted d;
H = whatever[F, G] = (xpart(E), whatever);
draw byLine(A, D, byyellow, SOLID_LINE, REGULAR_WIDTH);
byLineDefine(A, B, byred, SOLID_LINE, REGULAR_WIDTH);
byLineDefine(B, D, byblack, SOLID_LINE, REGULAR_WIDTH);
byLineDefine(D, C, byblack, DASHED_LINE, REGULAR_WIDTH);
byLineDefine(C, A, byblue, SOLID_LINE, REGULAR_WIDTH);
draw byNamedLineSeq(0)(AB,BD,DC,CA);
draw byLine(E, G, byblue, SOLID_LINE, REGULAR_WIDTH);
byLineDefine(E, F, byred, SOLID_LINE, REGULAR_WIDTH);
byLineDefine(F, G, byblack, SOLID_LINE, REGULAR_WIDTH);
byLineDefine(G, H, byblack, DASHED_LINE, REGULAR_WIDTH);
byLineDefine(H, E, byyellow, SOLID_LINE, REGULAR_WIDTH);
draw byNamedLineSeq(0)(EF,FG,GH,HE);
label.top(btex First etex, (xpart(1/2[B, C]), ypart(A) + 1/4u));
label.top(btex Second etex, (xpart(1/2[F, H]), ypart(E)));
draw byLabelsOnPolygon(B, A, C, D)(ALL_LABELS, 0);
draw byLabelsOnPolygon(F, E, H, G)(ALL_LABELS, 0);
}
\drawCurrentPictureInMargin
\problem{I}{n}{any triangle, the square of the side subtending an acute angle, is less than the sum of the squares of the sides containing that angle, by twice the rectangle contained by either of these sides, and the part of it intercepted between the foot of the perpendicular let fall on it from the opposite angle, and the angular point of the acute angle.\\
First.\\
$\drawProportionalLine{CA}^2 < \drawProportionalLine{BD,DC}^2 + \drawProportionalLine{AB}^2$ by $2 \cdot \drawProportionalLine{BD,DC} \cdot \drawProportionalLine{BD}$.\\
Second.\\
$\drawProportionalLine{EG}^2 < \drawProportionalLine{EF}^2 + \drawProportionalLine{FG}^2$ by $2 \cdot \drawProportionalLine{FG} \cdot \drawProportionalLine{FG,GH}$.
}

\begin{center}
First, suppose the perpendicular to fall within the triangle, then \byref{prop:II.VII}\\
$\drawProportionalLine{BD,DC}^2 + \drawProportionalLine{BD}^2 = 2 \cdot \drawProportionalLine{BD,DC} \cdot \drawProportionalLine{BD} + \drawProportionalLine{DC}^2$,\\
add to each $\drawProportionalLine{AD}^2$ then,\\
$\drawProportionalLine{BD,DC}^2 + \drawProportionalLine{BD}^2 + \drawProportionalLine{AD}^2 = 2 \cdot \drawProportionalLine{BD,DC} \cdot \drawProportionalLine{BD} + \drawProportionalLine{DC}^2 + \drawProportionalLine{AD}^2$\\
$\therefore$ \byref{prop:I.XLVII}\\
$\drawProportionalLine{BD,DC}^2 + \drawProportionalLine{AB}^2 = 2 \cdot \drawProportionalLine{BD,DC} \cdot \drawProportionalLine{BD} + \drawProportionalLine{CA}^2$,\\
and $\therefore \drawProportionalLine{CA}^2 < \drawProportionalLine{BD,DC}^2 + \drawProportionalLine{AB}^2$ by $2 \cdot \drawProportionalLine{BD,DC} \cdot \drawProportionalLine{BD}$

Next suppose the perpendicular to fall without the triangle, then \byref{prop:II.VII}\\
$\drawProportionalLine{FG,GH}^2 + \drawProportionalLine{FG}^2 = 2 \cdot \drawProportionalLine{FG,GH} \cdot \drawProportionalLine{FG} + \drawProportionalLine{GH}^2$,\\
add to each $\drawProportionalLine{HE}^2$ then\\
$\drawProportionalLine{FG,GH}^2 + \drawProportionalLine{FG}^2 + \drawProportionalLine{HE}^2= 2 \cdot \drawProportionalLine{FG,GH} \cdot \drawProportionalLine{FG} + \drawProportionalLine{GH}^2 + \drawProportionalLine{HE}^2$\\
$\therefore$ \byref{prop:I.XLVII}\\
$\drawProportionalLine{EF} + \drawProportionalLine{FG}^2 = 2 \cdot \drawProportionalLine{FG,GH} \cdot \drawProportionalLine{FG}^2 + \drawProportionalLine{EG}^2$,\\
$\therefore \drawProportionalLine{EG}^2 < \drawProportionalLine{EF}^2 + \drawProportionalLine{FG}^2$ by $2 \cdot \drawProportionalLine{FG,GH} \cdot \drawProportionalLine{FG}$.
\end{center}

\qed

\startproblem{Prop. XIV. prob.}\label{prop:II.XIV}
\defineNewPicture[1/3]{
path A;
pair a, b, c, d, e, f;
pair B, C, D, E, F, G, H;
numeric w, h, s, r;
a := (0, 0);
b := (u, 1/4u);
c := (2u, 1/5u);
d := (11/5u, -u);
e := (7/8u, -2u);
f := (1/8u, -3/4u);
A := a--b--c--d--e--f--cycle;
s := 0;
for i := 1 step 1 until length(A) - 1:
	s := s + 1/2(
		abs((point 0 of A) - (point i of A))
		*distanceToLine((point i + 1 of A), (point 0 of A)--(point i of A))
		);
endfor;
w := 5/2u;
h := s/w;
B := (w, 0);
C := (w, -h);
D := (0, -h);
E := (0, 0);
F := (-h, 0);
G := 1/2[B, F];
r := abs(B - G);
H := (D--(E shifted ((E-D)*10))) intersectionpoint ((fullcircle scaled 2r) shifted G);
byPointLabelRemove(a,b,c,d,e,f);
forsuffixes i=a,b,c,d,e,f:
	i := i shifted (xpart(G)-xpart(1/2[urcorner(A),ulcorner(A)]), r - ypart(lrcorner(A)) + 1/2u);
endfor;
draw byPolygon(a,b,c,d,e,f)(byyellow);
draw byPolygon(B,C,D,E)(byred);
draw byLine(H, G, byred, SOLID_LINE, REGULAR_WIDTH);
draw byLine(H, E, byblue, SOLID_LINE, REGULAR_WIDTH);
draw byLine(E, D, byyellow, SOLID_LINE, REGULAR_WIDTH);
byLineDefine(H, F, byblack, SOLID_LINE, THIN_WIDTH);
byLineDefine(H, B, byblack, SOLID_LINE, THIN_WIDTH);
byLineDefine(F, E, byblack, DASHED_LINE, REGULAR_WIDTH);
byLineDefine(E, G, byblue, DASHED_LINE, REGULAR_WIDTH);
byLineDefine(G, B, byblack, SOLID_LINE, REGULAR_WIDTH);
draw byNamedLineSeq(1)(FE,EG,GB,HB,HF);
draw byArc.G(G, B, F, r, byred, 0, 0, 0, 0);
byLineDefine(B, F, byblack, SOLID_LINE, REGULAR_WIDTH);
draw byLabelsOnPolygon(B, C, D, E, F, H)(ALL_LABELS, 0);
draw byLabelsOnPolygon(H, G, B)(OMIT_FIRST_LABEL+OMIT_LAST_LABEL, 0);
}
\drawCurrentPictureInMargin
\problem[4]{T}{o}{draw a right line of which the square shall be equal to a given rectilinear figure.\\
To draw \drawSizedLine{HE} such that, $\drawSizedLine{HE}^2 = \drawPolygon{abcdef}$
}

\begin{center}
Make $\drawPolygon{BCDE} = \drawPolygon{abcdef}$ \byref{prop:I.XLV},
produce \drawSizedLine{EG,GB} until $\drawSizedLine{FE} = \drawSizedLine{ED}$;\\
take $\drawSizedLine{FE,EG} = \drawSizedLine{GB}$ \byref{prop:I.X}.

Describe
\drawFromCurrentPicture[bottom]{
startTempScale(1/3);
draw byNamedLineFull(B, F, 0, 0,  0, 1)(BF);
startAutoLabeling;
draw byNamedArc(G);
stopAutoLabeling;
stopTempScale;
}
\byref{post:I.III},\\
and produce \drawSizedLine{ED} to meet it: draw \drawSizedLine{HG}.

$\drawSizedLine{BG}^2 \mbox{ or } \drawSizedLine{HG}^2 = \drawSizedLine{FE} \cdot \drawSizedLine{EG,GB} + \drawSizedLine{EG}^2$ \byref{prop:II.V},\\
but $\drawSizedLine{HG}^2 = \drawSizedLine{HE} ^2 + \drawSizedLine{EG}^2$ \byref{prop:I.XLVII};

$\therefore \drawSizedLine{HE}^2 + \drawSizedLine{EG}^2 = \drawSizedLine{FE} \cdot \drawSizedLine{EG,GB} + \drawSizedLine{EG}^2$

$\therefore \drawSizedLine{HE}^2 = \drawSizedLine{FE} \cdot \drawSizedLine{EG,GB}$, and

$\therefore \drawSizedLine{HE}^2 = \drawPolygon{BCDE} = \drawPolygon{abcdef}$.
\end{center}

\qed

\part{Book III}

\chapter*{Definitions}

\startdefinition{}\label{def:III.I}

\begin{center}
Los círculos iguales son aquellos cuyos diámetros son iguales.
\end{center}

\startdefinition{}\label{def:III.II}

\defineNewPicture{
	pair O, A, B;
	numeric r;
	r := 3/4u;
	O := (0, 0);
	A := (-r, -r);
	B := (r, -r);
	draw byCircleR(O, r, byblack, 0, 0, -1);
	draw byLine(A, B, byblack, SOLID_LINE, REGULAR_WIDTH);
}
\drawCurrentPictureInMargin
\begin{center}
Se dice que una línea recta toca un círculo cuando se encuentra con el círculo, y siendo prolongada no lo corta.
\end{center}

\startdefinition{}\label{def:III.III}

\defineNewPicture{
	pair A, B, C;
	numeric r[];
	r1 := 2/3u;
	r2 := 1/2r1;
	r3 := 2/5r1;
	A := (0, 0);
	B := A shifted (dir(110) scaled (r1-r2));
	C := A shifted (dir(-130) scaled (r1+r3));
	fill (fullcircle scaled 2r1) shifted A withcolor byyellow;
	draw byCircleR(A, r1, byyellow, 0, 0, 0);
	fill (fullcircle scaled 2r2) shifted B withcolor byred;
	draw byCircleR(B, r2, byblack, 0, 0, -1/2);
	fill (fullcircle scaled 2r3) shifted C withcolor byblue;
	draw byCircleR(C, r3, byblue, 0, 0, -1/2);
}
\drawCurrentPictureInMargin
\begin{center}
Se dice que los círculos se tocan entre sí, cuando se encuentra pero no se cortan entre ellos.
\end{center}

\startdefinition{}\label{def:III.IV}

\defineNewPicture{
	pair O, A, B, C, D, E, F;
	numeric r;
	r := 3/4u;
	O := (0, 0);
	A := dir(170) scaled r;
	B := dir(-110) scaled r;
	C := A xscaled -1;
	D := B xscaled -1;
	E := 1/2[A, B];
	F := 1/2[C, D];
	draw byLine(A, B, byblack, SOLID_LINE, REGULAR_WIDTH);
	draw byLine(C, D, byblack, SOLID_LINE, REGULAR_WIDTH);
	byLineDefine(O, E, byred, SOLID_LINE, REGULAR_WIDTH);
	byLineDefine(O, F, byblue, SOLID_LINE, REGULAR_WIDTH);
	draw byNamedLineSeq(0)(OE,OF);
	draw byCircleR(O, r, byblack, 0, 0, 0);
}
\drawCurrentPictureInMargin
\begin{center}
Se dice que las líneas rectas están igualmente distantes del centro de un círculo cuando las perpendiculares que se dibujan hacia ellas desde el centro son iguales.
\end{center}

\startdefinition{}\label{def:III.V}

\begin{center}
Y la línea recta sobre la cual cae la perpendicular mayor se dice que está más lejos del centro.
\end{center}

\startdefinition{}\label{def:III.VI}

\defineNewPicture{
	pair A, B;
	numeric r;
	r := 3/4u;
	A := (0, 0);
	B := (0, -1/8u);
	draw byFilledCircleSegment(A, r, 1/2, 4 - 1/2, byred);
	draw byFilledCircleSegment(B, r, 4 - 1/2, 8 + 1/2, byblue);
}
\drawCurrentPictureInMargin
\begin{center}
Un segmento de un círculo es la figura contenida por una línea recta y la parte de la circunferencia que corta.
\end{center}

\startdefinition{}\label{def:III.VII}

\begin{center}
Un ángulo en un segmento es el ángulo contenido por dos líneas rectas dibujadas desde cualquiera en la circunferencia del segmento hasta las extremidades de la línea recta que es la base del segmento.
\end{center}

\startdefinition{}\label{def:III.VIII}

\defineNewPicture{
	pair O, A, B, C, D;
	numeric r, b, e;
	r := u;
	b := -1;
	e := 5;
	O := (0, 0);
	A := dir(100) scaled r;
	B := dir(30) scaled r;
	C := point b of fullcircle scaled 2r;
	D := point e of fullcircle scaled 2r;
	byAngleDefine(C, A, D, byblue, SOLID_SECTOR);
	byAngleDefine(C, B, D, byyellow, SOLID_SECTOR);
	draw byNamedAngleResized();
	draw byArcBE(O, b, e, r, byblack, 0, 0, 0, 0);
	draw byLine(C, A, byblack, SOLID_LINE, REGULAR_WIDTH);
	draw byLine(C, B, byblack, SOLID_LINE, REGULAR_WIDTH);
	draw byLine(D, A, byblack, SOLID_LINE, REGULAR_WIDTH);
	draw byLine(D, B, byblack, SOLID_LINE, REGULAR_WIDTH);
	draw byLine(C, D, byblack, SOLID_LINE, REGULAR_WIDTH);
}
\drawCurrentPictureInMargin
\begin{center}
Se dice que un ángulo se encuentra en la parte de la circunferencia, o el arco, interceptado entre las líneas rectas que contienen el ángulo.
\end{center}

\startdefinition{}\label{def:III.IX}

\defineNewPicture{
	pair O, A, B, C;
	numeric r, b, e;
	r := 3/4u;
	b := -3/2;
	e := 9/2;
	O := (0, 0);
	A := dir(80) scaled r;
	B := point b of fullcircle scaled 2r;
	C := point e of fullcircle scaled 2r;
	byAngleDefine(C, A, B, byblue, SOLID_SECTOR);
	draw byNamedAngleResized();
	draw byArcBE.Op(O, b, e, r, byblack, 1, 0, 0, 0);
	draw byArcBE.Om(O, e, b + 8, r, byblack, 0, 0, 0, 0);
	draw byLine(C, A, byblack, SOLID_LINE, REGULAR_WIDTH);
	draw byLine(B, A, byblack, SOLID_LINE, REGULAR_WIDTH);
}
\drawCurrentPictureInMargin
\begin{center}
Un sector de un círculo es la figura contenida por dos radios y el arco entre ellos.
\end{center}

\startdefinition{}\label{def:III.X}

\defineNewPicture{
	pair O;
	numeric r, b, e;
	r := 2/3u;
	b := 1;
	e := 3;
	O := (0, 0);
	draw byFilledCircleSector(O, r, b, e, byyellow);
	draw byArcBE(O, e, b + 8, r, byblack, 0, 0, -1, 0);
}
\drawCurrentPictureInMargin
\begin{center}
Segmentos similares de círculos son aquellos que contienen ángulos iguales.

Los círculos que tienen el mismo centro se llaman \emph{círculos concéntricos}.
\end{center}

\startdefinition{}\label{def:III.XI}

\defineNewPicture{
	pair M, N, A, B, C, D, E, F;
	numeric r[], b, e;
	r1 := 3/2u;
	r2 := u;
	b := 1;
	e := 3;
	M := (0, 0);
	N := (0, -1/3u);
	A := (point b of fullcircle scaled 2r1) shifted M;
	B := (point 1/3[b,e] of fullcircle scaled 2r1) shifted M;
	C := (point e of fullcircle scaled 2r1) shifted M;
	D := (point b of fullcircle scaled 2r2) shifted N;
	E := (point 1/3[b,e] of fullcircle scaled 2r2) shifted N;
	F := (point e of fullcircle scaled 2r2) shifted N;
	draw byPolygon(A,B,C)(byred);
	draw byPolygon(D,E,F)(byred);
	draw byArcBE(M, b, e, r1, byblack, 0, 0, 0, 1);
	draw byArcBE(N, b, e, r2, byblack, 0, 0, 0, 1);
}
\drawCurrentPictureInMargin
\begin{center}
Proposición I. Problema.

Para encontrar el centro de un círculo dado .

Dibuja dentro del círculo cualquier línea recta , haz  $=$ , dibuja  ⊥ ; biseca , y el punto de bisección es el centro.

Sí es posible, deje que cualquier otro punto como el punto de encuentro de , ,  sea el centro.

Porque en  y 

 $=$  (hip. L. 1, def. 15.)  $=$  (const.) y  común,  $=$  (L. 1, pr. 8.), y por lo tanto son ángulos rectos; pero  $=$ 
Right angle

 (const.)  $=$   lo cual es absurdo; por lo tanto, el punto asumido no es el centro del círculo; de la misma manera se puede demostrar que otro punto que no esté en  es el centro, por lo tanto el centro está en , y por consiguiente el punto donde  es bisecada es el centro.

Q. E. D.
\end{center}

\startdefinition{}\label{def:III.XII}

\defineNewPicture{
	pair O;
	numeric r[];
	r1 := 1/3u;
	r2 := 1/2u;
	r3 := 3/4u;
	O := (0, 0);
	draw byFilledCircleSegment(O, r3, 0, 8, byred);
	draw byFilledCircleSegment(O, r2, 0, 8, white);
	draw byCircleR.OII(O, r2, byblack, 0, 0, 0);
	draw byFilledCircleSegment(O, r1, 0, 8, byblue);
}
\drawCurrentPictureInMargin
\begin{center}
Proposición II. Teorema.

La línea recta () que une dos puntos en la circunferencia de un círculo , se encuentra completamente dentro del círculo.

Encuentra el centro de  (L. 3. pr. 1.);
                     \\ desde el centro dibuja  a cualquier punto en ,
 \\ encontrando la circunferencia desde el centro;
                     \\ dibuja  y .

Entonces  $=$  (L. 1. pr. 5.)
                     \\ pero  >  o >  (L. 1. pr. 16.)
                     \\ $\therefore$  >  (L. 1. pr. 19.)
                     \\ pero  $=$ ,
                     \\ $\therefore$  > ;
                     \\ $\therefore$  < ;
                     \\ $\therefore$ cada punto en  se encuentra dentro del círculo.

Q. E. D.
\end{center}

\startproblem{Prop. I. prob.}\label{prop:III.I}

\defineNewPicture{
pair A, B, C, D, E, F, G;
numeric r, a;
r := 9/4u;
F := (0, 0);
A := F shifted (dir(170)*r);
B := F shifted (dir(-95)*r);
D := 1/2[A, B];
C := F shifted (dir(angle(A-B) - 90)*r);
E := F shifted (dir(angle(A-B) +90)*r);
G := F shifted (dir(-45)*1/2r);
a := -angle(G-D);
forsuffixes i=A, B, D, C, E, F, G:
	i := i rotated a;
endfor;
byAngleDefine(A, D, F, byblue, SOLID_SECTOR);
byAngleDefine(F, D, G, byyellow, SOLID_SECTOR);
byAngleDefine(G, D, B, byblack, SOLID_SECTOR);
draw byNamedAngleResized();
draw byLine(D, G)(byblue, DASHED_LINE, REGULAR_WIDTH);
draw byLine(A, D)(byred, SOLID_LINE, REGULAR_WIDTH);
draw byLine(D, B)(byred, DASHED_LINE, REGULAR_WIDTH);
draw byLine(E, C)(byblack, SOLID_LINE, REGULAR_WIDTH);
draw byMarkLine(1/2, byblack)(EC);
byLineDefine(A, G, byblue, SOLID_LINE, REGULAR_WIDTH);
byLineDefine(B, G, byblack, DASHED_LINE, REGULAR_WIDTH);
draw byNamedLineSeq(0)(AG,BG);
draw byCircleR(F, r, byblue, 0, 0, 0);
draw byLabelsOnPolygon(A, G, B)(ALL_LABELS, 0);
draw byLabelsOnPolygon(E, D, A)(OMIT_FIRST_LABEL+OMIT_LAST_LABEL, 0);
draw byLabelsOnPolygon(E, F, C)(OMIT_FIRST_LABEL+OMIT_LAST_LABEL, 4);
draw byLabelLineEnd(E, C, 0);
draw byLabelLineEnd(C, E, 0);
}
\drawCurrentPictureInMargin
\problem{S}{i}{ una línea recta (\drawUnitLine{AD,DB}) dibujada a través del centro de un círculo \drawUnitLine{AD} biseca una cuerda (\drawUnitLine{DB}) que no pasa a través del centro, es perpendicular a ella; o, si es perpendicular a ella, la biseca.}

\begin{center}
Dibuja \drawUnitLine{EC} y \drawUnitLine{AD,DB} al centro del círculo.

En \drawUnitLine{EC} y \drawUnitLine{AG}  \\ \drawUnitLine{DG} $=$ \drawUnitLine{BG}, \drawLine[bottom][triangleADG]{AG,DG,AD} común, y
                     \\ \drawLine[middle][triangleDGB]{DB,DG,BG} $=$ \drawUnitLine{AG} $\therefore$ \drawUnitLine{BG} $=$ \drawUnitLine{AD} (L. 1. pr. 8.)
                     \\ y $\therefore$ \drawUnitLine{DB} ⊥ \drawUnitLine{DG} (L. 1. def. 10.)
                     \\ Otra vez deje \drawAngle{ADF,FDG} ⊥ \drawAngle{GDB}

Entonces en \drawAngle{FDG,GDB} y \drawAngle{GDB}  \\ \drawAngle{FDG,GDB} $=$ \drawUnitLine{EC} (L. 1. pr. 5.)
                     \\ \drawUnitLine{EC} $=$ \drawUnitLine{EC} (hip.)
                     \\ y  $=$   \\ $\therefore$  $=$  (L. 1. pr. 26.)
                     \\ y $\therefore$  biseca .

Q. E. D. \byref{\hypref,def:I.XV} \byref{\constref} \byref{prop:I.VIII} \byref{\constref} \byref{ax:I.XI}
\end{center}

\qed

\starttheorem{Prop. II. theor.}\label{prop:III.II}

\defineNewPicture{
pair A, B, D, E, F;
numeric r;
r := 9/4u;
D := (0, 0);
A := (dir(185) scaled r) shifted D;
B := (dir(-70) scaled r) shifted D;
E := 3/5[A, B];
F := (dir(angle(E-D)) scaled r) shifted D;
byAngleDefine(D, A, B, byblue, SOLID_SECTOR);
byAngleDefine(A, E, D, byyellow, SOLID_SECTOR);
byAngleDefine(A, B, D, byblack, SOLID_SECTOR);
draw byNamedAngleResized();
draw byLine(D, E, byblack, SOLID_LINE, REGULAR_WIDTH);
draw byLine(E, F, byblack, DASHED_LINE, REGULAR_WIDTH);
draw byLine(A, B, byred, SOLID_LINE, REGULAR_WIDTH);
byLineDefine(A, D, byyellow, SOLID_LINE, REGULAR_WIDTH);
byLineDefine(B, D, byblue, SOLID_LINE, REGULAR_WIDTH);
draw byNamedLineSeq(0)(AD,BD);
draw byCircleR(D, r, byred, 0, 0, 0);
draw byLabelsOnPolygon(B, F, A, D)(ALL_LABELS, 0);
draw byLabelsOnPolygon(A, E, F)(OMIT_FIRST_LABEL+OMIT_LAST_LABEL, 0);
}
\drawCurrentPictureInMargin
\problem{S}{i}{ en un círculo dos líneas rectas que no pasan por el centro se cortan entre sí, no se bisecan entre sí.}

\begin{center}
Si una de las líneas pasa por el centro, es evidente que no puede ser bisecada por la otra, que no pasa por el centro.

Pero si  ninguna de las líneas \drawAngle{A} o \drawAngle{B} pasa a tráves del centro, dibuja \drawAngle{E} desde el centro a su intersección.

Si \drawAngle{A} es bisecada, \drawAngle{B} ⊥ a esta (L. 3. pr. 3.)
                     \\ $\therefore$  $=$ 
Right angle

 y si  es
                     \\ bisecada,  ⊥  (L. 3. pr. 3.)
                     \\ $\therefore$  $=$ 
Right angle
 y $\therefore$  $=$ ; una parte
                     \\ igual al todo; lo que es absurdo:
                     \\ $\therefore$  y   \\ no se bisecan la uno a la otra.

Q. E. D. \byref{prop:III.I} \byref{prop:I.V} \byref{prop:I.XVI} \byref{prop:I.XIX}
\end{center}

\qed

\starttheorem{Prop. III. theor.}\label{prop:III.III}

\defineNewPicture[1/2]{
pair A, B, C, D, E, F;
numeric r;
r := 9/4u;
E := (0, 0);
A := (dir(-90 - 60) scaled r) shifted E;
B := (dir(-90 + 60) scaled r) shifted E;
C := (dir(90) scaled r) shifted E;
D := (dir(-90) scaled r) shifted E;
F = whatever[A, B] = whatever[C, D];
byAngleDefine(E, A, F, byblue, SOLID_SECTOR);
byAngleDefine(A, F, E, byblack, SOLID_SECTOR);
byAngleDefine(E, F, B, byyellow, SOLID_SECTOR);
byAngleDefine(F, B, E, byred, SOLID_SECTOR);
draw byNamedAngleResized();
draw byLine(C, E, byblack, DASHED_LINE, REGULAR_WIDTH);
draw byLine(E, F, byblack, SOLID_LINE, REGULAR_WIDTH);
draw byLine(F, D, byblack, DASHED_LINE, REGULAR_WIDTH);
byLineDefine(A, F, byred, SOLID_LINE, REGULAR_WIDTH);
byLineDefine(F, B, byred, DASHED_LINE, REGULAR_WIDTH);
draw byNamedLineSeq(0)(AF,FB);
byLineDefine(A, E, byyellow, SOLID_LINE, REGULAR_WIDTH);
byLineDefine(E, B, byblue, SOLID_LINE, REGULAR_WIDTH);
draw byNamedLineSeq(0)(AE,EB);
draw byCircleR(E, r, byblue, 0, 0, 0);
draw byLabelsOnCircle(A, B)(E);
draw byLabelsOnPolygon(D, F, A)(OMIT_FIRST_LABEL+OMIT_LAST_LABEL, 2);
draw byLabelsOnPolygon(A, E, C)(OMIT_FIRST_LABEL+OMIT_LAST_LABEL, 0);
}
\drawCurrentPictureInMargin
\problem{S}{i}{ dos círculos \drawUnitLine{EF} se intersecan, no tienen el mismo centro.}

\begin{center}
Suponga que es posible que dos círculos que se cruzan tengan un centro común; desde dicho supuesto centro dibuje \drawUnitLine{AF,FB} hasta el punto de intersección y \drawUnitLine{AE} encontrando las circunferencias de los círculos.

Entonces \drawUnitLine{EB} $=$ \drawLine[bottom][triangleAEF]{AE,EF,AF} (L. 1. def. 15.)
                 \\ y \drawLine[bottom][triangleFEB]{EF,EB,FB} $=$ \drawUnitLine{AE} (L. 1. def. 15.)
                 \\ $\therefore$ \drawUnitLine{EB} $=$ \drawUnitLine{EF}; una parte
                 \\ igual al todo, lo cual es absurdo:
                 \\ $\therefore$ los círculos que se intersecan en cualquier punto no pueden tener el mismo centro.

Q. E. D. \drawUnitLine{AF} \drawUnitLine{FB} \drawAngle{AFE} \drawAngle{EFB} \drawUnitLine{EF} \drawUnitLine{AF,FB} \drawUnitLine{EF} \drawUnitLine{AF,FB} \triangleAEF \triangleFEB \drawAngle{A} \drawAngle{B} \drawAngle{AFE} \drawAngle{EFB} \drawUnitLine{AE} \drawUnitLine{EB} \drawUnitLine{AF} \drawUnitLine{FB} \drawUnitLine{EF} \drawUnitLine{AF,FB} \byref{prop:I.VIII} \byref{def:I.X} \byref{prop:I.V} \byref{\hypref} \byref{prop:I.XXVI}
\end{center}

\qed

\starttheorem{Prop. IV. theor.}\label{prop:III.IV}

\defineNewPicture{
pair A, B, C, D, E, F;
numeric r;
r := 9/4u;
F := (0, 0);
A := (dir(-175)*r) shifted F;
B := (dir(-140)*r) shifted F;
C := (dir(-50)*r) shifted F;
D := (dir(-10)*r) shifted F;
E = whatever[A, C] = whatever[B, D];
byAngleDefine(F, E, D, byblue, SOLID_SECTOR);
byAngleDefine(D, E, C, byyellow, SOLID_SECTOR);
draw byNamedAngleResized();
draw byLine(E, F, byblack, DASHED_LINE, REGULAR_WIDTH);
draw byLine(B, D, byred, SOLID_LINE, REGULAR_WIDTH);
draw byLine(A, C, byblack, SOLID_LINE, REGULAR_WIDTH);
draw byCircleR(F, r, byblue, 0, 0, 0);
draw byLabelsOnCircle(A, B, C, D)(F);
draw byLabelLineEnd(F, E, 0);
draw byLabelsOnPolygon(C, E, B)(OMIT_FIRST_LABEL+OMIT_LAST_LABEL, 0);
}
\drawCurrentPictureInMargin
\problem{S}{i}{ dos círculos \drawUnitLine{AC} se tocan internamente, no tienen el mismo centro}

\begin{center}
Porque, si es posible, deje que ambos círculos tengan el mismo centro; desde tal supuesto centro dibuje \drawUnitLine{BD} cortando ambos círculos, y \drawUnitLine{EF} hasta el punto de contacto.

Entonces \drawUnitLine{AC} $=$ \drawUnitLine{EF} (L. 1. def. 15.)
                 \\ y \drawAngle{FED,DEC} $=$ \drawUnitLine{BD} (L. 1. def. 15.)
                 \\ $\therefore$ \drawUnitLine{EF} $=$ \drawUnitLine{BD}; una parte
                 \\ igual al todo, lo cual es absurdo;

por lo tanto, el punto supuesto no es el centro de ambos círculos; y de la misma manera se puede demostrar que no hay otro punto.

Q. E. D. \drawAngle{FED} \drawAngle{FED} \drawAngle{FED,DEC} \drawUnitLine{AC} \drawUnitLine{BD} \byref{prop:III.III} \byref{prop:III.III}
\end{center}

\qed

\starttheorem{Prop. V. theor.}\label{prop:III.V}

\defineNewPicture{
pair M, N, E, F, G, C;
numeric r[], s;
path c[];
r1 := 2u;
r2 := 2u;
s := u;
M := (1/2s, 0);
N := (-1/2s, 0);
c1 := (fullcircle scaled 2r1) shifted M;
c2 := (fullcircle scaled 2r2) shifted N;
E := 1/2[M, N];
C := (subpath(0, 4) of c1) intersectionpoint (subpath(0, 4) of c2);
G := point 7/2 of c2;
F := c1 intersectionpoint (E--G);
byLineDefine(C, E, byyellow, SOLID_LINE, REGULAR_WIDTH);
byLineDefine(E, F, byblack, SOLID_LINE, REGULAR_WIDTH);
byLineDefine(F, G, byblack, DASHED_LINE, REGULAR_WIDTH);
draw byNamedLineSeq(0)(CE,EF,FG);
draw byArcBE.Ma(M, 4, 0, r1, byred, 0, 0, 0, 0);
draw byArcBE.Na(N, 4, 0, r2, byblue, 0, 0, 0, 0);
draw byArcBE.Nb(N, 4, 8, r2, byblue, 0, 0, 0, 0);
draw byArcBE.Mb(M, 4, 8, r1, byred, 0, 0, 0, 0);
draw byLabelLineEnd(C, E, 0);
draw byLabelLineEnd(G, E, 0);
draw byLabelsOnPolygon(C, E, F)(OMIT_FIRST_LABEL+OMIT_LAST_LABEL, 0);
draw byLabelsOnPolygon(C, F, E)(OMIT_FIRST_LABEL+OMIT_LAST_LABEL, 1);
}
\drawCurrentPictureInMargin
\starttheorem{S}{i desde cualquier punto dentro de un círculo \drawUnitLine{CE} que no es el centro, se dibujan líneas}{ }

\begin{center}

\drawUnitLine{EF,FG}
\drawUnitLine{CE}
\drawUnitLine{EF}


                    a la circunferencia; el mayor de esas líneas es la (\drawUnitLine{CE}) que pasa por el centro, y la menor es la parte restante (\drawUnitLine{EF,FG}) del diámetro.

De las otras, la (\drawUnitLine{EF}) que está más cerca de la línea que pasa por el centro, es mayor que la (\drawUnitLine{EF,FG}) que es más lejana.

\emph{Fig. 2.} Las dos líneas ( y ) que forman ángulos iguales con el que pasa por el centro, en lados opuestos, son iguales entre sí; y no se puede dibujar una tercera línea igual a ellas, desde el mismo punto hasta la circunferencia.

Al centro del círculo dibuja  y ; entonces  $=$  (L. 1. def. 15.)  $=$  +  >  (L. 1. pr. 20.) de la misma manera se puede demostrar que  es mayor que , o cualquier otra línea trazada desde el mismo punto a la circunferencia. De nuevo, por (L. 1. pr. 20.)  +  >  $=$  + , toma  de ambos; $\therefore$  >  \byref{def:I.XV}, de la misma manera se puede demostrar que  es menor que cualquier otra línea dibujada desde el mismo punto a la circunferencia. Otra vez, en  y ,  común,  > , y  $=$ 

$\therefore$  >  (L. 1. pr. 24.) y de la misma manera, puede probarse que  es más grande que cualquier otra línea trazada desde el mismo punto a la circunferencia más lejana desde .

Si  $=$  entonces  $=$ , si no toma  $=$  dibuja , entonces en  y ,  común,  $=$  y  $=$  $\therefore$  $=$  (L. 1. pr. 4.) $\therefore$  $=$  $=$  una parte igual al todo, lo cual es absurdo:

$\therefore$  $=$ ; y ninguna otra línea es igual a  dibujada desde el mismo punto a la circunferencia; porque si estuviera más cerca del que pasa por el centro sería más grande, y si fuera más distante sería menos.

Q. E. D. \byref{def:I.XV}
\end{center}

\qed

\starttheorem{Prop. VI. theor.}\label{prop:III.VI}

\defineNewPicture[1/4]{
pair M, N, B, C, E, F;
numeric r[], a;
path c[];
a := 80;
r1 := 9/4u;
r2 := 7/4u;
M := (0, 0);
N := M shifted (dir(a)*(r1-r2));
c1 := (fullcircle scaled 2r1) shifted M;
c2 := (fullcircle scaled 2r2) shifted N;
F := 1/2[M, N];
C :=c1 intersectionpoint (M--(M shifted (dir(a)*2r1)));
B := point -3/2 of c1;
E := c2 intersectionpoint (F--B);
byLineDefine(C, F, byyellow, SOLID_LINE, REGULAR_WIDTH);
byLineDefine(F, E, byblue, DASHED_LINE, REGULAR_WIDTH);
byLineDefine(E, B, byblue, SOLID_LINE, REGULAR_WIDTH);
draw byNamedLineSeq(0)(CF,FE,EB);
draw byCircle.M(M, C, byred, 0, 0, 0);
draw byCircle.N(N, C, byblack, 0, 0, -1);
draw byLabelsOnCircle(B, C)(M);
draw byLabelsOnPolygon(E, F, C)(OMIT_FIRST_LABEL+OMIT_LAST_LABEL, 0);
draw byLabelPoint(E, angle(B-F)+45, 2);
byPointLabelRemove(M, N);
}
\drawCurrentPictureInMargin
\problem{E}{l}{ texto original de esta proposición se divide aquí en tres partes. \drawUnitLine{FE,EB} \drawUnitLine{CF} \drawUnitLine{CF} \drawUnitLine{FE} \drawUnitLine{CF} \drawUnitLine{FE,EB} \drawUnitLine{FE} \drawUnitLine{FE,EB} \byref{def:I.XV} \byref{def:I.XV}}

\begin{center}

\end{center}

\qed

\starttheorem{Prop. VII. theor.}\label{prop:III.VII}

\defineNewPicture[1/2]{
pair A, B, C, D, E, F, G, H, K, M, N, d;
numeric r;
r := 2u;
E := (0, 0);
A := E shifted (dir(90)*r);
D := E shifted (dir(-90)*r);
F := 2/3[E, D];
B := E shifted (dir(20)*r);
C := E shifted (dir(-5)*r);
G := E shifted (dir(15)*r);
H := E shifted (dir(-170)*r);
K := E shifted (dir(175)*r);
M = whatever[E, H] = whatever[F, K];
N := D;
byAngleDefine(B, E, C, byblack, SOLID_SECTOR);
byAngleDefine(C, E, F, byyellow, SOLID_SECTOR);
draw byNamedAngleResized(BEC, CEF);
draw byLine(F, B, byred, SOLID_LINE, REGULAR_WIDTH);
draw byLine(E, C, byblue, DASHED_LINE, REGULAR_WIDTH);
draw byLine(F, C, byblue, SOLID_LINE, REGULAR_WIDTH);
draw byLine(E, B, byred, DASHED_LINE, REGULAR_WIDTH);
byLineDefine(A, E, byblack, DASHED_LINE, REGULAR_WIDTH);
byLineDefine(E, F, byblack, SOLID_LINE, REGULAR_WIDTH);
byLineDefine(F, D, byyellow, SOLID_LINE, REGULAR_WIDTH);
draw byNamedLineSeq(0)(AE,EF,FD);
draw byCircleR(E, r, byblue, 0, 0, 0);
draw byLabelsOnCircle(A, B, C, D)(E);
draw byLabelsOnPolygon(D, F, E, A)(OMIT_FIRST_LABEL+OMIT_LAST_LABEL, 0);
label.top(btex Fig. 1 etex, (0, r + 1/4cm));
d := (0, -2r -2u);
draw image(
byAngleDefine(G, F, E, byyellow, SOLID_SECTOR);
byAngleDefine(K, F, E, byred, SOLID_SECTOR);
draw byNamedAngleResized(GFE,KFE);
draw byLine(F, G, byred, SOLID_LINE, REGULAR_WIDTH);
draw byLine(E, G, byred, DASHED_LINE, REGULAR_WIDTH);
draw byLine(E, M, byyellow, SOLID_LINE, REGULAR_WIDTH);
draw byLine(M, H, byyellow, DASHED_LINE, REGULAR_WIDTH);
byLineDefine(F, M, byblue, SOLID_LINE, REGULAR_WIDTH);
byLineDefine(M, K, byblue, DASHED_LINE, REGULAR_WIDTH);
draw byNamedLineSeq(0)(FM,MK);
byLineDefine(A, E, byblack, DASHED_LINE, REGULAR_WIDTH);
byLineDefine(E, F, byblack, SOLID_LINE, REGULAR_WIDTH);
byLineDefine(F, N, byblack, SOLID_LINE, REGULAR_WIDTH);
draw byNamedLineSeq(0)(AE,EF,FN);
draw byCircleR(E, r, byblue, 0, 0, 0);
draw byLabelsOnCircle(G, K, H)(E);
draw byLabelsOnPolygon(D, F, M, H)(OMIT_FIRST_LABEL+OMIT_LAST_LABEL, 0);
draw byLabelsOnPolygon(M, E, A)(OMIT_FIRST_LABEL+OMIT_LAST_LABEL, 0);
label.top(btex Fig. 2 etex, (0, r+1/4cm));
) shifted d;
}
\drawCurrentPictureInMargin
\problem{S}{i desde un punto sin círculo, se dibujan líneas rectas}{ }

\begin{center}

\drawFromCurrentPicture{
startTempScale(1/3);
draw byNamedCircle(E);
draw byLabelPoint(F, 0, 0);
stopTempScale;
}
\drawUnitLine{EF,AE}
\drawUnitLine{FB} etc.



                    a la circunferencia; de aquellas que caen sobre la concavidad de la circunferencia, la mayor (\drawUnitLine{FC}) es la que pasa por el centro, y la línea (\drawUnitLine{EF,AE}) que está más cerca de la mayor es mayor que la (\drawUnitLine{FD}) que está más distante.

Dibuja \drawUnitLine{FB} y \drawUnitLine{FC} al centro.

Entonces, \drawUnitLine{FM,MK} que pasa a través del centro, es mayor; porque desde \drawUnitLine{FG} $=$ \drawUnitLine{EB}, si \drawUnitLine{EC} se agrega a ambas, \drawUnitLine{AE} $=$ \drawUnitLine{EB} + \drawUnitLine{EF,AE}; pero > \drawUnitLine{EF} (L. 1. pr. 20.) $\therefore$ \drawUnitLine{EB} es mayor que cualquier otra línea dibujada desde el mismo punto a la concavidad de la circunferencia.

De nuevo en \drawUnitLine{FB} y \drawUnitLine{EF,AE}, \drawUnitLine{FC} $=$ \drawUnitLine{EF},
 \\ y \drawUnitLine{FC} común, pero \drawUnitLine{EC} > \drawUnitLine{FD},
 \\ $\therefore$ \drawUnitLine{EF} > \drawUnitLine{EF} (L. 1. pr. 24.);
                     \\ y de la misma manera \drawUnitLine{FC} se puede mostrar > que cualquier otra línea más alejada de \drawUnitLine{FD}. \drawUnitLine{FD} \drawLine[middle][triangleEFB]{FB,EF,EB} \drawLine[middle][triangleEFC]{FC,EF,EC} \drawUnitLine{EF} \drawAngle{BEC,CEF} \drawAngle{CEF} \drawUnitLine{EB} \drawUnitLine{EC} \drawUnitLine{FB} \drawUnitLine{FC} \drawUnitLine{FB} \drawUnitLine{EF,AE} \drawAngle{KFE} \drawAngle{GFE} \drawUnitLine{FM,MK} \drawUnitLine{FG} \drawUnitLine{FM} \drawUnitLine{FG} \drawUnitLine{EM,MH} \drawLine[middle][triangleEFM]{EM,EF,FM} \drawLine[middle][triangleEFG]{FG,EF,EG} \drawUnitLine{EF} \drawAngle{KFE} \drawAngle{GFE} \drawUnitLine{FG} \drawUnitLine{FM} \drawUnitLine{EG} \drawUnitLine{EM} \drawUnitLine{EG} \drawUnitLine{EM,MH} \drawUnitLine{EM} \drawUnitLine{FG} \drawUnitLine{FM,MK} \drawUnitLine{FG} \byref{def:I.XV} \byref{prop:I.XX} \byref{prop:I.XX} \byref{ax:I.III} \byref{prop:I.XXIV} \byref{prop:I.IV}
\end{center}

\qed

\starttheorem{Prop. VIII. theor.}\label{prop:III.VIII}

\defineNewPicture[1/4]{
pair M, D, A, E, F;
numeric r;
r := 7/4u;
M := (0, 0);
D := M shifted (dir(90)*3/2r);
A := (dir(-90)*r) shifted M;
E := (dir(-140)*r) shifted M;
F := (dir(-170)*r) shifted M;
byAngleDefine(D, M, F, byyellow, SOLID_SECTOR);
byAngleDefine(F, M, E, byblack, SOLID_SECTOR);
draw byNamedAngleResized();
draw byLine(D, E, byred, SOLID_LINE, REGULAR_WIDTH);
draw byLine(M, E, byred, DASHED_LINE, REGULAR_WIDTH);
draw byLine(M, F, byblue, DASHED_LINE, REGULAR_WIDTH);
byLineDefine(M, A, byblack, DASHED_LINE, REGULAR_WIDTH);
byLineDefine(D, M, byblack, SOLID_LINE, REGULAR_WIDTH);
byLineDefine(D, F, byblue, SOLID_LINE, REGULAR_WIDTH);
draw byNamedLineSeq(0)(MA, DM,DF);
draw byCircle.M(M, E, byblack, 0, 0, 0);
draw byLabelsOnCircle(F, E, A)(M);
draw byLabelsOnPolygon(F, D, M, A)(OMIT_FIRST_LABEL+OMIT_LAST_LABEL, 0);
}
\drawCurrentPictureInMargin
\problem{D}{e}{ esas líneas que caen en la circunferencia convexa, la menor (\drawUnitLine{DM,MA}) es la que siendo prolongada pasaría por el centro, y la línea que está más cerca de la menor es menor que la que es más lejana.}

\begin{center}
Porque desde \drawUnitLine{DE} + \drawUnitLine{DF} > \drawUnitLine{MF} (L. 1. pr. 20.)
                     \\ y \drawUnitLine{ME} $=$ \drawUnitLine[0.75cm]{DM,MA},
 \\ $\therefore$ \drawUnitLine[0.75cm]{MA} > \drawUnitLine[0.75cm]{ME} \byref{prop:I.XX}
                     \\ Y otra vez desde \drawUnitLine[0.75cm]{DM} + \drawUnitLine{DM,MA} >
 \\ \drawUnitLine[0.75cm]{DM} + \drawUnitLine[0.75cm]{ME} (L. 1. pr. 21.),
                     \\ y \drawUnitLine[0.75cm]{DE} $=$ \drawUnitLine{DM,MA},
 \\ $\therefore$ \drawLine[middle][triangleDFM]{DM,MF,DF} < \drawLine[middle][triangleDEM]{DM,ME,DE}. Y así de otras. \drawUnitLine[0.5cm]{MF} \drawUnitLine[0.5cm]{ME} \drawUnitLine[0.5cm]{DM} \drawAngle{DMF,FME} \drawAngle{DMF} \drawUnitLine{DE} \drawUnitLine{DF} \drawUnitLine{DE} \drawUnitLine{DM,MA} \drawUnitLine{DG} \drawUnitLine{KM} \drawUnitLine{DK} \drawUnitLine{DG,GM} \drawUnitLine{KM} \drawUnitLine{GM} \drawUnitLine{DK} \drawUnitLine{DG} \drawUnitLine[0.75cm]{HM} \drawUnitLine[0.75cm]{DH} \drawUnitLine[0.75cm]{KM} \drawUnitLine[0.75cm]{DK} \drawUnitLine[0.75cm]{HM} \drawUnitLine[0.75cm]{KM} \drawUnitLine{DK} \drawUnitLine{DH} \drawUnitLine{DO,ON} \drawUnitLine{DH} \drawAngle{HDM} \drawAngle{MDN} \drawUnitLine{DO} \drawUnitLine{DH} \drawUnitLine{BM,BO} \drawLine[middle][triangleDMO]{DM,DO,BO,BM} \drawLine[middle][triangleDMH]{HM,DH,DM} \drawUnitLine{DO} \drawUnitLine{DH} \drawUnitLine{DM} \drawAngle{MDN} \drawAngle{HDM} \drawUnitLine{BM,BO} \drawUnitLine{HM} \drawUnitLine{HM} \drawUnitLine{BM} \drawUnitLine{BM} \drawUnitLine{BM,BO} \drawUnitLine{DH} \drawUnitLine{DO} \drawUnitLine{DO,ON} \drawUnitLine{DO,ON} \drawUnitLine{DH} \drawUnitLine{DH} \drawUnitLine{DO,ON} \byref{prop:I.XXIV} \byref{prop:I.XX} \byref{ax:I.V} \byref{prop:I.XXI} \byref{prop:I.IV}
\end{center}

\qed

\starttheorem{Prop. IX. theor.}\label{prop:III.IX}

\defineNewPicture[1/4]{
pair D, A, B, C, F, L, H;
numeric r;
r := 7/4u;
D := (0, 0);
A := (dir(170)*r) shifted D;
B := (dir(-90)*r) shifted D;
C := (dir(-45)*r) shifted D;
L := (dir(45)*r) shifted D;
H := (dir(45 + 180)*r) shifted D;
F := 2/4[D, L];
draw byLine(D, A, byyellow, DASHED_LINE, REGULAR_WIDTH);
draw byLine(D, B, byyellow, SOLID_LINE, REGULAR_WIDTH);
draw byLine(D, C, byblue, SOLID_LINE, REGULAR_WIDTH);
byLineDefine(D, F, byblack, SOLID_LINE, REGULAR_WIDTH);
byLineDefine(F, L, byred, DASHED_LINE, REGULAR_WIDTH);
byLineDefine(D, H, byred, SOLID_LINE, REGULAR_WIDTH);
draw byNamedLineSeq(0)(DH, DF, FL);
draw byCircle.D(D, A, byblue, 0, 0, 0);
draw byLabelsOnCircle(A, B, C, H, L)(D);
draw byLabelsOnPolygon(A, D, F, L)(OMIT_FIRST_LABEL+OMIT_LAST_LABEL, 0);
}
\drawCurrentPictureInMargin
\problem{A}{demás,}{ las líneas que forman ángulos iguales con los que pasan por el centro son iguales, ya sea que caigan en la circunferencia cóncava o convexa; y no se puede dibujar una tercera línea igual desde el mismo punto a la circunferencia.}

\begin{center}
Porque si \drawUnitLine{DA} > \drawUnitLine{DB}, pero haciendo \drawUnitLine{DC} $=$ \drawFromCurrentPicture[middle][pointF]{
startGlobalRotation(-lineAngle.DF);
draw byNamedPointLines(F)("");
stopGlobalRotation;
};
 \\ haz \drawUnitLine{DF} $=$ \drawUnitLine{DH,DF,FL}, y dibuja \drawUnitLine{DF,FL}.
 \\ Entonces en \drawUnitLine{DC} y \drawUnitLine{DF,FL} tenemos \drawUnitLine{DB} $=$ \drawUnitLine{DC},
 \\ y \drawUnitLine{DB} común, y también  $=$ ,
 \\ $\therefore$  $=$  (L. 1. pr. 4.);
                     \\ pero  $=$ ;
 \\ $\therefore$  $=$ , lo cual es absurdo.
                     \\ $\therefore$  no es $=$ , ni a cualquier parte
                     \\ de , $\therefore$  no es > .
 \\ Tampoco es  > , ellas son
                     \\ $\therefore$ $=$ una a la otra.

Y cualquier otra línea trazada desde el mismo punto a la circunferencia debe estar en el mismo lado con una de estas líneas, y debe ser más o menos lejana que desde la línea que pasa por el centro y, por lo tanto, no puede ser igual a ella.

Q. E. D. \byref{prop:III.VIII} \byref{\hypref}
\end{center}

\qed

\starttheorem{Prop. X. theor.}\label{prop:III.X}

\defineNewPicture[1/5]{
pair P, G, H, B, d, dd;
pair Pd, Gd, Hd, Bd, Pdd;
numeric r, t[];
path cr[], crd[];
r := 7/4u;
d := (0, -9/4r);
P := (0, 0);
cr1 := ((fullcircle scaled 7/3r xscaled 4/5) rotated 45) shifted P;
cr2 := ((fullcircle scaled 7/3r xscaled 4/5) rotated -45) shifted P;
H := (subpath (0, 2) of cr1) intersectionpoint cr2;
B := (subpath (0, -2) of cr1) intersectionpoint cr2;
G := (subpath (-2, -4) of cr1) intersectionpoint cr2;
Pd := P shifted d;
crd1 := (fullcircle scaled 2r) shifted Pd;
dd := (0, -3/2r);
crd2 := crd1 shifted dd;
Pdd := Pd shifted dd;
t1 := xpart(crd2 intersectiontimes (subpath (-2, -4) of crd1));
t2 := xpart(crd2 intersectiontimes (subpath (0, -2) of crd1));
Bd := point t1 of crd2;
Hd := point t2 of crd2;
Gd := point -2 of crd1;
crd2 := subpath (t1, t2 + 8) of crd2;
crd2 := crd2 .. (point -2 of crd1) .. cycle;
draw byLine(P, B, byyellow, SOLID_LINE, REGULAR_WIDTH);
draw byLine(P, G, byblack, SOLID_LINE, REGULAR_WIDTH);
draw byLine(P, H, byblue, SOLID_LINE, REGULAR_WIDTH);
draw byArbitraryFigure.fI(cr1, byred, 0, 0);
draw byArbitraryFigure.fII(cr2, byblue, 0, 0);
draw byLine(Pdd, Gd, byblack, SOLID_LINE, REGULAR_WIDTH);
byLineDefine(Pdd, Bd, byyellow, SOLID_LINE, REGULAR_WIDTH);
byLineDefine(Pdd, Hd, byblue, SOLID_LINE, REGULAR_WIDTH);
draw byNamedLineSeq(0)(PddBd,PddHd);
draw byArbitraryFigure.fdI(crd1, byred, 0, 0);
draw byArbitraryFigure.fdII(crd2, byblue, 0, 0);
byCircleDefineR.PI(P, r, byred, 0, 0, 0);
byCircleDefineR.PII(P, r, byblue, 0, 0, 0);
draw byLabelsOnCircle(G, B, H)(PI);
draw byLabelsOnPolygon(G, P, H)(OMIT_FIRST_LABEL+OMIT_LAST_LABEL, 0);
byPointLabelDefine(Gd, "G");
byPointLabelDefine(Hd, "H");
byPointLabelDefine(Bd, "B");
byPointLabelDefine(Dd, "D");
byPointLabelDefine(Pdd, "P");
draw byLabelLineEnd(Gd, Pdd, 0);
draw byLabelPoint(Bd, 180, 2);
draw byLabelPoint(Hd, 0, 2);
draw byLabelsOnPolygon(Hd, Pdd, Bd)(OMIT_FIRST_LABEL+OMIT_LAST_LABEL, 0);
}
\drawCurrentPictureInMargin
\problem{S}{i}{ se toma un punto dentro de un círculo \drawFromCurrentPicture{draw byNamedCircle(PII);}, desde el cual se pueden dibujar más de dos líneas rectas iguales (\drawFromCurrentPicture[middle][circlePI]{draw byNamedCircle(PI);}, \drawUnitLine{PG}, \drawUnitLine{PB}) a la circunferencia, ese punto debe ser el centro del círculo.}

\begin{center}
Porque si se supone que el punto \drawUnitLine{PH} en el que se encuentran más de dos líneas rectas iguales no es el centro, algún otro punto \drawUnitLine{PG} debe ser; unir estos dos puntos por \drawUnitLine{PB}, y prolongar en ambos sentidos a la circunferencia.

Entonces, dado que se dibujan más de dos líneas rectas iguales desde un punto que no es el centro, hasta la circunferencia, dos de ellas al menos deben estar en el mismo lado del diámetro \drawUnitLine{PH}; y dado que desde un punto \drawUnitLine{PG}, que no es el centro, se dibujan líneas rectas a la circunferencia; la más grande es \drawUnitLine{PB}, que pasa por el centro: y \drawUnitLine{PH} que está más cerca de , >  que está más lejana (L. 3. pr. 8.); pero  $=$  (hip.) que es absurdo.

Lo mismo puede demostrarse en cualquier otro punto, diferente de , que debe ser el centro del círculo.

Q. E. D. \byref{def:I.XV} \byref{prop:III.V} \byref{prop:III.VII,prop:III.VIII}
\end{center}

\qed

\starttheorem{Prop. XI. theor.}\label{prop:III.XI}

\defineNewPicture[1/2]{
pair M, N, A, D, F, G, H, K;
numeric r[];
path cr[];
r1 := 9/4u;
r2 := 2/3r1;
M := (0, 0);
N := M shifted (0, +r1-r2);
cr1 := (fullcircle scaled 2r1) shifted M;
cr2 := (fullcircle scaled 2r2) shifted N;
A := (0, r1) shifted M;
F := 1/2[M,N] shifted (dir(-20)*1/3r2);
G := 1/2[M,N] shifted (dir(-20 + 180)*1/3r2);
D := cr2 intersectionpoint (F--10[F, G]);
H := cr1 intersectionpoint (F--10[F, G]);
K := cr1 intersectionpoint (F--10[G, F]);
draw byPolygon(A,F,G)(byyellow);
draw byLine(A, G, byred, SOLID_LINE, REGULAR_WIDTH);
draw byLine(A, F, byblue, DASHED_LINE, REGULAR_WIDTH);
byLineDefine(H, D, byyellow, SOLID_LINE, REGULAR_WIDTH);
byLineDefine(D, G, byyellow, DASHED_LINE, REGULAR_WIDTH);
byLineDefine(G, F, byblack, SOLID_LINE, REGULAR_WIDTH);
byLineDefine(F, K, byblue, SOLID_LINE, REGULAR_WIDTH);
draw byNamedLineSeq(0)(HD, DG, GF, FK);
byPointLabelDefine(M, "F");
byPointLabelDefine(N, "G");
draw byCircle.M(M, A, byblack, 0, 0, 0.5);
draw byCircle.N(N, A, byblue, 0, 0, -0.5);
draw byLabelsOnPolygon(K, F, G, D)(OMIT_FIRST_LABEL+OMIT_LAST_LABEL, 0);
draw byLabelsOnPolygon(A, D, N)(OMIT_FIRST_LABEL+OMIT_LAST_LABEL, 0);
draw byLabelsOnCircle(A, H)(M);
}
\drawCurrentPictureInMargin
\problem{X}{X}{Un círculo \drawFromCurrentPicture{
startAutoLabeling;
draw byNamedPolygon(AFG);
stopAutoLabeling;
draw byNamedLineFull(A, A, 1, 1,  0, 0)(GF);
} no puede intersecar a otro  en más de dos puntos.}

\begin{center}
Si es posible, deje que se cruce en tres puntos;
                 \\ desde el centro de  dibuja ,   \\ y  a los puntos de intersección;
                 \\ $\therefore$  $=$  $=$   \\ (L. 1. def. 15.),
                 \\ pero a medida que los círculos se intersecan, no tienen el mismo centro (L. 3. pr. 5.):
                 \\ $\therefore$ el punto supuesto no es el centro de , y

$\therefore$ como ,  y  son dibujadas desde un punto y no desde el centro, no son iguales (L. 3. pr. 7, 8); pero antes se demostró que eran iguales, lo cual es absurdo; por lo tanto, los círculos no se intersecan en tres puntos.

Q. E. D. \byref{prop:I.XX}
\end{center}

\qed

\starttheorem{Prop. XII. theor.}\label{prop:III.XII}

\defineNewPicture[1/4]{
pair M, N, A, C, D, F, G;
numeric r[];
path cr[];
r1 := 3/2u;
r2 := 2u;
M := (0, 0);
N := (0, -r1-r2);
cr1 := (fullcircle scaled 2r1) shifted M;
cr2 := (fullcircle scaled 2r2) shifted N;
A := M shifted (0, -r1);
F := M shifted (dir(185)*1/2r1);
G := N shifted (dir(175)*1/2r2);
C := cr1 intersectionpoint (F--G);
D := cr2 intersectionpoint (F--G);
byLineDefine(F, C, byred, SOLID_LINE, REGULAR_WIDTH);
byLineDefine(C, D, byblack, SOLID_LINE, REGULAR_WIDTH);
byLineDefine(D, G, byblue, SOLID_LINE, REGULAR_WIDTH);
byLineDefine(A, F, byyellow, DASHED_LINE, REGULAR_WIDTH);
byLineDefine(A, G, byyellow, SOLID_LINE, REGULAR_WIDTH);
draw byNamedLineSeq(0)(FC,CD,DG,AG,AF);
draw byCircle.M(M, A, byblue, 0, 0, -1/2);
draw byCircle.N(N, A, byred, 0, 0, -1/2);
byPointLabelDefine(M, "F");
byPointLabelDefine(N, "G");
draw byLabelsOnPolygon(C, F, A)(OMIT_FIRST_LABEL+OMIT_LAST_LABEL, 0);
draw byLabelsOnPolygon(A, G, D)(OMIT_FIRST_LABEL+OMIT_LAST_LABEL, 0);
draw byLabelsOnPolygon(A, C, F)(OMIT_FIRST_LABEL+OMIT_LAST_LABEL, 0);
draw byLabelsOnPolygon(G, D, D)(OMIT_FIRST_LABEL+OMIT_LAST_LABEL, 0);
draw byLabelPoint(A, angle(F-A)-45, 2);
}
\drawCurrentPictureInMargin
\problem{S}{i}{ dos círculos \drawUnitLine{FC,CD,DG} y \drawUnitLine{FC,CD,DG} se tocan internamente, la línea recta que une sus centros, al ser prolongada, pasará por un punto de contacto.}

\begin{center}
Porque si es posible, deja \drawUnitLine{AF} unir sus centros y prolonga en ambos sentidos; desde un punto de contacto dibuja \drawUnitLine{AG} al centro de \drawUnitLine{AF}, y desde el mismo punto de contacto dibuja \drawUnitLine{AG} al centro de \drawUnitLine{FC,CD,DG}.

Porque en \drawUnitLine{FC}; \drawUnitLine{AF} + \drawUnitLine{DG} > \drawUnitLine{AG}  \\ (L. 1. pr. 20.),
                 \\ y \drawUnitLine{FC} $=$ \drawUnitLine{DG} como el radio de
                 \\ \drawUnitLine{FC,CD,DG},
 \\ pero  +  > ; quita
                 \\  que es común,
                 \\ y  > ;
 \\ pero  $=$ ,
 \\ porque son radios de ,
 \\ y $\therefore$  >  una parte mayor que el todo, lo cual es absurdo.

Por lo tanto, los centros no están tan ubicados, de modo que una línea que los une puede pasar por cualquier punto que no sea un punto de contacto.

Q. E. D. \byref{prop:I.XX} \byref{def:I.XV} \byref{def:I.XV}
\end{center}

\qed

\starttheorem{Prop. XIII. theor.}\label{prop:III.XIII}

\defineNewPicture{
pair M, N, F, G, H, B, D;
numeric r[];
path cr[];
r1 := 7/4u;
r2 := 3/4r1;
M := (0, 0);
N := (dir(120)*(r1-r2)) shifted M;
cr1 := (fullcircle scaled 2r1) shifted M;
cr2 := (fullcircle scaled 2r2) shifted N;
t1 := xpart(cr1 intersectiontimes (M--10[M, N]));
t2 := xpart(cr2 intersectiontimes (M--10[M, N]));
cr2 := (subpath (t2 + 2/3, t2 - 2/3 + 8) of cr2) .. tension 3/2 .. cycle;
B := point (t1 - 1/2) of cr1;
D := point (t1 + 1/2) of cr1;
G := 3/4[B, 1/2[M, N]];
H := 5/4[B, 1/2[M, N]];
draw byLine(D, G, byblack, SOLID_LINE, REGULAR_WIDTH);
byLineDefine(D, H, byred, SOLID_LINE, REGULAR_WIDTH);
byLineDefine(B, G, byblue, DASHED_LINE, REGULAR_WIDTH);
byLineDefine(G, H, byblue, SOLID_LINE, REGULAR_WIDTH);
draw byNamedLineSeq(0)(BG,GH,DH);
draw byCircleR(M, r1, byyellow, 0, 0, 0.5);
draw byArbitraryFigure.fI(cr2, byblue, 0, 0);
byCircleDefineR(M, r1, byyellow, 0, 0, 0);
byCircleDefineR(N, r2, byblue, 0, 0, 0);
byPointLabelDefine(M, "H");
byPointLabelDefine(N, "G");
draw byLabelsOnCircle(D, B)(M);
draw byLabelsOnPolygon(B, G, H, D)(OMIT_FIRST_LABEL+OMIT_LAST_LABEL, 0);
}
\drawCurrentPictureInMargin
\problem{S}{i}{ dos círculos \drawUnitLine{GH} y \drawUnitLine{DH} se tocan externamente, la línea recta \drawUnitLine{DG} que une sus centros pasa a través del punto de contacto.}

\begin{center}
Si es posible, deja \drawUnitLine{BG} unir sus centros, y que no pase por un punto de contacto; luego desde un punto de contacto dibuje \drawUnitLine{DG} y \drawUnitLine{GH} a los centros.

Porque \drawUnitLine{GH,BG} + \drawUnitLine{GH} > \drawUnitLine{DG} (L. 1. pr. 20.),
                 \\ y \drawUnitLine{GH,BG} $=$ \drawUnitLine{DH} (L. 1. def. 15.),
                 \\ y \drawUnitLine{GH} $=$ \drawUnitLine{DG} (L. 1. def. 15.),

$\therefore$ \drawUnitLine{DH} + \drawUnitLine{GH} > \drawUnitLine{DG}, una parte mayor que el todo, lo cual es absurdo.

Por lo tanto, los centros no están tan ubicados, de modo que la línea que los une puede pasar por cualquier punto que no sea el punto de contacto.

Q. E. D. \drawUnitLine{DH} \drawUnitLine{HA,AG} \drawUnitLine{CH} \drawUnitLine{CG} \drawUnitLine{CH} \drawUnitLine{HA} \drawUnitLine{AG} \drawUnitLine{CG} \drawUnitLine{CG} \drawUnitLine{CH} \drawUnitLine{HA,AG} \drawUnitLine{CG} \drawUnitLine{CH} \drawUnitLine{HA,AG} \byref{prop:III.XI} \byref{def:I.XV} \byref{def:I.XV} \byref{prop:I.XX} \byref{def:I.XV} \byref{def:I.XV} \byref{prop:I.XX}
\end{center}

\qed

\starttheorem{Prop. XIV. theor.}\label{prop:III.XIV}

\defineNewPicture{
pair A, B, C, D, E, F, G;
numeric r;
r := 9/4u;
E := (0, 0);
A := (dir(90-20)*r) shifted E;
B := (dir(90-130)*r) shifted E;
C := (dir(90+20)*r) shifted E;
D := (dir(90+130)*r) shifted E;
F = whatever[A, B] = whatever[E, E shifted ((A-B) rotated 90)];
G = whatever[C, D] = whatever[E, E shifted ((C-D) rotated 90)];
byAngleDefine(E, F, A, byyellow, SOLID_SECTOR);
byAngleDefine(E, G, C, byblack, ARC_SECTOR);
draw byNamedAngleResized();
draw byLine(E, A, byblack, SOLID_LINE, REGULAR_WIDTH);
draw byLine(E, C, byblue, SOLID_LINE, REGULAR_WIDTH);
byLineDefine(E, F, byblack, DASHED_LINE, REGULAR_WIDTH);
byLineDefine(E, G, byblue, DASHED_LINE, REGULAR_WIDTH);
draw byNamedLineSeq(0)(EF,EG);
byLineDefine(A, F, byred, SOLID_LINE, REGULAR_WIDTH);
byLineDefine(F, B, byred, DASHED_LINE, REGULAR_WIDTH);
draw byNamedLineSeq(0)(AF, FB);
byLineDefine(C, G, byyellow, SOLID_LINE, REGULAR_WIDTH);
byLineDefine(G, D, byyellow, DASHED_LINE, REGULAR_WIDTH);
draw byNamedLineSeq(0)(CG, GD);
draw byCircleR(E, r, byblue, 0, 0, 0);
draw byLabelsOnCircle(A, B, C, D)(E);
draw byLabelsOnPolygon(F, E, G)(OMIT_FIRST_LABEL+OMIT_LAST_LABEL, 0);
draw byLabelLineEnd(G, E, 0);
draw byLabelLineEnd(F, E, 0);
}
\drawCurrentPictureInMargin
\problem{U}{n}{ círculo no puede tocar a otro, ya sea externa o internamente en más puntos que uno.}

\begin{center}
\emph{Fig.} 1. Porque si es posible, deja \drawAngle{F} y  tocarse internamente en dos puntos; dibuja  uniendo sus centros, y prolonga hasta que pase por uno de los puntos de contacto (L. 3. pr. 11.);

dibuje  y ,
 \\ Pero  $=$  (L. 1. def. 15.),
                     \\ $\therefore$ si  es agregada a ambas
                     \\  $=$  + ;
 \\ pero  $=$  (L. 1. def. 15.),
                     \\ y $\therefore$  +  $=$ ; pero
                     \\  +  >  (L. 1. pr. 20.),
                     \\ lo que es absurdo.

\emph{Fig.} 2. Pero si los puntos de contacto son las extremidades de la línea recta que une los centros, esta línea recta debe ser bisecada en dos puntos diferentes para los dos centros; porque es el diámetro de ambos círculos, lo que es absurdo.

\emph{Fig.} 3. Siguiente, porque si es posible, deja  y  tocarse externamente en dos puntos; dibuja  uniendo los centros de los círculos, y pasando por uno de los puntos de contacto, y dibuja  y .

 $=$  (L. 1. def. 15.);
                     \\ y  $=$  (L. 1. def. 15.):
                     \\ $\therefore$  +  $=$ ; pero
                     \\  +  >  (L. 1. pr. 20.),
                     \\ lo que es absurdo.

Por lo tanto, no hay ningún caso en el que dos círculos puedan tocarse entre sí en dos puntos.

Q. E. D. \byref{prop:III.III} \byref{prop:III.III} \byref{\hypref} \byref{def:I.XV} \byref{prop:I.XLVII}
\end{center}

\qed

\starttheorem{Prop. XV. theor.}\label{prop:III.XV}

\defineNewPicture[2/3]{
pair A, D, E, F, G, M, N;
numeric r;
r := 9/4u;
E := (0, 0);
A := (r, 0) shifted E;
D := (-r, 0) shifted E;
F := (dir(90-30)*r) shifted E;
G := (dir(90+30)*r) shifted E;
M := (dir(90-60)*r) shifted E;
N := (dir(90+60)*r) shifted E;
byAngleDefine(N, E, G, byred, SOLID_SECTOR);
byAngleDefine(G, E, F, byyellow, SOLID_SECTOR);
byAngleDefine(F, E, M, byred, SOLID_SECTOR);
draw byNamedAngleResized();
draw byLine(E, M, byyellow, DASHED_LINE, REGULAR_WIDTH);
draw byLine(E, N, byyellow, SOLID_LINE, REGULAR_WIDTH);
draw byLine(E, F, byblue, DASHED_LINE, REGULAR_WIDTH);
draw byLine(E, G, byblack, DASHED_LINE, REGULAR_WIDTH);
draw byLine(D, E, byred, SOLID_LINE, REGULAR_WIDTH);
draw byLine(E, A, byblack, SOLID_LINE, REGULAR_WIDTH);
draw byLine(F, G, byred, DASHED_LINE, REGULAR_WIDTH);
draw byLine(M, N, byblue, SOLID_LINE, REGULAR_WIDTH);
draw byCircleR(E, r, byblack, 0, 0, 0);
draw byLabelsOnCircle(A, D, M, N, F, G)(E);
draw byLabelsOnPolygon(A, E, D)(OMIT_FIRST_LABEL+OMIT_LAST_LABEL, 0);
}
\drawCurrentPictureInMargin
\problem{X}{X}{
\drawUnitLine{DE,EA}
\drawUnitLine{MN}}

\begin{center}


                inscritas en un círculo están igualmente distantes del centro; y también, las líneas rectas igualmente distantes del centro son iguales.

Desde el centro de \drawUnitLine{EN} dibuja
                     \\ \drawUnitLine{EM} ⊥ a \drawUnitLine{EM} y \drawUnitLine{EA}  \\ ⊥ \drawUnitLine{EN}, une \drawUnitLine{DE} y \drawUnitLine{EN}.

Entonces \drawUnitLine{EM} $=$ la mitad de \drawUnitLine{DE,EA} (L. 3. pr. 3.)
                     \\ y \drawUnitLine{EN} $=$

1
/
2

\drawUnitLine{EM} (L. 3. pr. 3.)
                     \\ ya que \drawUnitLine{MN} $=$ \drawUnitLine{DE,EA} (hip.)
                     \\ $\therefore$ \drawUnitLine{MN} $=$ \drawUnitLine{MN},
 \\ y \drawUnitLine{FG} $=$ \drawUnitLine{EN} (L. 1. def. 15.)
                     \\ $\therefore$ \drawUnitLine{EM}2 $=$ \drawUnitLine{EG}2;
                     \\ pero puesto que \drawUnitLine{EF} es un ángulo recto
                     \\ \drawFromCurrentPicture{
draw byNamedAngle(NEG,GEF,FEM);
startAutoLabeling;
draw byNamedLineSeq(0)(MN,EM,EN);
stopAutoLabeling;
}2 $=$ \drawFromCurrentPicture{
draw byNamedAngle(GEF);
startAutoLabeling;
draw byNamedLineSeq(0)(FG,EF,EG);
stopAutoLabeling;
}2 + \drawUnitLine{EN}2 (L. 1. pr. 47.)
                     \\ y \drawUnitLine{EM}2 $=$ \drawUnitLine{EG}2 + \drawUnitLine{EF}2 por la misma razon,
                     \\ $\therefore$ \drawAngle{NEG,GEF,FEM}2 + \drawAngle{GEF}2 $=$ \drawUnitLine{MN}2 + \drawUnitLine{EF}2
 \\ $\therefore$ \drawUnitLine{BC}2 $=$ \drawUnitLine{FG}2,
                     \\ $\therefore$ \drawUnitLine{EL,LK} $=$ \drawUnitLine{EH}.

También si las líneas \drawUnitLine{FG} y \drawUnitLine{BC} están igualmente distante del centro; es decir, si las perpendiculares \drawUnitLine{EH} y \drawUnitLine{EL} están dadas igual entonces \drawUnitLine{MN} $=$ \drawUnitLine{EL,LK}.

Para, como en el caso anterior,
                     \\ \drawUnitLine{BC}2 + \drawUnitLine{MN}2 $=$ \drawUnitLine{BC}2 + \drawUnitLine{MN}2;
                     \\ pero \drawUnitLine{MN}2 $=$ \drawUnitLine{FG}2:

$\therefore$ \drawUnitLine{BC}2 $=$ \drawUnitLine{FG}2, los dobles de estos
                     \\  y  son también iguales.

Q. E. D. \byref{prop:I.XX} \byref{prop:I.XXIV} \byref{prop:III.XIV} \byref{prop:III.XV}
\end{center}

\qed

\starttheorem{Prop. XVI. theor.}\label{prop:III.XVI}

\defineNewPicture[1/5]{
pair A, B, C, D, E, F, G, H, K;
numeric r;
r :=2u;
D := (0, 0);
A := (0, -r);
B := (0, r);
C := (dir(190)*r) shifted D;
E := (4/3r, -r);
F := (4/3r, -1/3r);
G := 11/12[A, F];
H := (dir(angle(G-D))*r) shifted D;
K := (-r, -r);
byAngleDefine(C, A, D, byyellow, SOLID_SECTOR);
byAngleDefine(D, A, G, byblue, SOLID_SECTOR);
byAngleDefine(G, A, E, byred, SOLID_SECTOR);
byAngleDefine(D, C, A, byblack, SOLID_SECTOR);
byAngleDefine(A, G, D, byblack, ARC_SECTOR);
draw byNamedAngleResized();
draw byLine(A, C, byred, SOLID_LINE, REGULAR_WIDTH);
draw byLine(D, C, byblue, SOLID_LINE, REGULAR_WIDTH);
draw byLine(D, H, byblue, DASHED_LINE, REGULAR_WIDTH);
draw byLine(H, G, byblack, DASHED_LINE, REGULAR_WIDTH);
draw byLine(A, G, byred, DASHED_LINE, REGULAR_WIDTH);
draw byLine(G, F, byblack, DASHED_LINE, REGULAR_WIDTH);
draw byLine(B, D, byyellow, DASHED_LINE, REGULAR_WIDTH);
draw byLine(D, A, byblack, SOLID_LINE, REGULAR_WIDTH);
draw byLineFull(E, K, byyellow, 0, 0)(E, K, 0, 0, -1);
draw byCircle.D(D, A, byblue, 0, 0, 0);
draw byLabelsOnCircle(B, C)(D);
draw byLabelsOnPolygon(E, A, K, noPoint)(ALL_LABELS, 1);
draw byLabelsOnPolygon(F, G, A)(OMIT_FIRST_LABEL+OMIT_LAST_LABEL, 0);
draw byLabelsOnPolygon(C, D, B)(OMIT_FIRST_LABEL+OMIT_LAST_LABEL, 0);
draw byLabelPoint(H, angle(G-H)+45, 2);
}
\drawCurrentPictureInMargin
\problem{E}{l}{ diámetro es la línea recta mayor en un círculo: y, de todas los demás, la que está más cerca del centro es mayor que la más lejana.}

\begin{center}
El diametro \drawUnitLine{EK} es > cualquier línea \drawUnitLine{BD,DA}.
 \\ Dibuja \drawUnitLine{AG} y \drawUnitLine{AC}.
 \\ Luego \drawUnitLine{DA} $=$ \drawUnitLine{DC}  \\ y \drawUnitLine{DA} $=$ \drawUnitLine{DC},
 \\ $\therefore$ \drawAngle{CAD} + \drawAngle{C} $=$ \drawAngle{CAD}  \\ pero \drawUnitLine{AC} + \drawUnitLine{DA} > \drawUnitLine{EK} (L. 1. pr. 20.)
                         \\ $\therefore$ \drawUnitLine{DA} > \drawUnitLine{AG}.

De nuevo, la línea que está más cerca del centro es mayor que la más lejana.

Primero, deje que las líneas dadas sean \drawFromCurrentPicture{draw byNamedPointLines(G,"GF");} y \drawUnitLine{EK}, que están en el mismo lado del centro y no se cruzan;
                        
                            dibuja
                            


\drawAngle{DAG,GAE},
\drawAngle{DAG},
\drawUnitLine{DH,HG},
\drawUnitLine{AG}.




                        En \drawAngle{DAG} y \drawAngle{G},
 \\ \drawAngle{DAG} y \drawUnitLine{DA} $=$ \drawUnitLine{DH,HG} y \drawUnitLine{DH};
 \\ \drawUnitLine{DA} > \drawUnitLine{DH},
 \\ $\therefore$ \drawUnitLine{DH,HG} > \drawUnitLine{AG} (L. 1. pr. 24.)

Deja que las líneas dadas sean  y  y que estén en diferentes lados del centro o que se crucen; desde el centro dibuja  y  ⊥  y , haz  $=$ , y dibuja  ⊥ .

Ya que  y  están igualmente distantes
                     \\ del centro,  $=$  (L. 3. pr. 14.);
                     \\ pero  >  (Parte 1. L. 3. pr. 15.),
                     \\ $\therefore$  > .

Q. E. D. \byref{prop:I.V} \byref{prop:I.XVII} \byref{\hypref}
\end{center}

\qed

\startproblem{Prop. XVII. prob.}\label{prop:III.XVII}

\defineNewPicture[1/2]{
pair A, B, D, E, F;
numeric r[], a;
path cr[];
r1 := 6/4u;
r2 := 9/4u;
E := (0, 0);
cr1 := (fullcircle scaled 2r1) shifted E;
cr2 := (fullcircle scaled 2r2) shifted E;
A := (dir(50)*r2) shifted E;
D := (dir(50)*r1) shifted E;
F := cr2 intersectionpoint (D--D shifted (dir(angle(A-E) - 90)*r2));
B := cr1 intersectionpoint (E--F);
a := angle(B-E);
forsuffixes i=A, B, D, F:
i := ((i shifted -E) rotated -a) shifted E;
endfor;
byAngleDefine(A, B, E, byyellow, SOLID_SECTOR);
byAngleDefine(F, D, E, byyellow, SOLID_SECTOR);
byAngleDefine(F, E, A, byblue, SOLID_SECTOR);
draw byNamedAngleResized();
draw byLine(A, B, byblue, SOLID_LINE, REGULAR_WIDTH);
draw byLine(F, D, byblue, DASHED_LINE, REGULAR_WIDTH);
byLineDefine(A, D, byred, SOLID_LINE, REGULAR_WIDTH);
byLineDefine(D, E, byred, DASHED_LINE, REGULAR_WIDTH);
byLineDefine(F, B, byblack, SOLID_LINE, REGULAR_WIDTH);
byLineDefine(B, E, byblack, DASHED_LINE, REGULAR_WIDTH);
draw byNamedLineSeq(0)(AD,DE,BE,FB);
draw byCircleR.EI(E, r1, byred, 0, 0, -1);
draw byCircleR.EII(E, r2, byyellow, 0, 0, 0);
draw byLabelsOnCircle(A, F)(EII);
draw byLabelsOnPolygon(F, E, A)(OMIT_FIRST_LABEL+OMIT_LAST_LABEL, 0);
draw byLabelPoint(D, angle(A-E)+45, 2);
draw byLabelPoint(B, angle(F-E)-45, 2);
}
\drawCurrentPictureInMargin
\problem{Y}{}{ si alguna línea recta \drawUnitLine{DE,AD} es dibujada desde un punto en esa perpendicular al punto de contacto, esta corta el círculo.}

\begin{center}
Si es posible, deja que \drawUnitLine{FD}, se encuentre nuevamente con el círculo, ser ⊥ \drawUnitLine{DE}, y dibuja \drawUnitLine{DE,AD}.

Luego, porque \drawUnitLine{BE,FB} $=$ \drawUnitLine{FD},
 \\ \drawUnitLine{AB} $=$ \drawUnitLine{BE,FB} (L. 1. pr. 5.),
                     \\ y $\therefore$ cada uno de estos ángulos es agudo (L. 1. pr. 17.)
                     \\ pero \drawUnitLine{AB} $=$ 
Right angle

 (hip.), lo que es absurdo, por lo tanto
                     \\ \drawLine[bottom]{FD,FB,BE,DE} dibujada ⊥ \drawLine[bottom]{BE,DE,AD,AB} no vuelve a encontrarse
                     \\ con el círculo.

Deja \drawUnitLine{AD,DE} ser ⊥ \drawUnitLine{FB,BE} y deja \drawAngle{E} ser dibujada desde un punto \drawUnitLine{DE} entre \drawUnitLine{BE} y el círculo, que sí es posible, no corta el círculo.

Porque \drawAngle{B} $=$ 
Right angle

,
 \\ $\therefore$ \drawAngle{D} es un ángulo agudo; supone
                     \\ \drawUnitLine{FD} ⊥ , dibujada desde el centro del círculo, debe caer al lado de  el ángulo agudo.
                     \\ $\therefore$  que se supone que es un ángulo recto, es > ,
 \\ $\therefore$  > ;
 \\ pero  $=$ ,

y $\therefore$  > , una parte mayor que el todo, lo cual es absurdo. Por lo tanto, el punto no cae fuera del círculo y, por lo tanto, la línea recta  corta el círculo.

Q. E. D. \byref{prop:III.XVI} \byref{prop:I.IV}
\end{center}

\qed

\starttheorem{Prop. XVIII. theor.}\label{prop:III.XVIII}

\defineNewPicture{
pair B, C, D, F, G;
numeric r;
r := 7/4u;
F := (0, 0);
C := (r, 0);
G := (r, 6/5r);
D := 6/5[C, G];
B := (dir(angle(G-F))*r) shifted F;
draw byCircle.F(F, C, byyellow, 0, 0, 0);
byAngleDefine(F, C, G, byred, SOLID_SECTOR);
byAngleDefine(C, G, F, byyellow, SOLID_SECTOR);
draw byNamedAngleResized();
byLineDefine(F, B, byred, SOLID_LINE, REGULAR_WIDTH);
byLineDefine(B, G, byred, DASHED_LINE, REGULAR_WIDTH);
byLineDefine(C, D, byblue, DASHED_LINE, REGULAR_WIDTH);
byLineDefine(F, C, byblue, SOLID_LINE, REGULAR_WIDTH);
draw byNamedLineSeq(0)(BG,FB,FC,CD);
draw byLabelsOnCircle(C)(F);
draw byLabelsOnPolygon(C, F, G)(OMIT_FIRST_LABEL+OMIT_LAST_LABEL, 0);
draw byLabelPoint(B, angle(G-F)+45, 2);
draw byLabelPoint(G, angle(D-C)-90, 1);
draw byLabelPoint(D, angle(D-C)-90, 1);
}
\drawCurrentPictureInMargin
\problem{D}{ibujar}{ una tangente a un círculo dado \drawUnitLine{CD} desde un punto dado, ya sea dentro o fuera de su circunferencia.}

\begin{center}
Si el punto dado está en la circunferencia, como en \drawUnitLine{FC}, es claro que la línea recta \drawUnitLine{FB,BG} ⊥ \drawUnitLine{CD} el radio, será la tangente requerida (L. 3. pr. 16.)

Pero si el punto dado \drawAngle{G} está fuera de la circunferencia, dibuja \drawAngle{C}  \\ de ella al centro, cortando \drawUnitLine{FC}; y
                     \\ dibuja \drawUnitLine{FB,BG} ⊥ \drawUnitLine{FC}, traza \drawUnitLine{FB}  \\ concéntrico con \drawUnitLine{FB} radio $=$ \drawUnitLine{FB,BG},
 \\ entonces \drawUnitLine{FB,BG} será la tangente requerida.

En \drawUnitLine{CD} y \drawUnitLine{FC}  \\ \drawUnitLine{CD} $=$ ,  común,
                     \\ y  $=$ ,
 \\ $\therefore$ (L. 1. pr. 4.)  $=$  $=$ un ángulo recto,
                     \\ $\therefore$  es tangente a .

Q. E. D. \byref{prop:I.XVII} \byref{prop:I.XIX}
\end{center}

\qed

\starttheorem{Prop. XIX. theor.}\label{prop:III.XIX}

\defineNewPicture{
pair A, C, E, F, G;
numeric r;
r := 7/4u;
G := (0, 0);
A := (-r, 0) shifted G;
C := (r, 0) shifted G;
E := (r, 6/5r) shifted G;
F := (-1/7r, 1/2r) shifted G;
byAngleDefine(A, C, F, byblue, SOLID_SECTOR);
byAngleDefine(F, C, E, byyellow, SOLID_SECTOR);
draw byNamedAngleResized();
draw byLine(A, C, byyellow, SOLID_LINE, REGULAR_WIDTH);
draw byLine(C, E, byblue, SOLID_LINE, REGULAR_WIDTH);
draw byLine(C, F, byred, DASHED_LINE, REGULAR_WIDTH);
draw byCircleR(G, r, byred, 0, 0, 0);
draw byLabelsOnCircle(A, C)(G);
draw byLabelLineEnd(F, C, 0);
draw byLabelLineEnd(E, C, 0);
}
\drawCurrentPictureInMargin
\problem{S}{i}{ una línea recta \drawUnitLine{CE} es tangente a un círculo, la línea recta \drawUnitLine{AC} dibujada desde el centro hasta el punto de contacto es perpendicular a ella.}

\begin{center}
Por si es posible, deja \drawUnitLine{AC} ser ⊥ \drawUnitLine{CF},
 \\ entonces porque \drawUnitLine{CF} $=$ 
Right angle

, \drawUnitLine{CE} es agudo (L. 1. pr. 17.)
                 \\ $\therefore$ \drawAngle{FCE} > \drawAngle{ACF,FCE} (L. 1. pr. 19.);
                 \\ pero \drawAngle{FCE} $=$ \drawAngle{ACF,FCE},
 \\ y $\therefore$ \drawUnitLine{AC} > ,
 \\ una parte mayor que el todo, lo cual es absurdo.

$\therefore$  no es ⊥ ; y de la misma manera se puede demostrar que ninguna otra línea excepto  es perpendicular a .

Q. E. D. \byref{prop:III.XVIII} \byref{\hypref}
\end{center}

\qed

\starttheorem{Prop. XX. theor.}\label{prop:III.XX}

\defineNewPicture{
pair A, E, F, C;
r := 7/4u;
E := (0, 0);
A := (dir(80)*r) shifted E;
F := (dir(80 + 180)*r) shifted E;
C := (dir(-30)*r) shifted E;
byAngleDefine(C, A, F, byyellow, SOLID_SECTOR);
byAngleDefine(C, E, F, byblue, SOLID_SECTOR);
byAngleDefine(E, C, A, byred, SOLID_SECTOR);
draw byNamedAngleResized();
draw byLine(A, C, byblack, SOLID_LINE, THIN_WIDTH);
draw byLine(E, C, byblack, SOLID_LINE, REGULAR_WIDTH);
byLineDefine(E, F, byred, DASHED_LINE, REGULAR_WIDTH);
byLineDefine(E, A, byred, SOLID_LINE, REGULAR_WIDTH);
draw byNamedLineSeq(0)(EF,EA);
draw byCircleR(E, r, byblue, 0, 0, 0);
draw byLabelsOnCircle(F, A, C)(E);
draw byLabelsOnPolygon(F, E, A)(OMIT_FIRST_LABEL+OMIT_LAST_LABEL, 0);
}
\drawCurrentPictureInMargin
\problem{S}{i}{ una línea recta \drawUnitLine{EF,EA} es una tangente a un círculo, la línea recta \drawAngle{CAF}, dibujada perpendicular a ella desde un punto del contacto, pasa a través del centro del círculo.}

\begin{center}
Por, si es posible, deja el centro sin \drawUnitLine{EC}, y dibuja \drawUnitLine{EA} del supuesto centro al punto de contacto.

Porque \drawAngle{CAF} ⊥ \drawAngle{C} (L. 3. pr. 18.)

$\therefore$ \drawAngle{CEF} $=$ 
Right angle

, un ángulo recto;
                     \\ pero \drawAngle{CAF} $=$ 
Right angle

 (hip.), y $\therefore$ \drawAngle{C} $=$ \drawAngle{CEF},
 \\ una parte igual al todo, lo cual es absurdo.

Por lo tanto, el punto supuesto no es el centro; y de la misma manera se puede demostrar que ningún otro punto sin \drawAngle{CAF} es el centro.

Q. E. D. \drawAngle{BAF,CAF} \drawUnitLine{AF} \drawAngle{B} \drawAngle{BAF} \drawAngle{C} \drawAngle{CAF} \drawAngle{BAF} \drawAngle{B} \drawAngle{CAF} \drawAngle{C} \drawAngle{BAF,CAF} \drawAngle{BEF} \drawAngle{BAF} \drawAngle{B} \drawAngle{CEF} \drawAngle{CAF} \drawAngle{C} \drawAngle{BEF,CEF} \drawAngle{BAF,CAF} \drawAngle{FDC} \drawUnitLine{DG} \drawAngle{GEF,FEC} \drawAngle{GDF,FDC} \drawAngle{GEF} \drawAngle{GDF} \drawAngle{FEC} \drawAngle{FDC} \byref{prop:I.V} \byref{prop:I.XXXII} \byref{prop:I.V}
\end{center}

\qed

\starttheorem{Prop. XXI. theor.}\label{prop:III.XXI}

\defineNewPicture{
pair A, B, D, E, F;
numeric r;
r := 7/4u;
F := (0, 0);
B := (dir(-90 - 50)*r) shifted F;
D := (dir(-90 + 50)*r) shifted F;
A := (dir(90 + 25)*r) shifted F;
E := (dir(90 - 35)*r) shifted F;
byAngleDefine(B, A, D, byred, SOLID_SECTOR);
byAngleDefine(B, E, D, byblue, SOLID_SECTOR);
byAngleDefine(B, F, D, byyellow, SOLID_SECTOR);
draw byNamedAngleResized();
byLineDefine(B, F, byblue, SOLID_LINE, REGULAR_WIDTH);
byLineDefine(D, F, byred, SOLID_LINE, REGULAR_WIDTH);
draw byNamedLineSeq(0)(BF,DF);
draw byLine(B, D, byblack, DASHED_LINE, REGULAR_WIDTH);
draw byLine(B, A, byblack, SOLID_LINE, THIN_WIDTH);
draw byLine(B, E, byblack, SOLID_LINE, THIN_WIDTH);
draw byLine(D, A, byblack, SOLID_LINE, THIN_WIDTH);
draw byLine(D, E, byblack, SOLID_LINE, THIN_WIDTH);
draw byCircleR(F, r, byblue, 0, 0, 0);
draw byLabelsOnCircle(B, D, E, A)(F);
draw byLabelsOnPolygon(B, F, D)(OMIT_FIRST_LABEL+OMIT_LAST_LABEL, 0);
}
\drawCurrentPictureInMargin
\problem{E}{l}{ ángulo en el centro de un círculo, es el doble del ángulo en la circunferencia, cuando tienen la misma parte de la circunferencia como su base.}

\begin{center}
Deja el centro del círculo estar en \drawUnitLine{DF}  \\ un lado de \drawUnitLine{BF}.

Porque \drawAngle{F} $=$ \drawAngle{A},
 \\ \drawAngle{E} $=$ \drawAngle{A} (L. 1. pr. 5.).

Pero \drawAngle{E} $=$ \drawUnitLine{GA} + \drawUnitLine{GE},
 \\ \drawAngle{BAG} $=$ dos veces \drawAngle{BEG} (L. 1. pr. 32).

Deja que el centro esté dentro de \drawAngle{GAD}, el ángulo en la circufenrencia; dibuja \drawAngle{GED} desde el punto angular a través del centro del círculo;
                         \\ entonces \drawAngle{BAG,GAD} $=$ \drawAngle{BEG,GED}, y  $=$ ,
 \\ por la igualdad de los lados (L. 1. pr. 5).

Por lo tanto  +  +  +  $=$ dos veces .
 \\ Pero  $=$  + , y
                         \\  $=$  + ,
 \\ $\therefore$  $=$ dos veces .

Deja el centro ser sin  y
                         \\ dibuja , el diámetro.

Porque  $=$ dos veces ; y
                         \\  $=$ dos veces  (caso 1.);
                         \\ $\therefore$  $=$ dos veces .

Q. E. D. \byref{prop:III.XX}
\end{center}

\qed

\starttheorem{Prop. XXII. theor.}\label{prop:III.XXII}

\defineNewPicture{
pair A, B, C, D, E;
numeric r;
r := 7/4u;
E := (0, 0);
A := (dir(80)*r) shifted E;
B := (dir(10)*r) shifted E;
C := (dir(-100)*r) shifted E;
D := (dir(150)*r) shifted E;
byAngleDefine(D, A, C, byred, SOLID_SECTOR);
byAngleDefine(C, A, B, byblue, SOLID_SECTOR);
byAngleDefine(A, B, D, byyellow, SOLID_SECTOR);
byAngleDefine(D, B, C, byred, SOLID_SECTOR);
byAngleDefine(B, C, A, byblack, SOLID_SECTOR);
byAngleDefine(A, C, D, byyellow, SOLID_SECTOR);
byAngleDefine(C, D, B, byblue, SOLID_SECTOR);
byAngleDefine(B, D, A, byblack, SOLID_SECTOR);
draw byNamedAngleResized();
draw byLine(A, B, byblack, SOLID_LINE, THIN_WIDTH);
draw byLine(B, C, byblack, SOLID_LINE, THIN_WIDTH);
draw byLine(C, D, byblack, SOLID_LINE, THIN_WIDTH);
draw byLine(D, A, byblack, SOLID_LINE, THIN_WIDTH);
draw byLine(A, C, byred, SOLID_LINE, REGULAR_WIDTH);
draw byLine(B, D, byblack, SOLID_LINE, REGULAR_WIDTH);
draw byCircleR(E, r, byred, 0, 0, 1/2);
draw byLabelsOnCircle(A, B, C, D)(E);
}
\drawCurrentPictureInMargin
\problem{D}{eja}{ que el segmento sea mayor que un semi círculo, y dibuja \drawAngle{CDB,BDA} y \drawAngle{ABD,DBC} al centro.}

\begin{center}
\drawUnitLine{AC} $=$ dos veces \drawUnitLine{BD} o dos veces $=$ \drawAngle{CDB} (L. 3. pr. 20.);
                     \\ $\therefore$ \drawAngle{CAB} $=$ \drawAngle{DAC}.

Deja que el segmento sea un semicírculo, o menos que un semicírculo, dibuja \drawAngle{DBC} el diámetro, también dibuja \drawAngle{BCA,ACD}.

\drawAngle{DAC,CAB} $=$ \drawAngle{BCA,ACD} y \drawAngle{BCA,ACD} $=$ \drawAngle{CDB} (caso 1.)
                     \\ $\therefore$ \drawAngle{DBC} $=$ \drawAngle{CDB,BDA}.

Q. E. D. \drawAngle{ABD,DBC} \drawTwoRightAngles \byref{prop:I.XXXII}
\end{center}

\qed

\starttheorem{Prop. XXIII. theor.}\label{prop:III.XXIII}

\defineNewPicture{
pair A, B, C, D, M, N;
numeric r[], t[];
path cr[];
r1 := 14/6u;
r2 := 15/6u;
M := (0, 0);
N := (0, 1/3r1);
cr1 := (fullcircle scaled 2r1) shifted M;
cr2 := (fullcircle scaled 2r2) shifted N;
t1 := xpart(cr1 intersectiontimes (subpath (-2, 2) of cr2));
t2 := xpart(cr1 intersectiontimes (subpath (2, 6) of cr2));
t3 := xpart(cr2 intersectiontimes (subpath (-2, 2) of cr1));
t4 := xpart(cr2 intersectiontimes (subpath (2, 6) of cr1));
A := point t1 of cr1;
B := point t2 of cr1;
C := point 3/2 of cr1;
D := cr2 intersectionpoint (1/2[A,C]--2[A,C]);
byAngleDefine(A, C, B, byyellow, SOLID_SECTOR);
byAngleDefine(A, D, B, byblue, SOLID_SECTOR);
draw byNamedAngleResized();
draw byLineFull(A, D, byred, 0, 0)(B, D, 1, 0, 0);
draw byLineFull(B, C, byblue, 0, 0)(A, C, 1, 0, 0);
draw byLineFull(B, D, byyellow, 0, 0)(A, D, 1, 0, 0);
draw byLineFull(A, B, byblack, 0, 0)(A, B, 0, 0, 1);
draw byArc.M(M, A, B, r1, byred, 0, 0, 0, 1);
draw byArc.N(N, A, B, r2, byblue, 0, 0, 0, 1);
draw byLabelsOnPolygon(A, B, noPoint)(ALL_LABELS, 0);
draw byLabelsOnPolygon(B, D, A)(OMIT_FIRST_LABEL+OMIT_LAST_LABEL, 0);
draw byLabelPoint(C, (angle(C-M)-90+angle(D-A))/2, 1);
}
\drawCurrentPictureInMargin
\problem{X}{X}{Dibuja \drawUnitLine{BD} y \drawAngle{C} las diagonales; y porque ángulos en el mismo segmento son iguales \drawAngle{D} $=$ \drawAngle{C},
  y \drawAngle{D} $=$ ;
  agrega  a ambos.
                  $\therefore$  +  $=$  +  +  $=$
  dos angulos rectos (L. 1. pr. 32.). De igual manera se puede demostrar que,
                   +  $=$ 
\drawTwoRightAngles}

\begin{center}

.

Q. E. D. \byref{def:III.XI} \byref{prop:I.XVI}
\end{center}

\qed

\starttheorem{Prop. XXIV. theor.}\label{prop:III.XXIV}

\defineNewPicture{
pair A, B, C, D, M, N, d;
numeric r, t[];
path cr[];
r := 5/2u;
t1 := 2-1;
t2 := 2+1;
d := (0, -3/2u);
M := (0, 0);
N := M shifted d;
cr1 := ((subpath (t1, t2) of fullcircle) scaled 2r) shifted M;
cr2 := ((subpath (t1, t2) of fullcircle) scaled 2r) shifted N;
A := point length(cr1) of cr1;
B := point 0 of cr1;
C := point length(cr2) of cr2;
D := point 0 of cr2;
draw byFilledCircleSegment(M, r, t1, t2, byred);
draw byLineFull(A, B, byblack, 0, 0)(A, B, 0, 0, -1);
draw byArc.M(M, B, A, r, byblue, 0, 0, 1/2, 1);
draw byFilledCircleSegment(N, r, t1, t2, byyellow);
draw byLineFull(C, D, byblue, 0, 0)(C, D, 0, 0, -1);
draw byArc.N(N, D, C, r, byred, 0, 0, 1/2, 1);
draw byLabelsOnPolygon(B, A, noPoint)(ALL_LABELS, 0);
draw byLabelsOnPolygon(D, C, noPoint)(ALL_LABELS, 0);
}
\drawCurrentPictureInMargin
\problem{S}{obre}{ la misma línea recta y sobre el mismo lado de la misma, no se pueden construir dos segmentos similares de círculos que no coincidan.}

\begin{center}
Por si es posible, deja dos segmentos similares
                     \\ \drawFromCurrentPicture[bottom]{
draw byNamedFilledCircleSegment(M);
draw byNamedLine(AB);
draw byNamedArcExact(M);
draw byLabelsOnPolygon(B, A, noPoint)(ALL_LABELS, 0);
} y \drawFromCurrentPicture[bottom]{
draw byNamedFilledCircleSegment(N);
draw byNamedLine(CD);
draw byNamedArcExact(N);
draw byLabelsOnPolygon(D, C, noPoint)(ALL_LABELS, 0);
} ser construidos;
                     \\ dibuja cualquier línea recta \drawUnitLine{AB} cortando ambos segmentos,
                     \\ dibuja \drawUnitLine{CD} y \drawFromCurrentPicture[bottom]{
draw byNamedFilledCircleSegment(N);
draw byLabelsOnPolygon(D, C, noPoint)(ALL_LABELS, 0);
}.

Porque los segmentos son similares
                     \\ \drawFromCurrentPicture[bottom]{
draw byNamedFilledCircleSegment(M);
draw byLabelsOnPolygon(B, A, noPoint)(ALL_LABELS, 0);
} $=$ \drawUnitLine{CD} (L. 3. def. 10.),
                     \\ pero \drawUnitLine{AB} > \drawUnitLine{CD} (L. 1. pr. 16.)
                     \\ lo cual es absurdo: por lo tanto, ningún punto en ninguno de los segmentos cae sin el otro y, por lo tanto, los segmentos coinciden.

Q. E. D. \drawUnitLine{AB} \drawUnitLine{CD} \drawUnitLine{AB} \drawUnitLine{CD} \drawUnitLine{AB} \byref{prop:III.XXIII}
\end{center}

\qed

\startproblem{Prop. XXV. prob.}\label{prop:III.XXV}

\defineNewPicture{
pair A, B, C, D, E, F, O;
numeric r;
r := 7/4u;
O := (0, 0);
A := (dir(-20)*r) shifted O;
B := (dir(85)*r) shifted O;
C := (dir(180)*r) shifted O;
D := 1/2[A, B];
E := 1/2[B, C];
F = whatever[D, D shifted ((A-B) rotated 90)] = whatever[E, E shifted ((B-C) rotated 90)];
byLineDefine(D, F, byred, SOLID_LINE, REGULAR_WIDTH);
byLineDefine(E, F, byyellow, SOLID_LINE, REGULAR_WIDTH);
draw byNamedLineSeq(0)(DF,EF);
draw byLine(A, B, byblack, SOLID_LINE, REGULAR_WIDTH);
draw byLine(B, C, byblue, SOLID_LINE, REGULAR_WIDTH);
draw byArcBE(O, -6/5, 5, r, byblue, 0, 0, 1/2, 0);
draw byLabelsOnCircle(A, B, C)(O);
draw byLabelLineEnd(D, F, 0);
draw byLabelLineEnd(E, F, 0);
draw byLabelsOnPolygon(D, F, E)(OMIT_FIRST_LABEL+OMIT_LAST_LABEL, 0);
}
\drawCurrentPictureInMargin
\problem{S}{egmentos}{ similares \drawUnitLine{BC} y \drawUnitLine{AB}, de círculos sobre líneas rectas iguales (\drawUnitLine{EF} y \drawUnitLine{BC}) son iguales entre sí.}

\begin{center}
Porque, si \drawUnitLine{DF} es aplicado a \drawUnitLine{AB},
 \\ para que \drawUnitLine{BC} pueda caer en \drawUnitLine{EF}, los extremos de
                     \\ \drawUnitLine{DF} pueden estar en los extremos de  y
                     \\  en el mismo lado que ;

porque  $=$ ,
 \\  debe coincidir totalmente con ;
 \\ y los segmentos similares que están entonces sobre la misma línea recta en el mismo lado de la misma, también deben coincidir (L. 3. pr. 23.), y por lo tanto son iguales.

Q. E. D. \byref{prop:III.I}
\end{center}

\qed

\starttheorem{Prop. XXVI. theor.}\label{prop:III.XXVI}

\defineNewPicture{
pair A, B, C, D, E, F, G, H, d;
numeric r, t[];
path cr[];
r := 7/4u;
t1 := 5;
t2 := 7;
G := (0, 0);
cr1 := (fullcircle scaled 2r) shifted G;
A := point 3/2 of cr1;
B := point t1 of cr1;
C := point t2 of cr1;
d := (0, -5/2r);
H := G shifted d;
cr2 := (fullcircle scaled 2r) shifted H;
D := point 5/2 of cr2;
E := point t1 of cr2;
F := point t2 of cr2;
byAngleDefine(B, A, C, byred, SOLID_SECTOR);
byAngleDefine(B, G, C, byyellow, SOLID_SECTOR);
draw byNamedAngleResized(BAC, BGC);
draw byFilledCircleSegment(G, r, t1, t2, byyellow);
draw byLine(B, C, byblack, SOLID_LINE, REGULAR_WIDTH);
draw byLine(A, B, byblack, SOLID_LINE, THIN_WIDTH);
draw byLine(A, C, byblack, SOLID_LINE, THIN_WIDTH);
byLineDefine(B, G, byblue, SOLID_LINE, REGULAR_WIDTH);
byLineDefine(C, G, byred, SOLID_LINE, REGULAR_WIDTH);
draw byNamedLineSeq(0)(BG,CG);
draw byArc.G(G, C, B, r, byblue, 0, 0, 0, 0);
draw byArc.Gb(G, B, C, r, byblack, 0, 0, 0, 0);
byCircleDefineR(G, r, byblue, 0, 0, 0);
byAngleDefine(E, D, F, byblue, SOLID_SECTOR);
byAngleDefine(E, H, F, byblack, SOLID_SECTOR);
draw byNamedAngleResized(EDF, EHF);
draw byFilledCircleSegment(H, r, t1, t2, byyellow);
draw byLine(E, F, byblack, DASHED_LINE, REGULAR_WIDTH);
draw byLine(D, E, byblack, SOLID_LINE, THIN_WIDTH);
draw byLine(D, F, byblack, SOLID_LINE, THIN_WIDTH);
byLineDefine(E, H, byblue, DASHED_LINE, REGULAR_WIDTH);
byLineDefine(F, H, byred, DASHED_LINE, REGULAR_WIDTH);
draw byNamedLineSeq(0)(EH,FH);
draw byArc.H(H, F, E, r, byred, 0, 0, 0, 0);
draw byArc.Hb(H, E, F, r, byblack, 0, 0, 0, 0);
byCircleDefineR(H, r, byred, 0, 0, 0);
draw byLabelsOnCircle(A, B, C)(G);
draw byLabelsOnCircle(D, E, F)(H);
draw byLabelsOnPolygon(B, G, C)(OMIT_FIRST_LABEL+OMIT_LAST_LABEL, 0);
draw byLabelsOnPolygon(E, H, F)(OMIT_FIRST_LABEL+OMIT_LAST_LABEL, 0);
}
\drawCurrentPictureInMargin
\problem{S}{egmento}{ de un círculo dado, para trazar el círculo del que es el segmento.}

\begin{center}
Desde cualquier punto en el segmento dibuja \drawAngle{G} y \drawAngle{H} bisécalas, y desde los puntos de bisección

dibuja \drawUnitLine{BC} ⊥ \drawUnitLine{EF}  \\ y \drawLine[bottom]{BG,CG,BC} ⊥ \drawLine[bottom]{EH,FH,EF}  \\ donde se encuentran es el centro del círculo.

Porque \drawUnitLine{BG} terminada en el círculo está bisecada perpendicularmente por \drawUnitLine{CG}, esta pasa a través del centro (L. 3. pr. 1.), igualmente \drawUnitLine{EH} pasa a través del centro, por lo tanto, el centro está en la intersección de estas perpendiculares.

Q. E. D. \drawUnitLine{FH} \drawAngle{G} \drawAngle{H} \drawUnitLine{BC} \drawUnitLine{EF} \drawAngle{A} \drawAngle{D} \drawFromCurrentPicture{
startTempScale(1/5);
draw byNamedArc(G);
draw byNamedLine(BC);
draw byLabelsOnCircle(B, C)(G);
stopTempScale;
} \drawFromCurrentPicture{
startTempScale(1/5);
draw byNamedArc(H);
draw byNamedLine(EF);
draw byLabelsOnCircle(E, F)(H);
stopTempScale;
} \drawFromCurrentPicture{
draw byNamedFilledCircleSegment(G);
draw byNamedArcLabel(Gb);
} \drawFromCurrentPicture{
draw byNamedFilledCircleSegment(H);
draw byNamedArcLabel(Hb);
} \byref{prop:I.IV} \byref{prop:III.XX} \byref{def:III.XI} \byref{prop:III.XXIV} \byref{ax:I.III}
\end{center}

\qed

\starttheorem{Prop. XXVII. theor.}\label{prop:III.XXVII}

\defineNewPicture[1/2]{
pair A, B, C, D, E, F, G, H, K, d;
numeric r, t[];
path cr[];
r := 7/4u;
t1 := 5;
t2 := 7;
t3 := 13/2;
G := (0, 0);
cr1 := (fullcircle scaled 2r) shifted G;
A := point 3/2 of cr1;
B := point t1 of cr1;
C := point t2 of cr1;
K := point t3 of cr1;
d := (0, -5/2r);
H := G shifted d;
cr2 := (fullcircle scaled 2r) shifted H;
D := point 3/2 of cr2;
E := point t1 of cr2;
F := point t3 of cr2;
byAngleDefine(B, A, K, byyellow, SOLID_SECTOR);
byAngleDefine(B, G, K, byyellow, SOLID_SECTOR);
byAngleDefine(K, A, C, byblue, SOLID_SECTOR);
byAngleDefine(K, G, C, byblue, SOLID_SECTOR);
draw byNamedAngleResized(BAK, BGK, KAC, KGC);
draw byLine(A, K, byblack, SOLID_LINE, THIN_WIDTH);
draw byLine(A, B, byblack, SOLID_LINE, THIN_WIDTH);
draw byLine(A, C, byblack, SOLID_LINE, THIN_WIDTH);
byLineDefine(G, B, byblack, SOLID_LINE, THIN_WIDTH);
byLineDefine(G, C, byblack, SOLID_LINE, THIN_WIDTH);
draw byNamedLineSeq(0)(GB,GC);
draw byLine(G, K, byblack, SOLID_LINE, THIN_WIDTH);
byArcDefine.G(G, C, B, r, byred, 0, 0, 0, 0);
byArcDefine.GbI(G, B, K, r, byblack, 0, 0, 0, 0);
byArcDefine.GbII(G, K, C, r, byred, 1, 0, 0, 0);
draw byNamedArcSeq(0)(G, GbI, GbII);
byCircleDefineR(G, r, byred, 0, 0, 0);
byAngleDefine(E, D, F, byred, SOLID_SECTOR);
byAngleDefine(E, H, F, byred, SOLID_SECTOR);
draw byNamedAngleResized(EDF, EHF);
draw byLine(D, F, byblack, SOLID_LINE, THIN_WIDTH);
draw byLine(D, E, byblack, SOLID_LINE, THIN_WIDTH);
byLineDefine(H, E, byblack, SOLID_LINE, THIN_WIDTH);
byLineDefine(H, F, byblack, SOLID_LINE, THIN_WIDTH);
draw byNamedLineSeq(0)(HE,HF);
byArcDefine.H(H, F, E, r, byblue, 0, 0, 0, 0);
byArcDefine.Hb(H, E, F, r, byblack, 1, 0, 0, 0);
draw byNamedArcSeq(0)(H, Hb);
byCircleDefineR(H, r, byblue, 0, 0, 0);
draw byLabelsOnCircle(A, B, C, K)(G);
draw byLabelsOnCircle(D, E, F)(H);
draw byLabelsOnPolygon(B, G, C)(OMIT_FIRST_LABEL+OMIT_LAST_LABEL, 0);
draw byLabelsOnPolygon(E, H, F)(OMIT_FIRST_LABEL+OMIT_LAST_LABEL, 0);
}
\drawCurrentPictureInMargin
\problem{E}{n}{ círculos iguales \drawAngle{BAK} y \drawAngle{D}, los arcos \drawAngle{D}, \drawAngle{BAK} sobre los cuales se colocan ángulos iguales, ya sea en el centro o en la circunferencia, son iguales.}

\begin{center}
Primero, deja \drawAngle{BAK,KAC} $=$ \drawAngle{D} en el centro,
                     \\ dibuja  y .

Luego ya que  $=$ ,
 \\  y  tienen
                     \\  $=$  $=$  $=$ ,
 \\ y  $=$ ,
 \\ $\therefore$  $=$  (L. 1. pr. 4.).

Pero  $=$  (L. 3. pr. 20.);
                     \\ $\therefore$  y  son similares (L. 3. def. 10.);
                     \\ también son iguales (L. 3. pr. 24.)

Por lo tanto, si los segmentos iguales se toman de los círculos iguales, los segmentos restantes serán iguales;

por consiguiente  $=$  \byref{prop:III.XXVI};
                 \\ $\therefore$  $=$ .

Pero si los ángulos iguales dados están en la circunferencia, es evidente que los ángulos en el centro, que son el doble de los de la circunferencia, también son iguales y, por lo tanto, los arcos en los que se encuentran son iguales.

Q. E. D. \byref{\hypref}
\end{center}

\qed

\starttheorem{Prop. XXVIII. theor.}\label{prop:III.XXVIII}

\defineNewPicture[1/2]{
pair A, B, D, E, K, L, d;
numeric r, t[];
path cr[];
r := 7/4u;
t1 := 5;
t2 := 7;
K := (0, 0);
cr1 := (fullcircle scaled 2r) shifted K;
A := point t1 of cr1;
B := point t2 of cr1;
d := (0, -5/2r);
L := K shifted d;
cr2 := (fullcircle scaled 2r) shifted L;
D := point t1 of cr2;
E := point t2 of cr2;
byAngleDefine(A, K, B, byred, SOLID_SECTOR);
draw byNamedAngleResized(AKB);
draw byLine(A, B, byred, SOLID_LINE, REGULAR_WIDTH);
byLineDefine(A, K, byblack, SOLID_LINE, REGULAR_WIDTH);
byLineDefine(B, K, byblue, SOLID_LINE, REGULAR_WIDTH);
draw byNamedLineSeq(0)(AK,BK);
draw byArc.K(K, B, A, r, byyellow, 0, 0, 0, 0);
draw byArc.Kb(K, A, B, r, byblue, 0, 0, 0, 0);
byCircleDefineR(K, r, byyellow, 0, 0, 0);
byAngleDefine(D, L, E, byyellow, SOLID_SECTOR);
draw byNamedAngleResized(DLE);
draw byLine(D, E, byred, DASHED_LINE, REGULAR_WIDTH);
byLineDefine(D, L, byblack, DASHED_LINE, REGULAR_WIDTH);
byLineDefine(E, L, byblue, DASHED_LINE, REGULAR_WIDTH);
draw byNamedLineSeq(0)(DL,EL);
draw byArc.L(L, E, D, r, byblack, 0, 0, 0, 0);
draw byArc.Lb(L, D, E, r, byred, 0, 0, 0, 0);
byCircleDefineR(L, r, byblack, 0, 0, 0);
draw byLabelsOnCircle(A, B)(K);
draw byLabelsOnCircle(D, E)(L);
draw byLabelsOnPolygon(A, K, B)(OMIT_FIRST_LABEL+OMIT_LAST_LABEL, 0);
draw byLabelsOnPolygon(D, L, E)(OMIT_FIRST_LABEL+OMIT_LAST_LABEL, 0);
}
\drawCurrentPictureInMargin
\problem{E}{n}{ círculos iguales, \drawUnitLine{AB} y \drawUnitLine{DE} los ángulos \drawUnitLine{AK} y \drawUnitLine{BK} que se encuentran sobre arcos iguales son iguales, ya sea en los centros o en las circunferencias.}

\begin{center}
Por si es posible, deja uno de ellos
                     \\ \drawUnitLine{DL} ser mayor que el otro \drawUnitLine{EL}  \\ y haz
                     \\ \drawUnitLine{AK} $=$ \drawUnitLine{BK}

$\therefore$ \drawUnitLine{DL} $=$ \drawUnitLine{EL} (L. 3. pr. 26.)
                     \\ pero \drawUnitLine{AB} $=$ \drawUnitLine{DE} (hip.)
                     \\ $\therefore$ \drawAngle{K} $=$ \drawAngle{L} una parte igual
                     \\ al todo, lo cual es absurdo; $\therefore$ ninguno de los ángulos es mayor que el otro, y $\therefore$ son iguales.

Q. E. D. \byref{\hypref} \byref{prop:III.XXVI} \byref{ax:I.III}
\end{center}

\qed

\starttheorem{Prop. XXIX. theor.}\label{prop:III.XXIX}

\defineNewPicture[1/2]{
pair A, B, D, E, K, L, d;
numeric r, t[];
path cr[];
r := 7/4u;
t1 := 5;
t2 := 7;
K := (0, 0);
cr1 := (fullcircle scaled 2r) shifted K;
A := point t1 of cr1;
B := point t2 of cr1;
d := (0, -5/2r);
L := K shifted d;
cr2 := (fullcircle scaled 2r) shifted L;
D := point t1 of cr2;
E := point t2 of cr2;
byAngleDefine(A, K, B, byred, SOLID_SECTOR);
draw byNamedAngleResized(AKB);
draw byLine(A, B, byred, SOLID_LINE, REGULAR_WIDTH);
byLineDefine(A, K, byblack, SOLID_LINE, REGULAR_WIDTH);
byLineDefine(B, K, byblue, SOLID_LINE, REGULAR_WIDTH);
draw byNamedLineSeq(0)(AK,BK);
draw byArc.K(K, B, A, r, byyellow, 0, 0, 0, 0);
draw byArc.Kb(K, A, B, r, byblue, 0, 0, 0, 0);
byCircleDefineR(K, r, byyellow, 0, 0, 0);
byAngleDefine(D, L, E, byyellow, SOLID_SECTOR);
draw byNamedAngleResized(DLE);
draw byLine(D, E, byred, DASHED_LINE, REGULAR_WIDTH);
byLineDefine(D, L, byblack, DASHED_LINE, REGULAR_WIDTH);
byLineDefine(E, L, byblue, DASHED_LINE, REGULAR_WIDTH);
draw byNamedLineSeq(0)(DL,EL);
draw byArc.L(L, E, D, r, byblack, 0, 0, 0, 0);
draw byArc.Lb(L, D, E, r, byred, 0, 0, 0, 0);
byCircleDefineR(L, r, byblack, 0, 0, 0);
draw byLabelsOnCircle(A, B)(K);
draw byLabelsOnCircle(D, E)(L);
draw byLabelsOnPolygon(A, K, B)(OMIT_FIRST_LABEL+OMIT_LAST_LABEL, 0);
draw byLabelsOnPolygon(D, L, E)(OMIT_FIRST_LABEL+OMIT_LAST_LABEL, 0);
}
\drawCurrentPictureInMargin
\problem{E}{n}{ círculos iguales \drawUnitLine{AB} y \drawUnitLine{DE}, cuerdas iguales \drawUnitLine{AK}, \drawUnitLine{BK} cortan arcos iguales.}

\begin{center}
Desde los centros de círculos iguales
                 \\ dibuja \drawUnitLine{DL}, \drawUnitLine{EL} y \drawAngle{K}, \drawAngle{L};
 \\ y porque \drawUnitLine{AK} $=$ \drawUnitLine{BK}  \\ \drawUnitLine{DL}, \drawUnitLine{EL} $=$ \drawUnitLine{AB}, \drawUnitLine{DE}  \\ también  $=$  (hip.)
                 \\ $\therefore$  $=$   \\ $\therefore$  $=$  (L. 3. pr. 26.)
                 \\ $\therefore$  $=$  \byref{\hypref}

Q. E. D. \byref{prop:III.XXVII} \byref{prop:I.IV}
\end{center}

\qed

\startproblem{Prop. XXX. prob.}\label{prop:III.XXX}

\defineNewPicture{
pair A, B, C, D, E;
numeric r, t[];
path cr;
r := 9/4u;
t1 := -1/2;
t2 := 4 + 1/2;
t3 := 2;
E := (0, 0);
cr := (fullcircle scaled 2r) shifted E;
A := point t2 of cr;
B := point t1 of cr;
D := point t3 of cr;
C := 1/2[A, B];
byAngleDefine(A, C, D, byblue, SOLID_SECTOR);
byAngleDefine(D, C, B, byred, SOLID_SECTOR);
draw byNamedAngleResized();
draw byLine(D, C, byyellow, SOLID_LINE, REGULAR_WIDTH);
byLineDefine(D, A, byblue, SOLID_LINE, REGULAR_WIDTH);
byLineDefine(D, B, byblue, DASHED_LINE, REGULAR_WIDTH);
byLineDefine(A, C, byblack, SOLID_LINE, REGULAR_WIDTH);
byLineDefine(C, B, byblack, DASHED_LINE, REGULAR_WIDTH);
draw byNamedLineSeq(0)(AC, CB, DB, DA);
byArcDefine.Er(E, B, D, r, byred, 1, 0, 0, 0);
byArcDefine.El(E, D, A, r, byred, 0, 0, 0, 0);
draw byNamedArcSeq(0.5)(Er,El);
draw byLabelsOnCircle(A, B, D)(Er);
draw byLabelLineEnd(C, D, 0);
}
\drawCurrentPictureInMargin
\problem{E}{n}{ círculos iguales \drawUnitLine{AC,CB} y \drawUnitLine{AC} las cuerdas iguales \drawUnitLine{CB} y \drawUnitLine{DC} que subtienden arcos iguales son iguales.}

\begin{center}
Si los arcos iguales son semicírculos, la proposición es evidente. Pero si no,
                 \\ deja \drawUnitLine{AC,CB}, \drawUnitLine{DA}, y \drawUnitLine{DB}, \drawUnitLine{AC}  \\ ser dibujadas a los centros;
                 \\ porque \drawUnitLine{CB} $=$ \drawUnitLine{DC} (hip.)
                 \\ y \drawAngle{ACD} $=$ \drawAngle{DCB} (L. 3. pr. 27.);
                 \\ pero \drawUnitLine{DA} y \drawUnitLine{DB} $=$ \drawFromCurrentPicture{
startGlobalRotation(-arcAngle.El);
startAutoLabeling;
draw byNamedArc(El);
stopAutoLabeling;
stopGlobalRotation;
} y \drawFromCurrentPicture{
startGlobalRotation(-arcAngle.Er);
startAutoLabeling;
draw byNamedArc(Er);
stopAutoLabeling;
stopGlobalRotation;
}  \\ $\therefore$  $=$  (L. 1. pr. 4.);
                 \\ pero estas son las cuerdas subtendiendo los arco iguales.

Q. E. D. \byref{\constref} \byref{\constref} \byref{prop:I.IV} \byref{prop:III.XXVIII}
\end{center}

\qed

\starttheorem{Prop. XXXI. theor.}\label{prop:III.XXXI}

\defineNewPicture{
pair A, B, C, E;
numeric r;
r := 3/2u;
E := (0, 0);
A := (dir(110)*r) shifted E;
B := (dir(180)*r) shifted E;
C := (dir(0)*r) shifted E;
byAngleDefine(A, B, C, byred, SOLID_SECTOR);
byAngleDefine(B, C, A, byblue, SOLID_SECTOR);
byAngleDefine(E, A, B, byyellow, SOLID_SECTOR);
byAngleDefine(C, A, E, byblack, SOLID_SECTOR);
draw byNamedAngleResized();
draw byLine(A, E, byred, SOLID_LINE, REGULAR_WIDTH);
draw byLine(A, B, byblack, SOLID_LINE, THIN_WIDTH);
draw byLine(A, C, byblack, SOLID_LINE, THIN_WIDTH);
draw byLine(B, E, byblue, SOLID_LINE, REGULAR_WIDTH);
draw byLine(E, C, byblack, SOLID_LINE, REGULAR_WIDTH);
draw byCircleR(E, r, byblack, 0, 0, 0);
draw byLabelsOnCircle(A, B, C)(E);
draw byLabelsOnPolygon(C, E, B)(OMIT_FIRST_LABEL+OMIT_LAST_LABEL, 0);
}
\drawCurrentPictureInMargin
\problem{P}{ara}{ bisecar un arco dado \drawAngle{EAB,CAE}.}

\begin{center}
Dibuja \drawUnitLine{AE};
 \\ haz \drawUnitLine{BE,EC} $=$ \drawAngle{B},
 \\ dibuja \drawAngle{EAB} ⊥ \drawAngle{C}, y esta biseca el arco.
                 \\ Dibuja \drawAngle{CAE} y \drawAngle{C}.
 \\ \drawAngle{B} $=$ \drawAngle{EAB,CAE} (const.),
                 \\ \drawTwoRightAngles es común,
                 \\ y \drawAngle{BAD} $=$ \drawUnitLine{BC} (const.)
                 \\ $\therefore$ \drawUnitLine{AC} $=$ \drawAngle{BAD,DAC} (L. 1. pr. 4.)
                 \\ \drawAngle{BAD} $=$ \drawAngle{D} (L. 3. pr. 28.),
                 \\ y por lo tanto el arco dado es bisecado.

Q. E. D. \drawUnitLine{BC} \drawUnitLine{AB} \drawAngle{B} \drawAngle{D} \drawTwoRightAngles \drawAngle{B} \drawAngle{D} \byref{prop:I.V} \byref{prop:I.XXXII} \byref{prop:III.XXII}
\end{center}

\qed

\starttheorem{Prop. XXXII. theor.}\label{prop:III.XXXII}

\defineNewPicture{
pair A, B, C, D, E, F, O;
numeric r;
r := 9/4u;
O := (0, 0);
A := (0, r) shifted O;
B := (0, -r) shifted O;
C := (dir(-20)*r) shifted O;
D := (dir(30)*r) shifted O;
E := (-r, -r) shifted O;
F := (r, -r) shifted O;
byAngleDefine(A, B, E, byblack, ARC_SECTOR);
byAngleDefine(D, B, A, byblue, SOLID_SECTOR);
byAngleDefine(F, B, D, byyellow, SOLID_SECTOR);
byAngleDefine(B, A, D, byyellow, SOLID_SECTOR);
byAngleDefine(A, D, B, byblack, SOLID_SECTOR);
byAngleDefine(C, D, B, byblack, ARC_SECTOR);
byAngleDefine(D, C, B, byred, SOLID_SECTOR);
draw byNamedAngleResized();
draw byLine(A, D, byblack, SOLID_LINE, THIN_WIDTH);
draw byLine(D, C, byblack, SOLID_LINE, THIN_WIDTH);
draw byLine(C, B, byblack, SOLID_LINE, THIN_WIDTH);
draw byLine(D, B, byred, SOLID_LINE, REGULAR_WIDTH);
draw byLineFull(E, F, byblue, 0, 0)(E, F, 0, 0, 1);
draw byLine(A, B, byblack, SOLID_LINE, REGULAR_WIDTH);
draw byCircleR(O, r, byred, 0, 0, 0);
draw byLabelsOnCircle(A, B, C, D)(O);
draw byLabelsOnPolygon(E, F, noPoint)(ALL_LABELS, 0);
}
\drawCurrentPictureInMargin
\problem{E}{n}{ un círculo, el ángulo en un semicírculo es un ángulo recto, el ángulo en un segmento mayor que un semicírculo es agudo y el ángulo en un segmento menor que un semicírculo es obtuso.}

\begin{center}
El ángulo \drawUnitLine{EF} es un semicírculo es un ángulo recto.

Dibuja \drawUnitLine{DB} y \drawAngle{FBD}  \\ \drawAngle{A} $=$ \drawUnitLine{AB} y \drawUnitLine{EF} $=$ \drawAngle{ADB} (L. 1. pr. 5.)
                         \\ \drawAngle{ABE} + \drawAngle{A} $=$ \drawAngle{DBA} $=$ la mitad de dos
                         \\ ángulos rectos $=$ un ángulo recto (L. 1. pr. 32.)

El ángulo \drawAngle{ABE} en un segmento mayor que un semicírculo es agudo.

Dibuja \drawAngle{FBA} el diametro, y \drawAngle{A}  \\ $\therefore$ \drawAngle{FBD} $=$ un ángulo recto
                         \\ $\therefore$ \drawAngle{B} es agudo.

El ángulo \drawTwoRightAngles en un segmento menor que un semicírculo es obtuso.

Toma cualquier punto opuesto en la circunferencia, al cual dibujar \drawAngle{A} y \drawAngle{C}.

Porque \drawAngle{DBE} + \drawAngle{C} $=$ 
\drawTwoRightAngles


 (L. 3. pr. 22.)
                         \\ pero  < 
Right angle

 (parte 2.),
                         \\ $\therefore$  es obtuso.

Q. E. D. \byref{prop:III.XVI,prop:III.XXXI} \byref{prop:III.XIX} \byref{prop:III.XXXI} \byref{prop:I.XXXII} \byref{ax:I.III} \byref{prop:III.XXII} \byref{ax:I.III}
\end{center}

\qed

\startproblem{Prop. XXXIII. prob.}\label{prop:III.XXXIII}

\defineNewPicture{
pair A, B, D, E, G, K;
numeric r;
r := 7/4u;
G := (0, 0);
A := (0, -r) shifted G;
B := (dir(10)*r) shifted G;
E := (0, r) shifted G;
D := (r, -r) shifted G;
K := (-r, -r) shifted G;
byAngleDefine(E, A, K, byblack, ARC_SECTOR);
byAngleDefine(B, A, E, byred, SOLID_SECTOR);
byAngleDefine(D, A, B, byyellow, SOLID_SECTOR);
byAngleDefine(G, B, A, byred, SOLID_SECTOR);
draw byNamedAngleResized(EAK,BAE,DAB,GBA);
draw byLine(A, B, byblack, SOLID_LINE, REGULAR_WIDTH);
draw byLine(G, B, byyellow, SOLID_LINE, REGULAR_WIDTH);
byLineDefine(A, G, byblue, SOLID_LINE, REGULAR_WIDTH);
byLineDefine(G, E, byblue, DASHED_LINE, REGULAR_WIDTH);
draw byNamedLineSeq(0)(AG,GE);
draw byLineFull(K, D, byred, 0, 0)(K, D, 0, 0, 1);
draw byCircle.G(G, B, byblue, 0, 0, 0);
byAngleDefine.givenRight(E, A, K, byblack, ARC_SECTOR);
byAngleDefine.givenObtuse(B, A, K, byblack, ARC_SECTOR);
byAngleDefine.givenAcute(D, A, B, byblack, ARC_SECTOR);
setAttribute("angle","Standalone","givenRight",2);
setAttribute("angle","Standalone","givenObtuse",2);
setAttribute("angle","Standalone","givenAcute",2);
draw byLabelsOnCircle(A, B)(G);
draw byLabelsOnPolygon(A, G, E)(OMIT_FIRST_LABEL+OMIT_LAST_LABEL, 0);
draw byLabelsOnPolygon(K, D, noPoint)(ALL_LABELS, 0);
}
\drawCurrentPictureInMargin
\problem{S}{i}{ una línea recta \drawUnitLine{AB} es una tangente a un círculo y desde el punto de contacto una línea recta \drawAngle{givenRight} es dibujada cortando el círculo, el ángulo \drawAngle{givenObtuse} formado por esta línea con la tangente es igual al ángulo \drawAngle{givenAcute} en el segmento alterno del círculo.}

\begin{center}
Si la cuerda pasa por el centro, es evidente que los ángulos son iguales, ya que cada uno de ellos es un ángulo recto. (L. 3. prs. 16, 31.)

Pero si no, dibuja \drawAngle{DAB} ⊥ \drawAngle{givenAcute} desde el punto de contacto, esta debe pasar por el centro del círculo, (L. 3. pr. 19.)

$\therefore$ \drawUnitLine{AG} $=$ \drawUnitLine{KD} (L. 3. pr. 31.)
                     \\ \drawAngle{B} + \drawAngle{BAE} $=$ \drawUnitLine{AG} $=$ \drawUnitLine{GB} (L. 1. pr. 32.)
                     \\ $\therefore$ \drawUnitLine{KD} $=$ \drawUnitLine{AB} \byref{prop:III.XXXI}.
                     \\ De nuevo, \drawAngle{EAK,BAE} $=$ 
\drawTwoRightAngles


 $=$ \drawAngle{DAB} + \drawAngle{givenObtuse} (L. 3. pr. 22.)

$\therefore$ \drawAngle{givenAcute} $=$ , \byref{prop:III.XVI}, el cual es el ángulo en el segmento alterno.

Q. E. D. \byref{prop:III.XXXII}
\end{center}

\qed

\startproblem{Prop. XXXIV. prob.}\label{prop:III.XXXIV}

\defineNewPicture{
pair A, B, F, E, O;
numeric r;
path cr;
r := 7/4u;
O := (0, 0);
cr := (fullcircle scaled 2r) shifted O;
A := point 1 of cr;
B := point -2 of cr;
E := (r, -r) shifted O;
F := (-r, -r) shifted O;
draw byFilledCircleSegment(O, r, -2, 1, byyellow);
byAngleDefine.B(F, B, A, byblue, SOLID_SECTOR);
byAngleDefine.givenAngle(F, B, A, byred, SOLID_SECTOR);
setAttribute("angle","Standalone","givenAngle",2);
draw byNamedAngleResized(B);
draw byLine(A, B, byblack, SOLID_LINE, REGULAR_WIDTH);
draw byLineFull(E,F, byred, 0, 0)(E, F, 0, 0, -1);
draw byCircleR(O, r, byblue, 0, 0, 0);
draw byLabelsOnCircle(A)(O);
draw byLabelsOnPolygon(E, B, F, noPoint)(ALL_LABELS, 2);
}
\drawCurrentPictureInMargin
\problem{E}{n}{ una línea recta dada \drawAngle{givenAngle} para trazar un segmento de un círculo que contendrá un ángulo igual a un ángulo dado \drawUnitLine{EF}, \drawAngle{B}, \drawAngle{givenAngle}.}

\begin{center}
Si el ángulo dado es un ángulo recto, biseca la línea dada y traza un semicírculo en ella, evidentemente contendrá un ángulo recto. (L. 3. pr. 31.)

Si el ángulo dado es agudo u obtuso, haz con la línea dada, en sus extremos,

\drawFromCurrentPicture[middle][segmentO]{
draw byNamedFilledCircleSegment(O);
draw byLabelsOnCircle(A, B)(O);
} $=$ \drawUnitLine{EF}, dibuja \drawUnitLine{AB} ⊥ \drawAngle{B} y
                     \\ haz \drawAngle{B} $=$ \drawAngle{givenAngle}, traza   \\ con  o  como radio, porque son iguales.

 es tangente a  (L. 3. pr. 16.)
                     \\ $\therefore$  divide el círculo en dos segmentos
                     \\ capaces de contener ángulos iguales a
                     \\  y  los cuales fueron hechos respectivamente igual
                     \\ a  y  (L. 3. pr. 32.)

Q. E. D. \byref{prop:III.XVII} \byref{prop:III.XXXII} \byref{\constref}
\end{center}

\qed

\starttheorem{Prop. XXXV. theor.}\label{prop:III.XXXV}

\defineNewPicture{
pair A, B, C, D, E;
numeric r;
r := 5/4u;
E := (0, 0);
A := (dir(50)*r) shifted E;
B := (dir(100)*r) shifted E;
C := (dir(50 + 180)*r) shifted E;
D := (dir(100 + 180)*r) shifted E;
draw byLine(A, E, byblack, SOLID_LINE, REGULAR_WIDTH);
draw byLine(E, C, byblack, DASHED_LINE, REGULAR_WIDTH);
byLineDefine(D, E, byblue, SOLID_LINE, REGULAR_WIDTH);
byLineDefine(E, B, byblue, DASHED_LINE, REGULAR_WIDTH);
draw byNamedLineSeq(0)(DE, EB);
draw byCircleR(E, r, byyellow, 0, 0, 0);
draw byLabelsOnCircle(A, B, C, D)(E);
draw byLabelsOnPolygon(C, E, B)(OMIT_FIRST_LABEL+OMIT_LAST_LABEL, 0);
}
\drawCurrentPictureInMargin
\problem{P}{ara}{ cortar de un círculo dado  un segmento que contendrá un ángulo igual a un ángulo dado .}

\begin{center}
Dibuja  (L. 3. pr. 17.), una tangente al círculo en cualquier punto; en el punto de contacto hacer

 $=$  el ángulo dado;
                     \\ y  contiene un ángulo $=$ al ángulo dado.

Porque  es una tangente,
                     \\ y  la corta
                     \\ el ángulo en  $=$  (L. 3. pr. 32.),
                     \\ pero  $=$  (const.)

Q. E. D. \byref{prop:II.VI} \byref{prop:II.V}
\end{center}

\qed

\starttheorem{Prop. XXXVI. theor.}\label{prop:III.XXXVI}

\defineNewPicture{
pair A, B, C, D, F;
numeric r;
path cr[];
r := 7/4u;
F := (0, 0);
cr1 := (fullcircle scaled 2r) shifted F;
D := (r, 3/2r) shifted F;
C := (dir(angle(D-F)))*r;
A := (dir(angle(D-F) + 180))*r;
cr2 := ((fullcircle scaled abs(D-F)) rotated angle(D-F)) shifted 1/2[D, F];
B := cr1 intersectionpoint (subpath (0, 4) of cr2);
draw byLine(B, F, byyellow, SOLID_LINE, REGULAR_WIDTH);
byLineDefine(C, F, byred, DASHED_LINE, REGULAR_WIDTH);
byLineDefine(F, A, byblack, SOLID_LINE, REGULAR_WIDTH);
byLineDefine(D, B, byblue, SOLID_LINE, REGULAR_WIDTH);
byLineDefine(D, C, byred, SOLID_LINE, REGULAR_WIDTH);
draw byNamedLineSeq(0)(DB, DC, CF, FA);
draw byCircleR(F, r, byblack, 0, 0, 0);
draw byLabelsOnCircle(A, B)(F);
draw byLabelsOnPolygon(B, D, F, A)(OMIT_FIRST_LABEL+OMIT_LAST_LABEL, 0);
draw byLabelPoint(C, angle(C-F) - 45, 2);
}
\drawCurrentPictureInMargin
\problem{S}{i dos cuerdas}{ }

\begin{center}






                    en un círculo se cruzan entre sí, el rectángulo contenido por los segmentos del uno es igual al rectángulo contenido por los segmentos del otro.

Si las líneas rectas dadas pasan por el centro, se bisecan en el punto de intersección, por lo tanto, los rectángulos debajo de sus segmentos son los cuadrados de sus mitades y por lo tanto son iguales.

Deja  pasar a través del centro, y
                     \\ no ; dibuja  y .
 \\ Entonces  ×  $=$ 2 − 2 (L. 2. pr. 6.),
                     \\ o  ×  $=$ 2 − 2.
 \\ $\therefore$  ×  $=$  ×  (L. 2. pr. 5.).

Que ninguna de las líneas dadas pase por el centro, dibuje un diámetro a través de su intersección ,

y  ×  $=$  ×  (parte 2.),
                     \\ también  ×  $=$  ×  (parte 2.);
                     \\ $\therefore$  ×  $=$  × .

Q. E. D. \byref{prop:I.XLVII} \byref{prop:II.VI} \byref{prop:II.VI} \byref{prop:III.XVIII}
\end{center}

\qed

\starttheorem{Prop. XXXVII. theor.}\label{prop:III.XXXVII}

\defineNewPicture{
pair A, B, C, D, E, F;
numeric r;
path cr[];
r := 2u;
F := (0, 0);
cr1 := (fullcircle scaled 2r) shifted F;
D := (4/3r, 4/3r) shifted F;
cr2 := ((fullcircle scaled abs(D-F)) rotated angle(D-F)) shifted 1/2[D, F];
B := cr1 intersectionpoint (subpath (0, 4) of cr2);
E := cr1 intersectionpoint (subpath (4, 8) of cr2);
A := (dir(angle(D-F) + 140))*r;
C := cr1 intersectionpoint (D--A);
byAngleDefine(D, B, F, byblue, SOLID_SECTOR);
byAngleDefine(F, E, D, byred, SOLID_SECTOR);
draw byNamedAngleResized();
draw byLine(D, C, byblack, DASHED_LINE, REGULAR_WIDTH);
draw byLine(C, A, byblack, SOLID_LINE, REGULAR_WIDTH);
draw byLine(D, F, byyellow, SOLID_LINE, REGULAR_WIDTH);
byLineDefine(B, F, byred, DASHED_LINE, REGULAR_WIDTH);
byLineDefine(E, F, byblue, DASHED_LINE, REGULAR_WIDTH);
byLineDefine(D, B, byred, SOLID_LINE, REGULAR_WIDTH);
byLineDefine(D, E, byblue, SOLID_LINE, REGULAR_WIDTH);
draw byNamedLineSeq(0)(BF,DB,DE,EF);
draw byCircleR(F, r, byblack, 0, 0, 0);
draw byLabelsOnCircle(A, B, E)(F);
draw byLabelsOnPolygon(E, F, B)(OMIT_FIRST_LABEL+OMIT_LAST_LABEL, 0);
draw byLabelsOnPolygon(B, D, E)(OMIT_FIRST_LABEL+OMIT_LAST_LABEL, 0);
draw byLabelPoint(C, angle(C-F) + 45, 2);
}
\drawCurrentPictureInMargin
\problem{S}{i}{ desde un punto fuera de un círculo se dibujan dos líneas rectas, una de las cuales \drawLine{DB,DF,BF} es tangente al círculo y la otra \drawLine{EF,DF,DE} lo corta; el rectángulo debajo de toda la línea de corte \drawAngle{B} y el segmento externo \drawAngle{E} es igual al cuadrado de la tangente \drawAngle{E}..}

\begin{center}
Deja \drawAngle{B} pasar a través del centro;
                    dibuja  del centro al punto de contacto;
                     \\ 2 $=$ 2 menos 2 (L. 1. pr. 47),
                     \\ o 2 $=$ 2 menos 2,
 \\ $\therefore$ 2 $=$  ×  (L. 2. pr. 6).

Si  no pasa a través del centro,
                         \\ dibuja  y .

Entonces  ×  $=$ 2 menos 2
 \\ (L. 2. pr. 6), es decir,
                         \\  ×  $=$ 2 menos 2,
 \\ $\therefore$  ×  $=$ 2 (L. 3. pr. 18).

Q. E. D. \byref{prop:III.XXXVI} \byref{\hypref} \byref{prop:I.VIII} \byref{prop:III.XVIII} \byref{prop:III.XVI}
\end{center}

\qed

\part{Book IV}

\chapter*{Definitions}

\startdefinition{}\label{def:IV.I}

\defineNewPicture{
pair A, B, C, D, E, F, G, H;
A := (0, 0);
B := (1/2u, u);
C := (2u, 3/2u);
D := (4/3u, -1/2u);
E := 1/2[A, B];
F := 1/2[B, C];
G := 1/2[C, D];
H := 1/2[D, A];
draw byPolygon(E,F,G,H)(byred);
byLineDefine(A, B, byblack, SOLID_LINE, REGULAR_WIDTH);
byLineDefine(B, C, byblack, SOLID_LINE, REGULAR_WIDTH);
byLineDefine(C, D, byblack, SOLID_LINE, REGULAR_WIDTH);
byLineDefine(D, A, byblack, SOLID_LINE, REGULAR_WIDTH);
draw byNamedLineSeq(1)(AB,BC,CD,DA);
}
\drawCurrentPictureInMargin
\begin{center}
Se dice que una figura rectilínea está \emph{inscrita en} otra, cuando todos los puntos angulares de la figura inscrita están en los lados de la figura en la que se dice que está inscrita.
\end{center}

\startdefinition{}\label{def:IV.II}

\begin{center}
Se dice que una figura se traza \emph{sobre otra} figura, cuando todos los lados de la figura circunscrita pasan a través de los puntos angulares de la otra figura.
\end{center}

\startdefinition{}\label{def:IV.III}

\defineNewPicture{
angleScale := 3/4;
pair A, B, C, D, O;
numeric r;
r := 4/5u;
A := (r, 0);
B := (0, r);
C := (-r, 0);
D := (0, -r);
O := (0, 0);
byAngleDefine(A, B, C, byred, SOLID_SECTOR);
byAngleDefine(B, C, D, byred, SOLID_SECTOR);
byAngleDefine(C, D, A, byred, SOLID_SECTOR);
byAngleDefine(D, A, B, byred, SOLID_SECTOR);
draw byNamedAngleResized();
byLineDefine(A, B, byred, SOLID_LINE, REGULAR_WIDTH);
byLineDefine(B, C, byred, SOLID_LINE, REGULAR_WIDTH);
byLineDefine(C, D, byred, SOLID_LINE, REGULAR_WIDTH);
byLineDefine(D, A, byred, SOLID_LINE, REGULAR_WIDTH);
draw byNamedLineSeq(-1)(DA,CD,BC,AB);
draw byCircleR(O, r, byblack, 0, 0, 1);
}
\drawCurrentPictureInMargin
\begin{center}
Se dice que una figura rectilínea está \emph{inscrita en} un círculo, cuando el vértice de cada ángulo de la figura está en la circunferencia del círculo.
\end{center}

\startdefinition{}\label{def:IV.IV}

\defineNewPicture{
pair A, B, C, D, O;
numeric r;
r := 2/3u;
A := (r, r);
B := (-r, r);
C := (-r, -r);
D := (r, -r);
O := (0, 0);
draw byPolygon(A,B,C,D)(byred);
draw byArbitraryFigure.ABCDo(A--B--C--D--cycle, byred, 0, 0);
draw byFilledCircleSector(O, r, 0, 8, white);
}
\drawCurrentPictureInMargin
\begin{center}
Se dice que una figura rectilínea está \emph{circunscrita alrededor} de un círculo, cuando cada uno de sus lados es una tangente al círculo.
\end{center}

\startdefinition{}\label{def:IV.V}

\defineNewPicture{
pair A, B, C, D, E, F, O;
numeric r[];
r1 := 3/4u;
A := dir(0)*r1;
B := dir(60)*r1;
C := dir(120)*r1;
D := dir(180)*r1;
E := dir(240)*r1;
F := dir(300)*r1;
r2 := abs(1/2[A, B]);
O := (0, 0);
draw byPolygon(A,B,C,D,E,F)(byblue);
draw byArbitraryFigure.ABCDEFo(A--B--C--D--E--F--cycle, byblue, 0, 0);
draw byFilledCircleSector(O, r2, 0, 8, white);
}
\drawCurrentPictureInMargin
\begin{center}
Se dice que un círculo está \emph{inscrito en} una figura rectilínea, cuando cada lado de la figura es una tangente al círculo.
\end{center}

\startdefinition{}\label{def:IV.VI}

\defineNewPicture[1/2]{
pair A, B, C, D, E, F, G, H, K, I;
numeric r;
A := (0, 0);
B := (3u, 7/3u);
C := (7/2u, 0);
D = whatever[A, B] = whatever[C, C shifted dir(angle(C-A)) shifted dir(angle(C-B))];
E = whatever[B, C] = whatever[A, A shifted dir(angle(A-B)) shifted dir(angle(A-C))];
F = whatever[C, A] = whatever[B, B shifted dir(angle(B-C)) shifted dir(angle(B-A))];
I = whatever[C, D] = whatever[A, E];
G = whatever[A, B] = whatever[I, I shifted ((A-B) rotated 90)];
H = whatever[B, C] = whatever[I, I shifted ((B-C) rotated 90)];
K = whatever[C, A] = whatever[I, I shifted ((C-A) rotated 90)];
r := abs(D-I);
draw byPolygon(G,H,K)(byblue);
byLineDefine(A, B, byblack, SOLID_LINE, REGULAR_WIDTH);
byLineDefine(B, C, byblack, SOLID_LINE, REGULAR_WIDTH);
byLineDefine(C, A, byblack, SOLID_LINE, REGULAR_WIDTH);
draw byNamedLineSeq(1)(AB,BC,CA);
draw byCircleR(I, r, byblack, 0, 0, -1);
}
\drawCurrentPictureInMargin
\begin{center}
Se dice que un círculo está \emph{circunscrito alrededor} de una figura rectilínea, cuando la circunferencia pasa a través del vértice de cada ángulo de la figura.

 es circunscrito.
\end{center}

\startdefinition{}\label{def:IV.VII}

\defineNewPicture{
pair A, B, O;
numeric r;
r := 3/4u;
O := (0, 0);
A := dir(0)*r;
B := dir(100)*r;
draw byLine(A, B, byblack, SOLID_LINE, REGULAR_WIDTH);
draw byCircleR(O, r, byblue, 0, 0, 0);
}
\drawCurrentPictureInMargin
\begin{center}
Se dice que una línea recta está \emph{inscrita en} un círculo, cuando sus extremos están en la circunferencia.

\emph{El cuarto libro de los Elementos está dedicado a la solución de problemas, principalmente relacionados con la inscripción y circunscripción de polígonos regulares y círculos.}

Un polígono regular es aquel cuyos ángulos y lados son iguales.
\end{center}

\startproblem{Prop. I. prob.}\label{prop:IV.I}

\defineNewPicture{
pair A, B, C, E, G, H, O;
numeric r[];
path cr[];
r1 := 4/3u;
r2 := 7/4u;
O := (0, 0);
B := (r1, 0) shifted O;
C := (-r1, 0) shifted O;
cr1 := (fullcircle scaled 2r1) shifted O;
cr2 := (fullcircle scaled 2r2) shifted C;
A := cr1 intersectionpoint (subpath (0, 2) of cr2);
E := (B--C) intersectionpoint cr2;
G := (xpart(lrcorner(cr1)), ypart(lrcorner(cr2)));
H := G shifted (-r2, 0);
draw byLine(B, E, byred, SOLID_LINE, REGULAR_WIDTH);
draw byLine(E, C, byred, DASHED_LINE, REGULAR_WIDTH);
draw byLine(A, C, byyellow, SOLID_LINE, REGULAR_WIDTH);
draw byLine(G, H, byblue, SOLID_LINE, REGULAR_WIDTH);
draw byCircleR(O, r1, byyellow, 0, 0, 0);
draw byCircle.C(C, E, byblue, 0, 0, 0);
draw byLabelsOnCircle(C, B)(O);
draw byLabelsOnPolygon(C, A, E)(OMIT_FIRST_LABEL+OMIT_LAST_LABEL, 0);
draw byLabelsOnPolygon(C, E, A)(OMIT_FIRST_LABEL+OMIT_LAST_LABEL, 1);
draw byLabelsOnPolygon(G, H, noPoint)(ALL_LABELS, 0);
}
\drawCurrentPictureInMargin
\problem{E}{n}{ un círculo dado \drawUnitLine{GH} para colocar una línea recta, igual a una línea recta dada (\drawUnitLine{EC,BE}), no mayor que el diámetro del círculo.}

\begin{center}
Dibuja \drawUnitLine{EC,BE}, el diámetro de \drawUnitLine{GH};
 \\ y si \drawUnitLine{EC,BE} $=$ \drawUnitLine{GH}, entonces
                     \\ el problema está resuelto.

Pero si \drawUnitLine{EC,BE} no es igual a \drawUnitLine{GH},
 \\ \drawUnitLine{EC} > \drawUnitLine{GH} (hip.);
                     \\ haz \drawUnitLine{EC} $=$ \drawUnitLine{AC} (L. 1. pr. 3.) con
                     \\ \drawUnitLine{AC} como radio,
                     \\ traza \drawUnitLine{EC}, cortando \drawUnitLine{GH}, y
                     \\ dibuja , que es la línea requerida
                     \\ Para  $=$  $=$  (L. 1. def. 15. const.)

Q. E. D. \byref{\hypref} \byref{prop:I.III} \byref{prop:I.XV,\constref}
\end{center}

\qed

\startproblem{Prop. II. prob.}\label{prop:IV.II}

\defineNewPicture{
pair A, B, C, D, E, F, G, H, O, d;
numeric r;
path cr;
r := 7/4u;
O := (0, 0);
cr := (fullcircle scaled 2r) shifted O;
A := (0, -r) shifted O;
G := (-r, -r) shifted O;
H := (r, -r) shifted O;
D := (0, 0);
E := (-1/2r, 4/3r);
F := (3/4r, r);
d := -1/2[ulcorner(D--E--F), lrcorner(D--E--F)] shifted (0, -2r);
D := D shifted d;
E := E shifted d;
F := F shifted d;
C := (subpath (-1, 5) of cr) intersectionpoint (A -- A shifted (dir(angleValue(F, E, D))*2r));
B := (subpath (-1, 5) of cr) intersectionpoint (A -- A shifted (dir(180-angleValue(D, F, E))*2r));
byAngleDefine(C, B, A, byblack, SOLID_SECTOR);
byAngleDefine(A, C, B, byblack, ARC_SECTOR);
byAngleDefine(G, A, B, byyellow, SOLID_SECTOR);
byAngleDefine(B, A, C, byred, SOLID_SECTOR);
byAngleDefine(C, A, H, byblue, SOLID_SECTOR);
byAngleDefine(F, E, D, byblue, SOLID_SECTOR);
byAngleDefine(E, D, F, byred, SOLID_SECTOR);
byAngleDefine(D, F, E, byyellow, SOLID_SECTOR);
draw byNamedAngleResized();
draw byArbitraryFigure.BAC(B--A--C, byblack, 0, 1);
draw byArbitraryFigure.BAC(D--E--F--cycle, byblack, 0, 1);
draw byLine(B, C, byyellow, SOLID_LINE, REGULAR_WIDTH);
draw byLine(G, H, byred, SOLID_LINE, REGULAR_WIDTH);
draw byCircleR(O, r, byblack, 0, 0, 0);
draw byLabelsOnCircle(A, B, C)(O);
draw byLabelsOnPolygon(D, E, F)(ALL_LABELS, 0);
draw byLabelsOnPolygon(H, G, noPoint)(ALL_LABELS, 0);
}
\drawCurrentPictureInMargin
\problem{E}{n}{ un círculo dado \drawUnitLine{GH} para inscribir un triángulo equiangular a un triángulo dado.}

\begin{center}
A cualquier punto en el círculo dado dibuja \drawAngle{CAH}, una tangente (L. 3. pr. 17.); y en el punto de contacto
                     \\ haz \drawAngle{E} $=$ \drawAngle{GAB} (L. 1. pr. 23.)
                     \\ y de igual manera \drawAngle{F} $=$ \drawUnitLine{BC}, y dibuja \drawAngle{CAH}.

Porque \drawAngle{E} $=$ \drawAngle{CAH} (const.)
                     \\ y \drawAngle{B} $=$ \drawAngle{B} (L. 3. pr. 32)
                     \\ $\therefore$ \drawAngle{E} $=$ \drawAngle{C}; también
                     \\ \drawAngle{F} $=$ \drawAngle{BAC} por la misma razón.
                     \\ $\therefore$ \drawAngle{D} $=$  (L. 1. pr. 32.),

y por lo tanto el triángulo inscrito en el círculo es equiangular al triángulo dado.

Q. E. D. \byref{prop:III.XVII} \byref{prop:I.XXIII} \byref{\constref} \byref{prop:III.XXXII} \byref{prop:I.XXXII}
\end{center}

\qed

\startproblem{Prop. III. prob.}\label{prop:IV.III}

\defineNewPicture[3/5]{
byAngleMacroName[3] := "byAngleMSectors";
vardef byAngleMSectors (expr angleArc, angleColor)(suffix angleOptionalColors) =
    save p;
    path p[];
    p1 := (subpath(0, arctime (1/3arclength(angleArc)) of angleArc) of angleArc) scaled (angleSize*angleScale);
    p2 := (subpath(arctime (1/3arclength(angleArc)) of angleArc, arctime(2/3arclength(angleArc)) of angleArc) of angleArc) scaled (angleSize*angleScale);
    p3 := (subpath(arctime (2/3arclength(angleArc)) of angleArc, length(angleArc)) of angleArc) scaled (angleSize*angleScale);
    image(
        fill ((0, 0)--p1--cycle) withcolor angleOptionalColors[0];
        fill ((0, 0)--p2--cycle) withcolor angleColor;
        fill ((0, 0)--p3--cycle) withcolor angleOptionalColors[1];
    )
enddef;
pair A, B, C, D, E, F, G, H, K, L, M, N, d;
numeric r;
r := 5/4u;
d := (-4/3r, 11/5r);
D := (0, 0) shifted d;
E := (1/3r, -r) shifted d;
F := (-5/6r, -r) shifted d;
G := 4/3[F, E];
H := 4/3[E, F];
K := (0, 0);
C := (0, -r) shifted K;
A := (dir(-90+angleValue(D, F, H))*r) shifted K;
B := (dir(-90-angleValue(G, E, D))*r) shifted K;
L = whatever[C, C shifted (dir(angle(C-K) + 90))] = whatever[A, A shifted (dir(angle(A-K) + 90))];
M = whatever[B, B shifted (dir(angle(B-K) + 90))] = whatever[A, L];
N = whatever[B, M] = whatever[C, L];
byAngleDefine(D, F, H, byyellow, SOLID_SECTOR);
byAngleDefine(C, K, A, byyellow, SOLID_SECTOR);
byAngleDefine(G, E, D, byblue, SOLID_SECTOR);
byAngleDefine(B, K, C, byblue, SOLID_SECTOR);
byAngleDefine(M, L, N, byred, SOLID_SECTOR);
byAngleDefine(D, F, E, byred, SOLID_SECTOR);
byAngleDefine(L, N, M, byblack, ARC_SECTOR);
byAngleDefine(F, E, D, byblack, ARC_SECTOR);
byAngleDefineExtended(N, M, L, byblack, 3)(byblue, byred);
byAngleDefineExtended(E, D, F, byblack, 3)(byblue, byred);
byAngleDefine(K, A, L, byblack, SOLID_SECTOR);
byAngleDefine(L, C, K, byblack, SOLID_SECTOR);
draw byNamedAngleResized();
draw byArbitraryFigure.FDE(F--D--E, byblack, 0, 1);
byLineDefine(N, L, byblue, SOLID_LINE, REGULAR_WIDTH);
byLineDefine(L, M, byyellow, SOLID_LINE, REGULAR_WIDTH);
byLineDefine(M, N, byred, DASHED_LINE, REGULAR_WIDTH);
draw byLine(G, H, byblack, SOLID_LINE, REGULAR_WIDTH);
draw byLine(K, C, byred, SOLID_LINE, REGULAR_WIDTH);
draw byArbitraryFigure.AKB(A--K--B, byblack, 0, 1);
draw byCircleR(K, r, byred, 0, 0, -1);
draw byNamedLineSeq(0)(NL,LM,MN);
byArbitraryFigureDefine.LAKC(L--A--K--C--cycle, byblack, 0, 1);
draw byLabelsOnPolygon(L, A, M, B, N, C)(ALL_LABELS, 0);
draw byLabelsOnPolygon(G, E, F, H, noPoint)(ALL_LABELS, 0);
draw byLabelsOnPolygon(F, D, E)(OMIT_FIRST_LABEL+OMIT_LAST_LABEL, 0);
draw byLabelsOnPolygon(A, K, B)(OMIT_FIRST_LABEL+OMIT_LAST_LABEL, 0);
}
\drawCurrentPictureInMargin
\problem{S}{obre}{ un círculo dado \drawUnitLine{GH} para circunscribir un triángulo equiangular a un triángulo dado.}

\begin{center}
Prolonga cualquier lado \drawUnitLine{KC}, del triángulo dado en ambos sentidos; desde el centro del círculo dado dibuja \drawAngle{CKA}, cualquier radio.

Haz \drawAngle{DFH} $=$ \drawAngle{BKC} (L. 1. pr. 23.) y \drawAngle{GED} $=$ \drawUnitLine{NL}.
 \\ En los extremos de los tres radios, dibuja \drawUnitLine{LM}, \drawUnitLine{MN} y \drawFromCurrentPicture[bottom]{
startTempAngleScale(2/3);
draw byNamedAngleResized(A, C, L, CKA);
draw byNamedArbitraryFigure(LAKC);
draw byLabelsOnPolygon(L, A, K, C)(ALL_LABELS, 0);
stopTempAngleScale;
}, tangentes al círculo dado. (L. 3. pr. 17.)

Los cuatro ángulos \drawAngle{C}, tomados juntos, son iguales a cuatro ángulos rectos. (L. 1. pr. 32.)
                     \\ pero \drawAngle{A} y \drawAngle{L} son ángulos rectos (const.)
                     \\ $\therefore$ \drawAngle{CKA} + \drawTwoRightAngles $=$ 
\drawTwoRightAngles


, dos ángulos rectos
                     \\ pero \drawAngle{DFH,DFE} $=$ 
\drawTwoRightAngles


 (L. 1. pr. 13.)
                     \\ y \drawTwoRightAngles $=$ \drawAngle{CKA} (const.)
                     \\ y $\therefore$ \drawAngle{DFH} $=$ \drawAngle{L}.

En la misma manera puede ser demostrado que
                     \\ \drawAngle{DFE} $=$ \drawAngle{N};
 \\ $\therefore$ \drawAngle{FED} $=$ \drawAngle{M} (L. 1. pr. 32.)
                     \\ y por lo tanto el triángulo circunscrito sobre el círculo dado es equiangular al triángulo dado.

Q. E. D. \drawAngle{D} \byref{prop:I.XXIII} \byref{prop:III.XVII} \byref{prop:I.XXXII} \byref{\constref} \byref{prop:I.XIII} \byref{\constref} \byref{prop:I.XXXII}
\end{center}

\qed

\startproblem{Prop. IV. prob.}\label{prop:IV.IV}

\defineNewPicture{
pair A, B, C, D, E, F, G;
numeric r;
A := (0, 0);
B := (-3/2u, -3u);
C := (3u, -3u);
D = whatever[A, A shifted (unitvector(B-A) + unitvector(C-A))] = whatever[B, B shifted (unitvector(A-B) + unitvector(C-B))];
E = whatever[D, D shifted ((A-B) rotated 90)] = whatever[A, B];
F = whatever[D, D shifted ((B-C) rotated 90)] = whatever[B, C];
G = whatever[D, D shifted ((C-A) rotated 90)] = whatever[A, C];
r := abs(E-D);
byAngleDefine(B, E, D, byred, SOLID_SECTOR);
byAngleDefine(D, F, B, byred, SOLID_SECTOR);
byAngleDefine(E, B, D, byblue, SOLID_SECTOR);
byAngleDefine(D, B, F, byyellow, SOLID_SECTOR);
byAngleDefine(G, C, D, byblack, SOLID_SECTOR);
byAngleDefine(D, C, F, byblack, ARC_SECTOR);
byAngleDefine(E, D, F, byblack, ARC_SECTOR);
draw byNamedAngleResized();
draw byLine(E, D, byyellow, DASHED_LINE, REGULAR_WIDTH);
draw byLine(G, D, byred, DASHED_LINE, REGULAR_WIDTH);
draw byLine(F, D, byblack, DASHED_LINE, REGULAR_WIDTH);
byLineDefine(C, D, byblue, SOLID_LINE, REGULAR_WIDTH);
byLineDefine(B, D, byblue, DASHED_LINE, REGULAR_WIDTH);
draw byNamedLineSeq(0)(CD,BD);
byLineDefine(A, B, byyellow, SOLID_LINE, REGULAR_WIDTH);
byLineDefine(B, C, byblack, SOLID_LINE, REGULAR_WIDTH);
byLineDefine(C, A, byred, SOLID_LINE, REGULAR_WIDTH);
draw byNamedLineSeq(0)(AB,BC,CA);
draw byCircle.D(D, G, byblack, 0, 0, -1);
byLineDefine(B, F, byblack, SOLID_LINE, REGULAR_WIDTH);
byLineDefine(B, E, byyellow, SOLID_LINE, REGULAR_WIDTH);
draw byLabelsOnPolygon(B, E, A, G, C, F)(ALL_LABELS, 0);
draw byLabelsOnPolygon(E, D, G)(OMIT_FIRST_LABEL+OMIT_LAST_LABEL, 0);
}
\drawCurrentPictureInMargin
\problem{X}{X}{En un triángulo dado \drawFromCurrentPicture[bottom]{
startTempScale(1/4);
startAutoLabeling;
draw byNamedLineSeq(0)(CA,BC,AB);
stopAutoLabeling;
stopTempScale;
} para inscribir un círculo.}

\begin{center}
Biseca \drawAngle{EBD,DBF} y \drawAngle{GCD,DCF} (L. 1. pr. 9.) por \drawUnitLine{BD} y \drawUnitLine{CD}; desde el punto donde estas líneas se unen dibuja \drawUnitLine{FD}, \drawUnitLine{ED} y \drawUnitLine{GD} perpendiculares a \drawUnitLine{BC}, \drawUnitLine{AB} y \drawUnitLine{CA} respectivamente.

En \drawFromCurrentPicture{
startTempScale(3/4);
draw byNamedAngle(F, DBF);
startAutoLabeling;
draw byNamedLineSeq(0)(BD,FD,BF);
stopAutoLabeling;
stopTempScale;
} y \drawFromCurrentPicture{
startTempScale(3/4);
draw byNamedAngle(E, EBD);
startAutoLabeling;
draw byNamedLineSeq(0)(BE,ED,BD);
stopAutoLabeling;
stopTempScale;
}  \\ \drawAngle{DBF} $=$ \drawAngle{EBD}, \drawAngle{F} $=$ \drawAngle{E} y \drawUnitLine{BD} común,
                     \\ $\therefore$ \drawUnitLine{ED} $=$ \drawUnitLine{FD} (L. 1. pr. 4 y 26.)

De igual manera, se puede mostrar también
                     \\ que \drawUnitLine{GD} $=$ \drawUnitLine{FD},
 \\ $\therefore$ \drawUnitLine{FD} $=$ \drawUnitLine{ED} $=$ \drawUnitLine{GD};
 \\ por lo tanto, con cualquiera de estas líneas como radio, traza 

y pasará por las extremos de los otros dos; y los lados del triángulo dado, que son perpendiculares a los tres radios en sus extremos, tocan el círculo (L. 3. pr. 16.), que por lo tanto está inscrito en el triángulo dado.

Q. E. D. \byref{prop:I.IX} \byref{prop:I.IV,prop:I.XXVI} \byref{prop:III.XVI}
\end{center}

\qed

\startproblem{Prop. V. prob.}\label{prop:IV.V}

\defineNewPicture[1]{
def proptmp (expr a, b, c, s) =
pair A, B, C, D, E, F;
numeric r;
r := 7/4u;
F := s;
A := (dir(a)*r) shifted F;
B := (dir(b)*r) shifted F;
C := (dir(c)*r) shifted F;
D := 1/2[A, B];
E := 1/2[A, C];
byAngleDefine(A, D, F, byred, SOLID_SECTOR);
byAngleDefine(F, D, B, byblack, SOLID_SECTOR);
draw byNamedAngleResized();
draw byLine(D, F, byyellow, SOLID_LINE, REGULAR_WIDTH);
draw byLine(E, F, byyellow, DASHED_LINE, REGULAR_WIDTH);
draw byLine(A, F, byblack, DASHED_LINE, REGULAR_WIDTH);
byLineDefine(A, D, byblue, SOLID_LINE, REGULAR_WIDTH);
byLineDefine(D, B, byblue, DASHED_LINE, REGULAR_WIDTH);
draw byNamedLineSeq(0)(AD,DB);
if (sind(b)<>-sind(c)):
byLineDefine(B, F, byblack, SOLID_LINE, REGULAR_WIDTH);
byLineDefine(C, F, byblack, SOLID_LINE, THIN_WIDTH);
draw byNamedLineSeq(0)(BF,CF);
draw byLine(B, C, byred, SOLID_LINE, REGULAR_WIDTH);
else:
draw byLine(B, C, byblack, SOLID_LINE, REGULAR_WIDTH);
fi;
byLineDefine(C, E, byred, SOLID_LINE, REGULAR_WIDTH);
byLineDefine(E, A, byred, DASHED_LINE, REGULAR_WIDTH);
draw byNamedLineSeq(0)(CE,EA);
draw byCircleR(F, r, byblack, 0, 0, 0);
draw byLabelsOnCircle(A, B, C)(F);
draw byLabelsOnPolygon(B, D, A)(OMIT_FIRST_LABEL+OMIT_LAST_LABEL, 0);
draw byLabelsOnPolygon(A, E, C)(OMIT_FIRST_LABEL+OMIT_LAST_LABEL, 0);
draw byLabelsOnPolygon(C, F, B)(OMIT_FIRST_LABEL+OMIT_LAST_LABEL, 0);
enddef;
proptmp(100, 170, 10, (0, -8u));
proptmp(80, 160, -20, (0, -4u));
proptmp(100, 200, -30, (0, 0));
}
\drawCurrentPictureInMargin
\problem{P}{ara}{ trazar un círculo sobre un triángulo dado.}

\begin{center}
Haz \drawUnitLine{AD} $=$ \drawUnitLine{DB} y \drawUnitLine{CE} $=$ \drawUnitLine{EA} (L. 1. pr. 10.) Desde los puntos de bisección dibuja \drawUnitLine{DF} y \drawUnitLine{EF} ⊥ \drawUnitLine{AD} y \drawUnitLine{CE} respectivamente (L. 1. pr. 11.), y desde su punto de concurrencia dibuja \drawUnitLine{BF}, \drawUnitLine{AF} y \drawUnitLine{CF} y traza un círculo con cualquiera de ellos, y será el círculo requerido.

En \drawFromCurrentPicture{
draw byNamedAngle(FDB);
startAutoLabeling;
draw byNamedLineSeq(0)(DF,BF,DB);
stopAutoLabeling;
} y \drawFromCurrentPicture{
draw byNamedAngle(ADF);
startAutoLabeling;
draw byNamedLineSeq(0)(AF,DF,AD);
stopAutoLabeling;
}  \\ \drawUnitLine{DB} $=$ \drawUnitLine{AD} (const.),
                     \\ \drawUnitLine{DF} común,
                     \\ \drawAngle{FDB} $=$ \drawAngle{ADF} (const.),
                     \\ $\therefore$ \drawUnitLine{AF} $=$ \drawUnitLine{BF} (L. 1. pr. 4.).

De igual manera se puede demostrar que
                     \\ \drawUnitLine{CF} $=$ \drawUnitLine{AF}.

$\therefore$ \drawUnitLine{AF} $=$ \drawUnitLine{BF} $=$ \drawUnitLine{CF}; y, por lo tanto, un círculo descrito desde la concurrencia de estas tres líneas con cualquiera de ellas como un radio circunscribirá el triángulo dado.

Q. E. D. \byref{prop:I.X} \byref{prop:I.XI} \byref{\constref} \byref{\constref} \byref{prop:I.IV}
\end{center}

\qed

\startproblem{Prop. VI. prob.}\label{prop:IV.VI}

\defineNewPicture{
pair A, B, C, D, E;
numeric r;
r := 7/4u;
E := (0, 0);
A := (0, r);
B := (-r, 0);
C := (0, -r);
D := (r, 0);
byAngleDefine(D, A, B, byyellow, SOLID_SECTOR);
byAngleDefine(C, D, A, byblack, SOLID_SECTOR);
byAngleDefine(A, E, B, byblue, SOLID_SECTOR);
byAngleDefine(D, E, A, byred, SOLID_SECTOR);
draw byNamedAngleResized();
draw byLine(A, E, byblack, DASHED_LINE, REGULAR_WIDTH);
draw byLine(C, E, byblack, SOLID_LINE, THIN_WIDTH);
byLineDefine(B, E, byred, DASHED_LINE, REGULAR_WIDTH);
byLineDefine(D, E, byblue, DASHED_LINE, REGULAR_WIDTH);
draw byNamedLineSeq(0)(BE, DE);
byLineDefine(A, B, byblack, SOLID_LINE, REGULAR_WIDTH);
byLineDefine(B, C, byyellow, SOLID_LINE, REGULAR_WIDTH);
byLineDefine(C, D, byblue, SOLID_LINE, REGULAR_WIDTH);
byLineDefine(D, A, byred, SOLID_LINE, REGULAR_WIDTH);
draw byNamedLineSeq(0)(AB,BC,CD,DA);
draw byCircleR(E, r, byred, 0, 0, 1);
draw byLabelsOnCircle(A, B, C, D)(E);
draw byLabelsOnPolygon(C, E, B)(OMIT_FIRST_LABEL+OMIT_LAST_LABEL, 0);
}
\drawCurrentPictureInMargin
\problem{E}{e}{ un círculo dado \drawUnitLine{BC} para inscribir un cuadrado.}

\begin{center}
Dibuja los dos diámetros del círculo ⊥ entre sí y dibuja \drawUnitLine{AB}, \drawUnitLine{DA}, \drawUnitLine{CD} y \drawLine[middle][squareABCD]{DA,CD,BC,AB}

\drawAngle{A} es un cuadrado.

Porque desde \drawAngle{D} y \drawUnitLine{CD} son, cada uno de ellos, en un semicírculo, ángulos rectos (L. 3. pr. 31),
                     \\ $\therefore$ \drawUnitLine{AB} ∥ \drawUnitLine{DA} (L. 1. pr. 28):
                     \\ y de la misma manera \drawUnitLine{BC} ∥ \drawAngle{AEB}.

Y porque \drawAngle{DEA} $=$ \drawUnitLine{BE} (const.), y
                     \\ \drawUnitLine{AE} $=$ \drawUnitLine{DE} $=$ \drawUnitLine{AB} (L. 1. def. 15).
                     \\ $\therefore$ \drawUnitLine{DA} $=$ \square (L. 1. pr. 4);

y ya que los lados y ángulos adyacentes del paralelogramo \square son iguales, ellos son todos iguales (L. 1. pr. 34); y $\therefore$ , inscrito en el círculo dado, es un cuadrado.

Q. E. D. \byref{prop:III.XXXI} \byref{prop:I.XXVIII} \byref{\constref} \byref{def:I.XV} \byref{prop:I.IV} \byref{prop:I.XXXIV}
\end{center}

\qed

\startproblem{Prop. VII. prob.}\label{prop:IV.VII}

\defineNewPicture[1/4]{
pair A, B, C, D, E, F, G, H, K;
numeric r;
r := 7/4u;
E := (0, 0);
A := (0, r);
B := (-r, 0);
C := (0, -r);
D := (r, 0);
F := (r, r);
G := (-r, r);
H := (-r, -r);
K := (r, -r);
byAngleDefine(F, G, H, byred, SOLID_SECTOR);
byAngleDefine(G, H, K, byred, SOLID_SECTOR);
byAngleDefine(H, K, F, byred, SOLID_SECTOR);
byAngleDefine(K, F, G, byred, SOLID_SECTOR);
byAngleDefine(B, E, C, byblack, SOLID_SECTOR);
byAngleDefine(E, C, H, byyellow, SOLID_SECTOR);
draw byNamedAngleResized();
draw byLine(A, C, byred, DASHED_LINE, REGULAR_WIDTH);
draw byLine(B, D, byblue, DASHED_LINE, REGULAR_WIDTH);
byLineDefine(F, G, byblack, SOLID_LINE, REGULAR_WIDTH);
byLineDefine(G, H, byred, SOLID_LINE, REGULAR_WIDTH);
byLineDefine(H, K, byblue, SOLID_LINE, REGULAR_WIDTH);
byLineDefine(K, F, byyellow, SOLID_LINE, REGULAR_WIDTH);
draw byNamedLineSeq(0)(KF,HK,GH,FG);
draw byCircleR(E, r, byblue, 0, 0, -1);
draw byLabelsOnPolygon(H, B, G, A, F, D, K, C)(ALL_LABELS, 0);
draw byLabelsOnPolygon(A, E, D)(OMIT_FIRST_LABEL+OMIT_LAST_LABEL, 0);
}
\drawCurrentPictureInMargin
\problem{S}{obre}{ un círculo dado \drawUnitLine{HK} para circunscribir un cuadrado.}

\begin{center}
Dibuja dos diámetros del círculo dado perpendicularmente entre sí, y a través de sus extremos dibuja \drawUnitLine{GH}, \drawUnitLine{FG}, \drawUnitLine{KF}, y \drawLine[bottom][squareFGHK]{KF,HK,GH,FG} tangentes al círculo;

y \drawAngle{C} es un cuadrado.

\drawAngle{E} $=$ 
Right angle

 un ángulo recto (L. 3. pr. 18.)
                     \\ también \drawUnitLine{HK} $=$ 
Right angle

 (const.),

$\therefore$ \drawUnitLine{BD} ∥ \drawUnitLine{FG}; de la misma manera se puede demostrar que \drawUnitLine{BD} ∥ \drawUnitLine{GH}, y también que \drawUnitLine{KF} y \drawUnitLine{AC} ∥ \square;

$\therefore$ \drawAngle{C} es un paralelogramo, y
                     \\ porque \drawAngle{G} $=$ \drawAngle{F} $=$ \drawAngle{K} $=$ \drawAngle{H} $=$ \drawUnitLine{HK}  \\ todos ellos son ángulos rectos (L. 1. pr. 34):
                     \\ también es evidente que \drawUnitLine{GH}, \drawUnitLine{FG}, \drawUnitLine{KF} y \square son iguales.

$\therefore$  es un cuadrado.

Q. E. D. \byref{prop:III.XVIII} \byref{\constref} \byref{prop:I.XXXIV}
\end{center}

\qed

\startproblem{Prop. VIII. prob.}\label{prop:IV.VIII}

\defineNewPicture[1/2]{
pair A, B, C, D, E, F, G, H, K;
numeric r;
r := 7/4u;
G := (0, 0);
A := (-r, r);
B := (-r,-r);
C := (r, -r);
D := (r, r);
E := (0, r);
F := (-r, 0);
H := (0, -r);
K := (r, 0);
draw byPolygon(E,D,K,G)(byblack);
draw byPolygon(F,G,H,B)(byred);
draw byPolygon(G,K,C,H)(byblue);
draw byLine(F, G, byyellow, SOLID_LINE, REGULAR_WIDTH);
draw byLine(G, K, byyellow, DASHED_LINE, REGULAR_WIDTH);
draw byLine(G, H, byblack, SOLID_LINE, REGULAR_WIDTH);
draw byLine(E, G, byblack, DASHED_LINE, REGULAR_WIDTH);
byLineDefine(A, B, byblack, SOLID_LINE, REGULAR_WIDTH);
byLineDefine(A, E, byblue, DASHED_LINE, REGULAR_WIDTH);
byLineDefine(E, D, byblue, SOLID_LINE, REGULAR_WIDTH);
byLineDefine(D, K, byred, DASHED_LINE, REGULAR_WIDTH);
byLineDefine(K, C, byred, SOLID_LINE, REGULAR_WIDTH);
draw byNamedLineSeq(0)(AB,AE,ED,DK,KC);
draw byCircleR(G, r, byyellow, 0, 0, -1);
draw byLabelsOnPolygon(B, F, A, E, D, K, C, H)(ALL_LABELS, 0);
draw byLabelsOnPolygon(F, G, E)(OMIT_FIRST_LABEL+OMIT_LAST_LABEL, 0);
}
\drawCurrentPictureInMargin
\problem{P}{ara}{ inscribir un círculo en un cuadrado dado.}

\begin{center}
Haz \drawUnitLine{ED} $=$ \drawUnitLine{AE},
 \\ y \drawUnitLine{KC} $=$ \drawUnitLine{DK},
 \\ dibuja \drawUnitLine{FG,GK} ∥ \drawUnitLine{AE,ED},
 \\ y \drawUnitLine{EG,GH} ∥ \drawUnitLine{DK,KC}  \\ (L. 1. pr. 31.)

$\therefore$ \drawUnitLine{AE,ED} es un paralelogramo;
                     \\ y ya que \drawUnitLine{DK,KC} $=$ \drawUnitLine{ED} (hip.)
                     \\ \drawUnitLine{DK} $=$ \polygon

$\therefore$ \drawUnitLine{AE} es equilátero (L. 1. pr. 34.)

Del mismo modo, se puede demostrar que
                     \\ \drawUnitLine{GK} $=$ \drawUnitLine{GH} son paralelogramos equiláteros;
                     \\ $\therefore$ \drawUnitLine{FG} $=$  $=$  $=$ ,

y, por lo tanto, si se traza un círculo desde la concurrencia de estas líneas con cualquiera de ellas como radio, se inscribirá en el cuadrado dado. (L. 3. pr. 16.)

Q. E. D. \byref{prop:I.XXXI} \byref{\hypref} \byref{prop:I.XXXIV} \byref{prop:III.XVI}
\end{center}

\qed

\startproblem{Prop. IX. prob.}\label{prop:IV.IX}

\defineNewPicture{
pair A, B, C, D, E;
numeric r;
r := 7/4u;
E := (0, 0);
A := (dir(45+90)*r) shifted E;
B := (dir(45+180)*r) shifted E;
C := (dir(45+270)*r) shifted E;
D := (dir(45+360)*r) shifted E;
draw byPolygon(A,C,D)(byred);
draw byPolygon(A,B,C)(byyellow);
byAngleDefine(D, A, E, byyellow, SOLID_SECTOR);
byAngleDefine(E, A, B, byblack, SOLID_SECTOR);
byAngleDefine(A, B, E, byred, SOLID_SECTOR);
byAngleDefine(E, B, C, byblue, SOLID_SECTOR);
draw byNamedAngleResized();
draw byLine(A, E, byblue, SOLID_LINE, REGULAR_WIDTH);
draw byLine(E, C, byblue, DASHED_LINE, REGULAR_WIDTH);
byLineDefine(B, E, byblack, SOLID_LINE, REGULAR_WIDTH);
byLineDefine(E, D, byblack, DASHED_LINE, REGULAR_WIDTH);
draw byNamedLineSeq(0)(BE, ED);
draw byCircleR(E, r, byblack, 0, 0, 0);
draw byLabelsOnCircle(A, B, C, D)(E);
draw byLabelsOnPolygon(B, E, A)(OMIT_FIRST_LABEL+OMIT_LAST_LABEL, 1);
}
\drawCurrentPictureInMargin
\problem{P}{ara}{ trazar un círculo sobre un cuadrado dado \drawUnitLine{AE,EC}.}

\begin{center}
Dibuja las diagonales \drawUnitLine{BE,ED} y \drawUnitLine{AE,EC} intersecándose entre ellas, luego

porque \drawAngle{DAE} y \drawAngle{EAB} tienen
                 \\ sus lados iguales, y la base
                 \\ \drawAngle{DAE,EAB} común a ambos,
                 \\ \drawAngle{ABE,EBC} $=$ \drawAngle{DAE,EAB} (L. 1. pr. 8),
                 \\ o \drawAngle{ABE,EBC} es bisecado: de la misma manera se puede demostrar
                 \\ que \drawAngle{EAB} es bisecado;
                 \\ pero \drawAngle{ABE} $=$ \drawUnitLine{BE},
 \\ por consiguiente \drawUnitLine{AE} $=$ \drawUnitLine{AE} sus mitades,
                 \\ $\therefore$ \drawUnitLine{BE} $=$ \drawUnitLine{EC}; (L. 1. pr. 6.)
                 \\ y de la misma manera se puede demostrar que
                 \\ \drawUnitLine{ED} $=$  $=$  $=$ .

Si a partir de la confluencia de estas líneas con cualquiera de ellas como radio, se puede trazar un círculo, este circunscribirá el cuadrado dado.

Q. E. D. \byref{prop:I.VIII} \byref{prop:I.VI}
\end{center}

\qed

\startproblem{Prop. X. prob.}\label{prop:IV.X}

\defineNewPicture[1/2]{
pair A, B, C, D, E, F;
numeric r[];
path cr[];
r1 := 11/4u;
A := (0, 0);
cr1 := (fullcircle scaled 2r1) shifted A;
B := (r1, 0);
r2 := r1*sqrt(5)/2 - 1/2r1;
cr2 := (fullcircle scaled 2r2) shifted B;
C := (r2, 0);
D := cr1 intersectionpoint (subpath (0, 4) of cr2);
F = whatever[1/2[A, C], 1/2[A, C] shifted ((A-C) rotated 90)]
 = whatever[1/2[A, D], 1/2[A, D] shifted ((A-D) rotated 90)];
r3 := abs(F-A);
byAngleDefine(D, A, B, byblack, ARC_SECTOR);
byAngleDefine(C, D, A, byblack, SOLID_SECTOR);
byAngleDefine(B, D, C, byyellow, SOLID_SECTOR);
byAngleDefine(D, C, B, byblue, SOLID_SECTOR);
byAngleDefine(C, B, D, byred, SOLID_SECTOR);
draw byNamedAngleResized();
draw byLine(A, D, byyellow, SOLID_LINE, REGULAR_WIDTH);
draw byLine(B, D, byblue, SOLID_LINE, REGULAR_WIDTH);
draw byLine(C, D, byred, SOLID_LINE, REGULAR_WIDTH);
draw byLine(A, C, byblack, SOLID_LINE, REGULAR_WIDTH);
draw byLine(C, B, byblack, DASHED_LINE, REGULAR_WIDTH);
draw byCircleABC.F(A, C, D, byblue, 0, 0, 0);
draw byCircle.A(A, B, byred, 0, 0, 0);
draw byLabelsOnCircle(B)(A);
draw byLabelsOnCircle(A)(F);
draw byLabelPoint(D, angle(D-A)-15, 2);
draw byLabelsOnPolygon(A, C, D)(OMIT_FIRST_LABEL+OMIT_LAST_LABEL, 0);
}
\drawCurrentPictureInMargin
\problem{P}{ara}{ construir un triángulo isósceles, en el que cada uno de los ángulos en la base debe ser el doble del ángulo vertical.}

\begin{center}
Toma cualquier línea recta \drawLine{AD,BD,CB,AC} y divídela para que
                     \\ \drawLine{AD,CD,AC} × \drawAngle{BDC} $=$ \drawAngle{A}2 (L. 2. pr. 11.)

Con \drawAngle{CDA} como radio, traza \drawAngle{BDC} y coloca
                     \\ en este desde los extremos del radio, \drawAngle{CDA} $=$ \drawAngle{A}, (L. 4. pr. 1.); dibuja \drawAngle{CDA}.

Entonces \drawAngle{BDC} es el triángulo requerido.

Para, dibujar \drawAngle{CDA} y trazar
                     \\ \drawAngle{BDC,CDA} sobre \drawAngle{B} (L. 4. pr. 5.)

Desde \drawAngle{B} × \drawAngle{A} $=$ \drawAngle{CDA}2 $=$ \drawAngle{C}2,
 \\ $\therefore$ \drawAngle{A} es tangente a \drawAngle{CDA} (L. 3. pr. 37.)
                     \\ $\therefore$ \drawAngle{B} $=$ \drawAngle{BDC,CDA} (L. 3. pr. 32),
                     \\ agrega \drawAngle{C} a cada uno
                     \\ $\therefore$ \drawAngle{A} + \drawAngle{CDA} $=$ \drawAngle{A} + ;
 \\ pero  +  o  $=$  (L. 1. pr. 5):
                     \\ ya que  $=$  (L. 1. pr. 5.)
                     \\ consecuentemente  $=$  +  $=$  (L. 1. pr. 32.)
                     \\ $\therefore$  $=$  (L. 1. pr. 6.)
                     \\ $\therefore$  $=$  $=$  (const.)
                     \\ $\therefore$  $=$  (L. 1. pr. 5.)

$\therefore$  $=$  $=$  $=$  +  $=$ dos veces ; y en consecuencia, cada ángulo en la base es el doble del ángulo vertical.

Q. E. D. \byref{prop:II.XI} \byref{prop:IV.I} \byref{prop:IV.V} \byref{prop:III.XXXVII} \byref{prop:III.XXXII} \byref{prop:I.V} \byref{prop:I.V} \byref{prop:I.XXXII} \byref{prop:I.VI} \byref{\constref} \byref{prop:I.V}
\end{center}

\qed

\startproblem{Prop. XI. prob.}\label{prop:IV.XI}

\defineNewPicture[1/4]{
pair A, B, C, D, E, O;
numeric r;
r := 9/4u;
O := (0, 0);
A := (dir(90+0/5(360))*r) shifted O;
B := (dir(90+1/5(360))*r) shifted O;
C := (dir(90+2/5(360))*r) shifted O;
D := (dir(90+3/5(360))*r) shifted O;
E := (dir(90+4/5(360))*r) shifted O;
draw byArbitraryFigure.pg(C--E--B--D, byblack, 0, 1);
draw byPolygon(A,C,D)(byyellow);
byAngleDefineExtended(A, B, C, byblack, -1)(bytransparent);
byAngleDefineExtended(D, E, A, byblack, -1)(bytransparent);
byAngleDefine.Cl(B, C, A, byblack, -ARC_SECTOR);
byAngleDefine.Dr(A, D, E, byblack, -ARC_SECTOR);
byAngleDefine.Al(B, A, C, byblack, -ARC_SECTOR);
byAngleDefine.Ar(D, A, E, byblack, -ARC_SECTOR);
byAngleDefine(C, A, D, byblack, SOLID_SECTOR);
byAngleDefineExtended(E, C, A, byblack, 1)(byyellow);
byAngleDefine(D, C, E, byblue, SOLID_SECTOR);
byAngleDefine(A, D, B, byblack, ARC_SECTOR);
byAngleDefine(B, D, C, byred, SOLID_SECTOR);
draw byNamedAngleResized();
draw byNamedAngleDummySides(ADB,BDC);
byLineDefine(A, B, byblue, SOLID_LINE, REGULAR_WIDTH);
byLineDefine(B, C, byred, SOLID_LINE, REGULAR_WIDTH);
byLineDefine(C, D, byblack, SOLID_LINE, REGULAR_WIDTH);
byLineDefine(D, E, byred, DASHED_LINE, REGULAR_WIDTH);
byLineDefine(E, A, byyellow, SOLID_LINE, REGULAR_WIDTH);
draw byNamedLineSeq(0)(AB,BC,CD,DE,EA);
draw byCircleR(O, r, byblue, 0, 0, 1);
draw byLabelsOnCircle(A, B, C, D, E)(O);
}
\drawCurrentPictureInMargin
\problem{E}{n}{ un círculo dado \drawAngle{ECA,DCE} para inscribir un pentágono equilátero y equiangular.}

\begin{center}
Construye un triángulo isósceles, en el que cada uno de los ángulos en la base sea el doble del ángulo en el vértice, e inscribe en el círculo dado un triángulo \drawAngle{ADB,BDC} equiangular a él; (L. 4. pr. 2.)

Biseca \drawUnitLine{BC} y \drawUnitLine{AB} (L. 1. pr. 9.)
                     \\ dibuja \drawUnitLine{EA}, \drawUnitLine{DE}, \drawAngle{ECA} y \drawAngle{DCE}.

Porque cada uno de los ángulos
                     \\  \drawAngle{CAD}, \drawAngle{BDC}, \drawAngle{ADB}, \drawUnitLine{CD} y \drawUnitLine{BC} son iguales,

los arcos sobre los cuales se encuentran son iguales (L. 3. pr. 26.) y $\therefore$ \drawUnitLine{AB}, \drawUnitLine{EA}, \drawUnitLine{DE},  y  que subtienden estos arcos son iguales (L. 3. pr. 29.) y $\therefore$ el pentágono es equilátero, también es equiangular, ya que cada uno de sus ángulos se encuentra en arcos iguales. (L. 3. pr. 27).

Q. E. D. \byref{prop:IV.II} \byref{prop:I.IX} \byref{prop:III.XXVI} \byref{prop:III.XXIX} \byref{prop:III.XXVII}
\end{center}

\qed

\startproblem{Prop. XII. prob.}\label{prop:IV.XII}

\defineNewPicture[1/2]{
pair B, C, D, G, H, K, L, M, F;
numeric r[];
r1 := 5/2u;
F := (0, 0);
G := (dir(90+0/5(360))*r1) shifted F;
H := (dir(90+1/5(360))*r1) shifted F;
K := (dir(90+2/5(360))*r1) shifted F;
L := (dir(90+3/5(360))*r1) shifted F;
M := (dir(90+4/5(360))*r1) shifted F;
B := 1/2[H, K];
C := 1/2[K, L];
D := 1/2[L, M];
r2 := abs(F-B);
byAngleDefine(B, F, K, byred, SOLID_SECTOR);
byAngleDefine(K, F, C, byyellow, SOLID_SECTOR);
byAngleDefine(C, F, L, byblue, SOLID_SECTOR);
byAngleDefine(L, F, D, byred, SOLID_SECTOR);
byAngleDefine(F, K, B, byyellow, ARC_SECTOR);
byAngleDefine(C, K, F, byblack, SOLID_SECTOR);
byAngleDefine(F, C, K, byblue, ARC_SECTOR);
byAngleDefine(L, C, F, byblue, ARC_SECTOR);
byAngleDefine(F, L, C, byblack, SOLID_SECTOR);
byAngleDefine(D, L, F, byblack, ARC_SECTOR);
draw byNamedAngleResized();
draw byLine(F, K, byblue, SOLID_LINE, REGULAR_WIDTH);
draw byLine(F, C, byblack, DASHED_LINE, REGULAR_WIDTH);
draw byLine(F, L, byblack, SOLID_LINE, REGULAR_WIDTH);
byLineDefine(F, B, byred, DASHED_LINE, REGULAR_WIDTH);
byLineDefine(F, D, byyellow, DASHED_LINE, REGULAR_WIDTH);
draw byNamedLineSeq(0)(FB,FD);
byLineDefine(G, H, byblack, SOLID_LINE, REGULAR_WIDTH);
byLineDefine(H, B, byblue, DASHED_LINE, REGULAR_WIDTH);
byLineDefine(B, K, byblack, SOLID_LINE, REGULAR_WIDTH);
byLineDefine(K, C, byred, SOLID_LINE, REGULAR_WIDTH);
byLineDefine(C, L, byyellow, SOLID_LINE, REGULAR_WIDTH);
byLineDefine(L, M, byblack, SOLID_LINE, REGULAR_WIDTH);
byLineDefine(M, G, byblack, SOLID_LINE, REGULAR_WIDTH);
draw byNamedLineSeq(0)(GH,HB,BK,KC,CL,LM,MG);
draw byCircleR(F, r2, byred, 0, 0, -1);
draw byLabelsOnPolygon(M, D, L, C, K, B, H, G)(OMIT_FIRST_LABEL+OMIT_LAST_LABEL, 0);
draw byLabelsOnPolygon(B, F, D)(OMIT_FIRST_LABEL+OMIT_LAST_LABEL, 0);
}
\drawCurrentPictureInMargin
\problem{P}{ara}{ trazar un pentágono equilátero y equiangular sobre un círculo dado \drawLine{FB,FK,BK}.}

\begin{center}
Dibuja cinco tangentes a través de los vértices de los ángulos de cualquier pentágono regular inscrito en el círculo dado \drawLine{FK,FC,KC} (L. 3. pr. 17).

Estas cinco tangentes formarán el pentágono requerido.

Dibuja
                    


\drawAngle{FKB}
\drawAngle{CKF}
\drawAngle{BFK}
\drawAngle{KFC}



                    .
                    En \drawAngle{FKB,CKF} y \drawAngle{CKF}  \\ \drawAngle{BFK,KFC} $=$ \drawAngle{KFC} (L. 1. pr. 47),
                     \\ \drawAngle{FLC,DLF} $=$ \drawAngle{FLC}, y \drawAngle{CFL,LFD} común;
                     \\ $\therefore$ \drawAngle{CFL} $=$ \drawAngle{BFK,KFC} y \drawAngle{CFL,LFD} $=$ \drawAngle{KFC} (L. 1. pr. 8.)
                     \\ $\therefore$ \drawAngle{CFL} $=$ dos veces \drawAngle{FCK}, y \drawAngle{LCF} $=$ dos veces \drawAngle{CKF};

De la misma manera se puede demostrar que
                     \\ \drawAngle{FLC} $=$ dos veces \drawAngle{FLC,DLF}, y \drawAngle{FLC} $=$ dos veces \drawAngle{FKB,CKF};
 \\ pero \drawAngle{CKF} $=$ \drawAngle{CKF} (L. 3. pr. 27),
                     \\ $\therefore$ sus mitades \drawAngle{FLC} $=$ \drawAngle{FLC,DLF}, también \drawAngle{FKB,CKF} $=$ , y
                     \\  común;

$\therefore$  $=$  y  $=$ ,
 \\ $\therefore$  $=$ dos veces ;
 \\ De la misma manera, se puede demostrar
                     \\ que  $=$ dos veces ,
 \\ pero  $=$   \\ $\therefore$  $=$ ;

De la misma manera, se puede demostrar que los otros lados son iguales y, por lo tanto, el pentágono es equilátero, también es equiangular, porque

 $=$ dos veces  and  $=$ dos veces ,
 \\ y por lo tanto  $=$ ,
 \\ $\therefore$  $=$ ; de la misma manera se puede demostrar que los otros ángulos del pentágono trazado son iguales.

Q. E. D. \byref{prop:III.XVII} \byref{prop:I.XLVII} \byref{prop:I.VIII} \byref{prop:III.XXVII}
\end{center}

\qed

\startproblem{Prop. XIII. prob.}\label{prop:IV.XIII}

\defineNewPicture{
pair A, B, C, D, E, F, G, H, K, L, M;
numeric r[];
r1 := 5/2u;
F := (0, 0);
A := (dir(90+0/5(360))*r1) shifted F;
B := (dir(90+1/5(360))*r1) shifted F;
C := (dir(90+2/5(360))*r1) shifted F;
D := (dir(90+3/5(360))*r1) shifted F;
E := (dir(90+4/5(360))*r1) shifted F;
G := 1/2[A, B];
H := 1/2[B, C];
K := 1/2[C, D];
L := 1/2[D, E];
M := 1/2[E, A];
r2 := abs(F-G);
byAngleDefine(G, B, F, byyellow, SOLID_SECTOR);
byAngleDefine(F, B, H, byyellow, SOLID_SECTOR);
byAngleDefine(F, H, C, byblack, SOLID_SECTOR);
byAngleDefine(H, C, F, byblue, SOLID_SECTOR);
byAngleDefine(F, C, K, byblue, SOLID_SECTOR);
byAngleDefine(C, K, F, byblack, SOLID_SECTOR);
byAngleDefine(K, D, F, byred, SOLID_SECTOR);
byAngleDefine(F, D, E, byred, SOLID_SECTOR);
byAngleDefine(D, E, F, byblack, ARC_SECTOR);
byAngleDefine(F, E, M, byblack, ARC_SECTOR);
draw byNamedAngleResized();
draw byLine(F, G, byblack, SOLID_LINE, THIN_WIDTH);
draw byLine(F, M, byblack, SOLID_LINE, THIN_WIDTH);
draw byLine(F, A, byblack, SOLID_LINE, THIN_WIDTH);
draw byLine(F, H, byblue, DASHED_LINE, REGULAR_WIDTH);
draw byLine(F, K, byblue, SOLID_LINE, REGULAR_WIDTH);
draw byLine(F, C, byblack, SOLID_LINE, REGULAR_WIDTH);
draw byLine(F, D, byyellow, DASHED_LINE, REGULAR_WIDTH);
byLineDefine(F, B, byred, SOLID_LINE, REGULAR_WIDTH);
byLineDefine(F, E, byred, DASHED_LINE, REGULAR_WIDTH);
draw byNamedLineSeq(0)(FB,FE);
byLineDefine(A, B, byblack, SOLID_LINE, THIN_WIDTH);
byLineDefine(B, H, byblack, DASHED_LINE, REGULAR_WIDTH);
byLineDefine(H, C, byyellow, SOLID_LINE, REGULAR_WIDTH);
byLineDefine(C, K, byyellow, SOLID_LINE, REGULAR_WIDTH);
byLineDefine(K, D, byblack, DASHED_LINE, REGULAR_WIDTH);
byLineDefine(D, E, byblack, SOLID_LINE, THIN_WIDTH);
byLineDefine(E, A, byblack, SOLID_LINE, THIN_WIDTH);
draw byNamedLineSeq(0)(AB,BH,HC,CK,KD,DE,EA);
draw byCircle.F(F, H, byyellow, 0, 0, -1);
draw byLabelsOnPolygon(C, H, B, A, E, D, K)(ALL_LABELS, 0);
draw byLabelsOnPolygon(E, F, D)(OMIT_FIRST_LABEL+OMIT_LAST_LABEL, 2);
}
\drawCurrentPictureInMargin
\problem{P}{ara}{ inscribir un círculo en un pentágono equiangular y equilátero dado.}

\begin{center}
Sea \drawLine[bottom]{EA,DE,KD,CK,HC,BH,AB} un pentágono equiangular y equilátero dado; se requiere inscribir un círculo en él.

Haz \drawAngle{HCF} $=$ \drawAngle{FCK}, y \drawAngle{FDE} $=$ \drawAngle{KDF} (L. 1. pr. 9.)

Dibuja \drawUnitLine{FD}, \drawUnitLine{FC}, \drawUnitLine{FB}, \drawUnitLine{FE}, etcétera.
                     \\ Porque \drawUnitLine{KD,CK} $=$ \drawUnitLine{HC,BH}, \drawAngle{HCF} $=$ \drawAngle{FCK},
 \\ \drawUnitLine{FC} común a los dos triángulos
                     \\ \drawFromCurrentPicture{
startTempScale(1/2);
draw byNamedAngle(FBH,HCF);
startAutoLabeling;
draw byNamedLineSeq(0)(FB,FC,HC,BH);
stopAutoLabeling;
stopTempScale;
} y \drawFromCurrentPicture{
startTempScale(1/2);
draw byNamedAngle(FCK,KDF);
startAutoLabeling;
draw byNamedLineSeq(0)(FC,FD,KD,CK);
stopAutoLabeling;
stopTempScale;
};
 \\ $\therefore$ \drawUnitLine{FB} $=$ \drawUnitLine{FD} and \drawAngle{FBH} $=$ \drawAngle{KDF} (L. 1. pr. 4.)

Y porque \drawAngle{B} $=$ \drawAngle{D} $=$ dos veces \drawAngle{KDF}  \\ $\therefore$ $=$ dos veces \drawAngle{FBH}, por consiguiente \drawAngle{B} es bisecado por \drawUnitLine{FB}.

De manera similar, puede demostrarse que \drawAngle{DEF,FEM} está bisecado por \drawUnitLine{FE}, y que el ángulo restante del polígono está bisecado de manera similar.

Dibuja \drawUnitLine{FK}, \drawUnitLine{FH}, etcetera, perpendiculares a los lados del pentágono.

Entonces en los dos triángulos \drawFromCurrentPicture{
startTempScale(2/3);
draw byNamedAngle(H,HCF);
startAutoLabeling;
draw byNamedLineSeq(0)(FH,FC,HC);
stopAutoLabeling;
stopTempScale;
} y \drawFromCurrentPicture{
startTempScale(2/3);
draw byNamedAngle(FCK,K);
startAutoLabeling;
draw byNamedLineSeq(0)(CK,FC,FK);
stopAutoLabeling;
stopTempScale;
}  \\ tenemos \drawAngle{HCF} $=$ \drawAngle{FCK}, (const.), \drawUnitLine{FC} común,
                     \\ y \drawAngle{H} $=$ \drawAngle{K} $=$ un ángulo recto;
                     \\ $\therefore$ \drawUnitLine{FK} $=$ \drawUnitLine{FH}. (L. 1. pr. 26.)

Del mismo modo, se puede demostrar que los cinco perpendiculares a los lados del pentágono son iguales entre sí.

Traza  on cualquiera de los perpendiculares como radio, y será el círculo inscrito requerido. Porque si no toca los lados del pentágono, pero los corta, entonces una línea dibujada desde el extremo en ángulo recto hasta el diámetro de un círculo caerá dentro del círculo, lo cual se ha demostrado que es absurdo. (L. 3. pr. 16.)

Q. E. D. \byref{prop:I.IX} \byref{prop:I.IV} \byref{\constref} \byref{prop:I.XXVI} \byref{prop:III.XVI}
\end{center}

\qed

\startproblem{Prop. XIV. prob.}\label{prop:IV.XIV}

\defineNewPicture[1/2]{
pair A, B, C, D, E, F;
numeric r;
r := 9/4u;
F := (0, 0);
A := (dir(90+0/5(360))*r) shifted F;
B := (dir(90+1/5(360))*r) shifted F;
C := (dir(90+2/5(360))*r) shifted F;
D := (dir(90+3/5(360))*r) shifted F;
E := (dir(90+4/5(360))*r) shifted F;
byAngleDefine(C, F, B, byblue, SOLID_SECTOR);
byAngleDefine(D, F, C, byblack, ARC_SECTOR);
byAngleDefine(B, C, F, byblack, SOLID_SECTOR);
byAngleDefine(F, C, D, byyellow, SOLID_SECTOR);
byAngleDefine(C, D, F, byyellow, SOLID_SECTOR);
byAngleDefine(F, D, E, byred, SOLID_SECTOR);
draw byNamedAngleResized();
draw byLine(F, A, byyellow, DASHED_LINE, REGULAR_WIDTH);
draw byLine(F, C, byred, DASHED_LINE, REGULAR_WIDTH);
draw byLine(F, D, byblue, DASHED_LINE, REGULAR_WIDTH);
byLineDefine(F, B, byblack, SOLID_LINE, REGULAR_WIDTH);
byLineDefine(F, E, byyellow, SOLID_LINE, REGULAR_WIDTH);
draw byNamedLineSeq(0)(FB,FE);
byLineDefine(A, B, byblack, SOLID_LINE, THIN_WIDTH);
byLineDefine(B, C, byblue, SOLID_LINE, REGULAR_WIDTH);
byLineDefine(C, D, byred, SOLID_LINE, REGULAR_WIDTH);
byLineDefine(D, E, byblack, DASHED_LINE, REGULAR_WIDTH);
byLineDefine(E, A, byblack, SOLID_LINE, THIN_WIDTH);
draw byNamedLineSeq(0)(AB,BC,CD,DE,EA);
draw byCircleR(F, r, byred, 0, 0, 1);
draw byLabelsOnCircle(A, B, C, D, E)(F);
draw byLabelsOnPolygon(A, F, E)(OMIT_FIRST_LABEL+OMIT_LAST_LABEL, 1);
}
\drawCurrentPictureInMargin
\problem{P}{ara}{ trazar un círculo sobre un pentágono equilátero y equiangular dado.}

\begin{center}
Biseca \drawAngle{BCF,FCD} y \drawAngle{CDF,FDE} por \drawUnitLine{FC} y \drawUnitLine{FD},
                     \\ y desde el punto de sección, dibuja \drawUnitLine{FE}, \drawUnitLine{FA}, y \drawUnitLine{FB}.

\drawAngle{BCF,FCD} $=$ \drawAngle{CDF,FDE},
 \\ \drawAngle{FCD} $=$ \drawAngle{CDF}, $\therefore$ \drawUnitLine{FD} $=$ \drawUnitLine{FC} (L. 1. pr. 6);
                     \\ y ya que en \drawFromCurrentPicture{
draw byNamedAngle(BCF);
startAutoLabeling;
draw byNamedLineSeq(0)(FB,FC,BC);
stopAutoLabeling;
} y \drawFromCurrentPicture{
draw byNamedAngle(FCD);
startAutoLabeling;
draw byNamedLineSeq(0)(FC,FD,CD);
stopAutoLabeling;
},
 \\ \drawUnitLine{BC} $=$ \drawUnitLine{CD}, y \drawUnitLine{FC} común,
                     \\ también \drawAngle{BCF} $=$ \drawAngle{FCD};
 \\ $\therefore$ \drawUnitLine{FB} $=$ \drawUnitLine{FD} (L. 1. pr. 4).

De igual manera se puede demostrar que
                     \\ \drawUnitLine{FA} $=$ \drawUnitLine{FE} $=$ \drawUnitLine{FB}, y
                     \\ por consiguiente \drawUnitLine{FA} $=$ \drawUnitLine{FB} $=$ \drawUnitLine{FC} $=$ \drawUnitLine{FD} $=$ \drawUnitLine{FE}:

Por lo tanto, si se traza un círculo desde el punto donde estas cinco líneas se encuentran, con cualquiera de ellas como un radio, este circunscribirá el pentágono dado.

Q. E. D. \byref{prop:I.VI} \byref{prop:I.IV}
\end{center}

\qed

\startproblem{Prop. XV. prob.}\label{prop:IV.XV}

\defineNewPicture[1/2]{
pair A, B, C, D, E, F, G, H;
numeric r;
r := 9/4u;
G := (0, 0);
A := (dir(90)*r) shifted G;
B := (dir(150)*r) shifted G;
C := (dir(210)*r) shifted G;
D := (dir(270)*r) shifted G;
E := (dir(330)*r) shifted G;
F := (dir(30)*r) shifted G;
H := (dir(270)*r) shifted D;
byAngleDefine(A, G, F, byblack, -ARC_SECTOR);
byAngleDefine(B, G, A, byblack, -ARC_SECTOR);
byAngleDefine(C, G, B, byblack, -ARC_SECTOR);
byAngleDefine(D, G, C, byred, SOLID_SECTOR);
byAngleDefine(E, G, D, byblue, SOLID_SECTOR);
byAngleDefine(F, G, E, byblack, SOLID_SECTOR);
draw byNamedAngleResized();
draw byLine(D, H, byred, SOLID_LINE, REGULAR_WIDTH);
draw byLine(A, D, byblack, SOLID_LINE, REGULAR_WIDTH);
draw byLine(B, E, byyellow, SOLID_LINE, REGULAR_WIDTH);
draw byLine(C, F, byblue, SOLID_LINE, REGULAR_WIDTH);
draw byLine(A, B, byblack, SOLID_LINE, REGULAR_WIDTH);
draw byLine(B, C, byblack, SOLID_LINE, REGULAR_WIDTH);
draw byLine(C, D, byblack, DASHED_LINE, REGULAR_WIDTH);
draw byLine(D, E, byred, DASHED_LINE, REGULAR_WIDTH);
draw byLine(E, F, byblue, DASHED_LINE, REGULAR_WIDTH);
draw byLine(F, A, byblack, SOLID_LINE, REGULAR_WIDTH);
draw byCircle.D(D, G, byred, 0, 0, 0);
draw byCircleR(G, r, byyellow, 0, 0, 0);
byLineDefine(G, C, lineColor.CF, SOLID_LINE, REGULAR_WIDTH);
byLineDefine(G, D, lineColor.AD, SOLID_LINE, REGULAR_WIDTH);
byLineDefine(G, E, lineColor.BE, SOLID_LINE, REGULAR_WIDTH);
draw byLabelsOnCircle(A, B, F)(G);
draw byLabelPoint(G, angle(A + F - 2G), 3);
draw byLabelPoint(D, angle(D-G) + 45, 2);
draw byLabelsOnPolygon(D, C, G)(OMIT_FIRST_LABEL+OMIT_LAST_LABEL, 0);
draw byLabelsOnPolygon(G, E, D)(OMIT_FIRST_LABEL+OMIT_LAST_LABEL, 0);
}
\drawCurrentPictureInMargin
\problem{P}{ara}{ inscribir un hexágono equilátero y equiangular en un círculo dado \drawUnitLine{AD}.}

\begin{center}
Desde cualquier punto de la circunferencia del círculo dado, traza \drawUnitLine{CF} pasando por su centro y dibuja los diámetros \drawUnitLine{BE}, \drawUnitLine{CD} y \drawUnitLine{DE}; dibuja \drawUnitLine{EF}, \drawUnitLine{GD}, \drawFromCurrentPicture{
draw byNamedAngle(DGC);
startAutoLabeling;
draw byNamedLineSeq(0)(GC,GD,CD);
stopAutoLabeling;
}, etcétera, y el hexágono requerido está inscrito en el círculo dado.

Ya que \drawFromCurrentPicture{
draw byNamedAngle(EGD);
startAutoLabeling;
draw byNamedLineSeq(0)(GD,GE,DE);
stopAutoLabeling;
} pasa a través de los centros
                 \\ de los círculos, \drawAngle{DGC} y \drawAngle{EGD} son triángulos
                 \\ equiláteros, por lo tanto \drawTwoRightAngles $=$ \drawAngle{DGC,EGD,FGE} $=$ un tercio de dos ángulos
                 \\ rectos; (L. 1. pr. 32) pero \drawTwoRightAngles $=$ 
\drawTwoRightAngles


 (L. 1. pr. 13);

$\therefore$ \drawAngle{DGC} $=$ \drawAngle{EGD} $=$ \drawAngle{FGE} $=$ un tercio de 
\drawTwoRightAngles


 (L. 1. pr. 32), y los ángulos verticalmente opuestos a estos son todos iguales entre sí (L. 1. pr. 15), y se colocan en arcos iguales (L. 3. pr. 26), que están subtendidos por cuerdas iguales (L. 3. pr. 29); y dado que cada uno de los ángulos del hexágono es el doble del ángulo de un triángulo equilátero, también es equiangular.

Q. E. D. \drawTwoRightAngles \byref{prop:I.XXXII} \byref{prop:I.XIII} \byref{prop:I.XXXII} \byref{prop:I.XV} \byref{prop:III.XXVI} \byref{prop:III.XXIX}
\end{center}

\qed

\startproblem{Prop. XVI. prob.}\label{prop:IV.XVI}

\defineNewPicture{
pair A, B, C, D, E, F, G, H, O;
numeric r;
r := 9/4u;
O := (0, 0);
A := (dir(90 + 0*120)*r) shifted O;
C := (dir(90 + 1*120)*r) shifted O;
D := (dir(90 + 2*120)*r) shifted O;
B := (dir(90 + 1*72)*r) shifted O;
F := (dir(90 + 2*72)*r) shifted O;
G := (dir(90 + 3*72)*r) shifted O;
H := (dir(90 + 4*72)*r) shifted O;
draw byLine(C, F, byblack, DASHED_LINE, REGULAR_WIDTH);
byLineDefine(A, B, byred, SOLID_LINE, REGULAR_WIDTH);
byLineDefine(B, F, byblue, SOLID_LINE, REGULAR_WIDTH);
byLineDefine(F, G, byblack, SOLID_LINE, THIN_WIDTH);
byLineDefine(G, H, byblack, SOLID_LINE, THIN_WIDTH);
byLineDefine(H, A, byblack, SOLID_LINE, THIN_WIDTH);
byLineDefine(A, C, byyellow, SOLID_LINE, REGULAR_WIDTH);
byLineDefine(C, D, byblack, SOLID_LINE, THIN_WIDTH);
byLineDefine(D, A, byblack, SOLID_LINE, THIN_WIDTH);
draw byNamedLineSeq(0)(AB,BF,FG,GH,HA);
draw byNamedLineSeq(0)(AC,CD,DA);
draw byCircleR(O, r, byblue, 0, 0, 1);
draw byLabelsOnCircle(A, B, C, F)(O);
}
\drawCurrentPictureInMargin
\problem{P}{ara}{ inscribir un pentadecágono equilátero y equiangular en un círculo dado.}

\begin{center}
Sean \drawUnitLine{AB} y \drawUnitLine{BF} los lados de un pentágono equilátero inscrito en el círculo dado, y \drawUnitLine{AC} el lado de un triángulo equilátero inscrito.

El arco subtendido por
\drawUnitLine{AB} y \drawUnitLine{BF}



$=$

2
/
5

$=$

6
/
15




de toda la
circunferencia.

El arco subtendido por
\drawUnitLine{AC}



$=$

1
/
3

$=$

5
/
15




de toda la
circunferencia.

Su diferencia es $=$

1
/
15

$\therefore$ el arco subtendido por \drawUnitLine{CF} $=$

1
/
15

                diferencia de toda la circunferencia.

Por lo tanto, si se colocan líneas rectas iguales a \drawUnitLine{CF} en el círculo (L. 4. pr. 1), el pentadecágono equilátero y equiangular se inscribirá en el círculo.

Q. E. D.

Un proyecto de Nicholas Rougeux     Licencia

¿Problemas de visualización? ¿Errores ortográficos? \byref{prop:IV.I}
\end{center}

\qed

\part{Book V}

\chapter*{Definitions}

\startdefinition[1]{}\label{def:V.I} 
A less magnitude is said to be an aliquot part or submultiple of a greater magnitude, when the less measures the greater; that is, when the less is contained a certain number of times exactly in the greater.

\startdefinition[2]{}\label{def:V.II} 
A greater magnitude is said to be a multiple of a less, when the greater is measured be the less; that is, when the greater contains the less a certain number of times exactly.

\startdefinition[3]{}\label{def:V.III} 
Ratio is the relation which one quantity bears to another of the same kind, with respect to magnitude.

\startdefinition[4]{}\label{def:V.IV} 
Magnitudes are said to have a ratio to one another, when they are of the same kind; and the one which is not the greater can be multiplied so as to exceed the other.

\vskip \baselineskip

\begin{center}
\emph{The other definitions will be given throughout the book where their aid is first required.}
\end{center}

\chapter*{Axioms}

\startaxiom{}\label{ax:V.I}

\begin{center}
Se dice que una magnitud menor es una parte alícuota o submúltiplo de una magnitud mayor, cuando la menor mide la mayor; es decir, cuando la menor está contenida un cierto número de veces exactamente en la mayor.
\end{center}

\startaxiom{}\label{ax:V.II}

\begin{center}
Se dice que una magnitud mayor es un múltiplo de una menor, cuando el mayor es medida por la menor; es decir, cuando la mayor contiene a la menor exactamente un cierto número de veces.
\end{center}

\startaxiom{}\label{ax:V.III}

\begin{center}
Proporción es la relación que una cantidad tiene con otra del mismo tipo, con respecto a una magnitud.
\end{center}

\starttheorem{Prop. I. Theor.}\label{prop:V.I}

\defineNewPicture{
byMagnitudeSymbolDefine.ab("semicircleUp", byred, 1);
byMagnitudeSymbolDefine.cd("wedgeDown", byyellow, 0);
byMagnitudeSymbolDefine.ef("sectorDown", byblue, 1);
byMagnitudeDefine.A(0, false)(5)(ab);
byMagnitudeDefine.B(0, false)(1)(ab);
byMagnitudeDefine.C(0, false)(5)(cd);
byMagnitudeDefine.D(0, false)(1)(cd);
byMagnitudeDefine.E(0, false)(5)(ef);
byMagnitudeDefine.F(0, false)(1)(ef);
}
\drawCurrentPictureInMargin
\problem{S}{e}{ dice que las magnitudes tienen una relación entre sí, cuando son del mismo tipo; y la que no es la mayor puede multiplicarse para exceder a la otra.}

\begin{center}
\emph{Las otras definiciones se darán a lo largo del libro donde su ayuda es primero requerida.}
\end{center}

\qed

\starttheorem{Prop. II. Theor.}\label{prop:V.II}

\defineNewPicture{
byMagnitudeSymbolDefine.ab("circle", byyellow, 0);
byMagnitudeSymbolDefine.cd("sectorUp", byred, 1);
byMagnitudeSymbolDefine.e("circle", byblue, 0);
byMagnitudeSymbolDefine.f("sectorUp", byblack, 1);
byMagnitudeDefine.A(0, false)(3)(ab);
byMagnitudeDefine.B(0, false)(1)(ab);
byMagnitudeDefine.C(0, false)(3)(cd);
byMagnitudeDefine.D(0, false)(1)(cd);
byMagnitudeDefine.E(0, false)(4)(e);
byMagnitudeDefine.F(0, false)(4)(f);
}
\drawCurrentPictureInMargin
\problem{M}{últiplos}{ iguales o submúltiplos iguales de lo mismo, o de magnitudes iguales, son iguales.}

\begin{center}

\end{center}

\qed

\starttheorem{Prop. III. Theor.}\label{prop:V.III}

\defineNewPicture{
byMagnitudeSymbolDefine.a("square", byyellow, 0);
byMagnitudeSymbolDefine.be("square", byred, 0);
byMagnitudeSymbolDefine.c("rhombus", byblack, 0);
byMagnitudeSymbolDefine.df("rhombus", byblue, 0);
byMagnitudeDefine.A(0, false)(1, 2, 1)(a);
byMagnitudeDefine.B(0, false)(1)(be);
byMagnitudeDefine.C(0, false)(2, 2)(c);
byMagnitudeDefine.D(0, false)(1)(df);
byMagnitudeDefine.E(0, false)(4, 4, 4, 4)(be);
byMagnitudeDefine.F(0, false)(4, 4, 4, 4)(df);
}
\drawCurrentPictureInMargin
\problem{U}{n}{ múltiplo de una magnitud mayor, es mayor que el mismo múltiplo de una menor.}

\begin{center}

\end{center}

\qed

\startdefinition{Definition V.}\label{def:V.V}

\defineNewPicture{
byMagnitudeSymbolDefine.a("circle", byred, 0);
byMagnitudeSymbolDefine.b("square", byyellow, 0);
byMagnitudeSymbolDefine.c("rhombus", byblue, 0);
byMagnitudeSymbolDefine.d("wedgeDown", byblack, 0);
byMagnitudeDefine.A(0, false)(1)(a);
byMagnitudeDefine.B(0, false)(1)(b);
byMagnitudeDefine.C(0, false)(1)(c);
byMagnitudeDefine.D(0, false)(1)(d);
byMagnitudeDefine.Am(1, false)(2, 3, 4, 5, 6)(a);
byMagnitudeDefine.Bm(1, false)(2, 3, 4, 5, 6)(b);
byMagnitudeDefine.Cm(1, false)(2, 3, 4, 5, 6)(c);
byMagnitudeDefine.Dm(1, false)(2, 3, 4, 5, 6)(d);
}
\drawCurrentPictureInMargin
\begin{center}
Esa magnitud, de la cual un múltiplo es mayor que el mismo múltiplo de otra, es mayor que la otra.
\end{center}

\starttheorem{Prop. IV. Theor.}\label{prop:V.IV}

\defineNewPicture{
byMagnitudeSymbolDefine.I("circle", byyellow, 0);
byMagnitudeSymbolDefine.II("square", byblack, 0);
byMagnitudeSymbolDefine.III("rhombus", byred, 0);
byMagnitudeSymbolDefine.IV("wedgeDown", byblue, 0);
}
\drawCurrentPictureInMargin
\problem{S}{i}{ cualquier número de magnitudes son múltiplos iguales de tantas otras, cada uno de ellas: independientemente del múltiplo que sea una de las primeras de su parte, el mismo múltiplo de las primeras magnitudes tomadas juntas será de todas los demás tomadas juntas.}

\begin{center}
Sea 
Red dome


Red dome


Red dome


Red dome


Red dome

 el mismo múltiplo de 
Red dome

,
 \\ que 
Yellow home


Yellow home


Yellow home


Yellow home


Yellow home

 es de 
Yellow home

.
 \\ que 
Blue drop


Blue drop


Blue drop


Blue drop


Blue drop

 es de 
Blue drop

.

Entonces es evidente que
                     \\ 



Red dome


Red dome


Red dome


Red dome


Red dome



Yellow home


Yellow home


Yellow home


Yellow home


Yellow home



Blue drop


Blue drop


Blue drop


Blue drop


Blue drop





                    es el mismo múltiplo de
                    



Red dome



Yellow home



Blue drop
 que 
Red dome


Red dome


Red dome


Red dome


Red dome

 es de 
Red dome

;
 \\ porque hay tantas magnitudes
                     \\ en
                    



Red dome


Red dome


Red dome


Red dome


Red dome



Yellow home


Yellow home


Yellow home


Yellow home


Yellow home



Blue drop


Blue drop


Blue drop


Blue drop


Blue drop





$=$




Red dome



Yellow home



Blue drop
 como hay en 
Red dome


Red dome


Red dome


Red dome


Red dome

 $=$ 
Red dome

.

La misma demostración es válida en cualquier cantidad de magnitudes, que aquí se ha aplicado a tres.

$\therefore$ Si alguna cantidad de magnitudes, etcétera. \byref{prop:V.III} \byref{\hypref} \byref{def:V.V} \byref{def:V.V}
\end{center}

\qed

\starttheorem{Prop. V. Theor.}\label{prop:V.V}

\defineNewPicture{
byMagnitudeSymbolDefine.a("sectorDown", byblue, 1);
byMagnitudeSymbolDefine.b("semicircleDown", byyellow, 1);
byMagnitudeSymbolDefine.c("miniTriangleUp", byblack, 0);
byMagnitudeSymbolDefine.d("miniSquare", byred, 0);
byMagnitudeDefine.I(0, false)(1, 2, 1)(a, a, b);
byMagnitudeDefine.II(0, false)(1, 1)(c, d);
}
\drawCurrentPictureInMargin
\problem{S}{i}{ la primera magnitud es el mismo múltiplo de la segunda que la tercera es de la cuarta, y la quinta el mismo múltiplo de la segunda que la sexta es de la cuarta, entonces la primera, junto con la quinta, será el mismo múltiplo de la segunda que la tercera, junto con la sexta, es de la cuarta.}

\begin{center}
Sea 
Yellow circle


Yellow circle


Yellow circle

, la primera, el mismo múltiplo de 
Yellow circle

, la segunda, que 
Red drop


Red drop


Red drop

, la tercera, es de 
Red drop

, la cuarta; y sea 
Blue circle


Blue circle


Blue circle


Blue circle

, la quinta, el mismo múltiplo de 
Yellow circle

, la segunda, que 
Black drop


Black drop


Black drop


Black drop

, la quinta, es de 
Red drop

, la cuarta.

Entonces es evidente, que
                



Yellow circle


Yellow circle


Yellow circle



Blue circle


Blue circle


Blue circle


Blue circle




,
                la primera y la quinta juntas, son el mismo múltiplo de 
Yellow circle

, la segunda, que
                



Red drop


Red drop


Red drop



Black drop


Black drop


Black drop


Black drop




,
                la tercera y la sexta juntas, es del mismo múltiplo de 
Red drop

, la cuarta, porque hay tantas magnitudes en
                



Yellow circle


Yellow circle


Yellow circle



Blue circle


Blue circle


Blue circle


Blue circle





$=$ 
Yellow circle

 como hay en
                



Red drop


Red drop


Red drop



Black drop


Black drop


Black drop


Black drop





$=$ 
Red drop

.

$\therefore$ Si la primera magnitud, etcétera.
\end{center}

\qed

\starttheorem{Prop. VI. Theor.}\label{prop:V.VI}

\defineNewPicture{
byMagnitudeSymbolDefine.a("sectorDown", byyellow, 1);
byMagnitudeSymbolDefine.b("miniSquare", byred, 0);
byMagnitudeSymbolDefine.c("semicircleDown", byblack, 1);
byMagnitudeSymbolDefine.d("miniTriangleUp", byblue, 0);
byMagnitudeDefine.I(0, false)(1, 2, 1)(a);
byMagnitudeDefine.II(0, false)(1)(b);
byMagnitudeDefine.III(0, false)(2)(c);
byMagnitudeDefine.IV(0, false)(1)(d);
}
\drawCurrentPictureInMargin
\problem{S}{i}{ las primeras de cuatro magnitudes son el mismo múltiplo de la segunda que la tercera es de la cuarta, y si se toman múltiplos iguales cualesquiera que sean la primera y la tercera, serán múltiplos iguales; uno de la segunda y el otro de la cuarta.}

\begin{center}
Sea
                    




Yellow square



Yellow square


Yellow square



Yellow square






                    el mismo múltiplo de 
Red square
 que
                    




Black diamond


Black diamond



Black diamond


Black diamond






                    es de 
Blue diamond

;
 \\ toma
                    



Red square


Red square


Red square


Red square



Red square


Red square


Red square


Red square



Red square


Red square


Red square


Red square



Red square


Red square


Red square


Red square





                    el mismo mútliplo de
                    



Yellow square



Yellow square


Yellow square



Yellow square



,
 \\ que
                    



Blue diamond


Blue diamond


Blue diamond


Blue diamond



Blue diamond


Blue diamond


Blue diamond


Blue diamond



Blue diamond


Blue diamond


Blue diamond


Blue diamond



Blue diamond


Blue diamond


Blue diamond


Blue diamond





                    es de
                    



Black diamond


Black diamond



Black diamond


Black diamond



.

Entonces es evidente,
                     \\ que
                    



Red square


Red square


Red square


Red square



Red square


Red square


Red square


Red square



Red square


Red square


Red square


Red square



Red square


Red square


Red square


Red square





                    el mismo mútliplo de 
Red square
 que
                    



Blue diamond


Blue diamond


Blue diamond


Blue diamond



Blue diamond


Blue diamond


Blue diamond


Blue diamond



Blue diamond


Blue diamond


Blue diamond


Blue diamond



Blue diamond


Blue diamond


Blue diamond


Blue diamond





                    es de 
Blue diamond

;
 \\ porque
                    



Red square


Red square


Red square


Red square



Red square


Red square


Red square


Red square



Red square


Red square


Red square


Red square



Red square


Red square


Red square


Red square





                    contiene
                    



Yellow square



Yellow square


Yellow square



Yellow square





                    contiene 
Red square
 tantas veces como
                     \\ 



Blue diamond


Blue diamond


Blue diamond


Blue diamond



Blue diamond


Blue diamond


Blue diamond


Blue diamond



Blue diamond


Blue diamond


Blue diamond


Blue diamond



Blue diamond


Blue diamond


Blue diamond


Blue diamond





                    contiene
                    



Black diamond


Black diamond



Black diamond


Black diamond





                    contiene 
Blue diamond

.

El mismo razonamiento es aplicable en todos los caso.

$\therefore$ Si las primeras cuatro, etcétera.

Se dice que cuatro magnitudes 
Red circle

, 
Yellow square

, 
Blue diamond

, 
Black home

, son proporcionales cuando se toma cada múltiplo igual de la primera y la tercera, y cada múltiplo igual de la segunda y la cuarta, como,

de la primera

Trans square


Trans square


Trans square


Trans square


Trans square

 
Red circle


Red circle



Trans square


Trans square


Trans square


Trans square

 
Red circle


Red circle


Red circle



Trans square


Trans square


Trans square

 
Red circle


Red circle


Red circle


Red circle



Trans square


Trans square

 
Red circle


Red circle


Red circle


Red circle


Red circle



Trans square

 
Red circle


Red circle


Red circle


Red circle


Red circle


Red circle

etcétera.

de la segunda

Trans square


Trans square


Trans square


Trans square


Trans square

 
Yellow square


Yellow square



Trans square


Trans square


Trans square


Trans square

 
Yellow square


Yellow square


Yellow square



Trans square


Trans square


Trans square

 
Yellow square


Yellow square


Yellow square


Yellow square



Trans square


Trans square

 
Yellow square


Yellow square


Yellow square


Yellow square


Yellow square



Trans square

 
Yellow square


Yellow square


Yellow square


Yellow square


Yellow square


Yellow square

etcétera.

de la tercera

Trans diamond


Trans diamond


Trans diamond


Trans diamond


Trans diamond

 
Blue diamond


Blue diamond



Trans diamond


Trans diamond


Trans diamond


Trans diamond

 
Blue diamond


Blue diamond


Blue diamond



Trans diamond


Trans diamond


Trans diamond

 
Blue diamond


Blue diamond


Blue diamond


Blue diamond



Trans diamond


Trans diamond

 
Blue diamond


Blue diamond


Blue diamond


Blue diamond


Blue diamond



Trans diamond

 
Blue diamond


Blue diamond


Blue diamond


Blue diamond


Blue diamond


Blue diamond

etcétera.

de la cuarta

Trans square


Trans square


Trans square


Trans square


Trans square

 
Black home


Black home



Trans square


Trans square


Trans square


Trans square

 
Black home


Black home


Black home



Trans square


Trans square


Trans square

 
Black home


Black home


Black home


Black home



Trans square


Trans square

 
Black home


Black home


Black home


Black home


Black home



Trans square

 
Black home


Black home


Black home


Black home


Black home


Black home

etcétera.

Luego, tomando cada par de múltiplos iguales de la primera y la tercera, y cada par de múltiplos iguales de la segunda y la cuarta,

Si
                



Red circle


Red circle

 >, $=$ o < 
Yellow square


Yellow square



Red circle


Red circle

 >, $=$ o < 
Yellow square


Yellow square


Yellow square



Red circle


Red circle

 >, $=$ o < 
Yellow square


Yellow square


Yellow square


Yellow square



Red circle


Red circle

 >, $=$ o < 
Yellow square


Yellow square


Yellow square


Yellow square


Yellow square



Red circle


Red circle

 >, $=$ o < 
Yellow square


Yellow square


Yellow square


Yellow square


Yellow square


Yellow square

entonces será
                



Blue diamond


Blue diamond

 >, $=$ o < 
Black home


Black home



Blue diamond


Blue diamond

 >, $=$ o < 
Black home


Black home


Black home



Blue diamond


Blue diamond

 >, $=$ o < 
Black home


Black home


Black home


Black home



Blue diamond


Blue diamond

 >, $=$ o < 
Black home


Black home


Black home


Black home


Black home



Blue diamond


Blue diamond

 >, $=$ o < 
Black home


Black home


Black home


Black home


Black home


Black home

Es decir, si dos veces la primera es mayor, igual o menor que el doble de la segunda, dos veces la tercera será mayor, igual o menor que el doble de la cuarta; o, si dos veces la primera es mayor, igual o menor que tres veces la segunda, dos veces la tercera será mayor, igual o menor que tres veces la cuarta, y así sucesivamente, como se expresó anteriormente.

Si
                



Red circle


Red circle


Red circle

 >, $=$ o < 
Trans square


Trans square


Trans square


Trans square


Trans square

 
Yellow square


Yellow square



Red circle


Red circle


Red circle

 >, $=$ o < 
Trans square


Trans square


Trans square


Trans square

 
Yellow square


Yellow square


Yellow square



Red circle


Red circle


Red circle

 >, $=$ o < 
Trans square


Trans square


Trans square

 
Yellow square


Yellow square


Yellow square


Yellow square



Red circle


Red circle


Red circle

 >, $=$ o < 
Trans square


Trans square

 
Yellow square


Yellow square


Yellow square


Yellow square


Yellow square



Red circle


Red circle


Red circle

 >, $=$ o < 
Trans square

 
Yellow square


Yellow square


Yellow square


Yellow square


Yellow square


Yellow square

entonces será
                



Blue diamond


Blue diamond


Blue diamond

 >, $=$ o < 
Trans square


Trans square


Trans square


Trans square


Trans square

 
Black home


Black home



Blue diamond


Blue diamond


Blue diamond

 >, $=$ o < 
Trans square


Trans square


Trans square


Trans square

 
Black home


Black home


Black home



Blue diamond


Blue diamond


Blue diamond

 >, $=$ o < 
Trans square


Trans square


Trans square

 
Black home


Black home


Black home


Black home



Blue diamond


Blue diamond


Blue diamond

 >, $=$ o < 
Trans square


Trans square

 
Black home


Black home


Black home


Black home


Black home



Blue diamond


Blue diamond


Blue diamond

 >, $=$ o < 
Trans square

 
Black home


Black home


Black home


Black home


Black home


Black home

En otros términos, si tres veces la primera es mayor, igual o menor que el doble de la segunda, tres veces la tercera será mayor, igual o menor que el doble de la cuarta; o, si tres veces la primera es mayor, igual o menor que tres veces la segunda, entonces tres veces la tercera será mayor, igual o menor que tres veces la cuarta; o si tres veces la primera es mayor, igual o menor que cuatro veces la segunda, entonces tres veces la tercera será mayor, igual o menor que cuatro veces la cuarto, y así sucesivamente. De nuevo,

Si
                



Red circle


Red circle


Red circle


Red circle

 >, $=$ o < 
Yellow square


Yellow square



Red circle


Red circle


Red circle


Red circle

 >, $=$ o < 
Yellow square


Yellow square


Yellow square



Red circle


Red circle


Red circle


Red circle

 >, $=$ o < 
Yellow square


Yellow square


Yellow square


Yellow square



Red circle


Red circle


Red circle


Red circle

 >, $=$ o < 
Yellow square


Yellow square


Yellow square


Yellow square


Yellow square



Red circle


Red circle


Red circle


Red circle

 >, $=$ o < 
Yellow square


Yellow square


Yellow square


Yellow square


Yellow square


Yellow square

entonces será
                



Blue diamond


Blue diamond


Blue diamond


Blue diamond

 >, $=$ o < 
Black home


Black home



Blue diamond


Blue diamond


Blue diamond


Blue diamond

 >, $=$ o < 
Black home


Black home


Black home



Blue diamond


Blue diamond


Blue diamond


Blue diamond

 >, $=$ o < 
Black home


Black home


Black home


Black home



Blue diamond


Blue diamond


Blue diamond


Blue diamond

 >, $=$ o < 
Black home


Black home


Black home


Black home


Black home



Blue diamond


Blue diamond


Blue diamond


Blue diamond

 >, $=$ o < 
Black home


Black home


Black home


Black home


Black home


Black home

Y así sucesivamente, con cualquier otro múltiplo igual de las cuatro magnitudes, tomado de la misma manera.

Euclides expresa esta definición de la siguiente manera:—

Se dice que la primera de las cuatro magnitudes tiene la misma proporción con la segunda, que la tercera tiene con la cuarta, cuando se toma cualquier múltiplo igual de la primera y la tercera, y cualquier múltiplo igual de la segunda y la cuarta; si el múltiplo de la primera es menor que el de la segunda, el múltiplo de la tercera también es menor que el de la cuarta; o, si el múltiplo de la primera es igual al de la segunda, el múltiplo de la tercera también es igual al de la cuarta; o, si el múltiplo de la primera es mayor que el de la segunda, el múltiplo de la tercera también es mayor que el de la cuarta.

En el futuro, expresaremos esta definición en general, así:

Si M 
Red circle

 >, $=$ o < m 
Yellow square

,
cuando M 
Blue diamond

 >, $=$ o < m 
Black home

,

Luego deducimos que 
Red circle

, la primera, tiene la misma proporción con 
Yellow square

, la segunda, que 
Blue diamond

, la tercera, con 
Black home

 la cuarta: expresado en las demostraciones posteriores así:

Red circle

 : 
Yellow square

 :: 
Blue diamond

 : 
Black home

;
o así, 
Red circle

 : 
Yellow square

 $=$ 
Blue diamond

 : 
Black home

;
o así,
                            

Red circle


/

Yellow square



$=$


Blue diamond


/

Black home



: y se lee,

“
Red circle

 es a 
Yellow square

, commo 
Blue diamond

 es a 
Black home

.”

Y si 
Red circle

 : 
Yellow square

 :: 
Blue diamond

 : 
Black home

 inferiremos si
                 \\ M 
Red circle

 >, $=$ o < m 
Yellow square

, entonces
                 \\ M 
Blue diamond

 >, $=$ o < m 
Black home

.

Es decir, si la primera es para la segunda, como la tercera es para la cuarta; entonces, si M veces la primera es mayor, igual o menor que m veces la segunda, entonces M veces la tercera será mayor, igual o menor que m veces la cuarta, en el cual M y m no deben ser considerados múltiplos particulares, pero cada par de múltiplos de lo que sea; ni son tales marcas como 
Red circle

, 
Black home

, 
Yellow square

, etcétera consideradas más que representativas de magnitudes geométricas.

El estudiante debe comprender completamente esta definición antes de continuar. \byref{prop:V.IV} \byref{\constref} \byref{prop:V.A} \byref{prop:V.B}
\end{center}

\qed

\starttheorem{Prop. VII. Theor.}\label{prop:V.VII}

\defineNewPicture{
byMagnitudeSymbolDefine.I("circle", byred, 0);
byMagnitudeSymbolDefine.II("rhombus", byblue, 0);
byMagnitudeSymbolDefine.III("square", byyellow, 0);
}
\drawCurrentPictureInMargin
\problem{S}{i}{ la primera de las cuatro magnitudes tiene la misma proporción con la segunda, que la tercera tiene con la cuarta, entonces cualquier múltiplo igual de la primera y la tercera tendrá la misma proporción con cualquier múltiplo igual de la segunda y la cuarta; a saber, el múltiplo igual de la primera tendrá la misma proporción con el de la segunda, que el múltiplo igual de la tercera tiene con el de la cuarta.}

\begin{center}
Sea 
Yellow circle

: 
Black square

 :: 
Red diamond

 : 
Blue home

, entonces 3 
Yellow circle

 : 2 
Black square

 :: 3 
Red diamond

 : 2 
Blue home

, cada múltiplo igual de 3 
Yellow circle

 y 3 
Red diamond

 son múltiplos iguales de 
Yellow circle

 y 
Red diamond

, y cada múltiplo igual de 2 
Black square

 y 2 
Blue home

, son múltiplos iguales de 
Black square

 y 
Blue home

 (L. 5. pr. 3.)

Es decir, M veces 3 
Yellow circle

 y M veces 3 
Red diamond

 son múltiplos iguales de 
Yellow circle

 y 
Red diamond

, y m veces 2 
Black square

 y m 2 
Blue home

 son múltiplos iguales de 2 
Black square

 y 2 
Blue home

; pero 
Yellow circle

 : 
Black square

 :: 
Red diamond

 : 
Blue home

 (hip); $\therefore$ si M 3 
Yellow circle

 >, $=$ o < m 2 
Black square

, entonces M 3 
Red diamond

 >, $=$ o < m 2 
Blue home

 \byref{def:V.V} y por lo tanto 3 
Yellow circle

 : 2 
Black square

 :: 3 
Red diamond

 : 2 
Blue home

 \byref{def:V.V}

El mismo razonamiento es válido si se toma cualquier otro múltiplo igual de la primera y la tercera, y cualquier otro múltiplo igual de la segunda y la cuarta.

$\therefore$ Si la primera de las cuatro, etcétera.
\end{center}

\qed

\startdefinition{Definition VII.}\label{def:V.VII}

\defineNewPicture{
magnitudeScale := 4/5;
byMagnitudeSymbolDefine.i("circle", byred, 0);
byMagnitudeSymbolDefine.ii("square", byyellow, 0);
byMagnitudeSymbolDefine.iii("rhombus", byblue, 0);
byMagnitudeSymbolDefine.iv("wedgeDown", byblack, 0);
byMagnitudeDefine.I(0, false)(5)(i);
byMagnitudeDefine.II(0, false)(4)(ii);
byMagnitudeDefine.III(0, false)(5)(iii);
byMagnitudeDefine.IV(0, false)(4)(iv);
byMagnitudeSymbolDefine.Ia("wedgeDown", byred, 0);
byMagnitudeSymbolDefine.IIa("semicircleDown", byblack, 1);
byMagnitudeSymbolDefine.IIIa("square", byblue, 0);
byMagnitudeSymbolDefine.IVa("rhombus", byyellow, 0);
}
\drawCurrentPictureInMargin
\begin{center}
Proposición V. Teorema.

Si una magnitud es el mismo múltiplo de otra, que una magnitud tomada de la primera es de una magnitud tomada de la otra, el resto será el mismo múltiplo del resto, que el todo es del todo.

Sea
                    


Blue drop



Blue drop


Blue drop



Yellow dome




$=$
M′
                    


Black triangle



Red square
 y 
Yellow dome

 $=$ M′ 
Red square

,
                     \\ $\therefore$



Blue drop



Blue drop


Blue drop



Yellow dome




                    menos 
Yellow dome

 $=$
M′
                    


Black triangle



Red square




                    menos M′ 
Red square

,
                     \\ 



Blue drop



Blue drop


Blue drop




$=$ M′
                    (


Black triangle



Red square




                    menos 
Red square

),
                     \\ y $\therefore$



Blue drop



Blue drop


Blue drop




$=$ M′ 
Black triangle

.

$\therefore$ Si una magnitud, etcétera.
\end{center}

\starttheorem{Prop. VIII. Theor.}\label{prop:V.VIII}

\defineNewPicture{
byMagnitudeSymbolDefine.Ia("miniTriangleUp", byblack, 0);
byMagnitudeSymbolDefine.Ib("square", byred, 0);
byMagnitudeDefine.I(0, false)(1, 1)(Ia, Ib);
byMagnitudeSymbolDefine.II("square", byyellow, 0);
byMagnitudeSymbolDefine.III("circle", byblue, 0);
}
\drawCurrentPictureInMargin
\problem{S}{i}{ dos magnitudes son múltiplos iguales de otras dos, y si se toman múltiplos iguales de las dos primeras, los restos son iguales a estas otras, o múltiplos iguales de ellas.}

\begin{center}
Sea
                    


Yellow drop



Yellow drop


Yellow drop



Yellow drop




$=$ M′ 
Red square

; y 
Black dome


Black dome

 $=$ M′ 
Blue triangle

;
                     \\ entonces
                    


Yellow drop



Yellow drop


Yellow drop



Yellow drop




                    menos m′ 
Red square

 $=$

M′ 
Red square

 menos m′ 
Red square

 $=$ (M′ menos m′) 
Red square

,
                     \\ y 
Black dome


Black dome

 menos m′ 
Blue triangle

 $=$ M′ 
Blue triangle

 menos m′ 
Blue triangle

 $=$ (M′ menos m′) 
Blue triangle

.

Por lo tanto, (M′ menos m′) 
Red square

 y (M′ menos m′) 
Blue triangle

 son múltiplos iguales de 
Red square

 y 
Blue triangle

, e iguales a 
Red square

 y 
Blue triangle

, cuando M′ menos m′ $=$ 1.

$\therefore$ Si dos magnitudes son múltiplos iguales, etcétera.

Si la primera de las cuatro magnitudes tiene la misma relación con la segunda que la tercera con la cuarta, entonces si la primera es mayor que la segunda, la tercera también es mayor que la cuarta; y si es igual, igual; si es menor, menor.

Sea 
Red circle

 : 
Black square

 :: 
Blue home

 : 
Yellow diamond

; por lo tanto la quinta definición, si 
Red circle


Red circle

 > 
Black square


Black square

, entonces 
Blue home


Blue home

 > 
Yellow diamond


Yellow diamond

;
 \\ pero si 
Red circle

 > 
Black square

, entonces 
Red circle


Red circle

 > 
Black square


Black square
 y 
Blue home


Blue home

 > 
Yellow diamond


Yellow diamond

,
 \\ y $\therefore$ 
Blue home

 > 
Yellow diamond

.

Del mismo modo, si 
Red circle

 $=$, o < 
Black square

, entonces 
Blue home

 $=$, o < 
Yellow diamond

.

$\therefore$ Si la primera de las cuatro, etcétera.

Los geómetras hacen uso del término técnico “invertir,” por inversión, cuando hay cuatro proporcionales, y se infiere, que la segunda es a la primera coma la cuarta a la tercera.

Sea A : B :: C : D, entonces, por “inversión” es es inferido B : A :: D : C.

Si cuatro magnitudes son proporcionales, también son proporcionales cuando se toman inversamente.

Sea 
Blue home

 : 
Black dome

 :: 
Red square

 : 
Yellow diamond

,
 \\ entonces inversamente, 
Black dome

 : 
Blue home

 :: 
Yellow diamond

 : 
Red square

.

Si M 
Blue home

 < m 
Black dome

, entonces M 
Red square

 < m 
Yellow diamond
 por la quinta definición.

Sea M 
Blue home

 < m 
Black dome

, es decir, m 
Black dome

 > M 
Blue home

,
 \\ $\therefore$ M 
Red square

 < m 
Yellow diamond

, o, m 
Yellow diamond

 > M 
Red square

;
 \\ $\therefore$ si m 
Black dome

 > M 
Blue home

, entonces m 
Yellow diamond

 > M 
Red square

.

De la misma manera se puede mostrar,

que si m 
Black dome

 $=$ o < M 
Blue home

,
 \\ entonces m 
Yellow diamond

 $=$, o < M 
Red square

;
 \\ y por lo tanto, por la quinta definición, inferimos
                 \\ que 
Black dome

 : 
Blue home

 :: 
Yellow diamond

 : 
Red square

.

$\therefore$ Si cuatro magnitudes, etcétera.

Si la primera es el mismo múltiplo de la segunda, o la misma parte de ella, que la tercera es de la cuarta; la primera es para la segunda, como la tercera es para la cuarta.

Sea 


Blue square


Blue square



Blue square


Blue square



, la primera, el mismo múltiplo de 
Black circle

, la segunda,
                     \\ que 


Yellow diamond


Yellow diamond



Yellow diamond


Yellow diamond



, la tercera, es de 
Red home

, la cuarta.

Entonces 


Blue square


Blue square



Blue square


Blue square



 : 
Black circle

 :: 


Yellow diamond


Yellow diamond



Yellow diamond


Yellow diamond



 : 
Red home
 toma M 


Blue square


Blue square



Blue square


Blue square



, m 
Black circle

, M 


Yellow diamond


Yellow diamond



Yellow diamond


Yellow diamond



, m 
Red home

;
 \\ porque 


Blue square


Blue square



Blue square


Blue square



 es el mismo múltiplo de 
Black circle
 que 


Yellow diamond


Yellow diamond



Yellow diamond


Yellow diamond



 es de 
Red home

 (de acuerdo con la hipótesis);
                     \\ y M 


Blue square


Blue square



Blue square


Blue square



 se toma el mismo múltiplo de 


Blue square


Blue square



Blue square


Blue square
 que M 


Yellow diamond


Yellow diamond



Yellow diamond


Yellow diamond



 es de 


Yellow diamond


Yellow diamond



Yellow diamond


Yellow diamond



,
 \\ $\therefore$ (de acuerdo a la tercera preposición),
                     \\ M 


Blue square


Blue square



Blue square


Blue square



 es el mismo múltiplo de 
Black circle
 que M 


Yellow diamond


Yellow diamond



Yellow diamond


Yellow diamond



 es de 
Red home

.

Por consiguiente, si M 


Blue square


Blue square



Blue square


Blue square



 es de 
Black circle

 un múltiplo mayor que m 
Black circle

 es, entonces M 


Yellow diamond


Yellow diamond



Yellow diamond


Yellow diamond



 es un múltiplo mayor de 
Red home

 que m 
Red home

 es; es decir, si M 


Blue square


Blue square



Blue square


Blue square



 es mayor que m 
Black circle

, entonces M 


Yellow diamond


Yellow diamond



Yellow diamond


Yellow diamond



 será mayor que m 
Red home

; de la misma manera se puede mostrar, si M 


Blue square


Blue square



Blue square


Blue square



 es igual a m 
Black circle

, entonces
                     \\ M 


Yellow diamond


Yellow diamond



Yellow diamond


Yellow diamond



 será igual a m 
Red home

.

Y, generalmente, si M 


Blue square


Blue square



Blue square


Blue square



 >, $=$ o < m 
Black circle
 entonces M 


Yellow diamond


Yellow diamond



Yellow diamond


Yellow diamond



 será >, $=$ o < m 
Red home

;
 \\ $\therefore$ por la quinta definición,
                     \\ 


Blue square


Blue square



Blue square


Blue square



 : 
Black circle

 :: 


Yellow diamond


Yellow diamond



Yellow diamond


Yellow diamond



 : 
Red home

.

Siguiente, sea 
Black circle

 la misma parte de 


Blue square


Blue square



Blue square


Blue square
 que 
Red home

 es de 


Yellow diamond


Yellow diamond



Yellow diamond


Yellow diamond



.

En este caso también 
Black circle

 : 


Blue square


Blue square



Blue square


Blue square



 :: 
Red home

 : 


Yellow diamond


Yellow diamond



Yellow diamond


Yellow diamond



.

Porque
                     \\ 
Black circle

 es misma parte de 


Blue square


Blue square



Blue square


Blue square



 que 
Red home

 es de 


Yellow diamond


Yellow diamond



Yellow diamond


Yellow diamond



,
 \\ por lo tanto 


Blue square


Blue square



Blue square


Blue square



 es el mismo múltiplo de 
Black circle
 que 


Yellow diamond


Yellow diamond



Yellow diamond


Yellow diamond



 es de 
Red home

.

Por lo tanto, por el caso anterior,
                     \\ 


Blue square


Blue square



Blue square


Blue square



 : 
Black circle

 :: 


Yellow diamond


Yellow diamond



Yellow diamond


Yellow diamond



 : 
Red home

;
 \\ y $\therefore$ 
Black circle

 : 


Blue square


Blue square



Blue square


Blue square



 :: 
Red home

 : 


Yellow diamond


Yellow diamond



Yellow diamond


Yellow diamond



,
 \\  por la proposición B.

$\therefore$ Si la primera es el mismo múltiplo, etcétera.

Si la primera es para la segunda como la tercero para la cuarta, y si la primera es un múltiplo o una parte de la segunda; la tercera es el mismo múltiplo, o la misma parte de la cuarta.

Sea 


Yellow circle



Yellow circle


Yellow circle



 : 
Black square

 :: 


Red diamond


Red diamond



Red diamond


Red diamond



 : 
Blue home

;
 \\ y primero, sea 


Yellow circle



Yellow circle


Yellow circle



 múltiplo de 
Black square

;
 \\ 


Red diamond


Red diamond



Red diamond


Red diamond



 será el mismo múltiplo de 
Blue home

.

Yellow circle



Yellow circle


Yellow circle





Black square





Red diamond


Red diamond



Red diamond


Red diamond





Blue home

Red dome



Red dome


Red dome







Black drop


Black drop



Black drop


Black drop

Asume 


Red dome



Red dome


Red dome



 $=$ 


Yellow circle



Yellow circle


Yellow circle



.

Cualquier múltiplo que 


Yellow circle



Yellow circle


Yellow circle



 sea de 
Black square
 asume 


Black drop


Black drop



Black drop


Black drop



 el mismo múltiplo de 
Blue home

,
 \\ entonces, porque 


Yellow circle



Yellow circle


Yellow circle



 : 
Black square

 :: 


Red diamond


Red diamond



Red diamond


Red diamond



 : 
Blue home
 y del segundo y cuarto, hemos tomado múltiplos iguales,
                     \\ 


Yellow circle



Yellow circle


Yellow circle



 y 


Black drop


Black drop



Black drop


Black drop



, por lo tanto (L. 5. pr. 4),
                     \\ 


Yellow circle



Yellow circle


Yellow circle



 : 


Red dome



Red dome


Red dome



 :: 


Red diamond


Red diamond



Red diamond


Red diamond



 : 


Black drop


Black drop



Black drop


Black drop



, pero (const.),
                     \\ 


Yellow circle



Yellow circle


Yellow circle



 $=$ 


Red dome



Red dome


Red dome



 $\therefore$ (L. 5. pr. A.) 


Red diamond


Red diamond



Red diamond


Red diamond



 $=$ 


Black drop


Black drop



Black drop


Black drop
 y 


Black drop


Black drop



Black drop


Black drop



 es el mismo múltiplo de 
Blue home
 que 


Yellow circle



Yellow circle


Yellow circle



 es de 
Black square

.

Siguiente, sea 
Black square

 : 


Yellow circle



Yellow circle


Yellow circle



 :: 
Blue home

 : 


Red diamond


Red diamond



Red diamond


Red diamond



,
 \\ y también 
Black square

 una parte de 


Yellow circle



Yellow circle


Yellow circle



;
 \\ entonces 
Blue home

 será la misma parte de 


Red diamond


Red diamond



Red diamond


Red diamond



.

Inversamente (L. 5.), 


Yellow circle



Yellow circle


Yellow circle



 : 
Black square

 :: 


Red diamond


Red diamond



Red diamond


Red diamond



 : 
Blue home

,
 \\ pero 
Black square

 es una parte de 


Yellow circle



Yellow circle


Yellow circle



;
 \\ es decir, 


Yellow circle



Yellow circle


Yellow circle



 es un múltiplo de 
Black square

;
 \\ $\therefore$ por el caso anterior, 


Red diamond


Red diamond



Red diamond


Red diamond



 es el mismo múltiplo de 
Blue home
 es decir, 
Blue home

 es la misma parte de 


Red diamond


Red diamond



Red diamond


Red diamond
 que 
Black square

 es de 


Yellow circle



Yellow circle


Yellow circle



.

$\therefore$ Si la primera es para la segunda, etcétera.
\end{center}

\qed

\starttheorem{Prop. IX. Theor.}\label{prop:V.IX}

\defineNewPicture{
byMagnitudeSymbolDefine.I("rhombus", byblue, 0);
byMagnitudeSymbolDefine.II("circle", byred, 0);
byMagnitudeSymbolDefine.III("square", byyellow, 0);
}
\drawCurrentPictureInMargin
\problem{S}{ea}{ Red circle}

\begin{center}
 $=$ 
Blue diamond

 y 
Yellow square

 otra magnitud;
                     \\ entonces 
Red circle

 : 
Yellow square

 $=$ 
Blue diamond

 : 
Yellow square

 y 
Yellow square

 : 
Red circle

 $=$ 
Yellow square

 : 
Blue diamond

.

Porque 
Red circle

 $=$ 
Blue diamond

,
 \\ $\therefore$ M 
Red circle

 $=$ M 
Blue diamond

;

$\therefore$ si M 
Red circle

 >, $=$ or < m 
Yellow square

, entonces
                     \\ M 
Blue diamond

 >, $=$ o < m 
Yellow square

 ,
 \\ y $\therefore$ 
Red circle

 : 
Yellow square

 $=$ 
Blue diamond

 : 
Yellow square

 (L. 5. def. 5).

Del razonamiento anterior es evidente que,
                     \\ si m 
Yellow square

 >, $=$ o < M 
Red circle

, entonces
                     \\ m 
Yellow square

 >, $=$ o < M 
Blue diamond
 $\therefore$ 
Yellow square

 : 
Red circle

 $=$ 
Yellow square

 : 
Blue diamond

 (L. 5. def. 5).

$\therefore$ Las magnitudes iguales, etcétera.

Cuando de los múltiplos iguales de cuatro magnitudes (tomadas como en la quinta definición), el múltiplo de la primera es mayor que el de la segunda, pero el múltiplo de la tercera no es mayor que el múltiplo de la cuarta; entonces se dice que la primera tiene una proporción mayor con la segunda que la tercera magnitud tiene con la cuarta: y, por el contrario, se dice que la tercera tiene proporción menor con la cuarta que la primera tiene con la segunda.

Si, entre los múltiplos iguales de cuatro magnitudes, comparados como en la quinta definición, deberíamos encontrar 
Red circle


Red circle


Red circle


Red circle


Red circle

 > 
Yellow square


Yellow square


Yellow square


Yellow square

, pero 
Blue diamond


Blue diamond


Blue diamond


Blue diamond


Blue diamond

 $=$ o < 
Black square


Black square


Black square


Black square

, o si encontramos cualquier múltiplo M′ particular de la primera y la tercera, y un múltiplo m′ particular de la segundo y la cuarta, tal que M′ veces la primera es > m′ veces la segunda, pero M′ veces la tercera no es > m′ veces la cuarta, es decir $=$ o < m′ veces el cuarto; entonces se dice que la primera tiene a la segunda un proporción mayor que la tercera a la cuarta; o la tercera tiene a la cuarta, en tales circunstancias, una proporción menor que la primera tiene con la segunda: aunque varios otros múltiplos iguales pueden tender a mostrar que las cuatro magnitudes son proporcionales.

Esta definición en el futuro se expresará así:—

Si M′ 
Red home

 > m′ 
Black dome

, pero M′ 
Blue square

 $=$ o < m′ 
Yellow diamond

,
 \\ entonces 
Red home

 : 
Black dome

 > 
Blue square

 : 
Yellow diamond

.

En la expresión general anterior, M′ y m′ deben considerarse múltiplos particulares, no como los múltiplos M y m introducidos en la quinta definición, que en esa definición se consideran cada par de múltiplos que se pueden tomar. También debe observarse aquí que 
Red home

, 
Black dome

, 
Blue square

, y los símbolos similares deben considerarse simplemente los representantes de magnitudes geométricas.

De manera aritmética parcial, esto se puede establecer de la siguiente manera:

Tomemos cuatro números, 8, 7, 10, y 9.

Entre los múltiplos anteriores encontramos 16 > 14 y 20 > 18; es decir, dos veces el primero es mayor que dos veces el segundo, y dos veces el tercero es mayor que dos veces el cuarto; y 16 < 21 y 20 < 27; es decir, dos veces el primero es menor que tres veces el segundo, y dos veces el tercero es menor que de tres veces el cuarto; y entre los mismos múltiplos podemos encontrar 72 > 56 y 90 > 72: es decir, 9 veces el primero es mayor que 8 veces el segundo, y 9 veces el tercero es mayor que 8 veces el cuarto. Se podrían seleccionar muchos otros múltiplos iguales, lo que tendería a mostrar que los números 8, 7, 10, 9, eran proporcionales, pero no lo son, porque podemos encontrar un múltiplo del primero > que un múltiplo del segundo, pero el mismo múltiplo del tercero que se ha tomado del primero no es > que el mismo múltiplo del cuarto que se ha tomado del segundo; por ejemplo, 9 veces el primero es > que 10 veces el segundo, pero 9 veces el tercero no es > que 10 veces el cuarto, es decir, 72 > 70, pero 90 no es > 90, y encontramos que 8 veces el primero > que 9 veces el segundo, pero 8 veces el tercero no es mayor que 9 veces el cuarto, es decir 64 > 63, pero 80 no es > 81. Cuando se pueden encontrar múltiplos como estos, se dice que el primero (8) tiene el segundo (7) una proporción mayor que el tercero (10) con el cuarto (9), y por el contrario el tercero (10) se dice que tiene el cuarto (9) una proporción menor que el primero (8) tiene al segundo (7). \byref{prop:V.VIII}
\end{center}

\qed

\starttheorem{Prop. X. Theor.}\label{prop:V.X}

\defineNewPicture{
byMagnitudeSymbolDefine.I("wedgeDown", byblue, 0);
byMagnitudeSymbolDefine.II("circle", byred, 0);
byMagnitudeSymbolDefine.III("square", byyellow, 0);
}
\drawCurrentPictureInMargin
\problem{D}{e}{ magnitudes desiguales, la mayor tiene una proporción mayor con la misma que la menor; y la misma magnitud tiene una proporción mayor con la menor que con la mayor.}

\begin{center}
Sean 


Black triangle



Red square



 y 
Yellow square

 dos magnitudes desiguales, y 
Blue circle

 cualquier otra.

Primero demostraremos que 


Black triangle



Red square



 que es la mayor de las dos magnitudes desiguales, tiene mayor proporción a 
Blue circle

 que 
Yellow square

, la menor tiene a 
Blue circle

;

es decir, 


Black triangle



Red square



 : 
Blue circle

 > 
Yellow square

 : 
Blue circle

;
 \\ toma M′ 


Black triangle



Red square



, m′ 
Blue circle

, M′ 
Yellow square

, y m′ 
Blue circle

;
 \\ tal que M′ 
Black triangle

 y M′ 
Red square

 sean cada una > 
Blue circle

;
 \\ también toma m′ 
Blue circle

 el múltiplo menor de 
Blue circle

,
 \\ que hará m′ 
Blue circle

 > M′ 
Yellow square

 $=$ M′ 
Red square

;
 \\ $\therefore$ M′ 
Yellow square

 no es > m′ 
Blue circle

,
 \\ pero M′ 


Black triangle



Red square



 is > m′ 
Blue circle

, porque,
                     \\ como m′ 
Blue circle

 es el primer múltiplo que primero se convierte > M′ 
Red square

, que (m′ menos 1) 
Blue circle

 o m′ 
Blue circle

 menos 
Blue circle

 no es > M′ 
Red square

, y 
Blue circle

 no es > M′ 
Black triangle

,
 \\ $\therefore$ m′ 
Blue circle

 menos 
Blue circle

 + 
Blue circle

 debe ser < M′ 
Red square

 + M′ 
Black triangle

;
 \\ es decir, m′ 
Blue circle

 debe ser < M′ 
Red square

;
 \\ $\therefore$ M′ 


Black triangle



Red square



 es > m′ 
Blue circle

; pero se ha demostrado arriba que
                     \\ M′ 
Yellow square

 no es > m′ 
Blue circle

, por lo tanto, por la séptima definición,
                     \\ 


Black triangle



Red square



 tiene a 
Blue circle

 mayor proporción que 
Yellow square

 : 
Blue circle

.

A continuación demostraremos que 
Blue circle

 tiene mayor proporción a 
Yellow square

, la menor que esta tiene a 


Black triangle



Red square



, la mayor;
                     \\ o, 
Blue circle

 : 
Yellow square

 > 
Blue circle

 : 


Black triangle



Red square



.

Toma m′ 
Blue circle

, M′ 
Yellow square

, m′ 
Blue circle

, y M′ 


Black triangle



Red square



,
 \\ lo mismo como en el primer caso, de modo que
                     \\ M′ 
Black triangle

 y M′ 
Red square

 serán cada uno > 
Blue circle

, y m′ 
Blue circle

 el múltiplo menor de 
Blue circle

, que primero se convierte en mayor que M′ 
Red square

 $=$ M′ 
Yellow square

.

$\therefore$ m′ 
Blue circle

 menos 
Blue circle

 no es > M′ 
Red square

,
 \\ y 
Blue circle

 no es > M′ 
Black triangle

; por consiguiente
                     \\ m′ 
Blue circle

 menos 
Blue circle

 + 
Blue circle

 es < M′ 
Red square

 + M′ 
Black triangle

;
 \\ $\therefore$ m′ 
Blue circle

 is < M′ 


Black triangle



Red square



, y $\therefore$ por la séptima definición,
                     \\ 
Blue circle

 tiene a 
Yellow square

 mayor proporción que 
Blue circle

 tiene a 


Black triangle



Red square



.

$\therefore$ De magnitudes desiguales, etcétera.

La inventiva empleada en esta proposición para encontrar entre los múltiplos tomados, como en la quinta definición, un múltiplo del primero mayor que el múltiplo del segundo, pero el mismo múltiplo del tercero que se ha tomado del primero, no mayor que el El mismo múltiplo del cuarto que se ha tomado del segundo, puede ilustrarse numéricamente de la siguiente manera:—

El número 9 tiene mayor proporción a 7 que 8 tiene a 7: es decir, 9 : 7 > 8 : 7; o, 8 + 1 : 7 > 8 : 7.

El múltiplo de 1, que primero se convierte en mayor que 7, es 8 veces, por lo tanto, podemos multiplicar el primero y el tercero por 8, 9, 10 o cualquier otro número mayor; en este caso, multipliquemos el primero y el tercero por 8, y tenemos 64 + 8 y 64: nuevamente, el primer múltiplo de 7 que llega a ser mayor que 64 es 10 veces; luego, al multiplicar el segundo y el cuarto por 10, tendremos 70 y 70; entonces, arreglando los múltiplos, tenemos—

64 + 8

70

64

70

Consecuentemente, 64 + 8, o 72, es mayor que 70, pero 64 no es mayor que 70, $\therefore$ por la séptima definición, 9 tiene mayor proporción a 7 que 8 tiene a 7.

Lo anterior es meramente ilustrativo de la demostración anterior, ya que esta propiedad podría mostrarse de estos u otros números muy fácilmente de la siguiente manera; porque si un antecedente contiene este consecuente un mayor número de veces que otro antecedente contiene su consecuente, o cuando se forma una fracción de un antecedente para el numerador, y su consecuente para el denominador será mayor que otra fracción que se forma de otro antecedente para el numerador y su consecuente para el denominador, la razón del primer antecedente a su consecuente es mayor que la razón del último antecedente a su consecuente.

Por lo tanto, el número 9 tiene una relación mayor a 7, que 8 tiene a 7, ya que
                
9
/
7

                es mayor que
                
8
/
7
.

Nuevamente, 17 : 19 es una razón mayor que 13 : 15, porque
                

17
/
19

$=$

17 × 15
/
19 × 15

$=$

255
/
285
,
                
                y
                

13
/
15

$=$

13 × 19
/
15 × 19

$=$

247
/
285
,
                
                por lo tanto, es evidente que
                
255
/
285

                es mayor que
                
247
/
285
,
                $\therefore$

17
/
19

                es mayor que
                
13
/
15
,
                y de acuerdo con lo que se muestra arriba, 17 tiene a 19 una razón mayor que 13 tiene a 15.

De modo que los términos generales sobre los cuales existe una razón mayor, igual o menor son los siguientes:—

Si
                
A
/
B

                es mayor que
                
C
/
D
,
                se dice que A tiene a B una razón mayor que C tiene a D; si
                
A
/
B

                es igual a
                
C
/
D
,
                entonces A tiene la misma razón que C tiene a D; y si
                
A
/
B

                es menor que
                
C
/
D
,
                se dice que A tiene a B una razón menor que C tiene a D.

El alumno debe comprender perfectamente todo hasta esta proposición antes de continuar, para comprender completamente las siguientes proposiciones de este libro. Por lo tanto, recomendamos encarecidamente que el alumno comience de nuevo, y lea esto lentamente, y razone cuidadosamente en cada paso, a medida que avanza, especialmente evitando el sistema perjudicial de depender totalmente de la memoria. Siguiendo estas instrucciones, encontrará que las partes que generalmente presentan dificultades considerables no presentarán dificultades en el procesamiento del estudio de este importante libro. \byref{prop:V.VII} \byref{prop:V.VIII} \byref{prop:V.VIII} \byref{prop:V.VII} \byref{\hypref}
\end{center}

\qed

\starttheorem{Prop. XI. Theor.}\label{prop:V.XI}

\defineNewPicture{
byMagnitudeSymbolDefine.I("rhombus", byblue, 0);
byMagnitudeSymbolDefine.II("square", byblue, 0);
byMagnitudeSymbolDefine.III("miniTriangleUp", byblack, 0);
byMagnitudeSymbolDefine.IV("miniCircle", byblack, 0);
byMagnitudeSymbolDefine.V("circle", byred, 0);
byMagnitudeSymbolDefine.VI("wedgeDown", byyellow, 0);
}
\drawCurrentPictureInMargin
\problem{S}{ea}{ Blue diamond}

\begin{center}
 : 
Yellow square

 :: 
Red circle

 : 
Yellow square

, entonces 
Blue diamond

 $=$ 
Red circle

.

Porque, si no es 
Blue diamond

 > 
Red circle

, entonces
                     \\ 
Blue diamond

 : 
Yellow square

 > 
Red circle

 : 
Yellow square

 (L. 5. pr. 8),
                     \\ lo cual es absurdo según la hipótesis.
                     \\ $\therefore$ 
Blue diamond

 no es > 
Red circle

.

De la misma manera se puede demostrar que
                     \\ 
Red circle

 no es > 
Blue diamond

,
 \\ $\therefore$ 
Blue diamond

 $=$ 
Red circle

.

Nuevamente, sea 
Yellow square

 : 
Blue diamond

 :: 
Yellow square

 : 
Red circle

, entonces 
Blue diamond

 $=$ 
Red circle

.

Ya que (invert.) 
Blue diamond

 : 
Yellow square

 :: 
Red circle

 : 
Yellow square

,
 \\ por lo tanto, por el primer caso, 
Blue diamond

 $=$ 
Red circle

.

$\therefore$ Las magnitudes que tienen la misma razón, etcétera.

Esto puede mostrarse de otra manera, de la siguiente manera:—

Sea A : B $=$ A : C, entonces B $=$ C, ya que como la fracción
                
A
/
B

$=$ la fracción
                
A
/
C
,
                y el numerador de una es igual al numerador de la otra, por lo tanto, el denominador de estas fracciones es igual, es decir B $=$ C.

Nuevamente, si B : A $=$ C : A, B $=$ C. Ya que, como
                
B
/
A

$=$

C
/
A
,
                B debe ser $=$ C. \byref{def:V.V} \byref{def:V.V}
\end{center}

\qed

\starttheorem{Prop. XII. Theor.}\label{prop:V.XII}

\defineNewPicture{
byMagnitudeSymbolDefine.I("square", byred, 0);
byMagnitudeSymbolDefine.II("circle", byred, 0);
byMagnitudeSymbolDefine.III("semicircleDown", byblack, 1);
byMagnitudeSymbolDefine.IV("sectorUp", byblack, 1);
byMagnitudeSymbolDefine.V("rhombus", byyellow, 0);
byMagnitudeSymbolDefine.VI("wedgeDown", byyellow, 0);
byMagnitudeSymbolDefine.VII("miniCircle", byblue, 0);
byMagnitudeSymbolDefine.VIII("miniTriangleDown", byblue, 0);
byMagnitudeSymbolDefine.IX("miniTriangleUp", byblack, 0);
byMagnitudeSymbolDefine.X("miniCircle", byblack, 0);
}
\drawCurrentPictureInMargin
\problem{E}{sa}{ magnitud que tiene una razón mayor que otra tiene a la misma magnitud, es la mayor de las dos; y esa magnitud a la que la misma tiene una razón mayor, que esta tiene a otra magnitud, es la menor de las dos.}

\begin{center}
Sea 
Blue home

 : 
Yellow square

 > 
Red circle

 : 
Yellow square

, entonces 
Blue home

 > 
Red circle

.

Ya que si no es 
Blue home

 $=$ o < 
Red circle

;
 \\ entonces, 
Blue home

 : 
Yellow square

 $=$ 
Red circle

 : 
Yellow square

 (L. 5. pr. 7) or
                     \\ 
Blue home

 : 
Yellow square

 < 
Red circle

 : 
Yellow square

 (L. 5. pr. 8) y (invert.),
                     \\ lo cual es absurdo según la hipótesis.

$\therefore$ 
Blue home

 no es $=$ o < 
Red circle

, y
                     \\ $\therefore$ 
Blue home

 debe ser > 
Red circle

.

Nuevamente, sea 
Yellow square

 : 
Red circle

 > 
Yellow square

 : 
Blue home

,
 \\ entonces, 
Red circle

 < 
Blue home

.

Porque si no, 
Red circle

 debe ser > o $=$ 
Blue home

,
 \\ entonce 
Yellow square

 : 
Red circle

 < 
Yellow square

 : 
Blue home

 (L. 5. pr. 8) t (invert.);
                     \\ o 
Yellow square

 : 
Red circle

 $=$ 
Yellow square

 : 
Blue home

 (L. 5. pr. 7), lo que es absurdo (hip.);
                     \\ $\therefore$ 
Red circle

 no es > o $=$ 
Blue home

,
 \\ y $\therefore$ 
Red circle

 debe ser < 
Blue home

.

$\therefore$ Esa magnitud que tiene, etcétera. \byref{def:V.V}
\end{center}

\qed

\starttheorem{Prop. XIII. Theor.}\label{prop:V.XIII}

\defineNewPicture{
byMagnitudeSymbolDefine.I("wedgeDown", byblue, 0);
byMagnitudeSymbolDefine.II("semicircleDown", byblue, 1);
byMagnitudeSymbolDefine.III("square", byred, 0);
byMagnitudeSymbolDefine.IV("rhombus", byyellow, 0);
byMagnitudeSymbolDefine.V("sectorUp", byblack, 1);
byMagnitudeSymbolDefine.VI("circle", byblack, 0);
}
\drawCurrentPictureInMargin
\problem{R}{azones}{ que son lo mismo a la misma razón, son iguales entre sí.}

\begin{center}
Sea 
Blue diamond

 : 
Blue square

 $=$ 
Red circle

 : 
Yellow home

 y 
Red circle

 : 
Yellow home

 $=$ 
Black triangle

 : 
Black circle

,
                     \\ entonces 
Blue diamond

 : 
Blue square

 $=$ 
Black triangle

 : 
Black circle

.

Ya que M 
Blue diamond

 >, $=$, o < m 
Blue square

,
 \\ entonces M 
Red circle

 >, $=$, o < m 
Yellow home

,
 \\ y si M 
Red circle

 >, $=$, or < m 
Yellow home

,
 \\ entonces M 
Black triangle

 >, $=$, o < m 
Black circle

, (L. 5. def. 5);
                     \\ $\therefore$ si M 
Blue diamond

 >, $=$, o < m 
Blue square

, M 
Black triangle

 >, $=$, o < m 
Black circle

,
                     \\ y $\therefore$ (L. 5. def. 5) 
Blue diamond

 : 
Blue square

 $=$ 
Black triangle

 : 
Black circle

.

$\therefore$ Razones que son lo mismo, etcétera. \byref{def:V.V}
\end{center}

\qed

\starttheorem{Prop. XIV. Theor.}\label{prop:V.XIV}

\defineNewPicture{
byMagnitudeSymbolDefine.I("wedgeDown", byred, 0);
byMagnitudeSymbolDefine.II("semicircleDown", byblack, 1);
byMagnitudeSymbolDefine.III("square", byyellow, 0);
byMagnitudeSymbolDefine.IV("rhombus", byblue, 0);
}
\drawCurrentPictureInMargin
\problem{S}{i}{ cualquier número de magnitudes es proporcional como uno de los antecedentes es a su consecuente, entonces todos los antecedentes tomados juntos serán a todos los consecuentes.}

\begin{center}
Sea 
Red square

 : 
Red circle

 $=$ 
Black dome

 : 
Black drop

 $=$ 
Yellow diamond

 : 
Yellow home

 $=$ 
Blue circle

 : 
Blue triangle

 $=$ 
Black triangle

 : 
Black circle

;
                     \\ entonces 
Red square

 : 
Red circle

 $=$
 \\ 
Red square

 + 
Black dome

 + 
Yellow diamond

 + 
Blue circle

 + 
Black triangle

 : 
Red circle

 + 
Black drop

 + 
Yellow home

 + 
Blue triangle

 + 
Black circle

.

Por si M 
Red square

 > m 
Red circle

, entonces M 
Black dome

 > m 
Black drop

,
 \\ y M 
Yellow diamond

 > m 
Yellow home

 M 
Blue circle

 > m 
Blue triangle

,
                     \\ también M 
Black triangle

 > m 
Black circle

. (L. 5. def. 5.)

Por lo tanto M 
Red square

 + M 
Black dome

 + M 
Yellow diamond

 + M 
Blue circle

 + M 
Black triangle

,
                     \\ o M (
Red square

 + 
Black dome

 + 
Yellow diamond

 + 
Blue circle

 + 
Black triangle

) es mayor
                     \\ que m 
Red circle

 + m 
Black drop

 + m 
Yellow home

 + m 
Blue triangle

 + m 
Black circle

,
                     \\ o m (
Red circle

 + 
Black drop

 + 
Yellow home

 + 
Blue triangle

 + 
Black circle

).

De la misma manera se puede mostrar, si M veces uno de los antecedentes es igual o menor que m veces uno de los consecuentes, M veces todos los antecedentes tomados en conjunto, será igual o menor que m veces todos los consecuentes tomados juntos. Por lo tanto, según la quinta definición, como uno de los antecedentes es su consecuente, también lo son todos los antecedentes tomados en conjunto con todos los consecuentes tomados juntos.

$\therefore$ Si cualquier número de magnitudes, etcétera. \byref{prop:V.VIII} \byref{prop:V.XIII} \byref{prop:V.X} \byref{prop:V.VII} \byref{\hypref} \byref{prop:V.XI} \byref{prop:V.IX}
\end{center}

\qed

\starttheorem{Prop. XV. Theor.}\label{prop:V.XV}

\defineNewPicture{
byMagnitudeSymbolDefine.I("circle", byred, 0);
byMagnitudeSymbolDefine.II("square", byyellow, 0);
}
\drawCurrentPictureInMargin
\problem{S}{i}{ la primera tiene a la segunda la misma razón que la tercera a la cuarta, pero la tercera a la cuarta tiene una razón mayor que la quinta a la sexta; la primera también tendrá a la segunda una razón mayor que la quinta a la sexta.}

\begin{center}
Sea 
Blue home

 : 
Blue dome

 $=$ 
Red square

 : 
Yellow diamond

, pero 
Red square

 : 
Yellow diamond

 > 
Black drop

 : 
Black circle

,
 \\ entonces 
Blue home

 : 
Blue dome

 > 
Black drop

 : 
Black circle

.

Porque 
Red square

 : 
Yellow diamond

 > 
Black drop

 : 
Black circle

, hay algunos múltiplos (M′ y m′) de 
Red square

 y 
Black drop

, y de 
Yellow diamond

 y 
Black circle

, tal que M′ 
Red square

 > m′ 
Yellow diamond

,
 \\ pero M′ 
Black drop

 no es > m′ 
Black circle

, según la séptima definición.

Deja que se tomen estos múltiplos y toma los mismos múltiplos 
Blue home

 y 
Blue dome

.

$\therefore$ (L. 5. def. 5.) si M′ 
Blue home

 >, $=$, o < m′ 
Blue dome

;
 \\ entonces M′ 
Red square

 >, $=$, < m′ 
Yellow diamond

,
 \\ pero M′ 
Red square

 > m′ 
Yellow diamond

 (construcción);

$\therefore$ M′ 
Blue home

 > m′ 
Blue dome

,
 \\ pero M′ 
Black drop

 no es > m′ 
Black circle

 (conſtruction);
                     \\ y por lo tanto según la séptima definición,
                     \\ 
Blue home

 : 
Blue dome

 > 
Black drop

 : 
Black circle

.

$\therefore$ Si la primera tiene a la segunda, etcétera. \byref{prop:V.XII}
\end{center}

\qed

\starttheorem{Prop. XVI. Theor.}\label{prop:V.XVI}

\defineNewPicture{
byMagnitudeSymbolDefine.I("wedgeDown", byred, 0);
byMagnitudeSymbolDefine.II("semicircleDown", byblack, 1);
byMagnitudeSymbolDefine.III("square", byyellow, 0);
byMagnitudeSymbolDefine.IV("rhombus", byblue, 0);
}
\drawCurrentPictureInMargin
\problem{S}{i}{ la primera tiene la misma razón con la segunda que la tercera con la cuarta; entonces, si la primera es mayor que la tercera, la segundo será mayor que la cuarta; y si es igual, igual; y si es menor, menor.}

\begin{center}
Sea 
Red home

 : 
Black dome

 :: 
Yellow square

 : 
Blue diamond

, primero supongamos
                     \\ 
Red home

 > 
Yellow square

, entonces 
Black dome

 > 
Blue diamond

.

Ya que 
Red home

 : 
Black dome

 > 
Yellow square

 : 
Black dome

 (L. 5. pr. 8), por la
                     \\ hipótesis, 
Red home

 : 
Black dome

 $=$ 
Yellow square

 : 
Blue diamond

;
 \\ $\therefore$ 
Yellow square

 : 
Blue diamond

 > 
Yellow square

 : 
Black dome

 (L. 5. pr. 13),
                     \\ $\therefore$ 
Blue diamond

 < 
Black dome

 (L. 5. pr. 10.), or 
Black dome

 > 
Blue diamond

.

En segundo lugar, sea 
Red home

 $=$ 
Yellow square

, entonces 
Black dome

 $=$ 
Blue diamond

.

Ya que 
Red home

 : 
Black dome

 $=$ 
Yellow square

 : 
Black dome

 (L. 5. pr. 7),
                     \\ y 
Red home

 : 
Black dome

 $=$ 
Yellow square

 : 
Blue diamond

 (hip.);
                     \\ $\therefore$ 
Yellow square

 : 
Black dome

 $=$ 
Yellow square

 : 
Blue diamond

 (L. 5. pr. 11),
                     \\ y $\therefore$ 
Black dome

 $=$ 
Blue diamond

 (L. 5, pr. 9).

En tercer lugar 
Red home

 < 
Yellow square

, entonces 
Black dome

 < 
Blue diamond

;
 \\ porque 
Yellow square

 > 
Red home

 y 
Yellow square

 : 
Blue diamond

 $=$ 
Red home

 : 
Black dome

;
 \\ $\therefore$ 
Blue diamond

 > 
Black dome

, por el primer caso,
                     \\ es decir, 
Black dome

 < 
Blue diamond

.

$\therefore$ Si la primera tiene la misma razón, etcétera. \byref{prop:V.XV} \byref{\hypref} \byref{prop:V.XI} \byref{prop:V.XV} \byref{prop:V.XIV} \byref{prop:V.XIV}
\end{center}

\qed

\starttheorem{Prop. XVII. Theor.}\label{prop:V.XVII}

\defineNewPicture{
byMagnitudeSymbolDefine.I("wedgeDown", byred, 0);
byMagnitudeSymbolDefine.II("semicircleDown", byblack, 1);
byMagnitudeSymbolDefine.III("square", byyellow, 0);
byMagnitudeSymbolDefine.IV("rhombus", byblue, 0);
}
\drawCurrentPictureInMargin
\problem{S}{ean}{ Red circle}

\begin{center}
 y 
Yellow square

 dos magnitudes;
                     \\ entonces 
Red circle

 : 
Yellow square

 :: M′ 
Red circle

 : M′ 
Yellow square

.

Ya que 
Red circle

 : 
Yellow square

 $=$ 
Red circle

 : 
Yellow square


$=$ 
Red circle

 : 
Yellow square


$=$ 
Red circle

 : 
Yellow square

$\therefore$ 
Red circle

 : 
Yellow square

 :: 4 
Red circle

 : 4 
Yellow square

. (L. 5. pr. 12).

Un mismo razonamiento es generalmente aplicable, tenemos

Red circle

 : 
Yellow square

 :: M′ 
Red circle

 : M′ 
Yellow square

.

$\therefore$ Las magnitudes tienen la misma relación, etcétera.

El término técnico permutando o alternando, por permutación o alternancia, se usa cuando hay cuatro proporcionales, y se infiere que la primera tiene la misma razón con la tercera que la segunda con la cuarta; o que la primera es para la tercera como la segunda para la cuarta: como se muestra en la siguiente proposición:—

Sea 
Yellow circle

 : 
Black diamond

 :: 
Red home

 : 
Blue square

,
 \\ por “permutando” or “alternando” es
                 \\ inferido 
Yellow circle

 : 
Red home

 :: 
Black diamond

 : 
Blue square

.

Puede ser necesario aquí señalar que las magnitudes 
Yellow circle

, 
Black diamond

, 
Red home

, 
Blue square

, deben ser homogéneas, es decir, de la misma naturaleza o similitud de tipo; Por lo tanto, en tales casos, debemos comparar líneas con líneas, superficies con superficies, sólidos con sólidos, etc. Así, el alumno percibirá fácilmente que una línea y una superficie, una superficie y un sólido, u otras magnitudes heterogéneas, nunca pueden ubicarse en la relación de antecedente y consecuente. \byref{\hypref} \byref{def:V.V} \byref{def:V.V}
\end{center}

\qed

\starttheorem{Prop. XVIII. Theor.}\label{prop:V.XVIII}

\defineNewPicture{
byMagnitudeSymbolDefine.I("wedgeDown", byred, 0);
byMagnitudeSymbolDefine.II("semicircleDown", byblack, 1);
byMagnitudeSymbolDefine.III("square", byyellow, 0);
byMagnitudeSymbolDefine.IV("rhombus", byblue, 0);
byMagnitudeSymbolDefine.V("circle", byblack, 0);
}
\drawCurrentPictureInMargin
\problem{S}{i}{ cuatro magnitudes del mismo tipo son proporcionales, también son proporcionales cuando se toman alternativamente.}

\begin{center}
Sea 
Red home

 : 
Black dome

 :: 
Yellow square

 : 
Blue diamond

, entonces 
Red home

 : 
Yellow square

 :: 
Black dome

 : 
Blue diamond

.

Ya que M 
Red home

 : M 
Black dome

 :: 
Red home

 : 
Black dome

 (L. 5. pr. 15),
                     \\ y M 
Red home

 : M 
Black dome

 :: 
Yellow square

 : 
Blue diamond

 (hip.) y (L. 5. pr. 11);
                     \\ también m 
Yellow square

 : m 
Blue diamond

 :: 
Yellow square

 : 
Blue diamond

 (L. 5. pr. 15);
                     \\ $\therefore$ M 
Red home

 : M 
Black dome

 :: m 
Yellow square

 : m 
Blue diamond

 (L. 5. pr. 14),
                     \\ y $\therefore$ si M 
Red home

 >, $=$, or < m 
Yellow square

,
 \\ entonces M 
Black dome

 >, $=$, o < m 
Blue diamond

 (L. 5. pr. 14);
                     \\ por lo tanto por la quinta definición,
                     \\ 
Red home

 : 
Yellow square

 :: 
Black dome

 : 
Blue diamond

.

$\therefore$ Si cuatro magnitudes del mismo tipo, etcétera.

Dividendo, por división, cuando hay cuatro proporcionales, y se infiere, que el excedente de la primera sobre la segunda es para la segunda, como el excedente de la tercera sobre la cuarta, es para la cuarta.

Sea A : B :: C : D;
                 \\ por “dividendo” es inferido
                 \\ A menos B : B :: C menos D : D.

De acuerdo con lo anterior, se supone que, A es mayor que B, y C mayor que D; si este no es el caso, pero para tener B mayor que A, y D mayor que C, B y D pueden hacerse pasar por antecedentes, y A y C como consecuentes, por “inversión”

B : A :: D : C;
                 \\ entonces, por “dividendo,” inferimos
                 \\ B menos A : A :: D menos C : C. \byref{prop:V.XVII} \byref{\hypref} \byref{prop:V.XI} \byref{prop:V.IX}
\end{center}

\qed

\starttheorem{Prop. XIX. Theor.}\label{prop:V.XIX}

\defineNewPicture{
byMagnitudeSymbolDefine.I("wedgeDown", byred, 0);
byMagnitudeSymbolDefine.II("semicircleDown", byblack, 1);
byMagnitudeSymbolDefine.III("square", byblue, 0);
byMagnitudeSymbolDefine.IV("rhombus", byyellow, 0);
}
\drawCurrentPictureInMargin
\problem{S}{i}{ las magnitudes, tomadas en conjunto, son proporcionales, también serán proporcionales cuando se toman por separado: es decir, si dos magnitudes juntas tienen para una de ellas la misma razón que las otras dos tienen para una de ellas, la restante de las primeras dos tendrá a la otra la misma razón que la restante de las dos últimas tiene a la otra de estas.}

\begin{center}
Sea 
Red home

 + 
Black dome

 : 
Black dome

 :: 
Yellow square

 + 
Blue diamond

 : 
Blue diamond

,
 \\ entonces 
Red home

 : 
Black dome

 :: 
Yellow square

 : 
Blue diamond

.

Toma M 
Red home

 > m 
Black dome

  a cada una agrega M 
Black dome

,
 \\ entonces tenemos M 
Red home

 + M 
Black dome

 > m 
Black dome

 + M 
Black dome

,
 \\ o M (
Red home

 + 
Black dome

) > (m + M) 
Black dome

:
 \\ pero porque 
Red home

 + 
Black dome

 : 
Black dome

 :: 
Yellow square

 + 
Blue diamond

 : 
Blue diamond

 (hip.),
                     \\ y M (
Red home

 + 
Black dome

) > (m + M) 
Black dome

;
 \\ $\therefore$ M (
Yellow square

 + 
Blue diamond

) > (m + M) 
Blue diamond

 (L. 5. def. 5);
                     \\ $\therefore$ M 
Yellow square

 + M 
Blue diamond

 > m 
Blue diamond

 + M 
Blue diamond

;
 \\ $\therefore$ M 
Yellow square

 > m 
Blue diamond

, tomando M 
Blue diamond

 de ambos lados:
                     \\ es decir, cuando M 
Red home

 > m 
Black dome

, entonces M 
Yellow square

 > m 
Blue diamond

.

De la misma manera se puede demostrar que si
                     \\ M 
Red home

 $=$ or < m 
Black dome

, entonces M 
Yellow square

 $=$ or < m 
Blue diamond

;
 \\ y $\therefore$ 
Red home

 : 
Black dome

 :: 
Yellow square

 : 
Blue diamond

 (L. 5. def. 5).

$\therefore$ Si las magnitudes, tomadas en conjunto, etcétera.

El término componendo, por composición, se usa cuando hay cuatro proporcionales; y se infiere que la primera junto con la segunda es a la segunda como la tercera junto con la cuarta es a la cuarta.

Sea A : B :: C : D;
                 \\ entonces, por el término “componendo,” es inferido que
                 \\ A + B : B :: C + D : D.

Por “inversión” B y D pueden convertirse en la primera y la tercera, A y C la segunda y la cuarta como

B : A :: D : C,
                 \\ entonces, por “componendo,” inferimos que
                 \\ B + A : A :: D + C : C. \byref{\hypref} \byref{prop:V.XI} \byref{prop:V.XXII} \byref{prop:V.XXIII}
\end{center}

\qed

\starttheorem{Prop. XX. Theor.}\label{prop:V.XX}

\defineNewPicture{
byMagnitudeSymbolDefine.I("wedgeDown", byblue, 0);
byMagnitudeSymbolDefine.II("semicircleDown", byred, 1);
byMagnitudeSymbolDefine.III("square", byyellow, 0);
byMagnitudeSymbolDefine.IV("rhombus", byblue, 0);
byMagnitudeSymbolDefine.V("sectorUp", byred, 1);
byMagnitudeSymbolDefine.VI("circle", byyellow, 0);
}
\drawCurrentPictureInMargin
\problem{S}{i}{ las magnitudes, tomadas por separado, son proporcionales, también deben ser proporcionales cuando se toman conjuntamente: es decir, si la primera es a la segunda como la tercera es a la cuarta, la primera y la segundo juntas serán a la segunda como la tercera y la cuarta juntas son a la cuarta.}

\begin{center}
Sea 
Red home

 : 
Black dome

 :: 
Yellow square

 : 
Blue diamond

,
 \\ entonces 
Red home

 + 
Black dome

 : 
Black dome

 :: 
Yellow square

 + 
Blue diamond

 : 
Blue diamond

;
 \\ porque si no, sea 
Red home

 + 
Black dome

 : 
Black dome

 :: 
Yellow square

 + 
Black circle

 : 
Black circle

,
 \\ suponiendo 
Black circle

 no es $=$ 
Blue diamond

;
 \\ $\therefore$ 
Red home

 : 
Black dome

 :: 
Yellow square

 : 
Black circle

 (L. 5. pr. 17);
                 \\ pero 
Red home

 : 
Black dome

 :: 
Yellow square

 : 
Blue diamond

 (hip.);
                 \\ $\therefore$ 
Yellow square

 : 
Black circle

 :: 
Yellow square

 : 
Blue diamond

 (L. 5. pr. 11);
                 \\ $\therefore$ 
Black circle

 $=$ 
Blue diamond

 (L. 5. pr. 9),
                 \\ lo cual es contrario a la suposición;
                 \\ $\therefore$ 
Black circle

 no es desigual 
Blue diamond

;
 \\ es decir 
Black circle

 $=$ 
Blue diamond

;
 \\ $\therefore$ 
Red home

 + 
Black dome

 : 
Black dome

 :: 
Yellow square

 + 
Blue diamond

 : 
Blue diamond

.

$\therefore$ Si las magnitudes, tomadas por separado, etcétera. \byref{prop:V.XI} \byref{prop:V.XIV}
\end{center}

\qed

\starttheorem{Prop. XXI. Theor.}\label{prop:V.XXI}

\defineNewPicture{
byMagnitudeSymbolDefine.I("wedgeDown", byyellow, 0);
byMagnitudeSymbolDefine.II("wedgeUp", byred, 0);
byMagnitudeSymbolDefine.III("square", byblue, 0);
byMagnitudeSymbolDefine.IV("rhombus", byblue, 0);
byMagnitudeSymbolDefine.V("sectorUp", byred, 1);
byMagnitudeSymbolDefine.VI("circle", byyellow, 0);
}
\drawCurrentPictureInMargin
\problem{S}{i}{ una magnitud completa es para un todo, como una magnitud tomada de la primera, es a una magnitud tomada de la otra; el resto será para el resto, como el todo para el todo.}

\begin{center}
Sea 
Red home

 + 
Black dome

 : 
Blue square

 + 
Yellow diamond

 :: 
Red home

 : 
Blue square

,
 \\ entonces 
Black dome

 : 
Yellow diamond

 :: 
Red home

 + 
Black dome

 : 
Blue square

 + 
Yellow diamond

,

Ya que 
Red home

 + 
Black dome

 : 
Red home

 :: 
Blue square

 + 
Yellow diamond

 : 
Blue square

 (alter.),

$\therefore$ 
Black dome

 : 
Red home

 :: 
Yellow diamond

 : 
Blue square

 (divid.),
                     \\ de nuevo 
Black dome

 : 
Yellow diamond

 :: 
Red home

 : 
Blue square

 (alter.),
                     \\ pero 
Red home

 + 
Black dome

 : 
Blue square

 + 
Yellow diamond

 :: 
Red home

 : 
Blue square

 (hip.);
                     \\ por lo tanto 
Black dome

 : 
Yellow diamond

 :: 
Red home

 + 
Black dome

 : 
Blue square

 + 
Yellow diamond
 (L. 5. pr. 11).

$\therefore$ Si una magnitud completa es para un todo, etcétera.

El término “convertendo,” por conversión, es utilizado por los geómetras, cuando hay cuatro proporcionales, y se infiere que la primera es a su excedente sobre la segunda, como la tercera es a su excedente sobre sobre la cuarta. Ver la siguiente proposición:—

Si cuatro magnitudes son proporcionales, también son proporcionales por conversión: es decir, la primera es a su excedente sobre la segunda, como la tercera es a su excedente sobre sobre la cuarta.

Sea 
Blue circle


Black drop

 : 
Black drop

 :: 
Red square


Yellow diamond

 : 
Yellow diamond

,
 \\ entonces 
Blue circle


Black drop

 : 
Blue circle

 :: 
Red square


Yellow diamond

 : 
Red square

,

Porque 
Blue circle


Black drop

 : 
Black drop

 :: 
Red square


Yellow diamond

 : 
Yellow diamond

;
 \\ por lo tanto 
Blue circle

 : 
Black drop

 :: 
Red square

 : 
Yellow diamond

 (divid.),

$\therefore$ 
Black drop

 : 
Blue circle

 :: 
Yellow diamond

 : 
Red square

 (inver.),

$\therefore$ 
Blue circle


Black drop

 : 
Blue circle

 :: 
Red square


Yellow diamond

 : 
Red square

 (compo.).

$\therefore$ Si cuatro magnitudes, etcétera.

“Ex æquali” (sc. distancia), or ex æquo de igualdad de distancia: cuando hay cualquier cantidad de magnitudes más de dos, y tantas otras, de manera que sean proporcionales cuando se toman dos y dos de cada rango, y se infiere que la primera es a la última del primer rango de magnitudes, como la primera es a la última de las otras: “de esto hay los dos tipos siguientes, que surgen del diferente orden en que se toman las magnitudes, dos y dos.”

“Ex æquali,” de la igualdad. Este término se usa simplemente por sí mismo, cuando la primera magnitud es a la segunda del primer rango, como la primera a la segunda del otro rango; y como la segunda es a la tercera del primer rango, así la segunda a la tercera del otro; y así sucesivamente en orden: y la inferencia es como se menciona en la definición anterior; donde esto se llama proposición ordenada. Se demuestra en el Libro 5, pr. 22.

Por lo tanto, si hay dos rangos de magnitudes,

A, B, C, D, E, F, el primer rango,
                 \\ y L, M, N, O, P, Q, el segundo,
                 \\ tal que A : B :: L : M, B : C :: M : N,
                 \\ C : D :: N : O, D : E :: O : P, E : F :: P : Q;
                 \\ inferimos por el término “ex æquali” que
                 \\ A : F :: L : Q.

“Ex æquali in proporione perturbatâ ſeu inordinatâ,” de la igualdad en perturbación, o proporción desordenada. Este término se usa cuando la primera magnitud es a la segunda del primer rango como la penúltima es a la última del segundo rango; y como la segunda es a la tercera del primer rango, así la antepenúltima es a la penúltima del segundo rango; y como la tercer es a la cuarta del primer rango, así la ante antepenúltima es a la antepenúltima del segundo rango; y así sucesivamente en orden cruzado: y la inferencia está en la décimo octava definición. Se demuestra en L. 5. pr. 23.

Por lo tanto, si hay dos rangos de magnitudes,

A, B, C, D, E, F, el primer rango,
                 \\ y L, M, N, O, P, Q, el segundo,
                 \\ tal que A : B :: P : Q, B : C :: O : P,
                 \\ C : D :: N : O, D : E :: M : N, E : F :: L : M;
                 \\ el término “ex æquali in proporione perturbatâ ſeu inordinatâ” infiere que
                 \\ A : F :: L : Q. \byref{prop:V.VIII} \byref{\hypref} \byref{prop:V.XIII} \byref{\hypref} \byref{prop:V.XIII} \byref{prop:V.VII} \byref{\hypref} \byref{prop:V.XI} \byref{prop:V.IX}
\end{center}

\qed

\starttheorem{Prop. XXII. Theor.}\label{prop:V.XXII}

\defineNewPicture{
byMagnitudeSymbolDefine.I("wedgeDown", byred, 0);
byMagnitudeSymbolDefine.II("rhombus", byblue, 0);
byMagnitudeSymbolDefine.III("square", byyellow, 0);
byMagnitudeSymbolDefine.IV("rhombus", byred, 0);
byMagnitudeSymbolDefine.V("sectorUp", byblue, 1);
byMagnitudeSymbolDefine.VI("circle", byyellow, 0);
byMagnitudeSymbolDefine.Ia("wedgeDown", byblue, 0);
byMagnitudeSymbolDefine.IIa("rhombus", byblack, 0);
byMagnitudeSymbolDefine.IIIa("square", byyellow, 0);
byMagnitudeSymbolDefine.IVa("rhombus", byred, 0);
byMagnitudeSymbolDefine.Va("sectorUp", byblue, 1);
byMagnitudeSymbolDefine.VIa("circle", byblack, 0);
byMagnitudeSymbolDefine.VIIa("halfsquare", byyellow, 0);
byMagnitudeSymbolDefine.VIIIa("halfrhombusUp", byred, 0);
}
\drawCurrentPictureInMargin
\problem{S}{i}{ hay tres magnitudes, y otras tres, que, tomadas dos y dos, tienen la misma razón; entonces, si la primera es mayor que la tercera, la cuarto será mayor que la sexta; y si es igual, igual; y si es menor, menor.}

\begin{center}
Sean 
Blue home

, 
Red dome

, 
Yellow square

, las tres primeras magnitudes,
                     \\ y 
Blue diamond

, 
Red drop

, 
Yellow circle

, las otras tres,
                     \\ tal que 
Blue home

 : 
Red dome

 :: 
Blue diamond

 : 
Red drop

, y 
Red dome

 : 
Yellow square

 :: 
Red drop

 : 
Yellow circle

.

En ese caso, si 
Blue home

 >, $=$, o < 
Yellow square

, entonces 
Blue diamond

 >, $=$, o < 
Yellow circle

.

De la hipótesis, por alternando, tenemos
                     \\ 
Blue home

 : 
Blue diamond

 :: 
Red dome

 : 
Red drop

,
 \\ y 
Red dome

 : 
Red drop

 :: 
Yellow square

 : 
Yellow circle

;

$\therefore$ 
Blue home

 : 
Blue diamond

 :: 
Yellow square

 : 
Yellow circle

 (L. 5. pr. 11);

$\therefore$ si 
Blue home

 >, $=$, o < 
Yellow square

, entonces 
Blue diamond

 >, $=$, o < 
Yellow circle

 (L. 5. pr. 14).

$\therefore$ Si hay tres magnitudes, etcétera. \byref{prop:V.IV} \byref{prop:V.XX} \byref{def:V.V}
\end{center}

\qed

\starttheorem{Prop. XXIII. Theor.}\label{prop:V.XXIII}

\defineNewPicture{
byMagnitudeSymbolDefine.I("wedgeDown", byyellow, 0);
byMagnitudeSymbolDefine.II("semicircleDown", byblue, 1);
byMagnitudeSymbolDefine.III("square", byred, 0);
byMagnitudeSymbolDefine.IV("rhombus", byyellow, 0);
byMagnitudeSymbolDefine.V("sectorUp", byblue, 1);
byMagnitudeSymbolDefine.VI("circle", byred, 0);
byMagnitudeSymbolDefine.VII("halfsquare", byblack, 0);
byMagnitudeSymbolDefine.VIII("halfrhombusUp", byblack, 0);
}
\drawCurrentPictureInMargin
\problem{S}{i}{ hay tres magnitudes, y las otras tres que tienen la misma razón, tomadas dos y dos, pero en un orden cruzado; entonces si la primera magnitud es mayor que la tercera, la cuarta será mayor que la sexta; y si es igual, igual; y si es menor, menor.}

\begin{center}
Sean 
Yellow home

, 
Red home

, 
Blue square

, las tres primeras magnitudes,
                     \\ y 
Blue diamond

, 
Red drop

, 
Yellow circle

, las otras tres,
                     \\ tal que 
Yellow home

 : 
Red home

 :: 
Red drop

 : 
Yellow circle

, y 
Red home

 : 
Blue square

 :: 
Blue diamond

 : 
Red drop

.

En ese caso, si 
Yellow home

 >, $=$, o < 
Blue square

, entonces
                     \\ 
Blue diamond

 >, $=$, o < 
Yellow circle

.

En primer lugar, sea 
Yellow home

 > 
Blue square

:
 \\ entonces, porque 
Red home

 es cualquier otra magnitud,
                     \\ 
Yellow home

 : 
Red home

 > 
Blue square

 : 
Red home

 (L. 5. pr. 8);
                     \\ pero 
Red drop

 : 
Yellow circle

 :: 
Yellow home

 : 
Red home

 (hip.);
                     \\ $\therefore$ 
Red drop

 : 
Yellow circle

 > 
Blue square

 : 
Red home

 (L. 5. pr. 13);
                     \\ y porque 
Red home

 : 
Blue square

 :: 
Blue diamond

 : 
Red drop

 (hip.);
                     \\ $\therefore$ 
Blue square

 : 
Red home

 :: 
Red drop

 : 
Blue diamond

 (inv.),
                     \\ y se mostró que 
Red drop

 : 
Yellow circle

 > 
Blue square

 : 
Red home

,
 \\ $\therefore$ 
Red drop

 : 
Yellow circle

 > 
Red drop

 : 
Blue diamond

 (L. 5. pr. 13);
                     \\ $\therefore$ 
Yellow circle

 < 
Blue diamond

,
 \\ es decir 
Blue diamond

 > 
Yellow circle

.

En segundo lugar, sea 
Yellow home

 $=$ 
Blue square

; entonces 
Blue diamond

 $=$ 
Yellow circle
 Porque 
Yellow home

 $=$ 
Blue square

,
 \\ 
Yellow home

 : 
Red home

 $=$ 
Blue square

 : 
Red home

 (L. 5. pr. 7);
                     \\ pero 
Yellow home

 : 
Red home

 $=$ 
Red drop

 : 
Yellow circle

 (hip.),
                     \\ y 
Blue square

 : 
Red home

 $=$ 
Red drop

 : 
Blue diamond

 (hip. y inv.),
                     \\ $\therefore$ 
Red drop

 : 
Yellow circle

 $=$ 
Red drop

 : 
Blue diamond

 (L. 5. pr. 11),
                     \\ $\therefore$ 
Blue diamond

 $=$ 
Yellow circle

 (L. 5. pr. 9).

Después, sea 
Yellow home

 < 
Blue square

, entonces 
Blue diamond

 será < 
Yellow circle

;
 \\ porque 
Blue square

 > 
Yellow home

,
 \\ y se mostró que 
Blue square

 : 
Red home

 $=$ 
Red drop

 : 
Blue diamond

,
 \\ y 
Red home

 : 
Yellow home

 $=$ 
Yellow circle

 : 
Red drop

;
 \\ $\therefore$ por el primer caso 
Yellow circle

 es > 
Blue diamond

,
 \\ es decir, 
Blue diamond

 < 
Yellow circle

.

$\therefore$ Si hay tres magnitudes, etcétera. \byref{prop:V.XV} \byref{\hypref} \byref{prop:V.XI} \byref{\hypref} \byref{prop:V.IV} \byref{prop:V.XXI} \byref{def:V.V}
\end{center}

\qed

\starttheorem{Prop. XXIV. Theor.}\label{prop:V.XXIV}

\defineNewPicture{
byMagnitudeSymbolDefine.I("wedgeDown", byred, 0);
byMagnitudeSymbolDefine.II("semicircleDown", byblack, 1);
byMagnitudeSymbolDefine.III("square", byblue, 0);
byMagnitudeSymbolDefine.IV("rhombus", byyellow, 0);
byMagnitudeSymbolDefine.V("sectorUp", byred, 1);
byMagnitudeSymbolDefine.VI("circle", byblue, 0);
}
\drawCurrentPictureInMargin
\problem{S}{i}{ hay cualquier cantidad de magnitudes, y tantas otras, que, tomadas de dos y dos en orden, tienen la misma razón; la primera tendrá para la última de las primeras magnitudes la misma proporción que la primera de las otras tiene para la última de esas mismas.}

\begin{center}
N.B.—\emph{Esto suele citarse con las palabras “ex æquali,” o “ex æquo.”}

En primer lugar, sean las magnitudes 
Red home

, 
Blue diamond

, 
Yellow square

,
 \\ y como muchas otras 
Red diamond

, 
Blue drop

, 
Yellow circle

,
 \\ tal que
                 \\ 
Red home

 : 
Blue diamond

 :: 
Red diamond

 : 
Blue drop

,
 \\ y 
Blue diamond

 : 
Yellow square

 :: 
Blue drop

 : 
Yellow circle

;
 \\ entonces deberá 
Red home

 : 
Yellow square

 :: 
Red diamond

 : 
Yellow circle

.

Deje que estas magnitudes, así como cualquier múltiplo igual, cualesquiera que sean los antecedentes y los consecuentes de las razones, mantenerse como sigue:—

Red home

, 
Blue diamond

, 
Yellow square

, 
Red diamond

, 
Blue drop

, 
Yellow circle

,
 \\ y
                     \\ M 
Red home

, m 
Blue diamond

, N 
Yellow square

, M 
Red diamond

, m 
Blue drop

, N 
Yellow circle

,
 \\ porque 
Red home

 : 
Blue diamond

 :: 
Red diamond

 : 
Blue drop

;
 \\ $\therefore$ M 
Red home

 : m 
Blue diamond

 :: M 
Red diamond

 : m 
Blue drop

 (L. 5. p. 4).

Por la misma razón
                     \\ m 
Blue diamond

 : N 
Yellow square

 :: m 
Blue drop

 : N 
Yellow circle

;
 \\ y porque hay tres magnitudes,
                     \\ M 
Red home

, m 
Blue diamond

, N 
Yellow square

,
 \\ y otras tres M 
Red diamond

, m 
Blue drop

, N 
Yellow circle

,
 \\ que, tomadas dos y dos, tienen la misma razón;

$\therefore$ if M 
Red home

 >, $=$, < N 
Yellow square
 entonces M 
Red diamond

 >, $=$, < N 
Yellow circle

, por (L. 5. pr. 20);
                     \\ y $\therefore$ 
Red home

 : 
Yellow square

 :: 
Red diamond

 : 
Yellow circle

 \byref{\hypref}.

Después, sean cuatro magnitudes, 
Blue home

, 
Black diamond

, 
Yellow square

, 
Red diamond

,
 \\ y otras cuatro 
Blue drop

, 
Black circle

, 
Yellow rectangle

, 
Red triangle

,
 \\ que, tomadas dos y dos, tienen la misma razón,
                     \\ es decir, 
Blue home

 : 
Black diamond

 :: 
Blue drop

 : 
Black circle

,
 \\ 
Black diamond

 : 
Yellow square

 :: 
Black circle

 : 
Yellow rectangle

,
 \\ y 
Yellow square

 : 
Red diamond

 :: 
Yellow rectangle

 : 
Red triangle

,
 \\ entonces deberá 
Blue home

 : 
Red diamond

 :: 
Blue drop

 : 
Red triangle

;
 \\ porque 
Blue home

, 
Black diamond

, 
Yellow square

, son tres magnitudes,
                     \\ and 
Blue drop

, 
Black circle

, 
Yellow rectangle

, otras tres,
                     \\ que, tomadas dos y dos, tienen la misma razón;
                     \\ por lo tanto, por el caso anterior, 
Blue home

 : 
Yellow square

 :: 
Blue drop

 : 
Yellow rectangle

,
 \\ pero 
Yellow square

 : 
Red diamond

 :: 
Yellow rectangle

 : 
Red triangle

;
 \\ por lo tanto, de nuevo, por el primer caso, 
Blue home

 : 
Red diamond

 :: 
Blue drop

 : 
Red triangle

;
 \\ y así sucesivamente, sea cual sea el número de magnitudes.

$\therefore$ Si hay cualquier cantidad de magnitudes, etcétera. \byref{\hypref} \byref{prop:V.XXII} \byref{prop:V.XVIII} \byref{\hypref} \byref{prop:V.XXII}
\end{center}

\qed

\starttheorem{Prop. XXV. Theor.}\label{prop:V.XXV}

\defineNewPicture{
byMagnitudeSymbolDefine.Ia("wedgeDown", byred, 0);
byMagnitudeSymbolDefine.III("semicircleDown", byblack, 1);
byMagnitudeSymbolDefine.IIa("square", byblue, 0);
byMagnitudeSymbolDefine.IV("rhombus", byyellow, 0);
}
\drawCurrentPictureInMargin
\problem{S}{i}{ hay cualquier cantidad de magnitudes, y tantas otras, que, tomadas dos y dos en orden cruzado, tienen la misma razón; la primera tendrá a la última de las primeras magnitudes la misma proporción que la primera de las otras tendrá a la última de esas mismas.}

\begin{center}
N.B.—\emph{Esto suele citarse con las palabras “ex æquali in proporione perturbatâ;” or “ex æquo perturbato.”}

Primero, sean tres magnitudes 
Yellow home

, 
Blue dome

, 
Red square

,
 \\ y otras tres, 
Yellow diamond

, 
Blue drop

, 
Red circle

,
 \\ que, tomadas de dos en dos en orden cruzado, tienen las misma razón;
                


es decir, 
Yellow home

 : 
Blue dome

 :: 
Blue drop

 : 
Red circle

,
y 
Blue dome

 : 
Red square

 :: 
Yellow diamond

 : 
Blue drop

,
entonces deberá 
Yellow home

 : 
Red square

 :: 
Yellow diamond

 : 
Red circle

.

Sean estas magnitudes y sus respectivos múltiplos iguales, ordenadas como sigue:—

Yellow home

, 
Blue dome

, 
Red square

, 
Yellow diamond

, 
Blue drop

, 
Red circle

,
 \\ M 
Yellow home

, M 
Blue dome

, m 
Red square

, M 
Yellow diamond

, m 
Blue drop

, m 
Red circle

,
 \\ entonces 
Yellow home

 : 
Blue dome

 :: M 
Yellow home

 : M 
Blue dome

 (L. 5. pr. 15);
                     \\ y por la misma razón
                     \\ 
Blue drop

 : 
Red circle

 :: m 
Blue drop

 : m 
Red circle

;
 \\ pero 
Yellow home

 : 
Blue dome

 :: 
Blue drop

 : 
Red circle

 (hip.),
                     \\ $\therefore$ M 
Yellow home

 : M 
Blue dome

 :: 
Blue drop

 : 
Red circle

 (L. 5. pr. 11);
                     \\ y porque 
Blue dome

 : 
Red square

 :: 
Yellow diamond

 : 
Blue drop

 (hip.),
                     \\ $\therefore$ M 
Blue dome

 : m 
Red square

 :: M 
Yellow diamond

 : m 
Blue drop

 (L. 5. pr. 4);
                     \\ entonces porque hay tres magnitudes,
                     \\ M 
Yellow home

, M 
Blue dome

, m 
Red square

,
 \\ y otras tres, M 
Yellow diamond

, m 
Blue drop

, m 
Red circle

,
 \\ que tomadas de dos en dos en orden cruzado, tienen la misma razón;
                     \\ por consiguiente, si M 
Yellow home

 >, $=$, o < m 
Red square

,
 \\ entonces M 
Yellow diamond

 >, $=$, o < m 
Red circle

 (L. 5. pr. 21),
                     \\ y $\therefore$ 
Yellow home

 : 
Red square

 :: 
Yellow diamond

 : 
Red circle

 (L. 5. def. 5).

Después, sean cuatro magnitudes,
                     \\  
Yellow home

, 
Blue dome

, 
Red square

, 
Yellow diamond

,
 \\ y otras cuatro, 
Blue drop

, 
Red circle

, 
Black rectangle

, 
Black triangle

,
 \\ que tomadas de dos en dos en orden cruzado, tienen la misma razón; a saber,
                    



Yellow home

 : 
Blue dome

 :: 
Black rectangle

 : 
Black triangle

,

Blue dome

 : 
Red square

 :: 
Red circle

 : 
Black rectangle

,
y 
Red square

 : 
Yellow diamond

 :: 
Blue drop

 : 
Red circle

.
entonces deberá 
Yellow home

 : 
Yellow diamond

 :: 
Blue drop

 : 
Black triangle

.



                    Porque 
Yellow home

, 
Blue dome

, 
Red square

 son tres magnitudes,
                     \\ y 
Red square

, 
Black rectangle

, 
Black triangle

, otras tres,
                     \\ que tomadas de dos en dos en orden cruzado, tienen la misma razón,
                     \\ entonces, por el primer caso, 
Yellow home

 : 
Red square

 :: 
Red square

 : 
Black triangle

,
 \\ pero 
Red square

 : 
Yellow diamond

 :: 
Blue drop

 : 
Red circle

,
 \\ por lo tanto, de nuevo, por el primer caso, 
Yellow home

 : 
Yellow diamond

 :: 
Blue drop

 : 
Black triangle

;
 \\ y así sucesivamente, sea cual sea el número de magnitudes.

$\therefore$ Si hay cualquier cantidad de magnitudes, etcétera. \byref{prop:V.A,prop:V.XIV} \byref{prop:V.XIX} \byref{\hypref} \byref{prop:V.A} \byref{\hypref} \byref{\hypref} \byref{\hypref}
\end{center}

\qed

\part{Book VI}

\chapter*{Definitions}

\startdefinition{}\label{def:VI.I}

\defineNewPicture{
pair a, b, c, A, B, C, d, e, f, g, D, E ,F ,G;
a := (0, 0);
b := (u, 0);
c := (1/2u, u);
A := (a scaled 2) shifted (u, -u);
B := (b scaled 2) shifted (u, -u);
C := (c scaled 2) shifted (u, -u);
d := (0, 0);
e := (2/3u, 0);
f := (u, 2/3u);
g := (1/3u, 2/3u);
D := d scaled 2;
E := e scaled 2;
F := f scaled 2;
G := g scaled 2;
d := d shifted (0, -5/2u);
e := e shifted (0, -5/2u);
f := f shifted (0, -5/2u);
g := g shifted (0, -5/2u);
D := D shifted (u, -5/2u);
E := E shifted (u, -5/2u);
F := F shifted (u, -5/2u);
G := G shifted (u, -5/2u);
draw byPolygon(a,b,c)(byblue);
draw byPolygon(A,B,C)(byblue);
draw byPolygon(d,e,f,g)(byred);
draw byPolygon(D,E,F,G)(byred);
}
\drawCurrentPictureInMargin
\begin{center}
Se dice que las figuras rectilíneas son similares, cuando tienen varios ángulos iguales, cada uno para cada uno, y los lados alrededor de los ángulos iguales son proporcionales.
\end{center}

\startdefinition{}\label{def:VI.II}

\begin{center}
Se dice que dos lados de una figura son recíprocamente proporcionales a los dos lados de otra figura cuando uno de los lados del primero es al segundo, como el lado restante del segundo es al lado restante del primero.
\end{center}

\startdefinition{}\label{def:VI.III}

\begin{center}
Se dice que una línea recta se corta en un extremo y razón media, cuando el todo es al segmento mayor, como el segmento mayor es al menor.
\end{center}

\startdefinition{}\label{def:VI.IV}

\defineNewPicture{
pair d[];
d1 := (0, 0);
d2 := (5/2u, 0);
d3 := (5u, 0);
d4 := (15/2u, 0);
numeric h;
h := 6/5u;
pair A, B, C, H;
A := (0, 0) shifted d1;
B := (3/2u, 0) shifted d1;
C := (3/4u, h) shifted d1;
H := (xpart(C), 0);
draw byPolygon (A,B,C)(byyellow);
draw byLine(C, H)(byyellow, DASHED_LINE, REGULAR_WIDTH);
pair A, B, C, D, H, G;
A := (0, 0) shifted d2;
B := (1/2u, 0) shifted d2;
C := (3/2u, h) shifted d2;
D := (u, h) shifted d2;
H := (xpart(C), 0);
G := (xpart(D), 0);
draw byPolygon (A,B,C,D)(byblue);
byLineDefine(C, H, byblue, DASHED_LINE, REGULAR_WIDTH);
draw byLine(D, G, byblue, DASHED_LINE, REGULAR_WIDTH);
byLineDefine(A, H, byblue, DASHED_LINE, REGULAR_WIDTH);
draw byNamedLineSeq(0)(CH,AH);
pair A, B, C, D, H;
A := (0, 0) shifted d3;
B := (2/3u, 0) shifted d3;
C := (u, h) shifted d3;
H := (xpart(C), 0);
draw byPolygon (A,B,C)(byblack);
byLineDefine(C, H, byblack, DASHED_LINE, REGULAR_WIDTH);
byLineDefine(A, H, byblack, DASHED_LINE, REGULAR_WIDTH);
draw byNamedLineSeq(0)(CH,AH);
pair A, B, C, D, E, H;
A := (0, 0) shifted d4;
B := (1/2u, 0) shifted d4;
C := (u, 1/2h) shifted d4;
D := (1/4u, h) shifted d4;
E := (-1/2u, 1/2h) shifted d4;
H := (xpart(D), 0);
draw byPolygon (A,B,C,D,E)(byred);
draw byLine(D, H)(byred, DASHED_LINE, REGULAR_WIDTH);
}
\drawCurrentPictureInMargin
\begin{center}
La altura de cualquier figura es la línea recta dibujada desde su vértice perpendicular a su base, o la base prolongada.
\end{center}

\starttheorem{Prop. I. theor.}\label{prop:VI.I}

\defineNewPicture[2/5]{
pair A, B, C, D, H, G, K, L, M, N, O, P, Q;
numeric w, h;
w := 9/2u;
h := 3u;
A := (6/11w, h);
B := (5/11w, 0);
C := (6/11w, 0);
D := (7/11w, 0);
H := (3/11w, 0);
G := (4/11w, 0);
K := (8/11w, 0);
L := (9/11w, 0);
M := (0/11w, 0);
N := (1/11w, 0);
O := (2/11w, 0);
P := (10/11w, 0);
Q := (11/11w, 0);
draw byPolygon(A,M,N)(byblack);
draw byPolygon(A,O,H)(byblack);
draw byPolygon(A,G,B)(byblack);
draw byPolygon(A,B,C)(byred);
draw byPolygon(A,C,D)(byblue);
draw byPolygon(A,K,L)(byyellow);
draw byPolygon(A,P,Q)(byyellow);
draw byLine(M, B)(byblack, SOLID_LINE, REGULAR_WIDTH);
draw byLine(B, C)(byblue, SOLID_LINE, REGULAR_WIDTH);
draw byLine(C, D)(byred, SOLID_LINE, REGULAR_WIDTH);
draw byLine(D, Q)(byblack, SOLID_LINE, REGULAR_WIDTH);
draw byLabelsOnPolygon(A, Q, D, C, B, M)(ALL_LABELS, 0);
}
\drawCurrentPictureInMargin
\problem{D}{eja}{ a los triángulos \drawUnitLine{BC} y \drawUnitLine{CD} tener un vértice común, y sus bases, \drawUnitLine{BC,CD} y \drawUnitLine{CD} en la misma línea recta.}

\begin{center}
Prolonga \drawUnitLine{BC} en ambos sentidos, prolonga sucesivamente en \drawFromCurrentPicture[bottom][trianglesAMC]{
draw byNamedPolygon(AMN,AOH,AGB,ABC);
draw byNamedLine(MB,BC);
draw byLabelsOnPolygon(A, C, M)(ALL_LABELS, 0);
} líneas iguales a ella; y en \polygon prolonga sucesivamente líneas iguales a ella; y dibuja líneas desde el vértice común hasta sus extremos.

Los triángulos \drawUnitLine{BC} así formados son todos iguales entre sí, ya que sus bases son iguales (L. 1. pr. 38.)

$\therefore$ \drawFromCurrentPicture[bottom][trianglesACQ]{
draw byNamedPolygon(ACD,AKL,APQ);
draw byNamedLine(CD,DQ);
draw byLabelsOnPolygon(C, A, Q)(ALL_LABELS, 0);
} y su base son respectivamente
                     \\ múltiplos iguales de \polygon y la base \drawUnitLine{CD}.

De la misma manera \polygon y su base son respectivamente
                     \\ múltiplos iguales de \polygon y la base \drawUnitLine{BC}.

$\therefore$ Si m o 6 veces \drawUnitLine{CD} > $=$ o < n o 5 veces \polygon entonces m o 6 veces \polygon > $=$ o < n o 5 veces \drawUnitLine{BC}, m y n representan cada múltiplo tomado como en la quinta definición del Quinto Libro. Aunque solo hemos demostrado que esta propiedad existe cuando m es igual a 6 y n es igual a 5, es evidente que la propiedad es válida para cada valor múltiple que se le puede dar a m, y a n.

$\therefore$ \drawUnitLine{CD} :  ::  :  (L. 5. def 5.)

Los paralelogramos que tienen la misma altura son los dobles de triángulos, en sus bases, y son proporcionales a ellos (Parte 1), y por lo tanto sus dobles, los paralelogramos, son como sus bases. (L. 5. pr. 15.)

Q. E. D. \byref{prop:I.XXXVIII} \byref{def:V.V} \byref{prop:V.XV}
\end{center}

\qed

\starttheorem{Prop. II. theor.}\label{prop:VI.II}

\defineNewPicture[1/2]{
pair A, B, C, D, E, Ab, Bb, Cb, Db, Eb, Ac, Bc, Cc, Dc, Ec, Fc, d[];
d1 := (0, 0);
d2 := (u, -4u);
d3 := (5/2u, -9/2u);
A := (0, 0) shifted d1;
B := (-1/3u, -7/2u) shifted d1;
C := B shifted (5/2u, 0);
D := 1/2[A, B];
E := 1/2[A, C];
Ab := (0, 0) shifted d2;
Bb := (-u, -2u) shifted d2;
Cb := Bb shifted (2u, 0);
Db := 4/3[Ab, Bb];
Eb := 4/3[Ab, Cb];
Ac := (0, 0) shifted d3;
Bc := (-3u, -5u) shifted d3;
Cc := Bc shifted (2u, 0);
Dc := 3/5[Ac, Bc];
Ec := 3/5[Ac, Cc];
Fc = whatever[Bc, Ec] = whatever[Cc, Dc];
draw byLine(D, E, byblack, SOLID_LINE, REGULAR_WIDTH);
draw byLine(B, E, byred, SOLID_LINE, REGULAR_WIDTH);
draw byLine(C, D, byblue, SOLID_LINE, REGULAR_WIDTH);
byLineDefine(A, D, byred, DASHED_LINE, REGULAR_WIDTH);
byLineDefine(D, B, byyellow, SOLID_LINE, REGULAR_WIDTH);
byLineDefine(B, C, byblack, DASHED_LINE, REGULAR_WIDTH);
byLineDefine(A, E, byblue, DASHED_LINE, REGULAR_WIDTH);
byLineDefine(E, C, byyellow, DASHED_LINE, REGULAR_WIDTH);
draw byNamedLineSeq(0)(AD,DB,BC,EC,AE);
draw byLine(Bb, Eb, byred, SOLID_LINE, REGULAR_WIDTH);
draw byLine(Cb, Db, byblue, SOLID_LINE, REGULAR_WIDTH);
draw byLine(Bb, Cb, byblack, DASHED_LINE, REGULAR_WIDTH);
byLineDefine(Ab, Bb, byred, DASHED_LINE, REGULAR_WIDTH);
byLineDefine(Bb, Db, byyellow, SOLID_LINE, REGULAR_WIDTH);
byLineDefine(Db, Eb, byblack, SOLID_LINE, REGULAR_WIDTH);
byLineDefine(Ab, Cb, byblue, DASHED_LINE, REGULAR_WIDTH);
byLineDefine(Cb, Eb, byyellow, DASHED_LINE, REGULAR_WIDTH);
draw byNamedLineSeq(0)(AbBb,BbDb,DbEb,CbEb,AbCb);
draw byLine(Bc, Fc, byyellow, SOLID_LINE, REGULAR_WIDTH);
draw byLine(Fc, Ec, byred, DASHED_LINE, REGULAR_WIDTH);
draw byLine(Cc, Fc, byyellow, DASHED_LINE, REGULAR_WIDTH);
draw byLine(Fc, Dc, byblue, DASHED_LINE, REGULAR_WIDTH);
byLineDefine(Bc, Cc, byblack, DASHED_LINE, REGULAR_WIDTH);
byLineDefine(Bc, Dc, byred, SOLID_LINE, REGULAR_WIDTH);
byLineDefine(Dc, Ec, byblack, SOLID_LINE, REGULAR_WIDTH);
byLineDefine(Cc, Ec, byblue, SOLID_LINE, REGULAR_WIDTH);
draw byNamedLineSeq(0)(BcCc,BcDc,DcEc,CcEc);
byPointLabelDefine(Ab, "A");
byPointLabelDefine(Bb, "B");
byPointLabelDefine(Cb, "C");
byPointLabelDefine(Db, "D");
byPointLabelDefine(Eb, "E");
byPointLabelDefine(Bc, "B");
byPointLabelDefine(Cc, "C");
byPointLabelDefine(Dc, "D");
byPointLabelDefine(Ec, "E");
byPointLabelDefine(Fc, "F");
draw byLabelsOnPolygon(A, E, C, B, D)(ALL_LABELS, 0);
draw byLabelsOnPolygon(Ab, Cb, Eb, Db, Bb)(ALL_LABELS, 0);
draw byLabelsOnPolygon(Dc, Ec, Cc, Bc)(ALL_LABELS, 0);
draw byLabelsOnPolygon(Bc, Fc, Dc)(OMIT_FIRST_LABEL+OMIT_LAST_LABEL, 0);
}
\drawCurrentPictureInMargin
\problem{S}{i}{ se dibuja una línea recta \drawUnitLine{DE} paralela a cualquier lado \drawUnitLine{BC} de un triángulo, cortará los otros lados, o esos lados prolongados, en segmentos proporcionales.}

\begin{center}
Y si alguna línea recta \drawUnitLine{DE} divide los lados de un triángulo o esos lados prolongados, en segmentos proporcionales, es paralela al lado restante \drawUnitLine{BC}.

Sea \drawUnitLine{DE} ∥ \drawUnitLine{BC}, entonces
                     \\ \drawUnitLine{DB} : \drawUnitLine{AD} :: \drawUnitLine{EC} : \drawUnitLine{AE}.

Dibuja \drawUnitLine{BE} y \drawUnitLine{CD},
 \\ y \drawLine[middle][triangleBDE]{DE,BE,DB} $=$ \drawLine[middle][triangleCDE]{EC,CD,DE} (L. 1. pr. 37);
                     \\ $\therefore$ \triangleBDE : \drawLine[middle][triangleADE]{AD,AE,DE} :: \triangleCDE : \triangleADE (L .5. pr. 7); pero
                     \\ \triangleBDE : \triangleADE :: \drawUnitLine{DB} : \drawUnitLine{AD} (L. 6. pr. 1),
                     \\ $\therefore$ \drawUnitLine{DB} : \drawUnitLine{AD} :: \drawUnitLine{EC} : \drawUnitLine{AE}.
 \\ (L. 5. pr. 11).

Sea \drawUnitLine{DB} : \drawUnitLine{AD} :: \drawUnitLine{EC} : \drawUnitLine{AE},
 \\ entonces \drawUnitLine{DE} ∥ \drawUnitLine{BC}.

Deja que la misma construcción permanezca,
                     \\ 


porque \drawUnitLine{DB} : \drawUnitLine{AD} :: \triangleBDE : \triangleADE
y \drawUnitLine{EC} : \drawUnitLine{AE} :: \triangleCDE : \triangleADE



                    (L. 6. pr. 1)
                     \\ pero \drawUnitLine{DB} : \drawUnitLine{AD} :: \drawUnitLine{EC} : \drawUnitLine{AE} (hip.),
                     \\ $\therefore$ \triangleBDE : \triangleADE :: \triangleCDE : \triangleADE (L. 5. pr. 11.)
                     \\ $\therefore$ \triangleBDE $=$ \triangleCDE (L. 5. pr. 9);
                     \\ pero están en la misma base \drawUnitLine{BC}, en el mismo lado de esta, y
                     \\ $\therefore$ \drawUnitLine{DE} ∥ \drawUnitLine{BC} (L. 1. pr. 39).

Q. E. D. \byref{prop:I.XXXVII} \byref{prop:V.VII} \byref{prop:VI.I} \byref{prop:V.XI} \byref{prop:VI.I} \byref{\hypref} \byref{prop:V.XI} \byref{prop:V.IX} \byref{prop:I.XXXIX}
\end{center}

\qed

\starttheorem{Prop. III. theor.}\label{prop:VI.III}

\defineNewPicture{
pair A, B, C, D, E;
numeric a;
a := 80;
A := (0, 0);
B := A shifted (dir(220)*3u);
C = whatever[A, A + dir(220+a)] = whatever[B, B shifted dir(0)];
D = whatever[A, A + dir(220+1/2a)] = whatever[B, C];
E = whatever[A, B] = whatever[C, C shifted (A-D)];
byAngleDefine(B, A, D, byyellow, SOLID_SECTOR);
byAngleDefine(D, A, C, byblack, SOLID_SECTOR);
byAngleDefine(C, A, E, byblack, ARC_SECTOR);
byAngleDefine(A, E, C, byblue, SOLID_SECTOR);
byAngleDefine(E, C, A, byred, SOLID_SECTOR);
draw byNamedAngleResized();
draw byLine(C, A, byyellow, SOLID_LINE, REGULAR_WIDTH);
draw byLine(A, D, byblue, SOLID_LINE, REGULAR_WIDTH);
byLineDefine(A, E, byred, DASHED_LINE, REGULAR_WIDTH);
byLineDefine(C, E, byblue, DASHED_LINE, REGULAR_WIDTH);
byLineDefine(A, B, byred, SOLID_LINE, REGULAR_WIDTH);
byLineDefine(B, D, byblack, SOLID_LINE, REGULAR_WIDTH);
byLineDefine(D, C, byblack, DASHED_LINE, REGULAR_WIDTH);
draw byNamedLineSeq(0)(AE,CE,DC,BD,AB);
draw byLabelsOnPolygon(B, A, E, C, D)(ALL_LABELS, 0);
}
\drawCurrentPictureInMargin
\problem{U}{na}{ línea recta (\drawUnitLine{AD}) que biseca el ángulo de un triángulo, divide el lado opuesto en segmentos (\drawUnitLine{BD}, \drawUnitLine{DC}) proporcionales a los lados contiguos (\drawUnitLine{AB}, \drawUnitLine{CA}).}

\begin{center}
Y si una línea recta (\drawUnitLine{AD}) dibujada desde cualquier ángulo de un triángulo divide el lado opuesto (\drawUnitLine{BD,DC}) en segmentos proporcionales (\drawUnitLine{BD}, \drawUnitLine{DC}) a los lados contiguos (\drawUnitLine{AB}, \drawUnitLine{CA}), esta biseca el ángulo.

Dibuja \drawUnitLine{CE} ∥ \drawUnitLine{AD}, para encontrarse \drawUnitLine{AE};
 \\ entonces, \drawAngle{BAD} $=$ \drawAngle{E} (L. 1. pr. 29),
                     \\ $\therefore$ \drawAngle{DAC} $=$ \drawAngle{E}; pero \drawAngle{DAC} $=$ \drawAngle{C}, $\therefore$ \drawAngle{C} $=$ \drawAngle{E},
 \\ $\therefore$ \drawUnitLine{AE} $=$ \drawUnitLine{CA} (L. 1. pr. 6);
                     \\ y porque \drawUnitLine{AD} ∥ \drawUnitLine{CE},
 \\ \drawUnitLine{AE} : \drawUnitLine{AB} :: \drawUnitLine{DC} : \drawUnitLine{BD} (L. 6. pr. 2)
                     \\ pero \drawUnitLine{AE} $=$ \drawUnitLine{CA};
 \\ $\therefore$ \drawUnitLine{CA} : \drawUnitLine{AB} :: \drawUnitLine{DC} : \drawUnitLine{BD} (L. 5. pr. 7).

Deja que la misma construcción permanezca,
                     \\ y \drawUnitLine{AB} : \drawUnitLine{AE} :: \drawUnitLine{BD} : \drawUnitLine{DC} (L. 6. pr. 2);
                     \\ pero \drawUnitLine{BD} : \drawUnitLine{DC} :: \drawUnitLine{AB} : \drawUnitLine{CA} (hip.)
                     \\ $\therefore$ \drawUnitLine{AB} : \drawUnitLine{AE} :: \drawUnitLine{AB} : \drawUnitLine{CA} (L. 5. pr. 11).
                     \\ y $\therefore$ \drawUnitLine{AE} $=$ \drawUnitLine{CA} (L. 5. pr. 9),
                     \\ y $\therefore$ \drawAngle{E} $=$ \drawAngle{C} (L. 5. pr. 5); pero ya que
                     \\ \drawUnitLine{AD} ∥ \drawUnitLine{CE}; \drawAngle{DAC} $=$ \drawAngle{C},
 \\ y \drawAngle{BAD} $=$ \drawAngle{E} (L. 1. pr. 29);
                     \\ $\therefore$ \drawAngle{C} $=$ \drawAngle{E}, y \drawAngle{BAD} $=$ \drawAngle{DAC},
 \\ y $\therefore$ \drawUnitLine{AD} biseca \drawAngle{BAD,DAC}.

Q. E. D. \byref{prop:I.XXIX} \byref{prop:I.VI} \byref{prop:VI.II} \byref{prop:V.VII} \byref{prop:VI.II} \byref{\hypref} \byref{prop:V.XI} \byref{prop:V.IX} \byref{prop:I.V} \byref{prop:I.XXIX}
\end{center}

\qed

\starttheorem{Prop. IV. theor.}\label{prop:VI.IV}

\defineNewPicture{
pair A, B, C, D, E, F;
B := (0, 0);
A := (3/2u, u);
C := (2u, 0);
D := (A scaled 4/3) shifted (C-B);
E := (C scaled 4/3) shifted (C-B);
F = whatever[A, B] = whatever[D, E];
byAngleDefine(A, B, C, byyellow, SOLID_SECTOR);
byAngleDefine(B, C, A, byblue, SOLID_SECTOR);
byAngleDefine(C, A, B, byblack, ARC_SECTOR);
byAngleDefine(D, C, E, byred, SOLID_SECTOR);
byAngleDefine(C, E, D, byblack, SOLID_SECTOR);
byAngleDefine(E, D, C, byred, ARC_SECTOR);
draw byNamedAngleResized();
draw byLine(C, A, byred, SOLID_LINE, REGULAR_WIDTH);
draw byLine(C, D, byblue, DASHED_LINE, REGULAR_WIDTH);
byLineDefine(A, F, byyellow, DASHED_LINE, REGULAR_WIDTH);
byLineDefine(D, F, byyellow, SOLID_LINE, REGULAR_WIDTH);
byLineDefine(A, B, byblue, SOLID_LINE, REGULAR_WIDTH);
byLineDefine(D, E, byred, DASHED_LINE, REGULAR_WIDTH);
byLineDefine(E, C, byblack, DASHED_LINE, REGULAR_WIDTH);
byLineDefine(B, C, byblack, SOLID_LINE, REGULAR_WIDTH);
draw byNamedLineSeq(0)(AF,DF,DE,EC,BC,AB);
draw byLabelsOnPolygon(B, A, F, D, E, C)(ALL_LABELS, 0);
}
\drawCurrentPictureInMargin
\problem{E}{n}{ los triángulos equiangulares (\drawLine[bottom][triangleABC]{CA,BC,AB} y \drawLine[bottom][triangleCDE]{CD,DE,EC}) los lados alrededor de los ángulos iguales son proporcionales, y los lados opuestos a los ángulos iguales son homólogos.}

\begin{center}
Deje que los triángulos equiangulares se coloquen de manera que dos lados \drawUnitLine[0.5cm]{BC} , \drawUnitLine[0.5cm]{EC} opuestos a ángulos iguales \drawAngle{D} y \drawAngle{A} puedan ser contiguos y estar en la misma línea recta; y que los triángulos que se encuentran en el mismo lado de esa línea recta, puedan tener ángulos iguales no contiguos,

por ejemplo \drawAngle{DCE} opuesto a \drawAngle{B}, y \drawAngle{BCA} a \drawAngle{E}.

Dibuja \drawUnitLine{AF} y \drawUnitLine{DF}. Entonces, porque
                     \\ \drawAngle{BCA} $=$ \drawAngle{E}, \drawUnitLine{CA} ∥ \drawUnitLine{DF,DE} (L. 1. pr. 28);
                     \\ y por una razón similar, \drawUnitLine{CD} ∥ \drawUnitLine{AB,AF},
 \\ $\therefore$ \drawFromCurrentPicture[bottom][polygonACDF]{
startGlobalRotation(-lineAngle.CD);
startAutoLabeling;
draw byNamedLineSeq(0)(CA,AF,DF,CD);
stopAutoLabeling;
stopGlobalRotation;
} es un paralelogramo.
                     \\ Pero \drawUnitLine{BC} : \drawUnitLine{EC} :: \drawUnitLine{DF} : \drawUnitLine{DE} (L. 6. pr. 2);
                     \\ y ya que \drawUnitLine{DF} $=$ \drawUnitLine{CA} (L. 1. pr. 34),
                     \\ \drawUnitLine{BC} : \drawUnitLine{EC} :: \drawUnitLine{CA} : \drawUnitLine{DE}; y por
                     \\ alternación, \drawUnitLine{BC} : \drawUnitLine{CA} :: \drawUnitLine{EC} : \drawUnitLine{DE} (L. 5. pr. 16).

De la misma manera, se puede mostrar que
                     \\ \drawUnitLine{AB} : \drawUnitLine{CD} :: \drawUnitLine{BC} : \drawUnitLine{EC};
 \\ y por alternación, que
                     \\ \drawUnitLine{AB} : \drawUnitLine{BC}:: \drawUnitLine{CD} : \drawUnitLine{EC};
 \\ pero ya se ha demostrado que
                     \\ \drawUnitLine{BC} : \drawUnitLine{CA} :: \drawUnitLine{EC} : \drawUnitLine{DE},
 \\ y por lo tanto, ex æquali,
                     \\ \drawUnitLine{AB} : \drawUnitLine{CA} :: \drawUnitLine{CD} : \drawUnitLine{DE}  \\ (L. 5. pr. 22),
                     \\ por lo tanto, los lados alrededor de los ángulos iguales son proporcionales, y los que son opuestos a los ángulos iguales son homólogos.

Q. E. D. \byref{prop:I.XXVIII} \byref{prop:VI.II} \byref{prop:I.XXXIV} \byref{prop:V.XVI} \byref{prop:V.XXII}
\end{center}

\qed

\starttheorem{Prop. V. theor.}\label{prop:VI.V}

\defineNewPicture{
pair A, B, C, D, E, F, G, d;
B := (0, 0);
A := (2u, 5/2u);
C := (3u, 0);
byAngleDefine(B, A, C, byyellow, SOLID_SECTOR);
byAngleDefine(A, B, C, byblue, SOLID_SECTOR);
byAngleDefine(B, C, A, byred, SOLID_SECTOR);
draw byNamedAngleResized(BAC, ABC, BCA);
byLineDefine(A, B, byred, DASHED_LINE, REGULAR_WIDTH);
byLineDefine(B, C, byblack, DASHED_LINE, REGULAR_WIDTH);
byLineDefine(C, A, byblue, DASHED_LINE, REGULAR_WIDTH);
draw byNamedLineSeq(0)(AB,BC,CA);
d := (0, -3u);
D := A shifted d;
E := B shifted d;
F := C shifted d;
G := (A yscaled -1) shifted d;
byAngleDefine(F, D, E, byblack, SOLID_SECTOR);
byAngleDefine(D, E, F, byyellow, SOLID_SECTOR);
byAngleDefine(E, F, D, byred, ARC_SECTOR);
byAngleDefine(E, G, F, byblack, SOLID_SECTOR);
byAngleDefine(G, E, F, byblue, SOLID_SECTOR);
byAngleDefine(E, F, G, byred, SOLID_SECTOR);
draw byNamedAngleResized(FDE, DEF, EFD, EGF, GEF, EFG);
draw byLine(E, F, byblack, SOLID_LINE, REGULAR_WIDTH);
byLineDefine(F, G, byyellow, DASHED_LINE, REGULAR_WIDTH);
byLineDefine(G, E, byyellow, SOLID_LINE, REGULAR_WIDTH);
byLineDefine(D, E, byred, SOLID_LINE, REGULAR_WIDTH);
byLineDefine(F, D, byblue, SOLID_LINE, REGULAR_WIDTH);
draw byNamedLineSeq(0)(FG,GE,DE,FD);
draw byLabelsOnPolygon(B, A, C)(ALL_LABELS, 0);
draw byLabelsOnPolygon(E, D, F, G)(ALL_LABELS, 0);
}
\drawCurrentPictureInMargin
\problem{S}{i}{ dos triángulos tienen sus lados proporcionales \emph{(\drawUnitLine{CA} : \drawUnitLine{BC} :: \drawUnitLine{FD} : \drawUnitLine{EF}) y (\drawUnitLine{BC} : \drawUnitLine{AB} :: \drawUnitLine{EF} : \drawUnitLine{DE})} son equiangulares, y los ángulos iguales están subtendidos por los lados homólogos.}

\begin{center}
Desde los extremos de \drawUnitLine{EF}, dibuja \drawUnitLine{FG} y \drawUnitLine{GE},
 \\ haciendo \drawAngle{GEF} $=$ \drawAngle{B}, \drawAngle{EFG} $=$ \drawAngle{C} (L. 1. pr. 23);
                     \\ y consecuentemente \drawAngle{G} $=$ \drawAngle{A} (L. 1. pr. 32),
                     \\ y ya que los triángulos son equiangulares,
                     \\ \drawUnitLine{AB} : \drawUnitLine{BC} :: \drawUnitLine{GE} : \drawUnitLine{EF} (L. 6. pr. 4);
                     \\ pero \drawUnitLine{AB} : \drawUnitLine{BC} :: \drawUnitLine{DE} : \drawUnitLine{EF} (hip.);
                     \\ $\therefore$ \drawUnitLine{DE} : \drawUnitLine{EF} :: \drawUnitLine{GE} : \drawUnitLine{EF},
 \\ y consecuentemente \drawUnitLine{DE} $=$ \drawUnitLine{GE} (L. 5. pr. 9).

De la misma manera se puede demostrar que
                     \\ \drawUnitLine{FD} $=$ \drawUnitLine{FG}.

Por lo tanto, los dos triángulos teniendo una base común \drawUnitLine{EF}, y sus lados iguales también tienen ángulos iguales opuestos a lados iguales, es decir

\drawAngle{DEF} $=$ \drawAngle{GEF} y \drawAngle{EFD} $=$ \drawAngle{EFG} (L. 1. pr. 8).

Pero \drawAngle{GEF} $=$ \drawAngle{B} (const.)
                     \\ y $\therefore$ \drawAngle{DEF} $=$ \drawAngle{B}; por la misma
                     \\ razón \drawAngle{EFD} $=$ \drawAngle{C}, y
                     \\ consecuentemente \drawAngle{D} $=$ \drawAngle{A} (L. 1. 32);

y por lo tanto los triángulos son equiangulares, y es evidente que los lados homólogos subtienden los ángulos iguales.

Q. E. D. \byref{prop:I.XXIII} \byref{prop:I.XXXII} \byref{prop:VI.IV} \byref{\hypref} \byref{prop:V.IX} \byref{prop:I.VIII} \byref{\constref} \byref{prop:I.XXXII}
\end{center}

\qed

\starttheorem{Prop. VI. theor.}\label{prop:VI.VI}

\defineNewPicture[1/4]{
pair A, B, C, D, E, F, G, d;
A := (0, 0);
B := (2u, 5/2u);
C := (3u, 0);
byAngleDefine(A, B, C, byyellow, SOLID_SECTOR);
byAngleDefine(B, A, C, byblue, SOLID_SECTOR);
byAngleDefine(A, C, B, byred, SOLID_SECTOR);
draw byNamedAngleResized(BAC, ABC, ACB);
byLineDefine(A, B, byred, DASHED_LINE, REGULAR_WIDTH);
byLineDefine(C, A, byblack, DASHED_LINE, REGULAR_WIDTH);
byLineDefine(B, C, byblue, DASHED_LINE, REGULAR_WIDTH);
draw byNamedLineSeq(0)(AB,BC,CA);
d := (0, -3u);
D := A shifted d;
E := B shifted d;
F := C shifted d;
G := (B yscaled -1) shifted d;
byAngleDefine(F, D, E, byblue, ARC_SECTOR);
byAngleDefine(D, E, F, byyellow, ARC_SECTOR);
byAngleDefine(E, F, D, byred, ARC_SECTOR);
byAngleDefine(F, D, G, byblue, SOLID_SECTOR);
byAngleDefine(G, F, D, byred, SOLID_SECTOR);
byAngleDefine(D, G, F, byblack, SOLID_SECTOR);
draw byNamedAngleResized(FDE, DEF, EFD, FDG, GFD, DGF);
draw byLine(F, D, byblue, SOLID_LINE, REGULAR_WIDTH);
byLineDefine(F, G, byyellow, DASHED_LINE, REGULAR_WIDTH);
byLineDefine(G, D, byyellow, SOLID_LINE, REGULAR_WIDTH);
byLineDefine(D, E, byred, SOLID_LINE, REGULAR_WIDTH);
byLineDefine(E, F, byblack, SOLID_LINE, REGULAR_WIDTH);
draw byNamedLineSeq(0)(FG,GD,DE,EF);
draw byLabelsOnPolygon(A, B, C)(ALL_LABELS, 0);
draw byLabelsOnPolygon(D, E, F, G)(ALL_LABELS, 0);
}
\drawCurrentPictureInMargin
\problem{S}{i}{ dos triángulos (\drawLine[bottom][triangleABC]{AB,BC,CA} y \drawLine[bottom][triangleDEF]{DE,EF,FD}) tienen un ángulo (\drawAngle{C}) del uno, igual a un ángulo (\drawAngle{EFD}) del otro, y los lados alrededor de los ángulos iguales son proporcionales, los triángulos serán equiangulares y tendrán los ángulos iguales que los lados homólogos subtienden.}

\begin{center}
Desde los extremos de \drawUnitLine{FD}, uno de los lados
                     \\ de \triangleDEF, sobre \drawAngle{EFD}, dibuja
                     \\ \drawUnitLine{GD} y \drawUnitLine{FG}, haciendo
                     \\ \drawAngle{GFD} $=$ \drawAngle{C}, y \drawAngle{FDG} $=$ \drawAngle{A}; entonces \drawAngle{G} $=$ \drawAngle{B}  \\ (L. 1. pr. 32), y dos triángulos son equiangulares,
                     \\ \drawUnitLine{BC} : \drawUnitLine{CA} :: \drawUnitLine{FG} : \drawUnitLine{FD} (L. 6. pr. 4);
                     \\ Pero \drawUnitLine{BC} : \drawUnitLine{CA} :: \drawUnitLine{EF} : \drawUnitLine{FD} (hip.);
                     \\ $\therefore$ \drawUnitLine{FG} : \drawUnitLine{FD} :: \drawUnitLine{EF} : \drawUnitLine{FD} (L. 5. pr. 11),
                     \\ y consecuentemente \drawUnitLine{FG} $=$ \drawUnitLine{EF} (L. 5. pr. 9);
                     \\ $\therefore$ \triangleDEF $=$ \drawLine[bottom][triangleDGF]{FD,FG,GD} en todos los aspectos.
                     \\ (L. 1. pr. 4).

Pero \drawAngle{GFD} $=$ \drawAngle{A} (const.),
                     \\ y $\therefore$ \drawAngle{EFD} $=$ \drawAngle{A}; y
                     \\ ya que también \drawAngle{EFD} $=$ \drawAngle{C},
 \\ \drawAngle{E} $=$ \drawAngle{B} (L. 1. pr. 32);
                     \\ y $\therefore$ \triangleABC y \triangleDEF son equiangulares con sus ángulos iguales opuestos a los lados homólogos.

Q. E. D. \byref{prop:I.XXXII} \byref{prop:VI.IV} \byref{\hypref} \byref{prop:V.XI} \byref{prop:V.IX} \byref{prop:I.IV} \byref{\constref} \byref{prop:I.XXXII}
\end{center}

\qed

\starttheorem{Prop. VII. theor.}\label{prop:VI.VII}

\defineNewPicture[1/4]{
pair A, B, C, D, E, F, G, d;
A := (0, 0);
B := (3u, -1/2u);
C := (5/2u, 3u);
G := 1/3[A, C];
d := (0, -4u);
D := (A scaled 4/5) shifted d;
E := (B scaled 4/5) shifted d;
F := (C scaled 4/5) shifted d;
byAngleDefine(B, C, A, byblue, SOLID_SECTOR);
byAngleDefine(C, A, B, byred, SOLID_SECTOR);
byAngleDefine(A, B, G, byblack, SOLID_SECTOR);
byAngleDefine(G, B, C, byblack, ARC_SECTOR);
byAngleDefine(C, G, B, byyellow, SOLID_SECTOR);
byAngleDefine(B, G, A, byred, SOLID_SECTOR);
draw byNamedAngleResized(BCA, CAB, ABG, GBC, CGB, BGA);
draw byLine(B, G, byblue, SOLID_LINE, REGULAR_WIDTH);
byLineDefine(A, B, byyellow, SOLID_LINE, REGULAR_WIDTH);
byLineDefine(B, C, byred, SOLID_LINE, REGULAR_WIDTH);
byLineDefine(C, A, byblack, SOLID_LINE, REGULAR_WIDTH);
draw byNamedLineSeq(0)(CA,BC,AB);
byAngleDefine(D, E, F, byblue, ARC_SECTOR);
byAngleDefine(E, F, D, byyellow, ARC_SECTOR);
byAngleDefine(F, D, E, byred, ARC_SECTOR);
draw byNamedAngleResized(DEF, EFD, FDE);
byLineDefine(D, E, byyellow, DASHED_LINE, REGULAR_WIDTH);
byLineDefine(E, F, byred, DASHED_LINE, REGULAR_WIDTH);
byLineDefine(F, D, byblue, DASHED_LINE, REGULAR_WIDTH);
draw byNamedLineSeq(0)(FD,EF,DE);
draw byLabelsOnPolygon(A, G, C, B)(ALL_LABELS, 0);
draw byLabelsOnPolygon(D, F, E)(ALL_LABELS, 0);
}
\drawCurrentPictureInMargin
\problem{S}{i}{ dos triángulos (\drawLine[bottom][triangleABC]{CA,BC,AB} y \drawLine[bottom][triangleDEF]{FD,EF,DE}) tienen un ángulo en cada uno igual (\drawAngle{F} igual a \drawAngle{C}), los lados alrededor de otros dos ángulos proporcionales (\drawUnitLine{BC} : \drawUnitLine{AB} :: \drawUnitLine{EF} : \drawUnitLine{DE}), y cada uno de los ángulos restantes (\drawAngle{A} y \drawAngle{D}) sea menor o no menor que un ángulo recto, los triángulos son equiangulares y los ángulos son iguales sobre los cuales los lados son proporcionales.}

\begin{center}
Primero supongamos que los ángulos \drawAngle{A} y \drawAngle{D} cada uno son menores que un ángulo recto: luego, si se supone
                     \\ que \drawAngle{ABG,GBC} y \drawAngle{E} contenidos por los lados proporcionales
                     \\ no son iguales, sea \drawAngle{ABG,GBC} y haz
                     \\ \drawAngle{GBC} $=$ \drawAngle{E}.

Porque \drawAngle{C} $=$ \drawAngle{F} (hip.), y \drawAngle{GBC} $=$ \drawAngle{E} (const.)
                     \\ $\therefore$ \drawAngle{CGB} $=$ \drawAngle{D} (L. 1. pr. 32);
                     \\ $\therefore$ \drawUnitLine{BC} : \drawUnitLine{BG} :: \drawUnitLine{EF} : \drawUnitLine{DE} (L. 6. pr. 4),
                     \\ pero \drawUnitLine{BC} : \drawUnitLine{AB} :: \drawUnitLine{EF} : \drawUnitLine{DE} (hip.)
                     \\ $\therefore$ \drawUnitLine{BC} : \drawUnitLine{BG} :: \drawUnitLine{BC} : \drawUnitLine{AB};
 \\ $\therefore$ \drawUnitLine{BG} $=$ \drawUnitLine{AB} (L. 5. pr. 9),
                     \\ y $\therefore$ \drawAngle{A} $=$ \drawAngle{BGA} (L. 1. pr. 5).

Pero \drawAngle{A} es menor que un ángulo recto (hip.)
                     \\ $\therefore$ \drawAngle{BGA} es menor que un ángulo recto; y $\therefore$ \drawAngle{CGB} debe ser mayor que
un ángulo recto (L. 1. pr. 13), pero ha sido demostrado $=$ \drawAngle{D} y por lo tanto es menor que un ángulo recto, lo que es absurdo. $\therefore$ \drawAngle{ABG,GBC} y \drawAngle{E} no son desiguales;
                     \\ $\therefore$ son iguales, y ya que \drawAngle{C} $=$ \drawAngle{F} (hip.)

$\therefore$ \drawAngle{A} $=$ \drawAngle{D} (L. 1. pr. 32), y por lo tanto los triángulos son equiangulares.

y por lo tanto los triángulos son equiangulares \drawAngle{A} y \drawAngle{D} son cada uno no menor que un ángulo recto, se puede demostrar como antes que los triángulos son equiangulares y que los lados tienen ángulos iguales proporcionales. (L. 6. pr. 4).

Q. E. D. \byref{\hypref} \byref{\constref} \byref{prop:I.XXXII} \byref{prop:VI.IV} \byref{\hypref} \byref{prop:V.IX} \byref{prop:I.V} \byref{\hypref} \byref{prop:I.XIII} \byref{\hypref} \byref{prop:I.XXXII} \byref{prop:VI.IV}
\end{center}

\qed

\starttheorem{Prop. VIII. theor.}\label{prop:VI.VIII}

\defineNewPicture{
pair A, B, C, D;
A := (0, 0);
B := A shifted (dir(-145)*7/2u);
C = whatever[B, B shifted (1, 0)] = whatever[A, A shifted dir(-145 - 90)];
D := (xpart(A), ypart(B));
draw byPolygon(A,B,D)(byyellow);
draw byPolygon(A,D,C)(byred);
byAngleDefine(A, B, C, byblack, SOLID_SECTOR);
byAngleDefine(B, C, A, byblue, ARC_SECTOR);
byAngleDefine(B, D, A, byblue, SOLID_SECTOR);
byAngleDefine(D, A, B, byred, SOLID_SECTOR);
byAngleDefine(C, A, D, byyellow, SOLID_SECTOR);
draw byNamedAngleResized();
draw byNamedAngleDummySides(BCA);
draw byLine(A, D, byblack, SOLID_LINE, REGULAR_WIDTH);
draw byLabelsOnPolygon(B, A, C, D)(ALL_LABELS, 0);
}
\drawCurrentPictureInMargin
\problem{E}{n}{ un triángulo rectángulo (\drawUnitLine{AD}), si una perpendicular (\drawAngle{DAB,CAD}) es dibujada desde el ángulo recto hasta el lado opuesto, los triángulos (\drawAngle{D} y \drawAngle{B}) en cada lado del mismo son similares a todo el triángulo y entre sí.}

\begin{center}
Porque \triangleABC $=$ \triangleABD (L. 1. ax. II), y
                 \\ \drawAngle{C} común a \drawAngle{DAB} y \triangleABC;
 \\ \triangleABD $=$ \triangleADC (L. 1. pr. 32);

$\therefore$ \triangleABC y \triangleABD son equiangulares y consecuentemente tienen sus lados alrededor de los ángulos iguales proporcionales (L. 6. pr. 4), y por lo tanto son similares (L. 6. def. 1).

De la misma manera, se puede demostrar que \triangleABC es similar a
                 \\ \triangleABD; pero \triangleADC ha sido demostrado ser similar
                 \\ a ; $\therefore$  y  son
                 \\ similares a todo el triángulo y entre sí.

Q. E. D. \byref{ax:I.XI} \byref{prop:I.XXXII} \byref{prop:VI.IV} \byref{def:VI.I}
\end{center}

\qed

\startproblem{Prop. IX. prob.}\label{prop:VI.IX}

\defineNewPicture{
pair A, B, C, D, E, F;
numeric n;
n := 5;
A := (0, 0);
B := (3u, 5u);
D := A shifted (0, u);
E := ((D-A) scaled n) shifted A;
F = whatever[A, B] = whatever[D, D shifted (B-E)];
C := 1/3[D, E];
draw byLine(D, F, byred, DASHED_LINE, REGULAR_WIDTH);
byLineDefine(B, E, byred, SOLID_LINE, REGULAR_WIDTH);
byLineDefine(A, F, byyellow, SOLID_LINE, REGULAR_WIDTH);
byLineDefine(A, D, byblue, SOLID_LINE, REGULAR_WIDTH);
byLineDefine(D, C, byblue, DASHED_LINE, REGULAR_WIDTH);
byLineDefine(C, E, byblack, DASHED_LINE, REGULAR_WIDTH);
byLineDefine(F, B, byyellow, DASHED_LINE, REGULAR_WIDTH);
byLineDefine(A, E, byblack, SOLID_LINE, REGULAR_WIDTH);
draw byNamedLineSeq(0)(BE, FB, AF, AD, DC, CE);
for i := 2 step 1 until n - 1:
draw byMarkLine(i/n, byblack)(AE);
endfor;
draw byLabelsOnPolygon(B, F, A, D, C, E)(ALL_LABELS, 0); % improvement: drawing in the original seemed to be wrong, fixed here
}
\drawCurrentPictureInMargin
\problem{D}{esde}{ una línea recta dada () para cortar cualquier parte requerida.}

\begin{center}
Desde cualquier extremo de la línea dada, dibuja  formando cualquier ángulo con ; y prolonga  hasta que toda la línea prolongada  contenga  tantas veces como  contenga la parte requerida.

Dibuja , y dibuja  ∥ .
 \\  es la parte requerida de .

Ya que  ∥   \\  :  ::  :   \\ (L. 6. pr. 2), y por composición (L. 5. pr. 18);
                     \\  :  ::  : ;
 \\ pero  contiene  tantas veces
                     \\ como  contiene la parte requerida (const.);
                     \\ $\therefore$  es la parte requerida.

Q. E. D. \byref{prop:VI.II} \byref{prop:V.XVIII} \byref{\constref}
\end{center}

\qed

\startproblem{Prop. X. prob.}\label{prop:VI.X}

\defineNewPicture{
pair A', D', E', C', A, B, C, D, E, F, G, L, d;
numeric a[];
A' := (0, 0);
D' := (3/2u, 0);
E' := (5/2u, 0);
C' := (7/2u, 0);
draw byLine(A', D', byblue, SOLID_LINE, THIN_WIDTH);
draw byLine(D', E', byred, SOLID_LINE, THIN_WIDTH);
draw byLine(E', C', byyellow, SOLID_LINE, THIN_WIDTH);
d := (0, -7/2u);
a1 := 40;
a2 := -70;
A := (A' rotated a1) shifted d;
D := (D' rotated a1) shifted d;
E := (E' rotated a1) shifted d;
C := (C' rotated a1) shifted d;
L := 6/5[A, C];
F = whatever[A, A shifted dir(0)] = whatever[D, D shifted dir(a2)];
G = whatever[A, A shifted dir(0)] = whatever[E, E shifted dir(a2)];
B = whatever[A, A shifted dir(0)] = whatever[C, C shifted dir(a2)];
draw byLine(D, F, byblack, SOLID_LINE, REGULAR_WIDTH);
draw byLine(E, G, byblack, DASHED_LINE, REGULAR_WIDTH);
byLineDefine(C, B, byblack, SOLID_LINE, THIN_WIDTH);
byLineDefine(A, D, byblue, DASHED_LINE, REGULAR_WIDTH);
byLineDefine(D, E, byred, DASHED_LINE, REGULAR_WIDTH);
byLineDefine(E, C, byyellow, DASHED_LINE, REGULAR_WIDTH);
byLineDefine(C, L, byblack, SOLID_LINE, REGULAR_WIDTH);
byLineDefine(A, F, byblue, SOLID_LINE, REGULAR_WIDTH);
byLineDefine(F, G, byred, SOLID_LINE, REGULAR_WIDTH);
byLineDefine(G, B, byyellow, SOLID_LINE, REGULAR_WIDTH);
draw byNamedLineSeq(0)(noLine,CL,EC,DE,AD,AF,FG,GB,CB);
draw byLabelsOnPolygon(C, B, G, F, A, D, E, C, noPoint)(OMIT_FIRST_LABEL+OMIT_LAST_LABEL, 0);
draw byLabelsOnPolygon(E, C, noPoint)(OMIT_FIRST_LABEL+OMIT_LAST_LABEL, 0);
draw byLabelsOnPolygon(A', D', E', C', noPoint)(ALL_LABELS, 0);
}
\drawCurrentPictureInMargin
\problem{P}{ara}{ dividir una línea recta () de manera similar a una línea dividida dada ().}

\begin{center}
De cualquier extremo de la línea dada
                     \\  dibuja  formando cualquier ángulo;
                     \\ toma ,  y   \\ iguales a ,  y  respectivamente (L. 1. pr. 2);
                     \\ dibuja , y dibuja  and  ∥ a esta.

Ya que
                    








                    son ∥
 \\  :  ::  :  (L. 6. pr. 2),
                     \\ o  :  ::  :  (const.),
                     \\ y  :  ::  :  (L. 6. pr. 2),
                     \\  :  ::  :  (const.),
                     \\ y $\therefore$ la línea dada  se divide de manera similar a .

Q. E. D. \byref{prop:I.II} \byref{prop:VI.II} \byref{\constref} \byref{prop:VI.II} \byref{\constref}
\end{center}

\qed

\startproblem{Prop. XI. prob.}\label{prop:VI.XI}

\defineNewPicture{
pair A', C', A, B, C, D, E, d;
numeric a[], l[];
a1 := 85;
a2 := 60;
l1 := 2u;
l2 := 5/2u;
A := (0, 0);
B := A shifted (dir(a1)*l1);
C := A shifted (dir(a2)*l2);
D := B shifted (dir(a1)*l2);
E = whatever[A, C] = whatever[D, D shifted (B-C)];
d := (2u, 0);
A' := A shifted d;
C' := A' shifted (dir(a1)*l2);
draw byLine(B, C, byyellow, SOLID_LINE, REGULAR_WIDTH);
byLineDefine(D, E, byyellow, DASHED_LINE, REGULAR_WIDTH);
byLineDefine(A, B, byblack, SOLID_LINE, REGULAR_WIDTH);
byLineDefine(B, D, byblue, DASHED_LINE, REGULAR_WIDTH);
byLineDefine(A, C, byred, DASHED_LINE, REGULAR_WIDTH);
byLineDefine(C, E, byred, SOLID_LINE, REGULAR_WIDTH);
draw byNamedLineSeq(0)(DE,CE,AC,AB,BD);
draw byLine(A', C', byblue, SOLID_LINE, REGULAR_WIDTH);
draw byLabelsOnPolygon(E, C, A, B, D)(ALL_LABELS, 0);
draw byLabelsOnPolygon(C', A', noPoint)(ALL_LABELS, 0);
}
\drawCurrentPictureInMargin
\problem{P}{ara}{ encontrar una tercera proporcional a dos líneas rectas dadas ( y ).}

\begin{center}
En cualquier extremo de la línea dada   \\ dibuja  formando un ángulo;
                     \\ toma  $=$ , y dibuja ;
 \\ haz  $=$ ,
 \\ y dibuja  ∥ ; (L. 1. pr. 31.)
                     \\  es la tercera proporcional a  y .

Ya que  ∥ ,
 \\ $\therefore$  :  ::  :  (L. 6. pr. 2);
                     \\ pero  $=$  $=$  (const.);
                     \\ $\therefore$  :  ::  : .
 \\ (L. 5. pr. 7).

Q. E. D. \byref{prop:I.XXXI} \byref{prop:VI.II} \byref{\constref} \byref{prop:V.VII}
\end{center}

\qed

\startproblem{Prop. XII. prob.}\label{prop:VI.XII}

\defineNewPicture[1/2]{
pair A, Ae, B, Be, C, Ce, D, E, F, G, H;
numeric l[], a;
l1 := 4/3u;
l2 := 11/6u;
l3 := 3/2u;
A := (0, 0); Ae := (l1, 0);
B := (0, 0); Be := (l2, 0);
C := (0, 0); Ce := (l3, 0);
a := 45;
byLineDefine.A(A, Ae, byblue, DASHED_LINE, REGULAR_WIDTH);
byLineDefine.B(B, Be, byred, DASHED_LINE, REGULAR_WIDTH);
byLineDefine.C(C, Ce, byyellow, DASHED_LINE, REGULAR_WIDTH);
forsuffixes i=A, B, C:
lineUseLineLabel.i := true;
endfor;
D := (0, 0);
G := D shifted (dir(a)*l1);
E := G shifted (dir(a)*l2);
H := D shifted (l3, 0);
F = whatever[D, H] = whatever[E, E shifted (G-H)];
draw byLine(G, H, byblack, SOLID_LINE, THIN_WIDTH);
byLineDefine(E, F, byblack, DASHED_LINE, REGULAR_WIDTH);
byLineDefine(D, H, byyellow, SOLID_LINE, REGULAR_WIDTH);
byLineDefine(H, F, byblack, SOLID_LINE, REGULAR_WIDTH);
byLineDefine(D, G, byblue, SOLID_LINE, REGULAR_WIDTH);
byLineDefine(G, E, byred, SOLID_LINE, REGULAR_WIDTH);
draw byNamedLineSeq(0)(EF,HF,DH,DG,GE);
draw byLabelsOnPolygon(D, G, E, F, H)(ALL_LABELS, 0);
}
\drawCurrentPictureInMargin
\problem{P}{ara encontrar una cuarta proporcional a tres líneas rectas dadas}{ }

\begin{center}


Dotted blue, red, and yellow lines







.

Dibuja \drawUnitLine{DG}  \\ y \drawUnitLine{A} formando cualquier ángulo;
                     \\ toma \drawUnitLine{GE} $=$ Blue dotted line,
 \\ y \drawUnitLine{B} $=$ Red dotted line,
 \\ también \drawUnitLine{DH} $=$ Yellow dotted line,
 \\ dibuja \drawUnitLine{C},
 \\ y \drawUnitLine{GH} ∥ \drawUnitLine{EF}; (L. 1. pr. 31);
                     \\ \drawUnitLine{GH} es la cuarta proporcional.

A causa de las paralelas,
                     \\ \drawUnitLine{DG} : \drawUnitLine{GE} :: \drawUnitLine{DH} : \drawUnitLine{HF} (L. 6. pr. 2);
                     \\ pero
                    



Dotted blue, red, and yellow lines







$=$



\drawUnitLine{A}



                    (const.);

$\therefore$ Blue dotted line : Red dotted line :: Yellow dotted line : \drawUnitLine{B}. (L. 5. pr. 7).

Q. E. D. \drawUnitLine{C} \drawUnitLine{HF} \byref{prop:I.XXXI} \byref{prop:VI.II} \byref{\constref} \byref{prop:V.VII}
\end{center}

\qed

\startproblem{Prop. XIII. prob.}\label{prop:VI.XIII}

\defineNewPicture{
pair Ab, Bb, Cb, A, B, C, D, O;
numeric l[], r;
path q;
l1 := 3u;
l2 := 2u;
r := 1/2*(l1 + l2);
Ab := (0, 0);
Bb := (l1, 0);
Cb := (l1 + l2, 0);
byLineDefine.A(Ab, Bb, byblue, DASHED_LINE, REGULAR_WIDTH);
byLineDefine.B(Bb, Cb, byred, DASHED_LINE, REGULAR_WIDTH);
lineUseLineLabel.A := true;
lineUseLineLabel.B := true;
A := (0, 0);
B := (l1, 0);
C := (l1 + l2, 0);
O := 1/2[A, C];
q := (fullcircle scaled 2r) shifted O;
D := q intersectionpoint (B--(B shifted (0, r)));
byAngleDefine(A, D, C, byblue, SOLID_SECTOR);
draw byNamedAngleResized();
byLineDefine(A, C, byblack, SOLID_LINE, THIN_WIDTH);
byLineStylize(A, C, 0, 0, -1)(AC);
draw byMarkLine(1/2, byblue)(AC);
draw byLine(B, D, byblack, SOLID_LINE, REGULAR_WIDTH);
byLineDefine(A, B, byblue, SOLID_LINE, REGULAR_WIDTH);
byLineDefine(B, C, byred, SOLID_LINE, REGULAR_WIDTH);
byLineDefine(A, D, byyellow, SOLID_LINE, REGULAR_WIDTH);
byLineDefine(C, D, byyellow, DASHED_LINE, REGULAR_WIDTH);
draw byNamedLineSeq(1)(AB,BC,CD,AD);
draw byArc.O(O, C, A, r, byred, 0, 0, 0, 0);
draw byLabelsOnPolygon(C, B, A, noPoint)(ALL_LABELS, 0);
draw byLabelsOnCircle(D)(O);
}
\drawCurrentPictureInMargin
\problem{P}{ara encontrar una media proporcional entre dos líneas rectas dadas \emph{{}{ Dotted blue and red lines}

\begin{center}

}}.

Dibuja cualquier línea recta \drawFromCurrentPicture[bottom]{
draw byNamedLineFull(A, C, 0, 0,  0, -1)(AC);
draw byNamedArc(O);
draw byLabelsOnPolygon(C, A, noPoint)(ALL_LABELS, 0);
}, haz \drawUnitLine{AD} $=$ Blue dotted line,
 \\ y \drawUnitLine{CD} $=$ Red dotted line; biseca \drawAngle{D};
 \\ y desde el punto de bisección como centro, y la mitad
                     \\ de la línea como radio, traza un semicírculo \drawUnitLine{BD},
 \\ dibuja \drawUnitLine{BD} ⊥ \drawUnitLine{AB}:
 \\ \drawUnitLine{BC} es la media proporcional requerida.

Dibuja \drawUnitLine{A} y \drawUnitLine{B}.

Ya que  es un ángulo recto (L. 3. pr. 31),
                     \\ y  es ⊥ desde el lado opuesto,
                     \\ $\therefore$  es la media proporcional entre
                     \\  y  (L. 6. pr. 8),
                     \\ y $\therefore$ entre Blue dotted line y Red dotted line (const.).

Q. E. D. \byref{prop:III.XXXI} \byref{prop:VI.VIII} \byref{\constref}
\end{center}

\qed

\starttheorem{Prop. XIV. theor.}\label{prop:VI.XIV}

\defineNewPicture[1/4]{
pair A, B, C, D, E, F, G, H, d[];
numeric a;
d1 := (3/2u, 0);
d2 := (1/2u, -2u);
d3 := 2/3d1;
d4 := 3/2d2;
C := (0, 0);
E := C shifted d1;
G := C shifted d2;
B := C shifted (d1 + d2);
F := B shifted d3;
D := B shifted d4;
A := B shifted (d3 + d4);
H = whatever[A, F] = whatever[C, E];
draw byPolygon(C,E,B,G)(byyellow);
draw byPolygon(B,F,A,D)(byblue);
draw byPolygon(E,H,F,B)(byred);
draw byLine(B, G, byred, SOLID_LINE, REGULAR_WIDTH);
draw byLine(B, F, byblack, SOLID_LINE, REGULAR_WIDTH);
draw byLine(B, E, byblue, SOLID_LINE, REGULAR_WIDTH);
draw byLine(B, D, byyellow, SOLID_LINE, REGULAR_WIDTH);
draw byLabelsOnPolygon(B, G, C, E, H, F, A, D)(ALL_LABELS, 0);
}
\drawCurrentPictureInMargin
\problem{P}{roposición}{ XIV. Teorema. \drawUnitLine{BG} \drawUnitLine{BF} \drawUnitLine{BD} \drawUnitLine{BE} \drawUnitLine{BG} \drawUnitLine{BF} \drawUnitLine{BD} \drawUnitLine{BE} \drawUnitLine{BG,BF} \drawUnitLine{BD,BE} \drawUnitLine{BG} \drawUnitLine{BF} \drawUnitLine{BD} \drawUnitLine{BE} \drawUnitLine{BG} \drawUnitLine{BF} \drawUnitLine{BD} \drawUnitLine{BE} \byref{prop:I.XIII,prop:I.XIV,prop:I.XV} \byref{prop:V.VII} \byref{prop:VI.I} \byref{prop:VI.I} \byref{\hypref} \byref{prop:VI.I} \byref{prop:V.XI} \byref{prop:V.IX}}

\begin{center}

\end{center}

\qed

\starttheorem{Prop. XV. theor.}\label{prop:VI.XV}

\defineNewPicture{
pair A, B, C, D, E;
D := (0, 0);
B := (3u, 0);
A := (5/4u, -3/2u);
C := A shifted 3/2(A-D);
E := A shifted 3/2(A-B);
draw byPolygon(D,A,B)(byblue);
draw byPolygon(D,A,E)(byred);
draw byPolygon(A,B,C)(byyellow);
byAngleDefine(D, A, E, byblue, SOLID_SECTOR);
byAngleDefine(B, A, C, byred, SOLID_SECTOR);
draw byNamedAngleResized();
byLineDefine(B, D, byblack, DASHED_LINE, REGULAR_WIDTH);
byLineDefine(D, A, byyellow, SOLID_LINE, REGULAR_WIDTH);
byLineDefine(B, A, byblack, SOLID_LINE, REGULAR_WIDTH);
byLineDefine(A, C, byred, SOLID_LINE, REGULAR_WIDTH);
byLineDefine(A, E, byblue, SOLID_LINE, REGULAR_WIDTH);
draw byNamedLineSeq(0)(noLine,AE,BA,BD,DA,AC);
draw byLabelsOnPolygon(A, E, D, B, C)(ALL_LABELS, 0);
}
\drawCurrentPictureInMargin
\problem{P}{aralelogramos}{ iguales \drawAngle{DAE} y \drawAngle{BAC}, que tienen un ángulo en todo igual, tienen los lados alrededor de los ángulos iguales recíprocamente proporcionales \emph{(\drawUnitLine{AE} : \drawUnitLine{BA} :: \drawUnitLine{AC} : \drawUnitLine{DA})} \drawAngle{DAE} \drawAngle{BAC} \drawUnitLine{AE} \drawUnitLine{BA} \drawUnitLine{AC} \drawUnitLine{DA} \drawUnitLine{BD} \drawUnitLine{DA} \drawUnitLine{BA} \drawUnitLine{AC} \drawUnitLine{DA} \drawUnitLine{AE} \drawUnitLine{BA} \drawUnitLine{AC} \drawUnitLine{DA} \drawUnitLine{DA} \drawUnitLine{BA} \drawUnitLine{AC} \drawUnitLine{DA} \drawUnitLine{AE} \drawUnitLine{BA} \drawUnitLine{AC} \drawUnitLine{DA} \byref{prop:I.XIV} \byref{prop:VI.I} \byref{prop:V.VII} \byref{prop:VI.I} \byref{prop:V.XI} \byref{prop:VI.I} \byref{prop:VI.I} \byref{\hypref} \byref{prop:V.XI} \byref{prop:V.IX}}

\begin{center}

\end{center}

\qed

\starttheorem{Prop. XVI. theor.}\label{prop:VI.XVI}

\defineNewPicture{
pair A, B, C, D, Eb, Fb, Ee, Fe, E, F, G, H, d[];
numeric l[], h[];
l1 := -3u;
l2 := -2u;
h1 := -3/4l2;
h2 := -3/4l1;
A := (0, 0);
B := (l1, 0);
G := (0, h1);
F := (l1, h1);
d1 := (0, -h2 - 1/2u);
C := (0, 0) shifted d1;
D := (l2, 0) shifted d1;
H := (0, h2) shifted d1;
E := (l2, h2) shifted d1;
d2 := (0, h1 + u);
d3 := (0, h1 + 1/2u);
Eb := (0, 0) shifted d2; Ee := (-h2, 0) shifted d2;
Fb := (0, 0) shifted d3; Fe := (-h1, 0) shifted d3;
draw byPolygon(A,B,F,G)(byred);
draw byPolygon(C,D,E,H)(byyellow);
byLineDefine(A, B, byyellow, SOLID_LINE, REGULAR_WIDTH);
byLineDefine(A, G, byblack, SOLID_LINE, REGULAR_WIDTH);
draw byNamedLineSeq(0)(AB,AG);
byLineDefine(C, D, byblue, SOLID_LINE, REGULAR_WIDTH);
byLineDefine(C, H, byred, SOLID_LINE, REGULAR_WIDTH);
draw byNamedLineSeq(0)(CD,CH);
draw byLineWithName(Ee, Eb, byred, 1, 0)(E);
draw byLineWithName(Fe, Fb, byblack, 1, 0)(F);
draw byLabelsOnPolygon(F, G, A, B)(ALL_LABELS, 0);
draw byLabelsOnPolygon(E, H, C, D)(ALL_LABELS, 0);
draw byLabelLine(0)(E, F);
}
\drawCurrentPictureInMargin
\problem{Y}{}{ los paralelogramos que tienen un ángulo en todo igual, y los lados alrededor de ellos recíprocamente proporcionales, son iguales.}

\begin{center}
Deja que \drawUnitLine{AB} y \drawUnitLine{CD}; y \drawUnitLine{E} y \drawUnitLine{F}, se coloquen de manera que \drawUnitLine{AB} y \drawUnitLine{F} puedan ser líneas rectas continuas. Es evidente que pueden asumir esta posición. (L. 1. prs. 13, 14, 15.)

Completa \drawUnitLine{CD}.

Ya que \drawUnitLine{E} $=$ \drawUnitLine{AB};

$\therefore$ \drawUnitLine{CD} : \drawUnitLine{AG} :: \drawUnitLine{CH} : \drawUnitLine{F} (L. 5. pr. 7.)
                     \\ $\therefore$ \drawUnitLine{E} : \drawUnitLine{AB} :: \drawUnitLine{CD} : \drawUnitLine{E} (L. 6. pr. 1.)

La misma construcción restante:
                     \\ \drawUnitLine{F} : \drawUnitLine{AB} ::



\drawUnitLine{CD} : \drawUnitLine{CH} (L. 6. pr. 1.)
\drawUnitLine{AG} : \drawUnitLine{F} (hip.)
\drawUnitLine{AG} : \drawUnitLine{CH} (L. 6. pr. 1.)
 $\therefore$ \drawUnitLine{E} : \drawUnitLine{AB} :: \drawUnitLine{CD} : \drawUnitLine{CH} (L. 5. pr. 11.)
                     \\ y $\therefore$ \drawUnitLine{AG} $=$ \drawUnitLine{CH} (L. 5. pr. 9).

Q. E. D. \drawUnitLine{E} \drawUnitLine{AG} \drawUnitLine{F} \drawUnitLine{AB} \drawUnitLine{CD} \drawUnitLine{E} \drawUnitLine{F} \byref{\hypref} \byref{\constref} \byref{prop:VI.XIV} \byref{prop:VI.XIV} \byref{\constref} \byref{prop:V.VII}
\end{center}

\qed

\starttheorem{Prop. XVII. theor.}\label{prop:VI.XVII}

\defineNewPicture{
pair Ab, Ae, Bb, Be, Cb, Ce, Db, De;
pair A, B, C, D, E, F, G, H;
pair d[];
numeric l[], r;
r := 3/4;
l1 := 3u;
l2 := r*l1;
l3 := r*l2;
d1 := (0, 4/2u); d2 := (0, 3/2u); d3 := (0, 2/2u); d4 := (0, 1/2u);
Ae := (0, 0) shifted d1; Ab := (-l1, 0) shifted d1;
Be := (0, 0) shifted d2; Bb := (-l2, 0) shifted d2;
Ce := (0, 0) shifted d3; Cb := (-l3, 0) shifted d3;
De := (0, 0) shifted d4; Db := (-l2, 0) shifted d4;
d5 := (0, -l2);
A := (0, 0) shifted d5; B := (-l2, 0) shifted d5; C := (-l2, l2) shifted d5; D := (0, l2) shifted d5;
d6 := (0, -l2-l1-1/2u);
E := (0, 0) shifted d6; F := (-l3, 0) shifted d6; G := (-l3, l1) shifted d6; H := (0, l1) shifted d6;
draw byPolygon(A,B,C,D)(byred);
byLineDefine(A, B, byyellow, SOLID_LINE, REGULAR_WIDTH);
byLineDefine(A, D, byblue, SOLID_LINE, REGULAR_WIDTH);
draw byNamedLineSeq(0)(AB,AD);
draw byPolygon(E,F,G,H)(byyellow);
byLineDefine(E, F, byblack, SOLID_LINE, REGULAR_WIDTH);
byLineDefine(E, H, byred, SOLID_LINE, REGULAR_WIDTH);
draw byNamedLineSeq(0)(EF,EH);
draw byLineWithName(Ab, Ae, byred, 0, 0)(A);
draw byLineWithName(Bb, Be, byblue, 0, 0)(B);
draw byLineWithName(Cb, Ce, byblack, 0, 0)(C);
draw byLineWithName(Db, De, byyellow, 0, 0)(D);
draw byLabelLine(0)(A, B, C, D);
draw byLabelPolygon(1)(ABCD);
draw byLabelPolygon(1)(EFGH);
}
\drawCurrentPictureInMargin
\problem{P}{roposición}{ XV. Teorema. \drawUnitLine{A} \drawUnitLine{B} \drawUnitLine{B} \drawUnitLine{C} \drawUnitLine{D} \drawUnitLine{B} \drawUnitLine{A} \drawUnitLine{B} \drawUnitLine{B} \drawUnitLine{C} \drawUnitLine{A} \drawUnitLine{B} \drawUnitLine{D} \drawUnitLine{C} \drawUnitLine{A} \drawUnitLine{C} \drawUnitLine{B} \drawUnitLine{D} \drawUnitLine{D} \drawUnitLine{B} \drawUnitLine{B} \drawUnitLine{D} \drawUnitLine{B} \drawUnitLine{B} \drawUnitLine{B} \drawUnitLine{D} \drawUnitLine{B} \drawUnitLine{A} \drawUnitLine{C} \drawUnitLine{D} \drawUnitLine{B} \drawUnitLine{A} \drawUnitLine{B} \drawUnitLine{D} \drawUnitLine{C} \drawUnitLine{A} \drawUnitLine{B} \drawUnitLine{B} \drawUnitLine{C} \byref{prop:VI.XVI} \byref{prop:VI.XVI}}

\begin{center}

\end{center}

\qed

\startproblem{Prop. XVIII. prob.}\label{prop:VI.XVIII}

\defineNewPicture[1/4]{
pair C, D, E, F, L, d;
C := (0, 0);
D := (3/2u, 0);
E := (u, 3u);
F := (-u, 2u);
L := (5/2u, u);
draw byPolygon(D,C,F)(byyellow);
draw byPolygon(D,F,E)(byblue);
draw byPolygon(D,E,L)(byred);
byAngleDefineExtended(D, F, E, byred, 1)(byblack);
byAngleDefineExtended(D, E, L, byyellow, 1)(byred);
byAngleDefine(D, C, F, byred, ARC_SECTOR);
byAngleDefine(F, D, C, byblue, ARC_SECTOR);
byAngleDefine(E, D, F, byblack, ARC_SECTOR);
byAngleDefine(L, D, E, byyellow, ARC_SECTOR);
draw byNamedAngleResized(DFE, DEL, DCF, FDC, EDF, LDE);
draw byLine(D, F, byred, DASHED_LINE, REGULAR_WIDTH);
byLineDefine(D, E, byyellow, DASHED_LINE, REGULAR_WIDTH);
byLineDefine(C, D, byblack, DASHED_LINE, REGULAR_WIDTH);
byLineDefine(C, F, byblue, SOLID_LINE, THIN_WIDTH);
byLineDefine(F, E, byyellow, SOLID_LINE, REGULAR_WIDTH);
draw byNamedLineSeq(0)(DE,FE,CF,CD);
pair A, B, H, G, K;
numeric s;
s := 5/6;
d := (0, -7/2u);
A := (C scaled s) shifted d;
B := (D scaled s) shifted d;
H := (E scaled s) shifted d;
G := (F scaled s) shifted d;
K := (L scaled s) shifted d;
draw byPolygon(B,A,G)(byyellow);
draw byPolygon(B,G,H)(byblue);
draw byPolygon(B,H,K)(byred);
byAngleDefineExtended(B, G, H, byblack, 1)(byred);
byAngleDefineExtended(B, H, K, byyellow, 1)(byred);
byAngleDefine(B, A, G, byred, SOLID_SECTOR);
byAngleDefine(G, B, A, byblue, SOLID_SECTOR);
byAngleDefine(H, B, G, byblack, SOLID_SECTOR);
byAngleDefine(K, B, H, byyellow, SOLID_SECTOR);
draw byNamedAngleResized(BGH, BHK, BAG, GBA, HBG, KBH);
byLineDefine(B, G, byred, SOLID_LINE, REGULAR_WIDTH);
byLineDefine(A, B, byblack, SOLID_LINE, REGULAR_WIDTH);
byLineDefine(A, G, byblue, SOLID_LINE, REGULAR_WIDTH);
byLineDefine(G, H, byblue, DASHED_LINE, REGULAR_WIDTH);
draw byNamedLineSeq(0)(noLine,BG,AB,AG,GH);
draw byLabelsOnPolygon(F, E, L, D, C)(ALL_LABELS, 0);
draw byLabelsOnPolygon(G, H, K, B, A)(ALL_LABELS, 0);
}
\drawCurrentPictureInMargin
\problem{T}{riángulos}{ iguales, que tienen un ángulo en todo igual \emph{(\drawUnitLine{AB} $=$ \drawUnitLine{DF})}, tienen los lados alrededor de los ángulos iguales recíprocamente proporcionales \emph{(\drawUnitLine{DE} : \drawUnitLine{AB} :: \drawAngle{GBA} : \drawAngle{FDC})}. \drawAngle{A} \drawAngle{C} \drawUnitLine{BG} \drawAngle{G} \drawAngle{F} \drawAngle{HBG} \drawAngle{EDF} \drawAngle{KBH} \drawAngle{LDE} \drawAngle{H} \drawAngle{E} \polygon \drawUnitLine{AB} \drawUnitLine{AG} \drawUnitLine{CD} \drawUnitLine{CF} \drawUnitLine{AG} \drawUnitLine{BG} \drawUnitLine{CF} \drawUnitLine{DF} \drawUnitLine{BG} \drawUnitLine{GH} \drawUnitLine{DF} \drawUnitLine{FE} \drawUnitLine{AG} \drawUnitLine{GH} \drawUnitLine{CF} \drawUnitLine{FE} \polygon \polygon \drawUnitLine{AB} \byref{prop:I.XXXII} \byref{prop:VI.IV} \byref{prop:V.XXII} \byref{prop:VI.I}}

\begin{center}

\end{center}

\qed

\starttheorem{Prop. XIX. theor.}\label{prop:VI.XIX}

\defineNewPicture[1/4]{
pair A, B, C, D, E, F, G, d;
numeric s;
A := (5/2u, -3u);
B := (0, 0);
C := (-u, ypart(A));
d := (0, -4u);
s := 3/4;
D := (A scaled s) shifted d;
E := (B scaled s) shifted d;
F := (C scaled s) shifted d;
G := B shifted (unitvector(C-B) scaled ((abs(E-F)/abs(B-C))*abs(E-F)));
draw byPolygon(A,B,G)(byblue);
draw byPolygon(A,G,C)(byred);
byAngleDefine(A, B, C, byred, SOLID_SECTOR);
draw byNamedAngleResized(ABC);
draw byLine(A, G, byyellow, DASHED_LINE, REGULAR_WIDTH);
byLineDefine(A, B, byyellow, SOLID_LINE, REGULAR_WIDTH);
byLineDefine(B, G, byblack, DASHED_LINE, REGULAR_WIDTH);
byLineDefine(G, C, byblack, SOLID_LINE, REGULAR_WIDTH);
draw byNamedLineSeq(0)(AB,BG,GC);
draw byPolygon(D,E,F)(byyellow);
byAngleDefine(D, E, F, byblack, SOLID_SECTOR);
draw byNamedAngleResized(DEF);
byLineDefine(D, E, byred, SOLID_LINE, REGULAR_WIDTH);
byLineDefine(E, F, byblue, SOLID_LINE, REGULAR_WIDTH);
draw byNamedLineSeq(0)(DE,EF);
draw byLabelsOnPolygon(F, E, D)(ALL_LABELS, 0);
draw byLabelsOnPolygon(C, G, B, A)(ALL_LABELS, 0);
}
\drawCurrentPictureInMargin
\problem{Y}{}{ dos triángulos que tienen un ángulo de uno igual a un ángulo del otro, y los lados alrededor de los ángulos iguales recíprocamente proporcionales, son iguales. \drawAngle{E} \drawAngle{B} \drawUnitLine{BG,GC} \drawUnitLine{EF} \triangleDEF \triangleABC \drawUnitLine{BG,GC} \drawUnitLine{BG} \drawUnitLine{BG,GC} \drawUnitLine{EF} \drawUnitLine{EF} \drawUnitLine{BG} \drawUnitLine{AG} \drawUnitLine{BG,GC} \drawUnitLine{AB} \drawUnitLine{EF} \drawUnitLine{DE} \drawUnitLine{BG,GC} \drawUnitLine{EF} \drawUnitLine{AB} \drawUnitLine{DE} \drawUnitLine{BG,GC} \drawUnitLine{EF} \drawUnitLine{EF} \drawUnitLine{BG} \drawUnitLine{EF} \drawUnitLine{BG} \drawUnitLine{AB} \drawUnitLine{DE} \triangleDEF \drawAngle{E} \drawAngle{B} \triangleABC \triangleDEF \triangleABC \triangleABG \triangleABC \triangleABG \drawUnitLine{BG,GC} \drawUnitLine{BG} \triangleABC \triangleDEF \drawUnitLine{BG,GC} \drawUnitLine{BG} \drawUnitLine{EF} \drawUnitLine{BG,GC} \byref{prop:VI.IV} \byref{prop:V.XVI} \byref{\constref} \byref{prop:VI.XV} \byref{prop:V.VII} \byref{prop:VI.I} \byref{def:V.XI}}

\begin{center}

\end{center}

\qed

\starttheorem{Prop. XX. theor.}\label{prop:VI.XX}

\defineNewPicture{
pair A, B, C, D, E;
A := (-2u, 5/2u);
B := (-1/2u, 0);
C := (3/2u, ypart(B));
D := (2u, 3/2u);
E := (u, ypart(A));
draw byPolygon(B,E,A)(byblue);
draw byPolygon(B,D,E)(byred);
draw byPolygon(B,C,D)(byyellow);
byAngleDefine(B, D, E, byblue, SOLID_SECTOR);
byAngleDefine(B, C, D, byblack, SOLID_SECTOR);
byAngleDefine(B, D, C, byred, SOLID_SECTOR);
draw byNamedAngleResized(BDE, BCD, BDC);
draw byLine(B, D, byblack, DASHED_LINE, REGULAR_WIDTH);
byLineDefine(B, E, byblack, SOLID_LINE, REGULAR_WIDTH);
byLineDefine(B, C, byblue, SOLID_LINE, REGULAR_WIDTH);
byLineDefine(C, D, byblue, DASHED_LINE, REGULAR_WIDTH);
byLineDefine(D, E, byyellow, SOLID_LINE, REGULAR_WIDTH);
draw byNamedLineSeq(0)(BE,BC,CD,DE);
pair F, G, H, K, L, d;
numeric s;
s := 3/4;
d := (0, -3u);
F := (A scaled s) shifted d;
G := (B scaled s) shifted d;
H := (C scaled s) shifted d;
K := (D scaled s) shifted d;
L := (E scaled s) shifted d;
draw byPolygon(G,L,F)(byblue);
draw byPolygon(G,K,L)(byred);
draw byPolygon(G,H,K)(byyellow);
byAngleDefine(G, K, L, byblue, ARC_SECTOR);
byAngleDefine(G, H, K, byblack, ARC_SECTOR);
byAngleDefine(G, K, H, byred, ARC_SECTOR);
draw byNamedAngleResized(GKL, GHK, GKH);
draw byLine(G, K, byblack, DASHED_LINE, THIN_WIDTH);
byLineDefine(G, L, byblack, SOLID_LINE, THIN_WIDTH);
byLineDefine(G, H, byred, SOLID_LINE, THIN_WIDTH);
byLineDefine(H, K, byred, DASHED_LINE, REGULAR_WIDTH);
byLineDefine(K, L, byyellow, SOLID_LINE, REGULAR_WIDTH);
draw byNamedLineSeq(0)(GL,GH,HK,KL);
draw byLabelsOnPolygon(A, E, D, C, B)(ALL_LABELS, 0);
draw byLabelsOnPolygon(F, L, K, H, G)(ALL_LABELS, 0);
}
\drawCurrentPictureInMargin
\problem{D}{eja}{ que los triángulos se coloquen de manera que los ángulos iguales \drawUnitLine{BE} y \drawUnitLine{BD} puedan ser verticalmente opuestos, es decir, que \drawUnitLine{GL} y \drawUnitLine{GK} puedan estar en la misma línea recta. De donde también \drawAngle{C} y \drawAngle{H} deban estar en la misma línea recta (L. 1. pr. 14.)}

\begin{center}
Dibuja \drawUnitLine{BC}, entonces

\drawUnitLine{CD} : \drawUnitLine{GH} :: \drawUnitLine{HK} : \drawAngle{BDC} (L. 6. pr. 1.)
:: \drawAngle{GKH} : \drawAngle{BDE,BDC} (L. 5. pr. 7.)
:: \drawAngle{GKL,GKH} : \drawAngle{BDE} (L. 6. pr. 1.)
$\therefore$ \drawAngle{GKL} : \drawUnitLine{BD} :: \drawUnitLine{CD} : \drawUnitLine{GK} (L. 5. pr. 11.) \drawUnitLine{HK} \drawUnitLine{CD} \drawUnitLine{DE} \drawUnitLine{HK} \drawUnitLine{KL} \drawUnitLine{BD} \drawUnitLine{DE} \drawUnitLine{GK} \drawUnitLine{KL} \drawUnitLine{BD} \drawUnitLine{GK} \drawUnitLine{BD} \drawUnitLine{GK} \drawUnitLine{BE} \drawUnitLine{GL} \drawUnitLine{BE} \drawUnitLine{GL} \drawUnitLine{BC} \drawUnitLine{GH} \drawUnitLine{BC} \drawUnitLine{GH} \byref{prop:VI.VI} \byref{prop:V.XXII} \byref{prop:VI.VI} \byref{prop:VI.XIX} \byref{prop:V.XI} \byref{prop:V.XII}
\end{center}

\qed

\starttheorem{Prop. XXI. theor.}\label{prop:VI.XXI}

\defineNewPicture{
pair Aa, Ab, Ac, Ba, Bb, Bc, Ca, Cb, Cc, d[];
numeric s[];
Aa := (0, 0);
Ab := (0, 3/2u);
Ac := (-3u, 0);
s1 := 5/6;
d1 := (0, -2u);
Ba := (Aa scaled s1) shifted d1;
Bb := (Ab scaled s1) shifted d1;
Bc := (Ac scaled s1) shifted d1;
s2 := 4/6;
d2 := d1 shifted (0, -2u * s1) ;
Ca := (Aa scaled s2) shifted d2;
Cb := (Ab scaled s2) shifted d2;
Cc := (Ac scaled s2) shifted d2;
draw byPolygon.A(Aa,Ab,Ac)(byred);
draw byPolygon.B(Ba,Bb,Bc)(byblue);
draw byPolygon.C(Ca,Cb,Cc)(byyellow);
byPointLabelRemove(Aa, Ba, Ca, Ab, Bb, Cb, Ac, Bc, Cc);
}
\drawCurrentPictureInMargin
\problem{X}{X}{Que permanezca la misma construcción, y
                      \polygon : \polygon :: \polygon : \polygon (L. 6. pr. 1.)
                      y \polygon : \polygon ::  :  (L. 6. pr. 1.)}

\begin{center}
Pero  :  ::  : , (hip.)
                     \\ $\therefore$  :  ::  :  (L. 5. pr. 11);
                     \\ $\therefore$  $=$  (L. 5. pr. 9.)

Q. E. D. \byref{def:VI.I} \byref{prop:V.XI}
\end{center}

\qed

\starttheorem{Prop. XXII. theor.}\label{prop:VI.XXII}

\defineNewPicture[1/2]{
pair A, B, C, D, E, F, G, H, Ob, Oe, Pb, Pe;
pair K, L, Ma, Mb, Mc, Md, Na, Nb, Nc, Nd;
pair d[];
numeric s[];
s1 := 6/5;
A := (0, 0); B := (6/5u, 0); K := (u, u);
d1 := (7/4u, 0);
C := (A scaled s1) shifted d1;
D := (B scaled s1) shifted d1;
L := (K scaled s1) shifted d1;
d2 := (7/2u, 0);
Ob := (A scaled (s1*s1)) shifted d2;
Oe := (B scaled (s1*s1)) shifted d2;
E := (0, 0); F := (5/6u, 0); Ma := (4/3u, 2/3u); Mb := (u, 4/3u); Mc := (1/5u, 3/2u); Md := (-1/4u, 5/6u);
G := (E scaled s1) shifted d1;
H := (F scaled s1) shifted d1;
Na := (Ma scaled s1) shifted d1;
Nb := (Mb scaled s1) shifted d1;
Nc := (Mc scaled s1) shifted d1;
Nd := (Md scaled s1) shifted d1;
Pb := (E scaled (s1*s1)) shifted d2;
Pe := (F scaled (s1*s1)) shifted d2;
d3 := (1/4u, -5/2u);
forsuffixes i=E,F,Ma,Mb,Mc,Md,G,H,Na,Nb,Nc,Nd,Pb,Pe:
i := i shifted d3;
endfor;
draw byPolygon.AB(A,B,K)(byyellow);
draw byPolygon.CD(C,D,L)(byred);
draw byPolygon.EF(E,F,Ma,Mb,Mc,Md)(byblue);
draw byPolygon.GH(G,H,Na,Nb,Nc,Nd)(white);
draw byLine(A, B, byblack, SOLID_LINE, REGULAR_WIDTH);
draw byLine(C, D, byblue, SOLID_LINE, REGULAR_WIDTH);
draw byLineWithName(Ob, Oe, byblack, 1, 0)(O);
draw byLine(E, F, byred, SOLID_LINE, REGULAR_WIDTH);
draw byLine(G, H, byyellow, SOLID_LINE, REGULAR_WIDTH);
draw byLineWithName(Pb, Pe, byred, 1, 0)(P);
byPointLabelRemove(Ma, Mb, Mc, Md, Na, Nb, Nc, Nd, K, L);
draw byLabelLine(1)(AB,CD,EF,GH,O,P);
}
\drawCurrentPictureInMargin
\problem{S}{i}{ cuatro líneas rectas son proporcionales (\drawUnitLine{AB} : \drawUnitLine{CD} :: \drawUnitLine{EF} : \drawUnitLine{GH}), el rectángulo (\drawUnitLine{O} × \drawUnitLine{AB}) contenido por los extremos, es igual al rectángulo (\drawUnitLine{CD} × \drawUnitLine{P}) contenido por los medios.}

\begin{center}
Y si el rectángulo contenido por los extremos es igual al rectángulo contenido por los medios, las cuatro líneas rectas son proporcionales.

Desde los extremos de \drawUnitLine{EF} y \drawUnitLine{GH} dibuja \drawUnitLine{AB} y \drawUnitLine{CD} ⊥ a ellos e $=$ \drawUnitLine{EF} y \drawUnitLine{GH} respectivamente, completa los paralelogramos:

\drawUnitLine{CD} y \drawUnitLine{O}.

Y ya que,
                        


\drawUnitLine{GH} : \drawUnitLine{P} :: \drawUnitLine{AB} : \drawUnitLine{O} (hip.)
$\therefore$ \drawUnitLine{EF} : \drawUnitLine{P} :: \drawUnitLine{AB} : \drawUnitLine{O} (const.)



$\therefore$ \drawUnitLine{EF} $=$ \drawUnitLine{P} (L. 6. pr. 14),

es decir, el rectángulo contenido por los extremos es igual al rectángulo contenido por los medios.

Deja que la misma construcción permanezca; porque
                     \\ \drawUnitLine{AB} $=$ \drawUnitLine{O}, \drawUnitLine{EF} $=$ \drawUnitLine{P}  \\ y \drawUnitLine{AB} $=$ \drawUnitLine{CD},
 \\ $\therefore$ \drawUnitLine{EF} : \drawUnitLine{GH} ::  :  (L. 6. pr. 14).

Pero  $=$ ,
 \\ y  $=$  (const.)
                     \\ $\therefore$  :  ::  :  (L. 5. pr. 7).

Q. E. D. \byref{prop:VI.XI} \byref{\hypref} \byref{\constref} \byref{prop:VI.XX} \byref{prop:V.XI} \byref{\hypref} \byref{\constref} \byref{prop:V.XI}
\end{center}

\qed

\starttheorem{Prop. XXIII. theor.}\label{prop:VI.XXIII}

\defineNewPicture[1/2]{
pair A, B, C, D, E, F, G ,H, d[];
d1 := (u, 0);
d2 := d1 scaled -2;
d3 := (1/2u, 2u);
d4 := d3 scaled -2/3;
C := (0, 0);
G := C shifted d1;
B := C shifted d2;
E := C shifted d3;
D := C shifted d4;
H := C shifted d1 shifted d4;
A := C shifted d2 shifted d4;
F := C shifted d1 shifted d3;
draw byPolygon(A,B,C,D)(byyellow);
draw byPolygon(E,F,G,C)(byblue);
draw byPolygon(C,D,H,G)(byred);
byAngleDefine(B, C, D, byred, SOLID_SECTOR);
byAngleDefine(D, C, G, byyellow, SOLID_SECTOR);
byAngleDefine(G, C, E, byblack, SOLID_SECTOR);
draw byNamedAngleResized();
draw byLine(D, C, byblack, SOLID_LINE, REGULAR_WIDTH);
draw byLine(C, E, byred, SOLID_LINE, REGULAR_WIDTH);
draw byLine(B, C, byblue, SOLID_LINE, REGULAR_WIDTH);
draw byLine(C, G, byyellow, SOLID_LINE, REGULAR_WIDTH);
draw byLabelsOnPolygon(D, A, B, C, E, F, G, H)(ALL_LABELS, 0);
}
\drawCurrentPictureInMargin
\problem{S}{i}{ tres líneas rectas son proporcionales (\drawUnitLine{BC} : \drawUnitLine{CG} :: \drawAngle{BCD} : \drawAngle{DCG}) el cuadrado bajo los extremos es igual al cuadrado del medio.}

\begin{center}
Y si el rectángulo bajo de los extremos es igual al cuadrado del medio, las tres líneas rectas son proporcionales.

Supón que \drawTwoRightAngles $=$ \drawAngle{GCE}, y
ya que \drawAngle{BCD} : \drawAngle{GCE} :: \drawAngle{DCG} : \drawTwoRightAngles,
entonces \drawUnitLine{CE} : \drawUnitLine{DC} :: \polygon : \polygon,
$\therefore$ \drawUnitLine{BC} × \drawUnitLine{CG} $=$ \polygon × \polygon 
 (L. 6. pr. 16).

Pero \drawUnitLine{DC} $=$ \drawUnitLine{CE},
 \\ $\therefore$ \polygon × \polygon $=$ \drawUnitLine{BC} × \drawUnitLine{CG}, or $=$ \drawUnitLine{DC}2;
                     \\ por lo tanto, si las tres líneas rectas son proporcionales, el rectángulo contenido por los extremos es igual al cuadrado del medio.

Supón que \drawUnitLine{CE} $=$ , entonces
                     \\  ×  $=$  × .
 \\ $\therefore$  :  ::  :  (L. 6. pr. 16), y
                     \\ $\therefore$  :  ::  : .

Q. E. D. \byref{\hypref} \byref{prop:I.XIV} \byref{prop:VI.I} \byref{prop:VI.I}
\end{center}

\qed

\starttheorem{Prop. XXIV. theor.}\label{prop:VI.XXIV}

\defineNewPicture{
pair A, B, C, D, E, F, G, H, K, d[];
numeric s;
d1 := (3u, 0);
d2 := (u, 3u);
s := 3/5;
A := (0, 0);
B := A shifted d1;
C := A shifted d1 shifted d2;
D := A shifted d2;
G := s[A, D];
E := s[A, B];
H := s[B, C];
K := s[D, C];
F = whatever[G, H] = whatever[E, K];
draw byLine(F, H, byblack, SOLID_LINE, THIN_WIDTH);
draw byLine(E, F, byblack, SOLID_LINE, THIN_WIDTH);
draw byPolygon(G,F,K,D)(byyellow);
draw byPolygon(A,G,F)(byblue);
draw byPolygon(F,C,K)(byred);
draw byLine(G, F, byred, SOLID_LINE, REGULAR_WIDTH);
byLineDefine(A, E, byblack, SOLID_LINE, THIN_WIDTH);
byLineDefine(E, B, byblack, SOLID_LINE, THIN_WIDTH);
byLineDefine(B, H, byblack, SOLID_LINE, THIN_WIDTH);
byLineDefine(H, C, byblack, SOLID_LINE, THIN_WIDTH);
byLineDefine(D, K, byblue, SOLID_LINE, REGULAR_WIDTH);
byLineDefine(K, C, byblue, DASHED_LINE, REGULAR_WIDTH);
byLineDefine(D, G, byred, DASHED_LINE, REGULAR_WIDTH);
byLineDefine(G, A, byyellow, SOLID_LINE, REGULAR_WIDTH);
draw byNamedLineSeq(0)(GA,DG,DK,KC,HC,BH,EB,AE);
draw byLabelsOnPolygon(A, G, D, K, C, H, B, E)(ALL_LABELS, 0);
draw byLabelsOnPolygon(H, F, E)(OMIT_FIRST_LABEL+OMIT_LAST_LABEL, 0);
}
\drawCurrentPictureInMargin
\problem{X}{X}{En una línea recta dada (\drawFromCurrentPicture[bottom][parallelogramABCD]{
draw byNamedLine(AE,EB,BH,HC,FH,EF);
draw byNamedPolygon(GFKD,AGF,FCK);
draw byLabelsOnPolygon(D, C, B, A)(ALL_LABELS, 0);
}) para construir una figura rectilínea similar a una dada (\drawFromCurrentPicture[bottom][parallelogramFHCK]{
draw byNamedLine(FH,HC);
draw byNamedPolygon(FCK);
draw byLabelsOnPolygon(K, C, H, F)(ALL_LABELS, 0);
}) y colocada de manera similar.}

\begin{center}
Resuelve la figura dada en triángulos dibujando las líneas \drawFromCurrentPicture[bottom][parallelogramAEFG]{
draw byNamedLine(AE,EF);
draw byNamedPolygon(AGF);
draw byLabelsOnPolygon(G, F, E, A)(ALL_LABELS, 0);
} y \drawUnitLine{GF}.

En los extremos de \drawUnitLine{DK,KC} haz
                     \\ \drawFromCurrentPicture[bottom]{
startGlobalRotation(180-angle(A-C));
startAutoLabeling;
draw byNamedPolygon(AGF);
stopAutoLabeling;
stopGlobalRotation;
} $=$ \drawFromCurrentPicture[bottom]{
startGlobalRotation(180-angle(A-C));
startAutoLabeling;
draw byNamedPolygon(GFKD,AGF,FCK);
stopAutoLabeling;
stopGlobalRotation;
} y \drawUnitLine{GA} $=$ \drawUnitLine{GF}:
 \\ de nuevo en los extremos de \drawUnitLine{GA,DG} haz \drawUnitLine{DK,KC} $=$   \\ y  $=$ : de la misma manera haz
                     \\  $=$  y  $=$ .

Entonces  es similar a .

Es evidente por la construcción y (L. 1. pr. 32) que las figuras son 
equiangulares; y ya que los triángulos
                     \\  y  son equiangulares, entonces por (L. 6. pr. 4),

 :  ::  : 
y  :  ::  : .

De nuevo, porque  y  son equiangulares,
                     \\  :  ::  : ;
 \\ $\therefore$ ex æquali,
                     \\  :  ::  :  (L. 6. pr. 22.)

De manera similar, se puede demostrar que los lados restantes de las dos 
figuras son proporcionales.

$\therefore$ por (L. 6. def. 1.)
                     \\  es similar a   \\ y situadas de manera similar, en la línea dada .

Q. E. D. \byref{prop:VI.IV}
\end{center}

\qed

\startproblem{Prop. XXV. prob.}\label{prop:VI.XXV}

\defineNewPicture[1/2]{
pair d[];
pair A, B, C;
A := (5/3u, 2u);
B := (0, 0);
C := (9/4u, 0);
draw byPolygon(A,B,C)(byred);
pair L, E;
L := (xpart(B), -1/2ypart(A));
E := (xpart(C), ypart(L));
draw byPolygon(B,C,E,L)(byblue);
byAngleDefine(C, B, L, byred, SOLID_SECTOR);
draw byNamedAngleResized(CBL);
pair Da, Db, Dc, Dd, De;
Da := dir(0) scaled 1/2u;
Db := dir(72) scaled 2/5u;
Dc := dir(144) scaled 1/2u;
Dd := dir(-144) scaled 3/5u;
De := dir(-72) scaled 1/2u;
d1 := (3u, 3/2u);
byPointLabelRemove(Da,Db,Dc,Dd,De);
forsuffixes i=Da,Db,Dc,Dd,De:
	i := ((i rotated 36) scaled 3/2) shifted d1;
endfor;
draw byPolygon.D(Da,Db,Dc,Dd,De)(byblue);
pair F, M;
numeric a, l;
a := 	((abs(Da-Db)*distanceToLine(Dc, Da--Db))+
	(abs(Da-Dc)*distanceToLine(Dd, Da--Dc))+
	(abs(Da-Dd)*distanceToLine(De, Da--Dd)))/2;
l := a/abs(C-E);
F := C shifted (l, 0);
M := E shifted (l, 0);
draw byPolygon(C,F,M,E)(white);
byAngleDefine(F, C, E, byyellow, SOLID_SECTOR);
draw byNamedAngleResized(FCE);
draw byLine(C, E, byred, SOLID_LINE, REGULAR_WIDTH);
byLineDefine(B, C, byyellow, SOLID_LINE, REGULAR_WIDTH);
byLineDefine(C, F, byblack, DASHED_LINE, REGULAR_WIDTH);
draw byNamedLineSeq(0)(BC, CF);
pair K, G, H;
numeric s;
s := (a/(abs(B-C)*abs(B-L)))**(1/2);
d2 := (2/3u, -3u);
K := (A scaled s) shifted d2;
G := (B scaled s) shifted d2;
H := (C scaled s) shifted d2;
draw byPolygon(K,G,H)(byyellow);
draw byLine(G, H, byblue, SOLID_LINE, REGULAR_WIDTH);
draw byLabelsOnPolygon(C, F, M, E, L, B, A)(ALL_LABELS, 0);
draw byLabelsOnPolygon(G, K, H)(ALL_LABELS, 0);
}
\drawCurrentPictureInMargin
\problem{T}{riángulos}{ similares (\drawUnitLine{BC} y \triangleABC) están el uno con el otro en la proporción duplicada de sus lados homólogos.}

\begin{center}
Sean \drawUnitLine{CE} y \drawFromCurrentPicture[bottom][polygonCFME]{
startTempAngleScale(angleScale*1/2);
draw byNamedAngle(C);
startAutoLabeling;
draw byNamedPolygon(CFME);
stopAutoLabeling;
stopTempAngleScale;
} ángulos iguales, y \polygon y \drawAngle{B} lados homólogos de los triángulos similares \drawAngle{C} y \drawUnitLine{BC} y en \drawUnitLine{CF} la mayor de estas líneas toma \drawUnitLine{BC} proporcional tal que

\drawUnitLine{CF} : \drawUnitLine{GH} :: \drawUnitLine{GH} : \triangleABC;
 \\ dibuja \triangleKGH.

\polygon : \triangleABC :: \triangleKGH : \drawUnitLine{BC} (L. 6. pr. 4);
                     \\ $\therefore$ \drawUnitLine{GH} : \drawUnitLine{GH} :: \drawUnitLine{CF} : \triangleABC (L. 5. pr. 16, alt.),
                     \\ pero \triangleKGH : \drawUnitLine{BC} :: \drawUnitLine{CF} : \polygon (const.),
                     \\ $\therefore$ \polygon : \drawUnitLine{BC} :: \drawUnitLine{CF} : \triangleABC  \\ consecuentemente \triangleKGH $=$ \polygon porque tienen los lados sobre
                     \\ los ángulos iguales \polygon y \triangleABC recíprocamente proporcionales (L. 6. pr. 15);
                     \\ $\therefore$ \polygon : \triangleKGH :: \polygon : \polygon (L. 5. pr. 7),
                     \\ pero \polygon : \triangleKGH :: \triangleABC : \polygon (L. 6. pr. 1),
                     \\ $\therefore$  :  ::  : ,
 \\ es decir, los triángulos están entre sí en la proporción duplicada de sus lados homólogos
                     \\  y  (L. 5. def. 11).

Q. E. D. \byref{prop:I.XLV} \byref{prop:I.XXIX,prop:I.XIV} \byref{prop:VI.XIII} \byref{\constref} \byref{prop:VI.XX} \byref{prop:VI.I} \byref{prop:V.XI} \byref{\constref} \byref{prop:V.XIV} \byref{\constref}
\end{center}

\qed

\starttheorem{Prop. XXVI. theor.}\label{prop:VI.XXVI}

\defineNewPicture{
pair A, B, C, D, E, F, G, H, K, d[];
d1 := (3u, 0);
d2 := (u, 3u);
A := (0, 0);
B := A shifted d1;
C := A shifted d1 shifted d2;
D := A shifted d2;
F := 3/4[A, C];
E = whatever[A, B] = whatever[F, F shifted d2];
G = whatever[A, D] = whatever[F, F shifted d1];
H := 2/3[G, F];
K = whatever[A, B] = whatever[H, H shifted d2];
draw byPolygon(F,H,K,E)(byred);
draw byPolygon(G,D,C,B,E,F)(byblue);
byAngleDefine(B, A, D, byyellow, SOLID_SECTOR);
draw byNamedAngleResized();
draw byLine(K, H, byyellow, SOLID_LINE, REGULAR_WIDTH);
draw byLine(G, H, byyellow, DASHED_LINE, REGULAR_WIDTH);
draw byLine(A, C, byblack, SOLID_LINE, THIN_WIDTH);
byLineDefine(A, G, byred, SOLID_LINE, REGULAR_WIDTH);
byLineDefine(G, D, byred, DASHED_LINE, REGULAR_WIDTH);
byLineDefine(A, K, byblue, SOLID_LINE, REGULAR_WIDTH);
byLineDefine(K, E, byblue, DASHED_LINE, REGULAR_WIDTH);
byLineDefine(E, B, byblack, DASHED_LINE, REGULAR_WIDTH);
draw byNamedLineSeq(0)(GD,AG,AK,KE,EB);
draw byArbitraryFigure.AHC(A..H..C, byblack, 0, 0);
byLineDefine.KAt(K, A, byblack, SOLID_LINE, THIN_WIDTH);
byLineStylize(H, G, 1, 0, 1)(KAt);
byLineDefine.AGt(A, G, byblack, SOLID_LINE, THIN_WIDTH);
byLineStylize(K, H, 0, 0, 1)(AGt);
byLineDefine.GHt(G, H, byblack, SOLID_LINE, THIN_WIDTH);
byLineStylize(A, K, 0, 1, 1)(GHt);
byLineDefine.EAt(E, A, byblack, SOLID_LINE, THIN_WIDTH);
byLineStylize(F, G, 1, 0, 1)(EAt);
draw byLabelsOnPolygon(A, G, D, C, B, E, K)(ALL_LABELS, 0);
draw byLabelsOnPolygon(K, H, A)(OMIT_FIRST_LABEL+OMIT_LAST_LABEL, 0);
draw byLabelsOnPolygon(E, F, A)(OMIT_FIRST_LABEL+OMIT_LAST_LABEL, 0);
}
\drawCurrentPictureInMargin
\problem{X}{X}{Dibuja \drawFromCurrentPicture[bottom][parallelogramAEFG]{
draw byNamedPolygon(FHKE);
draw byNamedLine(KAt,AGt,GHt);
draw byLabelsOnPolygon(A, G, F, E)(ALL_LABELS, 0);
} y \drawFromCurrentPicture[bottom][parallelogramABCD]{
draw byNamedPolygon(GDCBEF);
draw byNamedLine(EAt,AC);
draw byNamedLineFull(E, F, 0, 1,  0, 1)(AGt);
draw byLabelsOnPolygon(A, D, C, B)(ALL_LABELS, 0);
}, y \drawFromCurrentPicture[bottom][lineAHC]{
startGlobalRotation(180-angle(A-C));
draw byNamedArbitraryFigure(AHC);
draw byLabelsOnPolygon(A, H, C, noPoint)(ALL_LABELS, 0);
stopGlobalRotation;
} y \drawUnitLine{KH}, resolviendo los polígonos en triángulos. Entonces porque los polígonos son similares, \drawUnitLine{AG} $=$ \drawLine[bottom][parallelogramAKHG]{AG,GH,KH,AK}, y \drawAngle{A} : \drawUnitLine{AC} ::  : }

\begin{center}
$\therefore$  y  son similares, y  $=$  (L. 6. pr. 6);
                     \\ pero  $=$  porque son ángulos de polígonos similares;
                     \\ por lo tanto los restantes  y  son iguales;
                     \\ por consiguiente  :  ::  : ,
 \\ causa de los triángulos similares,
                     \\ y  :  ::  : ,
 \\ causa de los polígonos similares,
                     \\ $\therefore$  :  ::  : ,
 \\ ex æquali (L. 5. pr. 22), y como estos lados proporcionales
                     \\ contienen ángulos iguales, los triángulos  y   \\ son similares (L. 6. pr. 6).

De igual manera se puede demostrar que los
                     \\ triángulos  y  son similares.

Pero  es a  en la proporción duplicada de
                     \\  a  (L. 6. pr. 19), y
                     \\  es a  en la misma manera, en la proporción duplicada
                     \\ de  a ;
                     \\ $\therefore$  :  ::  : , (L. 5. pr. 11);

De nuevo  es a  en la proporción duplicada de
                     \\  a , y  es a  en
                     \\ la proporción duplicada de  a ,

$\therefore$  :  ::  : 
::  : ;

y como uno de los antecedentes es a uno de los consecuentes, también lo es la suma de todos los antecedentes a la suma de todos los consecuentes; es decir, los triángulos similares tienen entre sí la misma relación como (L. 5. pr. 12).

Pero  es a  en la proporción duplicada de
                     \\  a ;

$\therefore$  es a  en la proporción
                     \\ duplicada de  a .

Q. E. D. \byref{prop:I.XXXI} \byref{prop:VI.XXIV} \byref{\hypref} \byref{prop:V.IX}
\end{center}

\qed

\starttheorem{Prop. XXVII. theor.}\label{prop:VI.XXVII}

\defineNewPicture[1/2]{
pair A, B, C, D, E, F, G, H, K;
numeric l, s;
l := 5u;
s := 1/5;
A := (0, 0);
B := (xpart(A), -l);
C := 1/2[A, B];
D := s[A, B];
E := (-1/2l, ypart(C));
F := (xpart(E), ypart(A));
G := (-s*l, ypart(D));
H := (xpart(G), ypart(B));
K = whatever[C, E] = whatever[G, H];
draw byPolygon(A,D,G,K,E,F)(byred);
draw byPolygon(C,D,G,K)(byblue);
draw byPolygon(B,C,K,H)(byyellow);
draw byLine(A, D, byyellow, SOLID_LINE, REGULAR_WIDTH);
draw byLine(D, C, byred, SOLID_LINE, REGULAR_WIDTH);
draw byLine(C, B, byblue, SOLID_LINE, REGULAR_WIDTH);
draw byLabelsOnPolygon(E, F, A, D, C, B, H)(ALL_LABELS, 0);
draw byLabelsOnPolygon(H, G, D)(OMIT_FIRST_LABEL+OMIT_LAST_LABEL, 0);
}
\drawCurrentPictureInMargin
\problem{F}{iguras}{ rectilíneas ( y ) que son similares a la misma figura () son similares entre sí.}

\begin{center}
Ya que  y  son similares, son equiangulares, y tienen los lados sobre los ángulos iguales proporcionales (L. 6. def. 1); y ya que las figuras  y  son también similares, son equiangulares, y tienen los lados sobre los ángulos iguales proporcionales; por lo tanto  y  también son equiangulares y tienen los lados sobre los ángulos iguales proporcionales (L. 5. pr. 11), y por lo tanto similares.

Q. E. D. \byref{prop:II.V}
\end{center}

\qed

\startproblem{Prop. XXVIII. prob.}\label{prop:VI.XXVIII}

\defineNewPicture{
pair A, B, C, D, E, F, G, H, d;
path a;
numeric r;
A := (0, 0);
B := (4u, 0);
C := 1/2[A, B];
D := C shifted (0, 3/2u);
r := 2u;
a := (fullcircle scaled (2r)) shifted D;
E := a intersectionpoint (A--C);
F := a intersectionpoint (D--2[D, C]);
byLineDefine(D, E, byyellow, SOLID_LINE, REGULAR_WIDTH);
byLineDefine(D, C, byred, SOLID_LINE, REGULAR_WIDTH);
byLineDefine(C, F, byblack, DASHED_LINE, REGULAR_WIDTH);
draw byNamedLineSeq(0)(CF,DC,DE);
byLineDefine(A, E, byred, DASHED_LINE, REGULAR_WIDTH);
byLineDefine(E, C, byblue, SOLID_LINE, REGULAR_WIDTH);
byLineDefine(C, B, byblue, DASHED_LINE, REGULAR_WIDTH);
draw byNamedLineSeq(0)(AE, EC, CB);
draw byArcBE.a(D, -1/2, -4 + 1/2, r, byred, 0, 0, 0, 0);
d := (0, 2u);
G := A shifted d;
H := C shifted d;
byLineDefine.G(G, H, byyellow, DASHED_LINE, REGULAR_WIDTH);
lineUseLineLabel.G := true;
draw byLabelsOnCircle(F)(a);
draw byLabelsOnPolygon(E, D, C, B)(OMIT_FIRST_LABEL+OMIT_LAST_LABEL, 0);
draw byLabelsOnPolygon(A, B, noPoint)(ALL_LABELS, 0);
draw byLabelPoint(E, angle(E-D) + 45, 2);
}
\drawCurrentPictureInMargin
\problem{S}{i}{ cuatro líneas rectas son proporcionales ( :  ::  : ), las figuras rectilíneas similares trazadas de manera similar en ellas también son proporcionales.}

\begin{center}
Y si cuatro figuras rectilíneas similares, trazadas de manera similar en cuatro líneas rectas, son proporcionales, las líneas rectas también son proporcionales.

Toma  una tercera proporcional a   \\ y , y  una tercera proporcional
                     \\ a  y  (L. 6. pr. 11);
                    


ya que  :  ::  :  (hip.),
 :  ::  :  (const.)

$\therefore$ ex æquali,
                     \\  :  ::  : ;
 \\ pero  :  ::  :  (L. 6. pr. 20),
                     \\ y  :  ::  : ;
 \\ $\therefore$  :  ::  :  (L. 5. pr. 11).

Deja que la misma construcción permanezca:

 :  ::  :  (hip.),
                     \\ $\therefore$  :  ::  :  (const.)
                     \\ y $\therefore$  :  ::  : . (L. 5. pr. 11).

Q. E. D. \byref{\hypref} \byref{prop:II.V} \byref{prop:I.XLVII} \byref{\constref}
\end{center}

\qed

\startproblem{Prop. XXIX. prob.}\label{prop:VI.XXIX}

\defineNewPicture[1/2]{
pair A, B, C, D, E, F, G, H, d;
d := (0, -1/2u);
G := (0, 0) shifted d;
H := (3/2u, 0) shifted d;
A := (0, 0);
B := (2u, ypart(A));
C := 1/2[A, B];
D := (xpart(B), abs(G-H));
E := C shifted (-abs(C-D), 0);
F := C shifted (abs(C-D), 0);
draw byLine.G(H, G, byblack, SOLID_LINE, REGULAR_WIDTH);
draw byLineFull(C, D, byred, 0, 0)(B, D, 1, 0, 0);
draw byLine(B, D, byred, DASHED_LINE, REGULAR_WIDTH);
byLineDefine(E, A, byyellow, DASHED_LINE, REGULAR_WIDTH);
byLineDefine(A, C, byblue, SOLID_LINE, REGULAR_WIDTH);
byLineDefine(C, B, byblue, DASHED_LINE, REGULAR_WIDTH);
byLineDefine(B, F, byyellow, SOLID_LINE, REGULAR_WIDTH);
draw byNamedLineSeq(1)(EA, AC, CB, BF);
draw byArcBE.a(C, 0, 4, abs(C-D), byred, 0, 0, 0, 0);
draw byLabelsOnCircle(D)(a);
draw byLabelsOnPolygon(F, B, C, A, noPoint)(ALL_LABELS, 0);
draw byLabelLine(0)(G);
}
\drawCurrentPictureInMargin
\problem{D}{eje}{ que dos de los lados  y  alrededor de los ángulos iguales se coloquen para que puedan formar una línea recta.}

\begin{center}
Ya que  +  $=$ 
\drawTwoRightAngles


,
 \\ y  $=$  (hip.),
                     \\  +  $=$ 
\drawTwoRightAngles


,
 \\ y $\therefore$  y  forman una línea recta (L. 1. pr. 14);
                     \\ completan .

Ya que  :  ::  :  (L. 6. pr. 1),
                     \\ y  :  ::  :  (L. 6. pr. 1),
                     \\  tiena a  una proporción compuesta de las proporciones de
                     \\  a , y de  a .

Q. E. D. \byref{prop:II.VI} \byref{prop:I.XLVII}
\end{center}

\qed

\startproblem{Prop. XXX. prob.}\label{prop:VI.XXX}

\defineNewPicture{
pair A, B, C, D, E, F, G, H;
numeric w;
w := 3u;
A := (0, 0);
B := (w, 0);
C := (0, w);
H := (w, w);
G := 1/2[A, C] shifted (0, -abs((1/2[A, C]) - B));
D := G shifted (abs(G-A), 0);
E = whatever[D, D shifted (0, 1)] = whatever[A, B];
F = whatever[C, H] = whatever[D, E];
draw byPolygon(A,C,F,E)(byyellow);
draw byPolygon(E,F,H,B)(byblue);
byLineDefine(A, E, byred, SOLID_LINE, REGULAR_WIDTH);
byLineDefine(E, B, byred, DASHED_LINE, REGULAR_WIDTH);
draw byNamedLineSeq(1)(AE, EB);
byLineDefine(C, A, byblue, SOLID_LINE, REGULAR_WIDTH);
byLineDefine(A, G, byblue, DASHED_LINE, REGULAR_WIDTH);
byLineDefine(G, D, byyellow, SOLID_LINE, REGULAR_WIDTH);
byLineDefine(D, E, byblack, SOLID_LINE, REGULAR_WIDTH);
draw byNamedLineSeq(0)(CA, DE, GD, AG);
byLineDefine.AGt(A, G, byblack, SOLID_LINE, THIN_WIDTH);
byLineStylize(A, D, 1, 0, -1)(AGt);
byLineDefine.GDt(G, D, byblack, SOLID_LINE, THIN_WIDTH);
byLineStylize(A, E, 0, 0, -1)(GDt);
byLineDefine.DEt(D, E, byblack, SOLID_LINE, THIN_WIDTH);
byLineStylize(G, E, 0, 1, -1)(DEt);
draw byLabelsOnPolygon(A, C, F, H, B, E, D, G)(ALL_LABELS, 0);
}
\drawCurrentPictureInMargin
\problem{X}{X}{En cualquier paralelogramo (\drawFromCurrentPicture[middle][rectangleCFDG]{
draw byNamedPolygon(ACFE);
draw byNamedLine(AGt,GDt,DEt);
draw byLabelsOnPolygon(G, C, F, D)(ALL_LABELS, 0);
}) los paralelogramos (\square y \drawLine{DE,GD,AG,AE}) que están sobre la diagonal son similares al todo y entre sí.}

\begin{center}
Como  y  tienen un
                     \\ ángulo común, son equiangulares;
                     \\ pero porque  ∥   \\  y  son similares (L. 6. pr. 4),
                     \\ $\therefore$  :  ::  : ;
 \\ y los lados opuestos restantes son iguales a esos,
                     \\ $\therefore$  y  tienen los lados sobre los ángulos
                     \\ iguales proporcionales y son por lo tanto similares.

De la misma manera se puede demostrar que los
                     \\ paralelogramos  y  son similares.

Ya que, por lo tanto, cada uno de los paralelogramos
                     \\  y  es similar a ,
 \\ son similares entre sí.

Q. E. D. \byref{prop:I.XLVI} \byref{prop:VI.XXIX} \byref{prop:I.XXXI} \byref{def:VI.III}
\end{center}

\qed

\starttheorem{Prop. XXXI. theor.}\label{prop:VI.XXXI}

\defineNewPicture[1/2]{
pair A, B, C, D, E, F, G, H, K, L;
numeric a, r, l[];
a := -125;
A := (0, 0);
B := A shifted (dir(a)*2u);
C = whatever[A, A shifted dir(a+90)] = whatever[B, B shifted dir(0)];
D = whatever[A, A shifted dir(-90)] = whatever[B, C];
l1 := abs(A-B);
l2 := abs(B-C);
l3 := abs(C-A);
r := 1/4;
E := A shifted (dir(a-90)*l1*r);
F := B shifted (dir(a-90)*l1*r);
G := B shifted (dir(-90)*l2*r);
H := C shifted (dir(-90)*l2*r);
K := C shifted (dir(a + 180)*l3*r);
L := A shifted (dir(a + 180)*l3*r);
draw byPolygon.AB(A,B,F,E)(byblue);
draw byPolygon.BC(B,C,H,G)(byred);
draw byPolygon.CA(C,A,L,K)(byyellow);
draw byLine(A, D, byblack, SOLID_LINE, REGULAR_WIDTH);
byLineDefine(A, B, byyellow, SOLID_LINE, REGULAR_WIDTH);
byLineDefine(B, D, byblue, DASHED_LINE, REGULAR_WIDTH);
byLineDefine(D, C, byblue, SOLID_LINE, REGULAR_WIDTH);
byLineDefine(C, A, byred, SOLID_LINE, REGULAR_WIDTH);
draw byNamedLineSeq(1)(AB,BD,DC,CA);
byPointLabelRemove(H, G, L, K, F, E);
draw byLabelsOnPolygon(B, F, E, A, L, K, C, H, G)(ALL_LABELS, 0);
draw byLabelsOnPolygon(A, D, C)(OMIT_FIRST_LABEL+OMIT_LAST_LABEL, 0);
}
\drawCurrentPictureInMargin
\problem{X}{X}{Para trazar una figura rectilínea que será similar a una figura rectilínea dada (\drawLine[bottom]{CA,DC,BD,AB}), e igual a otra (\drawFromCurrentPicture[bottom][figBC]{
startGlobalRotation(180-angle(B-C));
startAutoLabeling;
draw byNamedPolygon(BC);
stopAutoLabeling;
stopGlobalRotation;
}).}

\begin{center}
Sobre \drawFromCurrentPicture[bottom][figCA]{
startGlobalRotation(-angle(C-A));
startAutoLabeling;
draw byNamedPolygon(CA);
stopAutoLabeling;
stopGlobalRotation;
} traza \drawFromCurrentPicture[bottom][figAB]{
startGlobalRotation(-angle(A-B));
startAutoLabeling;
draw byNamedPolygon(AB);
stopAutoLabeling;
stopGlobalRotation;
} $=$ ,
 \\ y sobre  traza  $=$ ,
 \\ y teniendo  $=$  (L. 1. pr. 45), y entonces
                     \\  y  se ubicarán en la misma línea recta (L. 1. prs. 29, 14),

Entre  y  encuentra una media proporcional
                     \\  (L. 6. pr. 13), y sobre   \\ traza , similar a , y colocado de manera similar.

Entonces  $=$ .

Ya que  y  son similares, y
                     \\  :  ::  :  (const.),
                     \\  :  ::  :  (L. 6. pr. 20);
                     \\ pero  :  ::  :  (L. 6. pr. 1);
                     \\ $\therefore$  :  ::  :  (L. 5. pr. 11);
                     \\ pero  $=$  (const.),
                     \\ y $\therefore$  $=$  (L. 5. pr. 14);
                     \\ y  $=$  (const.); consecuentemente,
                     \\  que es similar a  es también $=$ .

Q. E. D. \byref{prop:VI.VIII} \byref{prop:VI.XX} \byref{prop:VI.XX}
\end{center}

\qed

\starttheorem{Prop. XXXII. theor.}\label{prop:VI.XXXII}

\defineNewPicture[1/2]{
pair A, B, C, D, E;
numeric s;
B := (0, 0);
C := (5/2u, 0);
A := (u, 2u);
s := 3/5;
D := (A scaled s) shifted C;
E := (C scaled s) shifted C;
byAngleDefine(C, A, B, byyellow, SOLID_SECTOR);
byAngleDefine(A, B, C, byred, SOLID_SECTOR);
byAngleDefine(B, C, A, byblue, SOLID_SECTOR);
byAngleDefine(E, D, C, byblack, SOLID_SECTOR);
byAngleDefine(D, C, E, byblack, ARC_SECTOR);
byAngleDefine(A, C, D, byyellow, SOLID_SECTOR);
draw byNamedAngleResized();
byLineDefine(A, B, byblue, SOLID_LINE, REGULAR_WIDTH);
byLineDefine(B, C, byyellow, SOLID_LINE, REGULAR_WIDTH);
byLineDefine(C, E, byyellow, DASHED_LINE, REGULAR_WIDTH);
byLineDefine(E, D, byred, DASHED_LINE, REGULAR_WIDTH);
byLineDefine(D, C, byblue, DASHED_LINE, REGULAR_WIDTH);
byLineDefine(C, A, byred, SOLID_LINE, REGULAR_WIDTH);
draw byNamedLineSeq(0)(ED,DC,noLine,CA,AB,BC,CE);
byLineDefine.ABt(A, B, byblack, SOLID_LINE, THIN_WIDTH);
byLineDefine.BCt(B, C, byblack, SOLID_LINE, THIN_WIDTH);
byLineDefine.CAt(C, A, byblack, SOLID_LINE, THIN_WIDTH);
byLineDefine.DCt(D, C, byblack, SOLID_LINE, THIN_WIDTH);
byLineDefine.CEt(C, E, byblack, SOLID_LINE, THIN_WIDTH);
byLineDefine.EDt(E, D, byblack, SOLID_LINE, THIN_WIDTH);
draw byLabelsOnPolygon(E, C, B, A, C, D)(ALL_LABELS, 0);
}
\drawCurrentPictureInMargin
\problem{X}{X}{Si los paralelogramos similares y con una posición similar (\drawFromCurrentPicture[bottom]{
startTempAngleScale(angleScale*3/5);
draw byNamedAngle(A, B, BCA);
draw byNamedLineSeq(0)(CAt,BCt,ABt);
draw byLabelsOnPolygon(C, B, A)(ALL_LABELS, 0);
stopTempAngleScale;
} y \drawFromCurrentPicture[bottom]{
startTempAngleScale(angleScale*3/5);
draw byNamedAngle(D, DCE);
draw byNamedLineSeq(0)(EDt,CEt,DCt);
draw byLabelsOnPolygon(C, D, E)(ALL_LABELS, 0);
stopTempAngleScale;
}) tienen un ángulo común, entonces están sobre la misma diagonal.}

\begin{center}
Porque, si es posible, deja que \drawUnitLine{AB}  \\ sea la diagonal de \drawUnitLine{CA} y
                     \\ dibuja \drawUnitLine{DC} ∥ \drawUnitLine{ED} (L. 1. pr. 31).

Ya que \drawUnitLine{BC} y \drawUnitLine{CE} están sobre la misma
                     \\ diagonal \drawUnitLine{AB}, y tienen \drawUnitLine{DC} común,
                     \\ son similares (L. 6. pr. 24);

$\therefore$ \drawAngle{A} : \drawAngle{ACD} :: \drawUnitLine{CA} : \drawUnitLine{ED};
 \\ pero \drawAngle{ACD} : \drawAngle{D} :: \drawAngle{A} : \drawAngle{D} (hip.),
                     \\ $\therefore$ \drawUnitLine{AB} : \drawUnitLine{CA} :: \drawUnitLine{DC} : \drawUnitLine{ED},
 \\ y $\therefore$ \drawAngle{B} $=$ \drawAngle{DCE} (L. 5. pr. 9.),
                     \\ lo que es absurdo.

$\therefore$ \drawAngle{A} no es la diagonal de \drawAngle{ACD}  \\ de la misma manera se puede demostrar que ninguna otra
                     \\ línea es excepto \drawAngle{BCA}.

Q. E. D. \drawAngle{ACD} \drawAngle{DCE} \drawAngle{BCA} \drawAngle{A} \drawAngle{B} \drawTwoRightAngles \drawUnitLine{BC} \drawUnitLine{CE} \byref{prop:I.XXIX} \byref{prop:I.XXIX} \byref{\hypref} \byref{prop:VI.VI} \byref{prop:I.XXXII} \byref{prop:I.XIV}
\end{center}

\qed

\starttheorem{Prop. XXXIII. theor.}\label{prop:VI.XXXIII}

\defineNewPicture[1/2]{
pair A, B, C, D, E, F, G, H, K, L, M, N;
numeric r, a[], ba, aa, q;
path c[];
q := 8/360;
r := 9/4u;
aa := 125;
ba := 205;
a1 := 30;
a2 := 35;
G := (0, 0);
A := (dir(aa)*r) shifted G;
B := (dir(ba)*r) shifted G;
C := (dir(ba + a1)*r) shifted G;
K := (dir(ba + 2a1)*r) shifted G;
L := (dir(ba + 3a1)*r) shifted G;
byAngleDefine(B, A, C, byyellow, SOLID_SECTOR);
byAngleDefine.BC(B, G, C, byblack, SOLID_SECTOR);
byAngleDefine.CK(C, G, K, byred, SOLID_SECTOR);
byAngleDefine.KL(K, G, L, byblue, SOLID_SECTOR);
draw byNamedAngleResized(BAC, BC, CK, KL);
draw byLine(A, B, byblack, SOLID_LINE, THIN_WIDTH);
draw byLine(A, C, byblack, SOLID_LINE, THIN_WIDTH);
draw byLine(G, C, byblack, SOLID_LINE, THIN_WIDTH);
draw byLine(G, K, byblack, SOLID_LINE, THIN_WIDTH);
byLineDefine(G, B, byblack, SOLID_LINE, THIN_WIDTH);
byLineDefine(G, L, byblack, SOLID_LINE, THIN_WIDTH);
draw byNamedLineSeq(0)(GB,GL);
byArcDefine.LB(G, L, B, r, byred, 0, 0, 0, 0);
byArcDefine.BC(G, B, C, r, byblack, 0, 0, 0, 0);
byArcDefine.CK(G, C, K, r, byred, 0, 0, 0, 0);
byArcDefine.KL(G, K, L, r, byblue, 0, 0, 0, 0);
draw byNamedArcSeq(0)(LB, BC, CK, KL);
byCircleDefineR(G, r, byred, 0, 0, 0);
H := (0, -1/2u - 2r);
D := (dir(aa)*r) shifted H;
E := (dir(ba)*r) shifted H;
F := (dir(ba + a2)*r) shifted H;
M := (dir(ba + 2a2)*r) shifted H;
N := (dir(ba + 3a2)*r) shifted H;
byAngleDefine(E, D, F, byyellow, ARC_SECTOR);
byAngleDefine.EF(E, H, F, byblack, ARC_SECTOR);
byAngleDefine.FM(F, H, M, byred, ARC_SECTOR);
byAngleDefine.MN(M, H, N, byblue, ARC_SECTOR);
draw byNamedAngleResized(EDF, EF, FM, MN);
draw byLine(D, E, byblack, SOLID_LINE, THIN_WIDTH);
draw byLine(D, F, byblack, SOLID_LINE, THIN_WIDTH);
draw byLine(H, F, byblack, SOLID_LINE, THIN_WIDTH);
draw byLine(H, M, byblack, SOLID_LINE, THIN_WIDTH);
byLineDefine(H, E, byblack, SOLID_LINE, THIN_WIDTH);
byLineDefine(H, N, byblack, SOLID_LINE, THIN_WIDTH);
draw byNamedLineSeq(0)(HE,HN);
byArcDefine.NE(H, N, E, r, byblue, 0, 0, 0, 0);
byArcDefine.EF(H, E, F, r, byyellow, 1, 0, 0, 0);
byArcDefine.FM(H, F, M, r, byred, 1, 0, 0, 0);
byArcDefine.MN(H, M, N, r, byblue, 1, 0, 0, 0);
draw byNamedArcSeq(0)(NE, EF, FM, MN);
byCircleDefineR(H, r, byblue, 0, 0, 0);
draw byLabelsOnCircle(B, C, K, L)(G);
draw byLabelsOnCircle(E, F, M, N)(H);
draw byLabelsOnPolygon(B, G, L)(OMIT_FIRST_LABEL+OMIT_LAST_LABEL, 0);
draw byLabelsOnPolygon(E, H, N)(OMIT_FIRST_LABEL+OMIT_LAST_LABEL, 0);
}
\drawCurrentPictureInMargin
\problem{D}{e}{ todos los rectángulos contenidos por los segmentos de una línea recta dada, el mayor es el cuadrado que se traza en la mitad de la línea.}

\begin{center}
Sea \drawAngle{BC} la línea dada, \drawAngle{EF} y \drawFromCurrentPicture[middle][arcBC]{
startGlobalRotation(180-angle(B-C));
startAutoLabeling;
draw byNamedArc(BC);
stopAutoLabeling;
stopGlobalRotation;
} segmentos desiguales, y \drawFromCurrentPicture[middle][arcEF]{
startGlobalRotation(180-angle(E-F));
startAutoLabeling;
draw byNamedArc(EF);
stopAutoLabeling;
stopGlobalRotation;
} y \drawFromCurrentPicture[middle][arcCK]{
startGlobalRotation(180-angle(C-K));
startAutoLabeling;
draw byNamedArc(CK);
stopAutoLabeling;
stopGlobalRotation;
} segementos iguales;
                 \\ entonces \drawFromCurrentPicture[middle][arcKL]{
startGlobalRotation(180-angle(K-L));
startAutoLabeling;
draw byNamedArc(KL);
stopAutoLabeling;
stopGlobalRotation;
} > \drawFromCurrentPicture[middle][arcFM]{
startGlobalRotation(180-angle(F-M));
startAutoLabeling;
draw byNamedArc(FM);
stopAutoLabeling;
stopGlobalRotation;
}.

Ya se ha demostrado (L. 2. pr. 5), que el cuadrado de la mitad de la línea es igual al rectángulo contenido por cualquier segmento desigual junto con el cuadrado de la parte intermedia entre el punto medio y el punto de la sección desigual. El cuadrado trazado en la mitad de la línea excede, por lo tanto, el rectángulo contenido por cualquier segmento desigual de la línea.

Q. E. D. \drawFromCurrentPicture[middle][arcMN]{
startGlobalRotation(180-angle(M-N));
startAutoLabeling;
draw byNamedArc(MN);
stopAutoLabeling;
stopGlobalRotation;
} \drawAngle{BC} \drawAngle{CK} \drawAngle{KL} \drawAngle{BC,CK,KL} \drawAngle{BC} \drawFromCurrentPicture[middle][arcBL]{
startGlobalRotation(180-angle(B-L));
startAutoLabeling;
draw byNamedArcSeq(0)(BC,CK,KL);
stopAutoLabeling;
stopGlobalRotation;
} \drawAngle{EF,FM,MN} \drawAngle{EF} \drawFromCurrentPicture[middle][arcEN]{
startGlobalRotation(180-angle(E-N));
startAutoLabeling;
draw byNamedArcSeq(0)(EF,FM,MN);
stopAutoLabeling;
stopGlobalRotation;
} \drawAngle{BC,CK,KL} \drawAngle{BC} \drawAngle{EF,FM,MN} \drawAngle{EF} \drawAngle{BC} \drawAngle{EF} \drawAngle{DCB,FCD} \drawFromCurrentPicture[bottom]{
startTempScale(1/2);
startAutoLabeling;
draw byNamedLineSeq(0)(AE,EC,BC,AB);
stopAutoLabeling;
stopTempScale;
} \drawAngle{DCB} \drawAngle{EBC} \drawAngle{FCD} \drawAngle{E} \drawAngle{CAE} \drawAngle{EAB} \drawLine[bottom]{CA,DC,BD,AB} \drawAngle{CAE} \drawAngle{EAB} \drawAngle{B} \drawAngle{E} \drawLine[bottom]{AD,BD,AB} \drawLine[middle]{CA,CE,DE,AD} \drawAngle{A} \drawLine[bottom]{CA,DC,BD,AB} \drawUnitLine{AD} \drawUnitLine{BD,DC} \drawUnitLine{AB} \drawUnitLine{CA} \drawUnitLine{AD} \drawUnitLine{AE} \drawUnitLine{CE} \drawAngle{D} \drawAngle{BCA,ECB} \drawAngle{B} \drawAngle{E} \drawLine[bottom]{AD,BD,AB} \drawLine[middle]{CA,CE,AE} \drawUnitLine{AB} \drawUnitLine{AD} \drawUnitLine{AE} \drawUnitLine{CA} \drawUnitLine{AB} \drawUnitLine{CA} \drawUnitLine{AD} \drawUnitLine{AE} \drawLine{DA,CD,BC,AB} \drawUnitLine{BE,ED} \drawUnitLine{AC} \drawUnitLine{BE,ED} \drawUnitLine{AC} \drawUnitLine{AB} \drawUnitLine{CD} \drawUnitLine{DA} \drawUnitLine{BC} \drawAngle{EAB} \drawAngle{DAC} \drawAngle{EAB,CAE} \drawAngle{CAE,DAC} \drawAngle{BCA} \drawAngle{D} \drawUnitLine{DA} \drawUnitLine{ED} \drawUnitLine{AC} \drawUnitLine{BC} \drawUnitLine{ED} \drawUnitLine{AC} \drawUnitLine{DA} \drawUnitLine{BC} \drawAngle{EAB} \drawAngle{DAC} \drawAngle{B} \drawAngle{ACD} \drawUnitLine{AB} \drawUnitLine{BE} \drawUnitLine{AC} \drawUnitLine{CD} \drawUnitLine{BE} \drawUnitLine{AC} \drawUnitLine{AB} \drawUnitLine{CD} \drawUnitLine{ED} \drawUnitLine{AC} \drawUnitLine{DA} \drawUnitLine{BC} \drawUnitLine{BE,ED} \drawUnitLine{AC} \drawUnitLine{AB} \drawUnitLine{CD} \drawUnitLine{DA} \drawUnitLine{BC} \byref{prop:III.XXVII} \byref{prop:III.XXVII} \byref{def:V.V} \byref{prop:III.XX} \byref{prop:V.XV} \byref{prop:I.IV,prop:III.XXIV,prop:III.XXVII,def:III.IX} \byref{prop:I.XXIX} \byref{\hypref} \byref{prop:I.XXIX} \byref{prop:I.VI} \byref{prop:V.VII} \byref{prop:VI.II} \byref{prop:V.XI} \byref{prop:IV.V} \byref{\hypref} \byref{prop:III.XXI} \byref{prop:I.XXXII} \byref{prop:VI.IV} \byref{prop:VI.XVI} \byref{prop:II.III} \byref{prop:III.XXXV} \byref{prop:IV.V} \byref{\constref,prop:III.XXXI} \byref{prop:III.XXI} \byref{prop:VI.IV} \byref{prop:VI.XVI} \byref{prop:I.XXIII} \byref{prop:III.XXI} \byref{prop:VI.IV} \byref{prop:VI.XVI} \byref{\constref} \byref{prop:III.XXI} \byref{prop:VI.IV} \byref{prop:VI.XVI} \byref{prop:II.I}
\end{center}

\qed

